%%% Attachments to the master thesis, if any. Each attachment must be
%%% referred to at least once from the text of the thesis. Attachments
%%% are numbered.
%%%
%%% The printed version should preferably contain attachments, which can be
%%% read (additional tables and charts, supplementary text, examples of
%%% program output, etc.). The electronic version is more suited for attachments
%%% which will likely be used in an electronic form rather than read (program
%%% source code, data files, interactive charts, etc.). Electronic attachments
%%% should be uploaded to SIS and optionally also included in the thesis on a~CD/DVD.
%%% Allowed file formats are specified in provision of the rector no. 72/2017.
\appendix
\chapter{Attachments}

Contents of the attachment archive:

\dirtree{%
  .1 {/}.
  .2 {bin}\mydtcomment{Prepared built binaries}.
  .3 {linux-x64}\mydtcomment{Built binaries for Linux}.
  .4 {dotnet}\mydtcomment{Patched \dotnet{}~6 SDK with managed QUIC implementation}.
  .4 {msquic}\mydtcomment{\libmsquic{} binaries used for performance comparison in \autoref{chap:04-evaluation}}.
  .4 {openssl}\mydtcomment{Modified \libopenssl{} binaries with QUIC-enabling API}.
  .4 {ThroughputTests}\mydtcomment{Self-contained build of benchmarking application from \autoref{chap:04-evaluation}}.
  .3 {win-x64}\mydtcomment{Built binaries for Windows}.
  .4 {dotnet}\mydtcomment{Patched \dotnet{}~6 SDK with managed QUIC implementation}.
  .4 {msquic}\mydtcomment{\libmsquic{} binaries used for performance comparison in \autoref{chap:04-evaluation}}.
  .4 {openssl}\mydtcomment{Modified \libopenssl{} binaries with QUIC-enabling API}.
  .4 {ThroughputTests}\mydtcomment{Self-contained build of benchmarking application from \autoref{chap:04-evaluation}}.
  .2 {certs}\mydtcomment{Certificates for use in tests and attached samples}.
  .3 {cert.crt}\mydtcomment{Public certificate file}.
  .3 {cert.key}\mydtcomment{Private key file}.
  .3 {cert.pfx}\mydtcomment{Certificate and private key combined in PFX format}.
  .2 {extern}\mydtcomment{Third-party source code}.
  .3 {openssl}\mydtcomment{Copy of the QUIC-enabled fork of \libopenssl{} by Akamai}.
  .2 {src}\mydtcomment{Root for all source code}.
  .3 {dotnet-runtime}\mydtcomment{\dotnet{} runtime fork with managed QUIC implementation}.
  .3 {supplementary}\mydtcomment{Root of additional \dotnet{} projects}.
  .4 {benchmark}\mydtcomment{Root of all benchmark projects}.
  .5 {ThroughputTests}\mydtcomment{Benchmarking application used in \autoref{chap:04-evaluation}}.
  .6 {run.ps1}\mydtcomment{Script used to automate running the benchmarking application}.
  .4 {samples}\mydtcomment{Root of all sample projects}.
  .5 {Echo}\mydtcomment{Simple echo server and client implementation from \autoref{chap:06-user-docs}}.
  .5 {hello-net6.0}\mydtcomment{Hello world application for testing \dotnet{}~6 installation}.
  .4 {ManagedQuic.sln}\mydtcomment{\dotnet{} Solution file for all projects in subdirectory}.
  .2 {tex}\mydtcomment{\LaTeX{} source code for this thesis}.
  .3 {measurements}\mydtcomment{Raw CSV data with measurements presented in \autoref{chap:04-evaluation}}.
  .4 {linux-lan}\mydtcomment{Measurements in the Linux LAN environment}.
  .4 {windows-loopback}\mydtcomment{Measurements in the Windows Loopback environment}.
  .2 {README.md}\mydtcomment{File describing the contents of the archive}.
}
