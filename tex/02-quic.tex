\chapter{QUIC Protocol}\label{chap:02-quic}

This chapter is intended to summarize the QUIC protocol specification and provide sufficient
knowledge about the protocol needed for designing our implementation. This text is based on version
27 of the draft specification documents from February 2020, more specifically on the documents
describing the core transport protocol~\cite{draft-ietf-quic-transport}, TLS
integration~\cite{draft-ietf-quic-tls}, and congestion control
mechanism~\cite{draft-ietf-quic-recovery}. Readers familiar with these documents may skip this
chapter.

We will start this chapter by first providing a high-level overview of QUIC and then providing a
more detailed description of the individual parts of the protocol.

\section{Introduction}

QUIC protocol provides reliable and secure transport of multiple streams of data over a single
connection. QUIC provides the following services:

\begin{itemize}

  \item Stream multiplexing

  \item Stream and connection-level flow control

  \item Low-latency connection establishment

  \item Connection migration and resilience to NAT rebinding

\end{itemize}

QUIC is implemented on top of UDP, which provides only unreliable transfer of datagrams. Therefore,
in addition to stream multiplexing, QUIC also implements loss recovery, congestion control,
transport security, and other features known from TCP or TLS protocols.

\subsection{Basic Concepts}

Throughout this chapter, and later when analyzing the QUIC protocol specification, this text will
use several terms in a very specific meaning. This meaning is the same as defined in the main
specification document~\autocite[Section~1.2]{draft-ietf-quic-transport}.

\begin{description}

  \ditem{\Gls{quic-packet}} A complete processable unit of QUIC that is transported using UDP
datagrams. Multiple QUIC packets can be encapsulated in a single UDP datagram but a single QUIC
packet may not be split into multiple UDP datagrams.

  \ditem{\Gls{out-of-order-packet}} A packet that does not arrive directly after the packet that was
sent before it. A packet can arrive out of order if it was delayed, if earlier packets were lost or
delayed, or if the sender intentionally skipped a packet number.

  \ditem{Endpoint} An entity that can participate in a QUIC connection by generating, receiving, and
processing QUIC packets. There are only two types of endpoints in QUIC\@: client and server.

  \ditem{Client} The endpoint initiating a QUIC connection.

  \ditem{Server} The endpoint accepting incoming QUIC connections.

  \ditem{Address} When used without qualification, an address is a tuple of IP version, IP address,
and UDP port number.

  \ditem{\Gls{network-path}} A pair of two network address --- one for each endpoint --- used to
exchange QUIC packets in a connection. Throughout its lifetime, a QUIC connection can change to a
different network-path, e.g., by a process called connection migration.

  \ditem{\Gls{cid}} An opaque identifier that is used to identify a QUIC connection at an endpoint.
Each endpoint sets a value for its peer to include in packets sent towards the endpoint.

  \ditem{Stream} A unidirectional or bidirectional channel of ordered bytes within a QUIC
connection. A QUIC connection can carry multiple streams simultaneously.

  \ditem{Application} An entity that uses QUIC to send and receive data. This can mean, e.g., a
higher-level protocol built on top of QUIC or an actual application which uses QUIC directly.

\end{description}

\subsection{Notational Conventions}

This chapter contains graphical diagrams of several elements, such as QUIC packets. In those
schematics, individual fields include length information as follows:

\begin{itemize}

  \litem{X (A)} Indicates that X is A bits long.

  \litem{X (i)} Literal \textit{i} indicates that X is encoded using the QUIC's variable-length
encoding. This encoding is described in \autoref{sec:02-variable-length-encoding}.

  \litem{X (A..B)} Indicates that X can be any length from A to B. The actual length is generally
stored in a different field.

  \litem{X (..)} Indicates that X can be any length. The actual length is generally stored in a
different field.

  \litem{X (A) = C} Indicates that X has a fixed value C.

  \litem{X (A) ...} Indicates that X is repeated zero or more times and each instance is A bits
long.

\end{itemize}

\subsection{Wire Encoding}\label{sec:02-wire-encoding}

The process of encoding QUIC packets to be sent via the network is optimized for size. There are two
nontrivial encodings used: variable-length integer encoding and packet number encoding. All values
sent over the network are encoded in big-endian order.

\subsubsection{Variable-Length Integer Encoding}\label{sec:02-variable-length-encoding}

Almost all numeric values stored in QUIC packets are encoded using a \textit{variable-length integer
encoding}. This encoding uses the first byte's two most significant bits to encode whether the value
is encoded as 1, 2, 4, or 8-byte integer. This encoding supports only positive numbers. The ranges
available for individual encoding lengths are listed in \autoref{tab:02-quic-varint-length}.

\begin{myTable} {tab:02-quic-varint-length} {Variable-length integer encoding lengths} {ccr}
  {Most significant bits & Encoding length (B) & Maximum value}
  00                     & 1                   & \num{63}         \\
  01                     & 2                   & \num{16383}      \\
  10                     & 4                   & \num{1073741824} \\
  11                     & 8                   & $2^{62}-1$       \\
\end{myTable}

In order to optimize the size of QUIC packets, the implementations are encouraged to always choose
the shortest encoding necessary to represent the given number.

\subsubsection{Packet Number Encoding}\label{sec:02-packet-number-encoding}

Packets in QUIC are sequentially numbered, and, therefore, packet numbers received by an endpoint
form a mostly ascending sequence with occasional reordering or gaps due to network unreliability.
This causes the packet number of a newly received packet to be close to that of the previously
received packet. QUIC leverages this fact by sending only the lower bytes of the packet number. The
number of bytes sent varies between one to four bytes and is always chosen so that the receiver can
reconstruct the original packet number using the previously received packet numbers and the bytes
from the arriving packet.

When determining how many bytes of the packet number need to be included in the packet, the sender
uses the highest acknowledged packet number. The encoding length must use enough bytes to be able to
represent at least twice the difference between the current packet number and the highest
acknowledged number.

As an example, consider the following situation: sender prepares to send packet \binary{4417} and
its highest acknowledged packet number is \binary{43a0}. The difference between the two numbers is
\binary{77} which is smaller than 128 and, therefore, requires only the least significant byte
(\binary{17}) of the packet number to be sent in the packet. Suppose that by the time the packet
arrives at the receiver, the highest packet number it previously received was \binary{43f5}. Based
on the encoded packet number length, the deconstructed packet number must be from the range
\binary{4375}--\binary{4475}. The only number from this range that matches the least significant
byte sent in the packet is \binary{4417} which is the correct packet number.

\section{QUIC Packets}

In QUIC connection, endpoints exchange \textit{QUIC packets} enclosed in UDP datagrams. A single UDP
datagram can contain multiple packets, although in most cases, they contain only one. Packing
multiple QUIC packets into a single UDP datagram is called \textit{coalescing}. A single QUIC packet
cannot span multiple UDP datagrams. QUIC uses a total of six different packet types:

\begin{enumerate}

  \litem{Version Negotiation} sent by the server when the client tries to establish a connection
with QUIC version that is not supported by the server;

  \litem{Retry} optionally used by servers for Client Address Validation when establishing new
connections, address validation is described in \autoref{sec:02-address-validation};

  \litem{Initial} used to initiate a new connection and exchange the initial information;

  \litem{Handshake} used during the connection handshake;

  \litem{0-RTT} when TLS 1.3 0-RTT mode of operation is enabled, 0-RTT packets carry \textit{early
data} --- application data that is sent before the TLS handshake is complete in order to reduce
latency; and

  \litem{1-RTT} main packet type used throughout the lifetime of QUIC connection. 1-RTT and 0-RTT
packets are the only packet types that carry application data.

\end{enumerate}

Version Negotiation and Retry packets are sent as one-time responses in particular scenarios and are
not individually numbered. QUIC is designed to let the server send these packets without maintaining
any state for the connection. These packets are not used in an established connection.

Packets of the other types --- Initial, Handshake, 0-RTT, and 1-RTT --- are individually numbered. The
packet numbers do not form a single sequence, as is the case with TCP packets. Instead, QUIC
organizes these packets into three separate \textit{packet number spaces} and uses a separate
sequence of numbers for each packet number space. These packet number spaces are:

\begin{enumerate}

  \litem{Initial} Used for initiating new connections and exchanging initial cryptographic
information. Contains only Initial packets.

  \litem{Handshake} Used during the connection handshake process. Contains only Handshake packets.

  \litem{Application} Used throughout the lifetime of the connection to transfer application data.
This packet space contains both 0-RTT and 1-RTT packets.

\end{enumerate}

In addition to being numbered separately, packets from each packet number space are processed
entirely independently of the other packet number spaces. For example, Initial packets can be
acknowledged only by another Initial packet. The four packet types also use different keys for
encryption and offer increasingly stronger level of protection, with Initial packets being the least
secure and 1-RTT packets the most secure. In their payload, the Initial, Handshake, 0-RTT, and 1-RTT
packets carry \textit{QUIC Frames} which are low-level protocol messages carrying, e.g.,
acknowledgments and parts of streams sent by the application.

Each packet type has a header and a payload. The packet header contains information such as
\gls{scid} and \gls{dcid} which are \glspl{cid} used to match the packets to the correct receiving
connection. QUIC optimizes the packet encoding to maximize the amount of application data sent in a
single UDP datagram. Therefore, two types of packet headers exist. 1-RTT packets use a \textit{short
  header}, and all other packet types use the \textit{long header}. The long header contains fields
like the protocol version identifier which are relevant only during connection establishment. The
short header reduces the packet size by including only the \gls{dcid}\@.

The type of the header is determined by the most significant bit of the first byte in the packet ---
the \textit{Header Form} bit. The second most significant bit --- the \textit{Fixed Bit} --- is
always set to 1 in a valid QUIC packet. The following sections describe the structure of the rest of
the packet for individual packet types.

\newcommand{\longFieldHeight}{1}

\subsection{Long Packet Header}

Long headers contain information that is necessary for the connection establishment. It includes
both the Source and Destination Connection IDs. It also contains \textit{Version} and \textit{Packet
Type} fields. A few leftover bits of the header are reserved for type-specific information.
\autoref{fig:02-long-header} illustrates the structure of the long header.

\begin{myFigure}{fig:02-long-header}{Long packet header structure}

  \begin{bytefield}[bitwidth=2.5em]{8}
    \bitbox{1}{H=1} & \bitbox{1}{F=1} & \bitbox{2}{T (2)} & \bitbox{4}{Type Specific Bits (4)} \\
    \wordbox[tlrb]{\longFieldHeight}{Version (32)} \\
    \wordbox[tlrb]{1}{Destination Connection ID Length (8)} \\
    \wordbox[tlrb]{\longFieldHeight}{Destination Connection ID (..) } \\
    \wordbox[tlrb]{1}{Source Connection ID Length (8)} \\
    \wordbox[tlrb]{\longFieldHeight}{Source Connection ID (..) } \\
  \end{bytefield}

  H = Header Form, F = Fixed Bit, T = Packet Type

\end{myFigure}

The semantics of the individual fields of the long header is as follows:

\begin{description}

    \ditem{Header Form (H)} Used to distinguish between packets with so-called \textit{long header}
and \textit{short header} format. In the case of the long header packets. The Header Form bit is set
to 1.

    \ditem{Fixed Bit (F)} A bit that is always set to 1 in a valid QUIC packet.

    \ditem{Packet Type (T)} Discriminator of the packet type. the possible values are listed in
\autoref{tab:02-packet-type}.

\begin{myTable}{tab:02-packet-type}{Values of the Packet Type field in long packet header} {cc}
  {Packet Type & Value}
  Initial      & 00 \\
  0-RTT        & 01 \\
  Handshake    & 10 \\
  Retry        & 11 \\
\end{myTable}

    \ditem{Version} Indicates which version of QUIC is in use. The location of this field will be
the same across all QUIC versions. However, the structure of the rest of the packet may be different
in future versions of QUIC.

    \ditem{Destination Connection ID Length} Length of the Destination Connection ID field.

    \ditem{Destination Connection ID} The Connection ID issued by the recipient of the packet.

    \ditem{Source Connection ID Length} Length of the Source Connection ID field.

    \ditem{Source Connection ID} The Connection ID issued by the sender of the packet.

    \ditem{Reserved Bits} Bits reserved for use in the future QUIC versions. In the initial QUIC
version, these bits must be set to 0.

\end{description}

\subsection{Version Negotiation Packet}\label{sec:02-version-negotiation-packet}

The Version Negotiation packet is sent by the server when it receives a long header packet
requesting a version of QUIC which is not supported by the server. As an exception to other long
header packets, the Version Negotiation packet is not discriminated by a specific value in the
Packet Type field in the header, but by a special value \binary{00000000} in the Version field.

After the header, the packet contains a list of supported versions, each one listed as a 32-bit
integer in big-endian. Also, since future QUIC versions may allow larger \glspl{cid} than 20 bytes,
a valid Version Negotiation packet can contain up to 255-byte \glspl{cid}. The structure of the
Version Negotiation packet is illustrated in \autoref{fig:02-version-negotiation-packet}.

\begin{myFigure}{fig:02-version-negotiation-packet}{Version Negotiation packet structure}

  \begin{bytefield}[bitwidth=2.5em]{8}
    \begin{rightwordgroup}{long \\ header}
      \bitbox{1}{H=1} & \colorbitbox{bytefieldunused}{7}{Unused} \\
      \wordbox[tlrb]{\longFieldHeight}{Version (32) = \binary{00000000}} \\
      \wordbox[tlrb]{1}{Destination Connection ID Length (8)} \\
      \wordbox[tlrb]{\longFieldHeight}{Destination Connection ID (0..2040) } \\
      \wordbox[tlrb]{1}{Source Connection ID Length (8)} \\
      \wordbox[tlrb]{\longFieldHeight}{Source Connection ID (0..2040) }
    \end{rightwordgroup} \\
    \wordbox[tlrb]{\longFieldHeight}{Supported Version 1 (32)} \\
    \wordbox[tlrb]{\longFieldHeight}{Supported Version 2 (32)} \\
    \wordbox[tlrb]{1}{...} \\
  \end{bytefield}

  H = Header Form

\end{myFigure}

The identifier of the initial QUIC version is \binary{00000001}. There are also unique identifiers
for draft versions of the protocol valid only until the QUIC specification is finalized.

\subsection{Retry Packet}\label{sec:02-retry-packet}

Retry packets are distinguished by bits 11 in the \textit{Packet Type} field of the long header. A
server sends Retry packets as part of the optional \textit{Address Validation} mechanism used to
protect against traffic amplification attack described in \autoref{sec:02-address-validation}.
\autoref{fig:02-retry-packet} illustrates the structure of the Retry packet.

\begin{myFigure}{fig:02-retry-packet}{Retry packet structure}

  \begin{bytefield}[bitwidth=2.5em]{8}
    \begin{rightwordgroup}{long \\ header}
      \bitbox{1}{H=1} & \bitbox{1}{F=1} & \bitbox{2}{T = 11} & \colorbitbox{bytefieldunused}{4}{Unused} \\
      \wordbox[tlrb]{\longFieldHeight}{Version (32)} \\
      \wordbox[tlrb]{1}{Destination Connection ID Length (8)} \\
      \wordbox[tlrb]{\longFieldHeight}{Destination Connection ID (0..160) } \\
      \wordbox[tlrb]{1}{Source Connection ID Length (8)} \\
      \wordbox[tlrb]{\longFieldHeight}{Source Connection ID (0..160) }
    \end{rightwordgroup} \\
    \wordbox[tlrb]{\longFieldHeight}{Retry Token (..)} \\
    \wordbox[tlrb]{\longFieldHeight}{Retry Integrity Tag (128)} \\
  \end{bytefield}

  H = Header Form, F = Fixed Bit, T = Packet Type

\end{myFigure}

The semantics of the fields specific to the Retry packet are:

\begin{description}

    \ditem{Retry Token} Contains an opaque token generated by the server. This token must be echoed
back in an Initial packet during the next connection attempt. The token must be difficult to guess
by the attacker and verifiable in a stateless manner, i.e., without saving it in memory for future
comparisons.

    \ditem{Retry Integrity Tag} Tag used to check the integrity of the packet. Detailed information
on how the integrity tag is calculated can be found in the TLS integration
specification~\autocite[Section~5.8]{draft-ietf-quic-tls}.

\end{description}

\subsection{Initial, Handshake, and 0-RTT Packets}

Initial, Handshake, and 0-RTT packets are almost identical in structure. All three types use the
\textit{Type-Specific Bits} from the long header to store \textit{Reserved Bits} and the
\textit{Packet Number Length}. After the long packet header, these packets contain the
\textit{Length} field containing the packet length, \textit{Packet Number} field, and the actual
payload consisting of QUIC frames. The only exception to this structure is the Initial packet, which
contains two additional fields just after the long header: \textit{Token Length} and \textit{Token},
which are used to carry the Retry Token from the Retry packet in case Address Validation is
requested by the server. \autoref{fig:02-initial-handshake-0rtt-packets} illustrates the structure
of these three packet types.

\begin{myFigure}{fig:02-initial-handshake-0rtt-packets}{Structure of the Initial, Handshake, and
0-RTT packets}

  \begin{bytefield}[bitwidth=2.5em]{8}
    \begin{rightwordgroup}{long \\ header}
      \bitbox{1}{H=1} & \bitbox{1}{F=1} & \bitbox{2}{T (2)} & \bitbox{2}{R = 00} & \bitbox{2}{L (2)} \\
      \wordbox[tlrb]{\longFieldHeight}{Version (32)} \\
      \wordbox[tlrb]{1}{Destination Connection ID Length (8)} \\
      \wordbox[tlrb]{\longFieldHeight}{Destination Connection ID (0..160) } \\
      \wordbox[tlrb]{1}{Source Connection ID Length (8)} \\
      \wordbox[tlrb]{\longFieldHeight}{Source Connection ID (0..160) }
    \end{rightwordgroup} \\
    \begin{leftwordgroup}{Initial \\ only}
      \colorbitbox{bytefieldunused}{8}{Token Length (i)} \\
      \colorwordbox{bytefieldunused}{tlrb}{\longFieldHeight}{Token (..)}
    \end{leftwordgroup} \\
    \wordbox[tlrb]{1}{Length (i)} \\
    \wordbox[tlrb]{1}{Packet Number (8..32)} \\
    \wordbox[tlrb]{\longFieldHeight}{Packet Payload (..)} \\
    \wordbox[tlrb]{\longFieldHeight}{Integrity Tag (16)} \\
  \end{bytefield}

  H = Header Form, F = Fixed Bit, T = Packet Type,\\ R = Reserved Bits, L = Packet Number Length

\end{myFigure}

The semantics of the new fields in these packets are:

\begin{description}
    \glsunset{aead}

    \ditemWithComment{Token Length}{Initial only} Length of the Token field.

    \ditemWithComment{Token}{Initial only} Contains an opaque token if the server provided one in
the Retry packet as part of Address Validation.

    \ditem{Reserved Bits (R)} Bits reserved for use in the future QUIC versions. In the initial QUIC
version, these bits must be set to 0.

    \ditem{Packet Number Length (L)} The length of the encoding used for the packet number.

    \ditem{Length} The length of the remainder of the packet. This includes the Packet Number, the
packet's payload, and the \gls{aead} integrity tag.

    \ditem{Packet Number} The sequence number of this packet in the respective packet number space.
This field uses the packet number encoding described in \autoref{sec:02-packet-number-encoding}.

    \ditem{Packet Payload} Serialized sequence of QUIC frames.

    \ditem{Integrity Tag} Opaque checksum produced by the \gls{aead} cipher during packet encryption.
Packet encryption is described in detail in \autoref{sec:02-packet-protection}.

    \glsreset{aead}
\end{description}

\subsection{1-RTT Packet}

1-RTT packets are the only packet that uses the short header to make more space for application data
in the UDP datagram. In addition to the Header Form, Fixed, Reserved Bits, and Packet Number Length
fields, which have the same meaning as in the long header, the short header contains \textit{Spin
Bit} and \textit{Key Phase Bit} fields.

1-RTT packets can be sent only after the connection has been successfully established. This implies
that Connection IDs used for the connection by both endpoints are already known, and there is no
need to repeat the \gls{scid} or specify the length of the \gls{dcid}\@. 1-RTT packets also lack the
Length field. It is assumed that 1-RTT packets fill the rest of the UDP datagram. Therefore, after
the short header, only the Packet Number field and the QUIC frame payload follows. The structure of
the 1-RTT frame is illustrated in \autoref{fig:02-1rtt-packet}.

\begin{myFigure}{fig:02-1rtt-packet}{1-RTT packet structure}

  \begin{bytefield}[bitwidth=2.5em]{8}
    \begin{rightwordgroup}{short \\ header}
      \bitbox{1}{H=0} & \bitbox{1}{F=1} & \bitbox{1}{S} & \bitbox{2}{R = 00} & \bitbox{1}{K} & \bitbox{2}{L (2)} \\
      \wordbox[tlrb]{\longFieldHeight}{Destination Connection ID (0..160) }
    \end{rightwordgroup} \\
    \wordbox[tlrb]{1}{Packet Number (8..32)} \\
    \wordbox[tlrb]{\longFieldHeight}{Packet Payload (..)} \\
    \wordbox[tlrb]{\longFieldHeight}{Integrity Tag (16)} \\
  \end{bytefield}

  H = Header Form, F = Fixed Bit, S = Spin Bit, \\ R = Reserved Bits, K = Key Phase Bit, L = Packet
Number Length

\end{myFigure}

The semantics of the fields specific to 1-RTT packets are:

\begin{description}

    \ditem{Spin Bit} A bit used for an optional QUIC feature which allows on-path nodes to measure
connection latency by observing changes in this bit. For a detailed description of the Spin Bit
feature, see the complete specification~\autocite[Section~17.3.1]{draft-ietf-quic-transport}.

    \ditem{Key Phase Bit} A bit used to communicate that the keys used for the packet encryption
need to be updated. After the keys have been used to encrypt a certain number of packets, it is
necessary to update them in order to maintain the level of protection in the connection.
\autoref{sec:02-key-update} describes the mechanism of Key Update in detail.

\end{description}

\section{QUIC Frames}\label{sec:02-quic-frames}

QUIC frames are low-level QUIC protocol messages carried in the payload of Initial, Handshake,
0-RTT, and 1-RTT packets. Examples of these frames include, e.g., \ACK{} frames carrying
acknowledgments for received packets, \STREAM{} frames carrying the application data, and \CRYPTO{}
frames carrying data for the TLS handshake.

During the lifetime of the connection, all QUIC packets must be acknowledged by sending an ACK frame
in another packet in the same packet number space. However, not all packets have to be acknowledged
immediately; e.g., acknowledging packets containing only ACK would cause an endless flood of ACK
packets. In such cases, sending an acknowledgement is delayed and sent later together with more
urgent data. Frame types that require immediate acknowledgments are called \textit{ack-eliciting
  frames}, and the packets with at least one such frame are called
\textit{\glspl{ack-eliciting-packet}}.

Because packets in different packet number spaces offer a different level of confidentiality, not
all frames can be sent in any packet type. For example, \STREAM{} frames --- which carry the
application data --- must not be sent in Initial and Handshake packets to avoid compromising
security. \autoref{tab:02-frame-types} lists all frame types used in QUIC, whether they are
ack-eliciting and which packets can carry them.

\begin{myTable}[\small] {tab:02-frame-types} {QUIC frame types}
  {l@{\hskip -0.1in}ccccc}
  {                               &               & \multicolumn{4}{c}{Allowed in packet type} \\
    \cmidrule(lr){3-6} Frame type & Ack-eliciting & Initial & Handshake & 0-RTT & 1-RTT}
  \PADDING{}                      &               & \checkmark{}     & \checkmark{}       & \checkmark{}   & \checkmark{}          \\
  \PING{}                         & \checkmark{}           & \checkmark{}     & \checkmark{}       & \checkmark{}   & \checkmark{}          \\
  \ACK{}                          &               & \checkmark{}     & \checkmark{}       &       & \checkmark{}          \\
  \RESETSTREAM{}                  & \checkmark{}           &         &           & \checkmark{}   & \checkmark{}          \\
  \STOPSENDING{}                  & \checkmark{}           &         &           & \checkmark{}   & \checkmark{}          \\
  \CRYPTO{}                       & \checkmark{}           & \checkmark{}     & \checkmark{}       &       & \checkmark{}          \\
  \NEWTOKEN{}                     & \checkmark{}           &         &           &       & \checkmark{}          \\
  \STREAM{}                       & \checkmark{}           &         &           & \checkmark{}   & \checkmark{}          \\
  \MAXDATA{}                      & \checkmark{}           &         &           & \checkmark{}   & \checkmark{}          \\
  \MAXSTREAMDATA{}                & \checkmark{}           &         &           & \checkmark{}   & \checkmark{}          \\
  \MAXSTREAMS{}                   & \checkmark{}           &         &           & \checkmark{}   & \checkmark{}          \\
  \DATABLOCKED{}                  & \checkmark{}           &         &           & \checkmark{}   & \checkmark{}          \\
  \STREAMDATABLOCKED{}            & \checkmark{}           &         &           & \checkmark{}   & \checkmark{}          \\
  \STREAMSBLOCKED{}               & \checkmark{}           &         &           & \checkmark{}   & \checkmark{}          \\
  \NEWCONNECTIONID{}              & \checkmark{}           &         &           & \checkmark{}   & \checkmark{}          \\
  \RETIRECONNECTIONID{}           & \checkmark{}           &         &           & \checkmark{}   & \checkmark{}          \\
  \PATHCHALLENGE{}                & \checkmark{}           &         &           & \checkmark{}   & \checkmark{}          \\
  \PATHRESPONSE{}                 & \checkmark{}           &         &           & \checkmark{}   & \checkmark{}          \\
  \CONNECTIONCLOSE{}              &               & \checkmark{}     & \checkmark{}       & \checkmark{}   & \checkmark{}          \\
  \HANDSHAKEDONE{}                & \checkmark{}           &         &           &       & \checkmark{}          \\
\end{myTable}

\section{QUIC Connection}

This section describes the details of \gls{cid} management and the QUIC connection lifetime.

\subsection{Connection ID}\label{sec:02-connection-id}

Traditional network protocols use the combination of remote endpoint IP address and port to identify
the connection. QUIC, on the other hand, uses a dedicated Connection ID identifier. This, in
essence, enables migrating the connection to different \glspl{network-path} and interfaces, e.g.,
from cellular data to a Wi-Fi network, because the connection identity does not depend on the peer's
IP address.

Connection IDs are opaque byte sequences between 8 to 20 bytes in length. Each endpoint in a QUIC
connection independently selects Connection IDs it will use to identify the QUIC connection. These
Connection IDs are then used to populate the Source Connection ID and Destination Connection ID
fields of the QUIC packets.

The endpoints exchange the first pair of Connection IDs during connection establishment. Additional
Connection IDs can be issued independently by each endpoint during the connection's lifetime using
the \NEWCONNECTIONID{} frame. These additional Connection IDs are primarily used when migrating the
connection to a new \gls{network-path} because QUIC specification requires that the same Connection
ID be used only on one \gls{network-path} in order to prevent correlation of the network traffic by
external observers.

Connection IDs can also be retired by the other endpoint using the \RETIRECONNECTIONID{} frame. By
retiring a Connection ID, the endpoint communicates that it will no longer use the Connection ID to
send packets and that the other endpoint should drop any incoming packets that use that Connection
ID\@. At the same time, retiring a Connection ID also serves as a request to the peer to issue a new
Connection ID as a replacement.

Alternatively, endpoints can use a zero-length Connection ID\@. In that case, the connection's
identity is tied to the remote endpoint's IP address and port. Using zero-length Connection ID saves
space in the sent datagrams but imposes several limitations on the connection. For example, if an
endpoint uses a zero-length Connection ID, it cannot issue additional connection IDs and, therefore,
it cannot migrate a connection to a new local address.

\subsection{Matching Packets to Connections}

When a packet arrives at an endpoint, it needs to be associated with an existing connection or ---
for servers --- potentially initiate a new connection. If packet contains a non-zero-length
Connection ID in the \gls{dcid} field, the \gls{cid} is used to find an existing connection. If the
packet uses a zero-length Connection ID, the local and remote addresses determine the target
connection.

\autoref{fig:02-connection-multiplexing} illustrates how server endpoint processes incoming packets
from client connections. The server maintains a table mapping between \gls{dcid}s corresponding
connections and dispatches the incoming packets accordingly. If a packet cannot be associated with
an existing connection, it may be a new connection attempt; otherwise, the packet is discarded.

\begin{myFigure}{fig:02-connection-multiplexing}{Multiple QUIC connections on the same machine port}

\input{img/02-socket-multiplexing.pdf_tex}

\end{myFigure}

In case the packet cannot be associated with an existing connection, client endpoints simply ignore
the packet. Server, on the other hand, check the packet type and version of the protocol it
requires. For valid Initial packets with supported versions, the server proceeds with the handshake
described in the previous section. For invalid Initial packets, the server responds with an Initial
packet containing a \CONNECTIONCLOSE{} frame signaling the refusal of the connection.

In case the client packet requests an unsupported QUIC version, the server replies with a Version
Negotiation packet (described in \autoref{sec:02-version-negotiation-packet}). After receiving such
a packet, the client can try again using one of the supported versions.
\autoref{fig:02-version-negotiation-flow} illustrates such an exchange. In the figure, the client
tries to establish a connection using an unsupported version denoted by X. The server then replies
with a Version Negotiation packet listing versions 1, 2, and 3. The client then tries to establish
the connection again using version 1.

\begin{myFigure}{fig:02-version-negotiation-flow}{Example Version Negotiation packet exchange}

\resizebox{\linewidth}{!}{\input{img/02-version-negotiation-flow.pdf_tex}}

\end{myFigure}

Lastly, the server may choose to send a Stateless Reset (see later in
\autoref{sec:02-stateless-reset}) for any other packet that it cannot match to an existing
connection.

\subsection{QUIC Transport Parameters}\label{sec:02-transport-parameters}

QUIC has several options for parameterizing the connection. These are called \textit{Transport
Parameters}, and they essentially represent constraints for the other endpoint. Each endpoint sets
the transport parameters for the other endpoint. QUIC leverages the extensibility of the TLS
protocol to exchange transport parameters during the connection handshake. Transport parameters
contain information such as

\begin{itemize}

  \item initial flow control limits,

  \item whether connection migration is allowed,

  \item maximum delay before sending an acknowledgment for \glspl{ack-eliciting-packet},

  \item maximum idle timeout before the connection is silently closed, and

  \item maximum size of a UDP packet the endpoint is willing to receive.

\end{itemize}

Many parameters have a default value, which is used when the given transport parameter is not sent.
Other transport parameters are mandatory. The exhaustive list of the transport parameters can be
found in the core transport specification~\autocite[Section~7.3]{draft-ietf-quic-transport}.

\subsection{Connection Establishment}\label{sec:02-connection-establishment}

To initiate a new connection, clients send an Initial packet to the server, which initiates the
handshake process. After the handshake completes, both peers have derived protection keys necessary
to send and receive 1-RTT packets with application data.

\autoref{fig:02-example-handshake-flow} illustrates a possible sequence of QUIC packets sent during
a connection handshake. The figure lists the contents of the individual QUIC frames, including the
TLS records sent inside \CRYPTO{} frames. However, contents of the \CRYPTO{} frames are listed only
for illustrative purposes because QUIC itself does not interpret their contents in any way. Instead,
the contents of \CRYPTO{} frames are forwarded to a TLS implementation.

\begin{myFigure} {fig:02-example-handshake-flow} {Example QUIC handshake flow}

\resizebox{\linewidth}{!}{\input{img/02-handshake-flow.pdf_tex}}

\end{myFigure}

In its first datagram, the client sends an Initial packet with initial information consisting of a
single \CRYPTO{}(CH) frame. This frame contains the \textit{Client Hello} TLS message.

The server replies with a UDP datagram containing three coalesced QUIC packets. The first is an
Initial packet that acknowledges the client's Initial packet using the \ACK{}(0) frame, and a
\CRYPTO{}(SH) frame with a \textit{Server Hello} message. The TLS implementation uses the contents
of Client Hello and Server Hello messages to derive the Handshake protection keys. The server then
advances the TLS handshake by sending another \CRYPTO{} frame in the Handshake packet. The server
also has enough information to derive the 1-RTT keys, so it can also start sending data on Stream 1
using the \STREAM{}(1, ``...'') frame in a 1-RTT packet.

Because the server's Initial packet contained an ack-eliciting \CRYPTO{}(SH) frame, the client needs
to acknowledge it by sending an Initial frame with an \ACK{}(0) frame. The client possesses both
Client and Server Hello messages and can derive the Handshake keys, enabling him to process the
server's Handshake packet. The Handshake packet needs to be separately acknowledged by another
\ACK{}(0) frame, and a reply from the TLS layer must be sent using another \CRYPTO{} frame. From the
information in the server's Handshake packet's \CRYPTO{} frame, the client derives 1-RTT protection
keys and processes the server's 1-RTT packet. In addition to sending an \ACK{}(0) frame for the
server's 1-RTT packet, the client can now start sending application data on stream 0 using a
\STREAM{}(0, ``...'') frame.

After the server detects that the handshake has completed, it sends a \HANDSHAKEDONE{} frame to
communicate it to the client. The figure also shows that servers sends \ACK{}(0) frame for the
clients Handshake packet but not for the client's last Initial packet because it was not
ack-eliciting.

\subsection{Connection Termination}

QUIC connection can be terminated in three ways:

\begin{itemize}

  \item Idle timeout

  \item Immediate close

  \item Stateless reset

\end{itemize}

\subsubsection{Idle Timeout}\label{sec:02-idle-timeout}

If idle timeout is enabled, the endpoint silently closes the connection if it does not receive a
packet from the peer for a specified time period. Each peer may advertise a timeout period using the
\MaxIdleTimeout{} transport parameter, but the effective value is the minimum of the two values.

In order to prevent timeouts, endpoints can send a \PING{} or another ack-eliciting frame to test
the liveness of the connection. However, sending \PING{} frames should be initiated by the
application protocol, not QUIC implementation, to prevent unnecessary network traffic.

\subsubsection{Immediate Close}

An immediate close can be initiated either by QUIC implementation or by the application. Either of
the endpoints can initiate an immediate close by sending a \CONNECTIONCLOSE{} frame. By sending a
\CONNECTIONCLOSE{} frame, the peer enters a \textit{closing state}, in which it includes the
\CONNECTIONCLOSE{} frame in all packets, it sends in reply to incoming packets. The closing state is
also entered if the endpoint receives a \CONNECTIONCLOSE{} frame from the peer. In that case, the
endpoint also echoes the \CONNECTIONCLOSE{} frame back to the other endpoint.

The closing state lasts until the endpoint is sure the other endpoint is also in the closing state
--- e.g., until it also receives \CONNECTIONCLOSE{} --- or until a \textit{closing timeout} expires.
The closing timeout period is calculated from the current estimate of the round-trip time of the
connection.

The \CONNECTIONCLOSE{} frame carries an error code and, optionally, a human-readable error phrase.
When initiated by QUIC, the error codes semantics are defined by the QUIC specification. However,
when initiated by the application protocol, the semantics of all possible error code values sent in
the frame are defined by the application protocol itself. This implies that the implementations must
require an error code when closing the connection and should not provide a default error code value.

\subsubsection{Stateless Reset}\label{sec:02-stateless-reset}

A stateless reset is an option of last resort for an endpoint that does not have access to the state
of a connection, possibly as a result of a crash or outage. An endpoint may send a stateless reset
in response to receiving a packet that it cannot associate with an active connection.

In such cases, the endpoint sends a specially crafted packet that ends with a Stateless Reset Token
generated for the \gls{dcid} from the incoming packet. The Stateless Reset Token requirements are
quite complex, and we encourage readers to read the full specification if they are interested in
details~\autocite[Section~10.4]{draft-ietf-quic-transport}.

\subsection{Connection Migration}

A novel feature of QUIC is migrating connections to a different \gls{network-path}. This is enabled
by using a dedicated Connection ID instead of using the endpoint's address to identify the
connection. In the initial QUIC version, only client endpoints can migrate the connection to a
different address.

Before migrating a connection, the endpoint can optionally check the reachability of the other
endpoint using the process called \textit{path validation}. Path validation consists of exchanging
\textit{probing packets}, and is described in \autoref{sec:02-path-validation}. The migration itself
is initiated by client by simply switching to the new local address for sending all outbound
packets. After the server receives the first non-probing packet from the new client's address, it
sends all future packets to that new address.

QUIC uses additional measures to prevent network traffic from being correlated by the outside
observers. In \autoref{sec:02-connection-id} we mentioned that each endpoint could use multiple
Connection IDs to refer to the same connection. QUIC requires that each Connection ID be used only
on one \gls{network-path}. Therefore, both endpoints must switch to using different \glspl{cid} when
the connection is migrated. Therefore, connection migration can only be initiated if both endpoints
issued additional Connection IDs using the \NEWCONNECTIONID{} frame.

The connection migration process is illustrated in \autoref{fig:02-connection-migration-flow}.
Client and servers use Connection IDs \textbf{C1} and \textbf{S1}, respectively. The client first
probes the server's reachability with a probing packet containing a \PATHCHALLENGE{} frame. Because
this packet is sent from a different local address, it uses a different Connection ID (\textbf{S2})
issued previously by the server. Likewise, the server uses a different Connection ID \textbf{C2} to
reply with a packet containing a \PATHRESPONSE{} frame, confirming the reachability from the
client's new local address. The server, meanwhile, still uses the old network path for all other
communication with the client. After confirming reachability, the client migrates the connection by
sending all packets via the new local address. After receiving the next non-probing packet from the
new address, the server switches to the new client's address for all outgoing packets. From that
point, Connection IDs \textbf{C1} and \textbf{S1} are no longer used.

\begin{myFigure}{fig:02-connection-migration-flow}{Flow of packets during connection migration}

\resizebox{\linewidth}{!}{\input{img/02-connection-migration-flow.pdf_tex}}

\end{myFigure}

Connection migration is an optional feature and can be disabled by sending the
\DisableActiveMigration{} transport parameter during the connection handshake.

\section{QUIC Streams}

QUIC can transport multiple streams of data in a single connection. Each stream is identified by its
\textit{Stream ID} and is processed independently of the other streams. Each QUIC packet can carry
data for one or more QUIC streams. \autoref{fig:02-stream-multiplexing} illustrates how QUIC may
pack two streams into QUIC packets such that those streams are transported in parallel.

\begin{myFigure}{fig:02-stream-multiplexing}{Stream multiplexing in QUIC}

  \input{img/02-stream-multiplexing.pdf_tex}

\end{myFigure}

\subsection{Streams Types}\label{sec:02-stream-types}

Streams transported by QUIC can be either unidirectional or bidirectional. Unidirectional streams
carry data from the initiator to its peer, and bidirectional streams carry data in both directions.
Both client and server can open new streams. QUIC recognizes four types of streams, and the type of
the stream is encoded in the two least significant bits of the Stream ID. The stream types and their
associated encoding is summarized in \autoref{tab:02-stream-id-type-map}.

\begin{myTable} {tab:02-stream-id-type-map} {Mapping of QUIC Stream types to Stream ID bits} {cc}
{Stream type & Least significant bits} Client-Initiated, Bidirectional & 00 \\ Server-Initiated,
Bidirectional & 01 \\ Client-Initiated, Unidirectional & 10 \\ Server-Initiated, Unidirectional & 11
\\
\end{myTable}

Bidirectional streams can be viewed as the combination of two unidirectional streams in opposing
directions. After opening a bidirectional stream, each direction of the stream behaves as separate
inbound and outbound unidirectional streams. This implies, e.g., that the sending and receiving
parts of the bidirectional stream can be closed independently of each other.

\subsection{Stream Lifetime}

Opening a stream does not require any particular action. Streams are opened simply by sending the
first \STREAM{} frame carrying data for that stream. However, streams of a particular type can be
opened only in ascending order of their Stream IDs. For example, a stream 2 must be opened before
opening stream 6. Sending data for higher-numbered streams will automatically open all
lower-numbered streams of the same stream type.

Streams can be closed either gracefully or abortively. Graceful stream close is signaled by a
\textit{Fin} bit in the \STREAM{} frame, signaling that data carried by this packet are the last
part of the stream. The stream then is gracefully closed once all stream data is confirmed received
by the other endpoint.

Abortive stream close is achieved using the \RESETSTREAM{} frame and, therefore, this action is also
referred to as \textit{resetting the stream}. Streams can be reset only by the sender. The receiver
can request aborting the stream by sending a \STOPSENDING{} frame which indicates that the
application no longer wishes to receive data from that particular stream. Both \RESETSTREAM{} and
\STOPSENDING{} frames carry an application-level error code which is reported to the applicaiton on
the other side.

After the stream is closed, its Stream ID may not be reused. Instead, the next available Stream ID
must be used. QUIC uses the variable-length integer encoding (see
\autoref{sec:02-variable-length-encoding}) and, therefore, there is no shortage of available stream
IDs, which range from 0 to $2^{62}-1$.

\subsection{Required Operations on Streams}

The implementation should provide the following operations on sending part of the stream:

\begin{itemize}

  \item write data;

  \item end the stream by specifying that all data has been written; and

  \item terminate the stream with an application-level error code.

\end{itemize}

On receiving part of the stream, application protocols must be able to:

\begin{itemize}

  \item read data; and

  \item abort reading with an application-level error code.

\end{itemize}

\section{Flow Control}

QUIC aims to be a general-purpose transport protocol to be used over a potentially untrusted
network, and as such, it needs to protect endpoints from malicious peers. To prevent malicious
senders from exhausting all available memory on the receiver by sending large amounts of data, or
fast senders from overwhelming slow receivers, QUIC employs a credit-based flow control scheme.

All QUIC streams are flow controlled both individually and together as an aggregate. Each endpoint
also controls the number of streams the other peer is allowed to open. All flow control limits are
communicated to the peer using three types of frames:

\begin{itemize}

\litem[]{\MAXSTREAMDATA{}} maximum offset of data sent on a stream with specified Stream ID\@.

\litem[]{\MAXDATA{}} the maximum sum of all offsets of data sent on all streams.

\litem[]{\MAXSTREAMS{}} the maximum number of streams of a particular stream type.

\end{itemize}

Endpoints can only increase the flow control limits. Their peers must ignore any attempts to
decrease the flow control limits to ensure consistency when two consecutive QUIC packets with flow
control updates are reordered during transit. In case the peer violates any of the control flow
limits mentioned above, the QUIC implementation must immediately terminate the connection.

\section{Loss Detection and Recovery}\label{sec:02-loss-detection}

Because UDP is an unreliable transport protocol, QUIC must implement measures to recover from packet
loss. The packet loss detection is implemented similarly to TPC --- endpoints send acknowledgments
for each received packet. However, an essential difference from TCP is that QUIC endpoints do not
retransmit entire lost packets with the same packet number. Instead, each QUIC frame in the original
packet is updated and sent in some future packet or dropped altogether if the information contained
in the frame is no longer relevant.

The endpoints communicate the acknowledgment using the \ACK{} frame containing ranges of received
packet numbers. A packet number can be sent multiple times (in \ACK{} frames in different packets)
until the endpoint can determine that the other endpoint received the acknowledgment.

For packets containing an \textit{ack-eliciting} frame (see \autoref{sec:02-quic-frames}),
acknowledgements must be sent within the period specified by the \MaxAckDelay{} transport parameter.
For other packets, such as packets containing only an \ACK{} frame, the acknowledgement can be
delayed until an \gls{ack-eliciting-packet} is received.

\autoref{fig:02-packet-loss-example} illustrates the loss detection and retransmission process in
action. In the figure, frame \STREAM{}(0,[0..9]) denotes a \STREAM{} frame carrying the first 10
bytes of the stream with Stream ID 0. When the server receives the acknowledgment for packet 3, but
not for packet 2 sent earlier, it infers that packet 2 never reached the receiver and retransmits
the bytes 10 to 19 of stream 0 in packet 4. The server does not have to retransmit the \ACK{}(1)
frame because it was sent in packet 3, which the client acknowledged. Therefore, packet 4
acknowledges only the client's packet 2.

\begin{myFigure}{fig:02-packet-loss-example}{Loss detection and retransmission example}

\resizebox{\linewidth}{!}{\input{img/02-retransmission-example.pdf_tex}}

\end{myFigure}

The exact criteria for a packet to be deemed lost by a QUIC endpoint are following:

\begin{enumerate}

  \item The packet was not acknowledged.

  \item A packet which was sent later has been acknowledged.

  \item Either the packet has been sent long enough in the past, or its packet number is
sufficiently smaller than the highest acknowledged packet number.

\end{enumerate}

The third condition is dependent on the values of particular constants. The specification recommends
that the packet is considered lost if the gap between its packet number and the highest acknowledged
packet number is at least 3, or if it was sent for longer than $9/8$ times the estimate of the
current round-trip time.

The first condition mentioned above requires receiving a packet from the peer to declare any packet
as lost. However, the loss of the last packet in a sequence could go undetected because there is no
following packet that can be acknowledged. In order to avoid possible deadlocks in such scenarios,
QUIC endpoint sends up to two ack-eliciting \textit{probe packets} if it does not receive a packet
from a peer in a period called \textit{probe timeout} (PTO for short). The PTO duration doubles each
time probe packets are sent until either a reply is received or the connection is terminated due to
idle timeout (see \autoref{sec:02-idle-timeout})

Similarly to TCP, QUIC also uses congestion control to manage the \textit{congestion window} --- the
amount of data that can be in-flight. The selection of the congestion control algorithm is left on
the implementation. As an example, The QUIC specification document for loss detection and
recovery~\autocite[Section~7]{draft-ietf-quic-recovery} describes a congestion control algorithm
similar to TCP NewReno~\cite{rfc6582} algorithm.

\section{Security}

This section describes the mechanisms used to ensure the security of the protocol. Besides
encrypting all packets sent throughout the lifetime of the connection, QUIC uses additional
mechanisms to ensure that the servers using the protocol are resistant to denial-of-service and
other cyber-attacks.

\subsection{TLS Integration}

Instead of designing a new handshake protocol, QUIC offloads the encryption negotiation to TLS
protocol (more precisely, TLS version 1.3). The low-level messages used in TLS, such as
\textit{Server Hello} and \textit{Client Hello} are transported by QUIC inside \CRYPTO{} frames (as
illustrated in \autoref{sec:02-connection-establishment}) and passed to a TLS implementation on the
other side. This way, QUIC can offer the same confidentiality level as conventional TLS connections.

The TLS protocol is extensible. Among the standard extensions which are also used by QUIC are
\gls{alpn}~\cite{rfc7301} and \gls{sni}~\cite{rfc6066}.

\Gls{alpn} is used when multiple application protocols or their multiple versions are supported on
the same TCP or UDP port. \Gls{alpn} allows the application layer to negotiate --- as part of the
TLS handshake --- which application protocol will be used in the established connection.

Clients use \gls{sni} to specify the hostname of the server to which they are connecting. When
multiple websites are hosted on the same IP address and port, \gls{sni} allows the server to
customize the security configuration for each hosted website. During connection establishment,
proper security configuration, such as the SSL certificate to be used, can be selected based on the
hostname provided by the client.

QUIC also uses a custom TLS extension to exchange transport parameters during the handshake. The
QUIC transport parameters were described in \autoref{sec:02-transport-parameters}.

\subsection{Packet Protection}\label{sec:02-packet-protection}

All QUIC packets of type Initial, Handshake, 1-RTT, and 0-RTT are encrypted to ensure the integrity
and confidentiality of the transmitted data. Negotiation of the cryptographic ciphers and the
encryption keys is handled by the TLS handshake. This section focuses on how the negotiated
encryption is applied to QUIC packets.

\subsubsection{Authenticated Encryption with Associated Data}

QUIC uses type of encryption called \gls{aead}~\cite{rfc5116}. This type of encryption ensures both
the confidentiality and authenticity of the encrypted data. In addition to encrypted data --- called
\textit{ciphertext} --- \gls{aead} encryption outputs an authentication tag, which is used to check
the integrity of the payload during decryption. The encryption can be authenticated by supplying
additional authentication data (AAD for short), which are not encrypted but influence the
authentication tag and, therefore, must also be supplied during decryption. As additional
protection, \gls{aead} also accepts a \textit{nonce} parameter, which is an additional input that is
supposed to be unique for each encrypted message. By using a unique nonce parameter for each
message, the algorithm ensures that two otherwise identical messages produce different ciphertexts.

The programming interface for \gls{aead} provides the following operations:

\begin{itemize}

  \item Encryption:

  \begin{itemize}

    \item input: plaintext, key, nonce, AAD (optional)

    \item output: ciphertext, authentication tag

  \end{itemize}

  \item Decryption:

  \begin{itemize}

    \item input: ciphertext, key, nonce, authentication tag, AAD (if provided during encryption)

    \item output: plaintext or error if the authentication tag does not match the rest of the input

  \end{itemize}

\end{itemize}

\subsubsection{Deriving QUIC Protection Keys}\label{sec:02-encryption-key-derivation}

QUIC derives multiple distinct keys that are used in different parts of the packet protection
procedure. Furthermore, each packet type uses a different set of such keys. This allows QUIC to
provide confidentiality and integrity protection even for Initial and Handshake packets, which are
sent before the TLS handshake negotiates the application-level encryption.

Each set of encryption keys is derived from a \textit{secret} --- a sequence of bytes known only to
the two endpoints. Secrets for deriving the Handshake and 1-RTT keys are produced by the TLS
handshake. Secrets for Initial keys are derived using the \gls{dcid} of the client's first Initial
packet and a version-specific \textit{initial salt} which ensures that newer versions of QUIC derive
different protection keys. The derivation process uses the HKDF-Extract and HKDF-Expand-Label
functions from RFC 5869~\cite{rfc5869} to derive secrets of the desired length. The following
equations show how QUIC derives distinct secrets for client and server.

\begin{equation*}
  \begin{split}
    initial\_secret & = \operatorname{HKDF-Extract}(dcid, initial\_salt) \\
    client\_secret & = \operatorname{HKDF-Expand-Label}(initial\_secret, \texttt{"client in"}, \texttt{""}, 32) \\
    server\_secret & = \operatorname{HKDF-Expand-Label}(initial\_secret, \texttt{"server in"}, \texttt{""}, 32) \\
  \end{split}
\end{equation*}

Once the necessary secrets are obtained, a total of three keys are derived using the following
equations.

\begin{equation*}
  \begin{split}
    key & = \operatorname{HKDF-Expand-Label}(secret, \texttt{"quic key"}, \texttt{""}, 32) \\
    iv & = \operatorname{HKDF-Expand-Label}(secret, \texttt{"quic iv"}, \texttt{""}, 12) \\
    hp & = \operatorname{HKDF-Expand-Label}(secret, \texttt{"quic hp"}, \texttt{""}, 32) \\
  \end{split}
\end{equation*}

The following subsections describe how the $key$, $iv$, and $hp$ keys are used in the actual process
of packet encryption.

\subsubsection{Packet Protection Procedure}

When encrypting the packets, QUIC first encrypts the packet payload --- the sequence of QUIC frames
--- using the \gls{aead} cipher negotiated by the TLS implementation. QUIC specification allows the
use of all \gls{aead} ciphers allowed in TLS 1.3. These ciphers are:

\begin{itemize}

  \item TLS\_AES\_128\_GCM\_SHA256

  \item TLS\_AES\_256\_GCM\_SHA384

  \item TLS\_CHACHA20\_POLY1305\_SHA256

  \item TLS\_AES\_128\_CCM\_SHA256

  \item TLS\_AES\_128\_CCM\_8\_SHA256

\end{itemize}


The parameters for \gls{aead} for packet payload protection are:

\begin{itemize}

    \litem{key} the $key$ derived in the previous section

    \litem{nonce} the $iv$ derived in the previous section, with the last 8 bytes XORed with the
packet number

    \litem{AAD} the contents of the packet header

    \litem{plaintext} the payload of the packet.

\end{itemize}

The produced authentication tag is appended to the packet as the Integrity Tag field and, in case of
the packet types with long headers, its size is included in the payload size in the Length field.

QUIC also protects the header of the packet. The header protection mechanism is more complicated
than that of the payload protection. The process requires calculating a \textit{header protection
  mask}, which is then applied using XOR to selected fields of the packet header and the Packet
Number field. The algorithm for calculating the header protection mask depends on the negotiated
cipher, but it always uses the $hp$ key and a 16-byte sample of the encrypted payload.

The parts of the packets protected by the payload encryption and header protection mechanisms are
illustrated in \autoref{fig:02-protected-fields-long} for long headers and in
\autoref{fig:02-protected-fields-short} for short headers.

\begin{myFigure}{fig:02-protected-fields-long}{Protected fields in the long packet header}

  \newcommand{\legendsquare}[1]{\textcolor{#1}{\rule{0.7em}{0.7em}}}

  \definecolor{hp}{rgb}{0.7, 1, 0.7}
  \definecolor{pp}{rgb}{0.7, 0.7, 1}

  \legendsquare{hp}~Header protection \hspace{1cm} \legendsquare{pp}~Payload encryption

  \vspace{5mm}

  \begin{bytefield}[bitwidth=2.5em]{8}
    \begin{rightwordgroup}{long \\ header}
      \bitbox{1}{H=1} & \bitbox{1}{F=1} & \bitbox{2}{T (2)} & \colorbitbox{hp}{2}{R = 00} & \colorbitbox{hp}{2}{L (2)} \\
      \wordbox[tlrb]{\longFieldHeight}{Version (32)} \\
      \wordbox[tlrb]{1}{Destination Connection ID Length (8)} \\
      \wordbox[tlrb]{\longFieldHeight}{Destination Connection ID (0..160) } \\
      \wordbox[tlrb]{1}{Source Connection ID Length (8)} \\
      \wordbox[tlrb]{\longFieldHeight}{Source Connection ID (0..160) }
    \end{rightwordgroup} \\
    \begin{leftwordgroup}{Initial \\ only}
      \colorbitbox{bytefieldunused}{8}{Token Length (i)} \\
      \colorwordbox{bytefieldunused}{tlrb}{\longFieldHeight}{Token (..)}
    \end{leftwordgroup} \\ \wordbox[tlrb]{1}{Length (i)} \\
    \colorwordbox{hp}{ltrb}{\longFieldHeight}{Packet Number (8..32)} \\
    \colorwordbox{pp}{tlrb}{\longFieldHeight}{Packet Payload (..)} \\
    \wordbox[tlrb]{\longFieldHeight}{Integrity Tag (16)} \\
  \end{bytefield}
\end{myFigure}

\begin{myFigure}{fig:02-protected-fields-short}{Protected fields in the short packet header}

  \newcommand{\legendsquare}[1]{\textcolor{#1}{\rule{0.7em}{0.7em}}}

  \definecolor{hp}{rgb}{0.7, 1, 0.7}
  \definecolor{pp}{rgb}{0.7, 0.7, 1}

  \legendsquare{hp}~Header protection \hspace{1cm} \legendsquare{pp}~Payload encryption

  \vspace{5mm}

  % this one has to be off-center to make the image aligned with the long header one
\hspace{1.8cm}\begin{bytefield}[bitwidth=2.5em]{8}
  \begin{rightwordgroup}{short \\ header}
    \bitbox{1}{H=0} & \bitbox{1}{F=1} & \bitbox{1}{S} & \colorbitbox{hp}{2}{R = 00} & \colorbitbox{hp}{1}{K} & \colorbitbox{hp}{2}{L (2)} \\
    \wordbox[tlrb]{\longFieldHeight}{Destination Connection ID (0..160) }
  \end{rightwordgroup} \\
  \colorwordbox{hp}{ltrb}{\longFieldHeight}{Packet Number (8..32)} \\
  \colorwordbox{pp}{tlrb}{\longFieldHeight}{Packet Payload (..)} \\
  \wordbox[tlrb]{\longFieldHeight}{Integrity Tag (16)} \\
  \end{bytefield}
\end{myFigure}

\subsubsection{Updating 1-RTT Protection Keys}\label{sec:02-key-update}

\gls{aead} ciphers slowly lose the levels confidentiality and integrity protection with each
encrypted packet. Therefore, QUIC --- like TLS --- tracks the number of packets encrypted using a
particular encryption key and updates the key before an attacker can gain substantial advantage from
observing the encrypted packets. Analysis in the specification document shows that updating the
protection keys after sending at most $2^{23}$ packets provides the same confidentiality level as
provided by TLS~\autocite[Appendix~B]{draft-ietf-quic-tls}.

The protection keys are updated using a process called \textit{\gls{key-update}}. The process is
signalled to the peer by flipping the \textit{Key Phase} bit in the 1-RTT packet header. After
observing a change in the \textit{Key Phase}, an endpoint derives new secret and $key$ and $iv$ keys
using the HKDF-Expand-Label function:

\begin{equation*}
  \begin{split}
    secret_{n+1} & = \operatorname{HKDF-Expand-Label}(secret_{n}, \texttt{"quic ku"}, \texttt{""}, 32) \\
    key_{n+1} & = \operatorname{HKDF-Expand-Label}(secret_{n+1}, \texttt{"quic key"}, \texttt{""}, 32) \\
    iv_{n+1} & = \operatorname{HKDF-Expand-Label}(secret_{n+1}, \texttt{"quic iv"}, \texttt{""}, 12) \\
  \end{split}
\end{equation*}

Because the \textit{Key Phase} bit is protected by header protection, the $hp$ key must remain
unchanged to ensure that the other endpoint can correctly remove the header protection.

\subsection{Client Address Validation}\label{sec:02-address-validation}

After receiving the first Initial packet from a new client, the server can request address
validation by sending a Retry packet (see \autoref{sec:02-retry-packet}). The Retry packet carries a
token, which the client must echo back to the server in all following Initial packets. As long as an
attacker cannot generate a valid token for its address, and the client can return that token, this
exchange proves to the server that the client has received the token.

\autoref{fig:02-client-address-validation-flow} illustrates the use of Retry packet to validate
client address during connection establishment. By default, clients do not fill the Token field of
the Initial packet. The server rejects the initial connection attempt and issues a Retry Token
(denoted ``ABCD'' in the figure). The client then tries again with another Initial with the provided
token, and the server proceeds with the usual handshake.

\begin{myFigure}{fig:02-client-address-validation-flow}{Client address validation using a Retry
packet.}

\resizebox{\linewidth}{!}{\input{img/02-client-address-validation-flow.pdf_tex}}

\end{myFigure}

\subsection{Path Validation}\label{sec:02-path-validation}

QUIC is layered on top of UDP, which is a connection-less protocol. This means that changes in
endpoint address can also happen without active migration on the endpoint's part, e.g., because of
NAT rebinding along the \gls{network-path}.

The other endpoint may also spoof the endpoint address in an attempt to perform
\textit{\gls{traffic-amplification-attack}}. For this reason, the amount of data sent to the new
endpoint address must be limited until \textit{\gls{path-validation}} determines that the address
belongs to the endpoint. Path validation is performed by sending a \textit{probing packet}
containing a \PATHCHALLENGE{} frame with an unpredictable token. The other endpoint must echo the
token back in a \PATHRESPONSE{} frame. After that, the new address is considered validated, and the
sending rate restrictions are lifted.
