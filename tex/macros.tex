%%% This file contains definitions of various useful macros and environments %%%
%%% Please add more macros here instead of cluttering other files with them. %%%

%%% Minor tweaks of style

% These macros employ a little dirty trick to convince LaTeX to typeset
% chapter headings sanely, without lots of empty space above them.
% Feel free to ignore.
\makeatletter
\def\@makechapterhead#1{
  {\parindent \z@ \raggedright \normalfont
   \Huge\bfseries \thechapter. #1
   \par\nobreak
   \vskip 20\p@
}}
\def\@makeschapterhead#1{
  {\parindent \z@ \raggedright \normalfont
   \Huge\bfseries #1
   \par\nobreak
   \vskip 20\p@
}}
\makeatother

% This macro defines a chapter, which is not numbered, but is included
% in the table of contents.
\def\chapwithtoc#1{
\chapter*{#1}
\addcontentsline{toc}{chapter}{#1}
}

% Draw black "slugs" whenever a line overflows, so that we can spot it easily.
\overfullrule=1mm

%%% macro for replacing
\ExplSyntaxOn
\NewDocumentCommand{\replace}{mmm}
 {
  \marian_replace:nnn {#1} {#2} {#3}
 }

\tl_new:N \l_marian_input_text_tl

\cs_new_protected:Npn \marian_replace:nnn #1 #2 #3
 {
  \tl_set:Nn \l_marian_input_text_tl { #1 }
  \tl_replace_all:Nnn \l_marian_input_text_tl { #2 } { #3 }
  \tl_use:N \l_marian_input_text_tl
 }
\ExplSyntaxOff

% define a macro \Autoref to allow multiple references to be passed to \autoref
% src: https://tex.stackexchange.com/questions/15728/multiple-references-with-autoref
\makeatletter

\newcommand\Autoref[1]{\@first@ref#1,@}
\def\@throw@dot#1.#2@{#1}% discard everything after the dot
\def\@set@refname#1{%    % set \@refname to autoefname+s using \getrefbykeydefault
    \edef\@tmp{\getrefbykeydefault{#1}{anchor}{}}%
    \xdef\@tmp{\expandafter\@throw@dot\@tmp.@}%
    \ltx@IfUndefined{\@tmp autorefnameplural}%
         {\def\@refname{\@nameuse{\@tmp autorefname}s}}%
         {\def\@refname{\@nameuse{\@tmp autorefnameplural}}}%
}
\def\@first@ref#1,#2{%
  \ifx#2@\autoref{#1}\let\@nextref\@gobble% only one ref, revert to normal \autoref
  \else%
    \@set@refname{#1}%  set \@refname to autoref name
    \@refname~\ref{#1}% add autoefname and first reference
    \let\@nextref\@next@ref% push processing to \@next@ref
  \fi%
  \@nextref#2%
}
\def\@next@ref#1,#2{%
   \ifx#2@ and~\ref{#1}\let\@nextref\@gobble% at end: print and+\ref and stop
   \else, \ref{#1}% print  ,+\ref and continue
   \fi%
   \@nextref#2%
}

\makeatother

%%% Functional foreach construct
%%% (https://stackoverflow.com/questions/2402354/split-comma-separated-parameters-in-latex)

\makeatletter

% #1 - Function to call on each comma-separated item in #3
% #2 - Parameter to pass to function in #1 as first parameter
% #3 - Comma-separated list of items to pass as second parameter to function #1
\def\foreach#1#2#3{%
  \@test@foreach{#1}{#2}#3,\@end@token%
}

% Internal helper function - Eats one input
\def\@swallow#1{}

% Internal helper function - Checks the next character after #1 and #2 and
% continues loop iteration if \@end@token is not found
\def\@test@foreach#1#2{%
  \@ifnextchar\@end@token%
    {\@swallow}%
    {\@foreach{#1}{#2}}%
}

% Internal helper function - Calls #1{#2}{#3} and recurses
% The magic of splitting the third parameter occurs in the pattern matching of the \def
\def\@foreach#1#2#3,#4\@end@token{%
  #1{#2}{#3}%
  \@test@foreach{#1}{#2}#4\@end@token%
}

%%% foreach usage:
% Example-function used in foreach, which takes two params and builds hrefs
%\def\makehref#1#2{\href{#1/#2}{#2}}

% Using foreach by passing #1=function, #2=constant parameter, #3=comma-separated list
%\foreach{\makehref}{http://stackoverflow.com}{2409851,2408268}

% Will in effect do
%\href{http://stackoverflow.com/2409851}{2409851}\href{http://stackoverflow.com/2408268}{2408268}

\makeatother

%% \citeauthor command variant with full name
\DeclareCiteCommand{\citeauthorfull}
  {\boolfalse{citetracker}%
   \boolfalse{pagetracker}%
   \DeclareNameAlias{labelname}{first-last}%
   \usebibmacro{prenote}}
  {\ifciteindex
     {\indexnames{labelname}}
     {}%
   \printnames{labelname}}
  {\multicitedelim}
  {\usebibmacro{postnote}}


%%% Various table goodies
\newcommand{\pulrad}[1]{\raisebox{1.5ex}[0pt]{#1}}
\newcommand{\mc}[1]{\multicolumn{1}{c}{#1}}

% Outputs red TODOs in the document. Requires \usepackage{color}.
%
% Usage: \todo{Document the TODO command.}
%
% Comment out second line to disable.
\newcommand{\todo}[1]{}
\renewcommand{\todo}[1]{{\color{red} TODO: {#1}}}

% the symbol of .NET
\newcommand{\dotnet}{.NET\@}

% prettier C# typesetting
\newcommand{\csharp}{%
  {\settoheight{\dimen0}{C}C\kern-.05em \resizebox{!}{\dimen0}{\raisebox{\depth}{\#}}}}

% special items for itemize and enumerate environments which have a label
\newcommand\litem[2][\itshape]{\item{#1 #2}:}

% special item for description lists to make the description go on the next line
\newcommand\ditem[1]{ \item[#1] }
\newcommand\ditemWithComment[2]{ \item[#1 \hfill \textnormal{\textit{(#2)}}] }

% macro for typesetting library names
\newcommand\libname[1]{\texttt{\itshape#1}}

% macros for library names used throughout the thesis
\newcommand{\libmsquic}{\libname{MsQuic}}
\newcommand{\libcurl}{\libname{libcurl}}
\newcommand{\libopenssl}{\libname{OpenSSL}}
\newcommand{\libschannel}{\libname{Schannel}}
\newcommand{\xUnit}{\libname{xUnit}}

% helper for composing captions from short and long entries
\newcommand{\myCaption}[2]{% short, long
  \def\temp{#1}\ifx\temp\empty% Use different short caption if provided
    \caption{#2}%
  \else%
    \caption[#1]{#2}%
  \fi%
}
\newcommand{\myCaptionInline}[2]{% short, long
  \def\temp{#1}\ifx\temp\empty% Use different short caption if provided
    #2%
  \else%
    [#1]#2%
  \fi%
}
\newcommand{\myCaptionOf}[3]{% env, short, long
  \def\temp{#2}\ifx\temp\empty% Use different short caption if provided
    \captionof{#1}{#3}%
  \else%
    \captionof{#1}[#2]{#3}%
  \fi%
}

% environment setting common attributes of figures
\newcommand{\figureArgs}{} % dummy macro to be redefined, used to make caption go under the figure
\newenvironment{myFigure}[3][]{% label, caption
  \renewcommand{\figureArgs}{\myCaption{#1}{#3}\label{#2}}%
  \figure[H]{}
  \centering
}{ %
  \figureArgs{}
  \endfigure{}
}

\newenvironment{mySubfigure}[3]{% [width, label, caption]
  \renewcommand{\figureArgs}{\caption{#3}\label{#2}}%
  \subfigure[H]{#1}
  \centering
}{ %
  \figureArgs{}
  \endsubfigure{}
}

%% Fix cmidrule and babel[czech] clash
%% src: https://tex.stackexchange.com/questions/111999/slovak-and-czech-babel-gives-problems-with-cmidrule-and-cline
\begingroup
    \makeatletter
    \catcode`\-=\active
    \AtBeginDocument{
    \def\@@@cmidrule[#1-#2]#3#4{\global\@cmidla#1\relax
        \global\advance\@cmidla\m@ne
        \ifnum\@cmidla>0\global\let\@gtempa\@cmidrulea\else
        \global\let\@gtempa\@cmidruleb\fi
        \global\@cmidlb#2\relax
        \global\advance\@cmidlb-\@cmidla
        \global\@thisrulewidth=#3
        \@setrulekerning{#4}
        \ifnum\@lastruleclass=\z@\vskip \aboverulesep\fi
        \ifnum0=`{\fi}\@gtempa
        \noalign{\ifnum0=`}\fi\futurenonspacelet\@tempa\@xcmidrule}
    }
\endgroup

% environment setting common attributes of tables
\newenvironment{myTable}[5][\normalsize]{% [fontsize, label, caption, columnspec, header]
  \table[H]
  #1
  \caption{#3}\label{#2}% for tables, the caption goes to the top
  \centering
  \tabular{#4}
  \toprule
  #5 \\
  \midrule
}{ %
  \bottomrule
  \endtabular{}
  \endtable{}
}


\definecolor{colorunimportant}{gray}{0.5}

% use VS like colors/
\definecolor{codecolorclass}{rgb}{0.12, 0.53, 0.62}
\definecolor{codecolorenum}{rgb}{0.40, 0.63, 0.23}
\definecolor{codecolorinterface}{rgb}{0.40, 0.63, 0.23}
\definecolor{codecolorkeyword}{rgb}{0, 0, 0.5}
\definecolor{codecolormethod}{rgb}{0.45, 0.32, 0.12}
\definecolor{codecolorkeywordcontrolflow}{rgb}{0.55, 0.03, 0.79}
\definecolor{codecolornamespace}{gray}{0}
\definecolor{codecolorcomment}{rgb}{0, 0.5, 0}
\definecolor{codecolorstring}{rgb}{0.64, 0.08, 0.08}
\definecolor{codecolorxmltag}{rgb}{0.64, 0.08, 0.08}
\definecolor{codecolorxmlstring}{rgb}{0, 0, 1}
\definecolor{codecolorxmlattribute}{rgb}{1, 0, 0}

% allow adding known classes to the listings for better syntax highlight
% https://tex.stackexchange.com/questions/172048/how-can-i-define-a-custom-class-of-keywords-in-listings
\makeatletter
\lst@InstallKeywords k{types}{typestyle}\slshape{typestyle}{}ld
\lst@InstallKeywords k{enums}{enumstyle}\slshape{enumstyle}{}ld
\lst@InstallKeywords k{attributes}{attributestyle}\slshape{attributestyle}{}ld
\makeatother

\lstloadlanguages{csh}

% general settings
\lstset{
  captionpos=t,
  showspaces=false,
  showtabs=false,
  keepspaces=true,
  breaklines=true,
  showstringspaces=false,
  breakatwhitespace=true,
  frame=bt,
  framesep=1em,
  columns=fullflexible,
  numbers=left,
  numberstyle=\tiny,
  escapeinside={(*@}{@*)},
  commentstyle=\color{codecolorcomment},
  keywordstyle=\color{codecolorkeyword}\bfseries,
  stringstyle=\color{codecolorstring},
  typestyle=\color{codecolorclass},
  enumstyle=\color{codecolorenum},
  basicstyle=\ttfamily\footnotesize
}

\newcommand{\lstsetcsharpsettings}{%
  \lstset{language=[Sharp]C,
    keywords={
      abstract, event, new, struct, as, explicit, null, switch, base, extern,
      object, this, bool, false, operator, break, out, true, byte, fixed, override,
      case, float, params, typeof, catch, private, uint, char, protected, ulong,
      checked, public, unchecked, class, readonly, unsafe, const, implicit, ref, ushort,
      continue, in, using, decimal, int, sbyte, virtual, default, interface, sealed, volatile,
      delegate, internal, short, void, do, is, sizeof, double, lock, stackalloc, async, await,
      long, static, unmanaged, Cdecl, enum, namespace, string, var
    },
    morekeywords=[2]{% Execution control flow keywords
      throw, finally, try, for, foreach, goto, if, return, while, else,
    },
    keywordstyle=\color{codecolorkeyword}\bfseries,
    keywordstyle=[2]{\color{codecolorkeywordcontrolflow}\bfseries},
    %morecomment=[s]{//}{},
    morestring=*[s]{\$"}{"},
    moredelim=*[s]{\{}{\}},
    moredelim=[is][\color{codecolormethod}]{|}{|}
  }
}

\newcommand{\lstsetcsharp}[5]{% [extra], label, caption, types, enums
  \lstsetcsharpsettings{}
  \lstset{
    caption={#3},
    label={#2},
    types={#4},
    enums={#5},
    #1
  }
}

\lstnewenvironment{myListingCsharp}[5][]{% [extra], label, caption, types, enums
  \lstsetcsharp{#1}{#2}{#3}{#4}{#5}%
}{
}

\lstnewenvironment{myListingCsharpNoPageBreak}[5][]{% [extra], label, caption, types, enums
  \noindent\minipage{\linewidth}\medskip
  \lstsetcsharp{#1}{#2}{#3}{#4}{#5}%
}{%
  \endminipage%
}

\lstnewenvironment{myListingC}[5][]{% [extra], label, caption, types, enums
  % \noindent\minipage{\linewidth}\medskip
  \lstset{
    caption={#3},
    label={#2},
    language=C,
    types={#4},
    enums={#5},
    moredelim=[is][\color{codecolormethod}]{|}{|},
    #1
  }
}{%
  % \endminipage
}

% custom XML language to get nicer formatting
\lstdefinelanguage{XML}
{
  basicstyle=\ttfamily\footnotesize,
  morestring=[b]",
  morecomment=[s]{<?}{?>},
  morecomment=[s]{<!--}{-->},
  commentstyle=\color{codecolorcomment},
  moredelim=[s][\color{black}]{/>}{<},
  moredelim=[s][\color{black}]{/>}{</},
  moredelim=[s][\color{black}]{>}{<},
  stringstyle=\color{codecolorxmlstring},
  identifierstyle=\color{codecolorxmltag},
}
\lstnewenvironment{myListingXml}[4][]{% [extra], label, caption, attributes
  \noindent\minipage{\linewidth}\medskip
  \lstset{
    caption={#3},
    captionpos=t,
    label={#2},
    language=XML,
    attributes={#4},
    attributestyle=\color{codecolorxmlattribute},
    #1
  }
}{%
  \endminipage
}

% general listing env

\newenvironment{myListingFrame}{%
  \begin{mdframed}[
    hidealllines=true,
    innerleftmargin=0,
    innerrightmargin=0
    ]
  \hrule\medskip
}{%
  \medskip\hrule
  \end{mdframed}
}

\newenvironment{myListing}[3][]{% [short caption], label, caption
  \begin{mdframed}[
    hidealllines=true,
    innerleftmargin=0,
    innerrightmargin=0
    ]
  \myCaptionOf{lstlisting}{#1}{#3}\label{#2}% for tables, the caption goes to the top
  \hrule\medskip
}{%
  \medskip\hrule
  \end{mdframed}
}

% settings for uniform verbatim environment
\DefineVerbatimEnvironment%
{myVerbatim}{Verbatim}
{frame=lines,fontsize=\footnotesize,framesep=1em,commandchars=\&\{\}}

\newenvironment{myListingVerbatim}[3][]{% [short caption], label, caption
  \mdframed[
    hidealllines=true,
    innerleftmargin=0,
    innerrightmargin=0
    ]%
  \myCaptionOf{lstlisting}{#1}{#3}\label{#2}% for tables, the caption goes to the top
  \myVerbatim
}{%
  \endmyVerbatim%
  \endmdframed%
}


% typesetting binary and other values
\newcommand{\binary}[1]{0x#1}

% macros for typesetting various code elements inside the text

\definecolor{bytefieldunused}{gray}{0.8}

\newcommand{\class}[1]{{\color{codecolorclass}\texttt{#1}}}
\newcommand{\genericClass}[2]{\texttt{\class{#1}\allowbreak{}<{#2}>}}
\newcommand{\interface}[1]{{\color{codecolorclass}\texttt{#1}}}
\newcommand{\keyword}[1]{{\color{codecolorkeyword}\texttt{\textbf{#1}}}}
\newcommand{\namespace}[1]{{\color{codecolornamespace}\texttt{#1}}}
\newcommand{\method}[1]{{\color{codecolormethod}{\texttt{#1}}}}
\newcommand{\enum}[1]{{\color{codecolorenum}{\texttt{#1}}}}
\newcommand{\xmltag}[1]{{\color{codecolorxmltag}{\texttt{#1}}}}

% macros for typesetting description lists with C# class members

\newcommand{\propgetset}{\{ \keyword{get}; \keyword{set}; \}}
\newcommand{\propget}{\{ \keyword{get}; \}}
\newcommand{\propset}{\{ \keyword{set}; \}}

\newcommand{\ditemsrcsize}{\footnotesize}


\newcommand{\ditemmethodseparator}{}

% method to apply in \foreach macro
\def\makeditemmethodarg#1#2{\ditemmethodseparator{}{#2}%
\renewcommand{\ditemmethodseparator}{, }%set proper separator for the next item
}

\newcommand{\ditemmethodmakearglist}[1]{% arglist
\renewcommand{\ditemmethodseparator}{}%clear the separator for the first item
\foreach{\makeditemmethodarg}{}{#1}%
}

\newcommand{\ditemmethod}[4][\class]% [type typeset] type, name, args (types only)
  {\ditem{\ditemsrcsize\texttt{#1{#2} \method{#3}(#4)}}}

\newcommand{\ditemmethodWithComment}[5][\class]% [type typeset] type, name, args (types only)
  {\ditemWithComment{\ditemsrcsize\texttt{#1{#2} \method{#3}(#4)}}{#5}}

\newcommand{\ditemctor}[2]% name, args
  {\ditem{\ditemsrcsize\texttt{\method{#1}(#2)}}}

\newcommand{\ditemproperty}[4][\class]% [type typeset] type, name, get/set
  {\ditem{\ditemsrcsize\texttt{#1{#2} \method{#3} #4}}}

% checkmark
\def\checkmark{\tikz\fill[scale=0.4](0,.35) -- (.25,0) -- (1,.7) -- (.25,.15) -- cycle;}

% colored bitboxes for the bytefield package
\newcommand{\colorbitbox}[3]{%
\rlap{\bitbox{#2}{\color{#1}\rule{\width}{\height}}}%
\bitbox{#2}{#3}}
\newcommand{\colorwordbox}[4]{%
\rlap{\wordbox[#2]{#3}{\color{#1}\rule{\width}{\height}}}%
\wordbox[#2]{#3}{#4}}

% macro for typesetting transport parameters
\newcommand{\quicTransportParameter}[1]{\texttt{#1}}
\newcommand{\MaxAckDelay}{\quicTransportParameter{max\_ack\_delay}}
\newcommand{\MaxIdleTimeout}{\quicTransportParameter{max\_idle\_timeout}}
\newcommand{\DisableActiveMigration}{\quicTransportParameter{disable\_active\_migration}}
\newcommand{\MaxUdpPayloadSize}{\quicTransportParameter{max\_udp\_payload\_size}}

% macro for typesetting QUIC frame types
\newcommand{\quicFrame}[1]{\texttt{#1}}
\newcommand{\PADDING}{\quicFrame{PADDING}}
\newcommand{\PING}{\quicFrame{PING}}
\newcommand{\ACK}{\quicFrame{ACK}}
\newcommand{\RESETSTREAM}{\quicFrame{RESET\_STREAM}}
\newcommand{\STOPSENDING}{\quicFrame{STOP\_SENDING}}
\newcommand{\CRYPTO}{\quicFrame{CRYPTO}}
\newcommand{\NEWTOKEN}{\quicFrame{NEW\_TOKEN}}
\newcommand{\STREAM}{\quicFrame{STREAM}}
\newcommand{\MAXDATA}{\quicFrame{MAX\_DATA}}
\newcommand{\MAXSTREAMDATA}{\quicFrame{MAX\_STREAM\_DATA}}
\newcommand{\MAXSTREAMS}{\quicFrame{MAX\_STREAMS}}
\newcommand{\DATABLOCKED}{\quicFrame{DATA\_BLOCKED}}
\newcommand{\STREAMDATABLOCKED}{\quicFrame{STREAM\_DATA\_BLOCKED}}
\newcommand{\STREAMSBLOCKED}{\quicFrame{STREAMS\_BLOCKED}}
\newcommand{\NEWCONNECTIONID}{\quicFrame{NEW\_CONNECTION\_ID}}
\newcommand{\RETIRECONNECTIONID}{\quicFrame{RETIRE\_CONNECTION\_ID}}
\newcommand{\PATHCHALLENGE}{\quicFrame{PATH\_CHALLENGE}}
\newcommand{\PATHRESPONSE}{\quicFrame{PATH\_RESPONSE}}
\newcommand{\CONNECTIONCLOSE}{\quicFrame{CONNECTION\_CLOSE}}
\newcommand{\HANDSHAKEDONE}{\quicFrame{HAND\-SHAKE\_DONE}}

% macros for commonly used class names
\newcommand{\ValueTaskOf}[1]{\genericClass{ValueTask}{#1}}
\newcommand{\MemoryOf}[1]{\genericClass{Memory}{#1}}
\newcommand{\ListOf}[1]{\genericClass{List}{#1}}
\newcommand{\ReadOnlySpanOf}[1]{\genericClass{ReadOnlySpan}{#1}}
\newcommand{\ReadOnlyMemoryOf}[1]{\genericClass{ReadOnlyMemory}{#1}}
\newcommand{\ReadOnlySequenceOf}[1]{\genericClass{ReadOnlySequence}{#1}}
\newcommand{\SpanOf}[1]{\genericClass{Span}{#1}}
\newcommand{\ChannelOf}[1]{\genericClass{Channel}{#1}}
\newcommand{\ArrayOf}[1]{\texttt{{#1}[]}}
\newcommand{\FuncOf}[1]{\genericClass{Func}{#1}}
\newcommand{\ArrayPoolOf}[1]{\genericClass{ArrayPool}{#1}}
\newcommand{\ValueTask}{\class{ValueTask}}

\newcommand{\Byte}{\keyword{byte}}
\newcommand{\Long}{\keyword{long}}
\newcommand{\Int}{\keyword{int}}
\newcommand{\bool}{\keyword{bool}}
\newcommand{\CancellationToken}{\class{CancellationToken}}

\newcommand{\Socket}{\class{Socket}}
\newcommand{\Stream}{\class{Stream}}
\newcommand{\TcpClient}{\class{TcpClient}}
\newcommand{\SslStream}{\class{SslStream}}
\newcommand{\QuicSocketContext}{\class{Quic\allowbreak{}Socket\allowbreak{}Context}}
\newcommand{\QuicServerSocketContext}{\class{Quic\allowbreak{}Server\allowbreak{}Socket\allowbreak{}Context}}
\newcommand{\QuicClientSocketContext}{\class{Quic\allowbreak{}Client\allowbreak{}Socket\allowbreak{}Context}}
\newcommand{\QuicConnectionContext}{\class{Quic\allowbreak{}Connection\allowbreak{}Context}}
\newcommand{\QuicReader}{\class{Quic\allowbreak{}Reader}}
\newcommand{\QuicWriter}{\class{Quic\allowbreak{}Writer}}
\newcommand{\CryptoSeal}{\class{Crypto\allowbreak{}Seal}}
\newcommand{\CryptoSealAlgorithm}{\class{Crypto\allowbreak{}Seal\allowbreak{}Algorithm}}
\newcommand{\ReceiveStream}{\class{Receive\allowbreak{}Stream}}
\newcommand{\SendStream}{\class{Send\allowbreak{}Stream}}
\newcommand{\RecoveryController}{\class{Recovery\allowbreak{}Controller}}
\newcommand{\RangeSet}{\class{RangeSet}}
\newcommand{\PacketNumberWindow}{\class{Packet\allowbreak{}Number\allowbreak{}Window}}
\newcommand{\PacketNumberSpace}{\class{Packet\allowbreak{}Number\allowbreak{}Space}}
\newcommand{\SentPacket}{\class{Sent\allowbreak{}Packet}}
\newcommand{\ICongestionController}{\interface{ICongestion\allowbreak{}Controller}}
\newcommand{\StreamCollection}{\class{Stream\allowbreak{}Collection}}
\newcommand{\ProcessPacketResult}{\enum{Process\allowbreak{}Packet\allowbreak{}Result}}
\newcommand{\TransportParameters}{\class{Transport\allowbreak{}Parameters}}
\newcommand{\ITls}{\interface{ITls}}
\newcommand{\QuicTlsProvider}{\class{QuicTlsProvider}}
\newcommand{\MockQuicTlsProvider}{\class{MockQuicTlsProvider}}
\newcommand{\OpenSslQuicTlsProvider}{\class{OpenSslQuicTlsProvider}}
\newcommand{\OpenSslTls}{\class{OpenSslTls}}
\newcommand{\MockTls}{\class{MockTls}}
\newcommand{\SSL}{\class{SSL}}
\newcommand{\SSLCTX}{\class{SSL\_CTX}}
\newcommand{\QuicTestBase}{\genericClass{Quic\allowbreak{}Test\allowbreak{}Base}{\class{T}}}
\newcommand{\QuicListenerTests}{\genericClass{Quic\allowbreak{}Listener\allowbreak{}Tests}{\class{T}}}
\newcommand{\QuicConnectionTests}{\genericClass{Quic\allowbreak{}Connection\allowbreak{}Tests}{\class{T}}}
\newcommand{\QuicStreamTests}{\genericClass{Quic\allowbreak{}Stream\allowbreak{}Tests}{\class{T}}}
\newcommand{\ManualTransmissionQuicTestBase}{\class{Manual\allowbreak{}Transmission\allowbreak{}Quic\allowbreak{}Test\allowbreak{}Base}}

\newcommand{\QuicConnection}{\class{Quic\allowbreak{}Connection}}
\newcommand{\QuicStream}{\class{Quic\allowbreak{}Stream}}
\newcommand{\QuicListener}{\class{Quic\allowbreak{}Listener}}

\newcommand{\QuicConnectionProvider}{\class{Quic\allowbreak{}Connection\allowbreak{}Provider}}
\newcommand{\QuicStreamProvider}{\class{Quic\allowbreak{}Stream\allowbreak{}Provider}}
\newcommand{\QuicListenerProvider}{\class{Quic\allowbreak{}Listener\allowbreak{}Provider}}
\newcommand{\QuicImplementationProvider}{\class{Quic\allowbreak{}Implementation\allowbreak{}Provider}}
\newcommand{\QuicImplementationProviders}{\class{Quic\allowbreak{}Implementation\allowbreak{}Providers}}

\newcommand{\MsQuicConnection}{\class{MsQuic\allowbreak{}Connection}}
\newcommand{\MsQuicStream}{\class{MsQuic\allowbreak{}Stream}}
\newcommand{\MsQuicListener}{\class{MsQuic\allowbreak{}Listener}}
\newcommand{\MsQuicImplementationProvider}{\class{MsQuic\allowbreak{}Implementation\allowbreak{}Provider}}

\newcommand{\ManagedQuicConnection}{\class{Managed\allowbreak{}Quic\allowbreak{}Connection}}
\newcommand{\ManagedQuicStream}{\class{Managed\allowbreak{}Quic\allowbreak{}Stream}}
\newcommand{\ManagedQuicListener}{\class{Managed\allowbreak{}Quic\allowbreak{}Listener}}
\newcommand{\ManagedQuicImplementationProvider}{\class{Managed\allowbreak{}Quic\allowbreak{}Implementation\allowbreak{}Provider}}

\newcommand{\QuicListenerOptions}{\class{Quic\allowbreak{}Listener\allowbreak{}Options}}
\newcommand{\ServerAuthenticationOptions}{\class{ServerAuthenticationOptions}}
\newcommand{\QuicClientConnectionOptions}{\class{QuicClientConnectionOptions}}

% macro for the System.Net.Quic.dll library
\newcommand{\SystemNetQuicDll}{\texttt{System.\allowbreak{}Net.\allowbreak{}Quic.dll}}

% macros for typesetting paths and filenames
\newcommand{\filename}[1]{\path{#1}}
