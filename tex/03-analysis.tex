\chapter{Analysis}\label{chap:03-analysis}

This chapter analyzes the protocol and selects the necessary subset needed to evaluate a \dotnet{}
implementation of the QUIC protocol. Afterward, we design the architecture and outline the
implementation of its major parts. Lastly, we investigate the means for unit testing and debugging
of the implementation.

\section{Implemented Feature Subset Selection}\label{sec:03-feature-selection}

The decision which features of the QUIC protocol we implement is guided by the goals we set in
\autoref{sec:01-goals}. For feature selection, these goals can be rephrased into the following:

\begin{enumerate}

  \item Support basic data transport, enabling some experimentation with QUIC as the transport for
application layer protocols. \textit{(goals 1 and 2)}

  \item Enable performance measurements that are representative of the potential full QUIC
implementation. \textit{(goals 1 and 3)}

\end{enumerate}

The first goal requires full implementation of the multiplexed stream abstraction as defined by the
QUIC specification. It also requires implementing loss detection and recovery to ensure that no data
gets lost during the transport.

In order to get representative performance measurements, we should implement all performance
affecting features. The most important are packet protection and flow control because they influence
performance throughout the lifetime of the QUIC connection. To summarize, the thesis should
implement at least the following features:

\begin{itemize}

    \item Connection lifetime support (establishment, termination)

    \item Stream multiplexing

    \item Packet protection

    \item Loss detection and recovery

    \item Flow control

\end{itemize}

On the other hand, we can disregard many QUIC features that react to one-time events or do not
otherwise influence the implementation's performance. These features include:

\begin{itemize}

    \item Connection migration, and therefore multiple Connection IDs support

    \item Complex (token-based) address validation

    \item Network path MTU detection

    \item Version negotiation

    \item 1-RTT key updates

    \item Advanced security measures (see Section 21 in transport
specification~\cite{draft-ietf-quic-transport})

\end{itemize}

The rest of the features form a grey area that can be implemented fully, partially, or even not at
all if convenient.

\section{Design Considerations}

Before we start the actual analysis, we will briefly outline the design principles used for the
actual design of the implementation and their rationale.

\subsection{Performance}

One of the critical factors in the decision between managed \dotnet{} implementation of QUIC or
using an external library like \libmsquic{} is performance. Therefore, the implementation design
decisions should focused on greater performance, possibly sacrificing maintainability if the
trade-off was justified.

As a general rule, the implementation will:

\begin{itemize}

    \litem{Avoid excessive heap allocations} Although heap allocation is cheap in \dotnet{}, the
actual price is paid during garbage collection. The more objects are allocated, the more frequent
garbage collections are. Each collection introduces a small stall into the program which could
disrupt internal timing of the QUIC implementation. Therefore, heap allocations on hot paths of the
code executions should be minimized.

    \litem{Prefer return codes over exceptions} Throwing an exception is an expensive operation, and
their frequent use would have negative impact on the performance.

\end{itemize}

\subsection{Testability}

The second design aspect we focusd on is the testability of the implementation. Ideally, the design
would minimize the need for live debugging of the implementation. This is especially important
because stopping the implementation on a breakpoint inevitably disrupts the connection, possibly
leading to termination because of idle timeout (see \autoref{sec:02-idle-timeout}).

The design intention was to allow writing deterministic automated tests that can inspect the packets
sent by the endpoint and verify that they are consistent with the behavior defined by the QUIC
specification.

\section{Target \dotnet{} API}

The public API, which was designed by the \dotnet{} team for use by other developers, uses similar
concepts like TCP and UDP \dotnet{} networking APIs. There is a \QuicListener{} class intended to be
used by servers for listening for incoming connections, similarly to using the \class{TcpListener}
class. The actual connection is represented by the \QuicConnection{} class, which can be compared to
\class{TcpClient} class. The individual QUIC streams are exposed using the \QuicStream{} class,
which implements the \Stream{} abstraction.

The API is expected to be used asynchronously using the \keyword{async}/\keyword{await} model. Most
methods return a \ValueTask{} which should be \keyword{await}ed to efficiently wait until the
operation completes. The full list of methods with detailed description can be found later in
\autoref{sec:06-api}.

\section{High-Level Architecture}\label{sec:03-high-level-architecture}

The \QuicConnection{} implementation will require some sort of background processing
thread\footnote{In order to reduce verbosity, this text will be using the term \textit{thread} to
mean both a dedicated \class{Thread} instance or a \class{Task} running on a thread-pool thread.} to
send acknowledgments for incoming packets and to resend data after being determined lost due to a
timeout. Because the correctness of a multithreaded code is hard to test, our implementation of
\QuicConnection{} will not interface with the underlying \Socket{} instance directly. Instead, the
implementation will provide an internal interface for exchanging the datagrams to be sent/received.
The actual sending of these datagrams will be handled by a separate class named
\QuicSocketContext{}. By separating the socket I/O from the connection logic implementation, we can
write unit tests that inspect the sent datagrams and assert that their content conforms to the
protocol specification.

On the server's side, the \QuicSocketContext{} class will also handle any stateless packet
processing, such as sending Retry and Version Negotiation packet. It will be also responsible for
matching packets to the appropriate \QuicConnection{} instances and queueing new connections to
\QuicListener{} to be read by the application. \autoref{fig:03-architecture} illustrates the
relationship between \QuicSocketContext{}, \QuicConnection{} and \QuicListener{} classes.

\begin{myFigure}{fig:03-architecture}{High-level background processing architecture.}

\resizebox{0.76\textwidth}{!}{\input{img/03-architecture.pdf_tex}}

\end{myFigure}

\subsection{Servicing the Socket}

The \QuicSocketContext{} class will implement the necessary background processing management. Its
responsibilies are:

\begin{itemize}

  \item routing incoming datagrams from \Socket{} to the proper \QuicConnection{} instance;

  \item calling timeout handlers after a timeout set by the connection expires; and

  \item sending outgoing datagrams provided by the \QuicConnection{}.

\end{itemize}

For performance reasons, it may be better to process timeouts and send the datagrams using one
thread and process received datagrams using another thread. However, the cost of synchronization of
the shared state may outweigh the performance gain. Our prototype implementation will use a single
thread to service both types of events but will allow for the possibility of future experimentation
with separate threads for sending and receiving.

\subsubsection{Processing Multiple Connections in Parallel}

For client connections, the background processing provided by the \QuicSocketContext{} class needs
to service only a single connection. However, on the server's side, there can be multiple
connections receiving datagrams from the same socket and, therefore, being served by a single
\QuicSocketContext{} instance. Servicing multiple \QuicConnection{} instances by only one thread
could limit the server's throughput.

In order to allow processing multiple QUIC connections in parallel, our implementation will use a
dedicated background \Task{} for each \QuicConnection{} instance. Our implementation separates the
per-connection logic into separate \QuicConnectionContext{} class. The \QuicSocketContext{} class
will be directly responsible only for stateless packet processing like sending a Version Negotiation
packet for incoming packets with unsupported versions. Packets belonging to existing connections
will be queued for processing by the appropriate \QuicConnectionContext{} instance using the
\ChannelOf{} class which provides efficient implemntation of a producer-consumer queue. A possible
runtime structure of the \QuicSocketContext{} servicing two separate connections is illustrated in
\autoref{fig:03-socket-context-architecture}. The two \QuicConnectionContext{} instances are
serviced by separate dedicated threads. The \QuicSocketContext{} itself does not need a dedicated
thread. Instead, its logic is performed in whatever thread that completes the pending asynchronous
socket receive call.

\begin{myFigure}{fig:03-socket-context-architecture}{Architecture of the server-side background
processing.}

\resizebox{0.88\textwidth}{!}{\input{img/03-server-socket-context.pdf_tex}}

\end{myFigure}

\subsubsection{Future Support for Connection Migration}

Lastly, we need to review the architecture for possible support for the connection migration
feature. Thanks to the \QuicConnectionContext{} not depending directly on a particular \Socket{}
instance, it is conceivable that a single \QuicConnectionContext{} instance could be part of two
\QuicSocketContext{}s --- one for the old endpoint address, and the other for the new one. Use of
the \ChannelOf{} class already ensures that the incoming datagrams could be queued concurrently by
both \QuicSocketContext{} instances. The only modification needed on the architecture level would be
allowing the \QuicConnectionContext{} to select the right \Socket{} from which the outgoing
datagrams should be sent.

\subsection{Public API Threading model}

The target API uses the \class{ValueTask} type designed for efficient asynchronous methods. The user
code will start an asynchronous operation that can be completed by a background thread servicing the
connection. This way, the user code cannot block the connection's background thread, which could
otherwise cause timeouts to be missed.

The \QuicConnection{} can be potentially used in a multithreaded environment, and the implementation
should allow concurrent usage of \QuicConnection{} and \QuicStream{} classes when it makes sense.
For example, it makes sense for an application code to process each \QuicStream{} in a different
thread. However, it does not make sense to concurrently write into one \QuicStream{} from two
threads without any synchronization. Our implementation will, therefore, provide the following
thread-safety guarantees for the API\@:

\begin{itemize}

  \item Individual streams can be used concurrently from different threads. However, each direction
of the stream (reading and writing) can be accessed only by one thread at a time and must be,
therefore, synchronized.

  \item Accepting/opening new streams on a connection can be done concurrently from multiple
threads.

  \item All other operations on \QuicConnection{} must be synchronized, including, e.g., starting
and aborting the connection.

\end{itemize}

\section{Packet Serialization/Deserialization}

Special care needs to be taken when implementing the QUIC packets' serialization to the wire format.
Inefficient data representation can negatively impact overall performance because processing
individual QUIC packets and frames will likely be a hot path in the implementation. The packet and
frame representation should be, therefore, carefully designed to avoid allocations.

The incoming packets can contain arbitrary encoding errors that should be handled efficiently and
gracefully, i.e., without throwing exceptions. Possible errors include values being outside of the
range of allowed values and incomplete or damaged packets.

\subsection{QUIC Packet and Frame Representation}\label{sec:03-data-representation}

Some packets contain many fields. Therefore, passing them around as individual variables would make
the implementation harder to maintain. The QUIC frames form coherent messages that should be
represented by individual \dotnet{} types. This gives us also an opportunity to keep the
serialization and deserialization code next to each other, making it easier to ensure that, e.g.,
the order of the serialized fields matches in both methods.

Representing QUIC frames as individual classes would introduce many heap allocations for every
received QUIC packet. This thesis, therefore, will model QUIC frames and the QUIC packet headers as
value types. Also, the payload of some frames may consist of large blocks of memory. Examples
include \STREAM{} and \CRYPTO{} frames, which can fill the entire payload of the QUIC packet.
Duplicating this block of memory into a separate \ArrayOf{\Byte{}} would be another unnecessary
memory allocation. \dotnet{} Core~2.1 introduced the \SpanOf{\class{T}} type, which can be used to
efficiently reference an arbitrary memory block, including memory allocated on the stack using the
\keyword{stackalloc} keyword.

The \SpanOf{\class{T}} type is a \keyword{ref struct} --- a special kind of value type which can be
stored only in local variables or inside another \keyword {ref struct}s. Limitations on the usage of
\keyword{ref struct}s also forbid their use as generic type arguments, e.g., for \FuncOf{} and other
generic delegates. This limitation is acceptable for \QuicConnection{} implementation because the
QUIC frames can be processed one after another right after being deserialized from the QUIC packet.

\subsection{QuicReader and QuicWriter}

Both serialization and deserialization require maintaining the current position in the buffer being
read from or written into. In order to simplify the serialization or deserialization code, our
implementation introduces \QuicReader{} and \QuicWriter{} classes as a primary means of reading and
writing QUIC primitives to memory. We have chosen \QuicReader{} and \QuicWriter{} to be reference
types allocated on the heap, because we expect to cache and reuse their instances multiple times.

The \QuicReader{} and \QuicWriter{} classes also take care of converting the endianity of the data
between big-endian used by QUIC and the endianity used by the machine running the application. This
is done by forwarding the calls to respective methods on the \class{BinaryPrimitives} class.

\section{Stream Implementation}\label{sec:03-stream-implementation}

QUIC is a transport protocol. Therefore, its entire purpose is transferring streams of data. Since
it is likely to be the hot path of the implementation, the internal handling of stream data must be
efficient and avoid unnecessary copying of stream data blocks.

QUIC recognizes four types of streams. These streams can have a sending part, receiving part, or
both. The fact which endpoint initiated the stream controls only which flow control limits apply to
that stream. Otherwise, all streams are handled equally. As mentioned in
\autoref{sec:02-stream-types}, bidirectional streams can be implemented as two unidirectional
streams. Therefore, our implementation will divide the QUIC stream logic into \SendStream{} and
\ReceiveStream{} classes, which will handle the sending and receiving behavior of the stream.

\subsection{Receiving Part of the Stream}\label{sec:03-receive-stream}

The implementation of \ReceiveStream{} must buffer the received data before delivering them to the
application in case the data are received out-of-order. It also needs to track the amount of
buffered memory and how much data was delivered to the application in order to correctly update flow
control limits for the peer.

\subsubsection{Stream Data Buffering}

Ideally, the stream implementation would be structured so that the stream data were copied straight
from the decrypted packet to the memory provided by the application. This would be possible if the
API used an event-based model with callbacks for incoming data. However, the current API design
utilizes a method-based model of the \Stream{} abstraction. If the application does not call the
\method{Read} or \method{ReadAsync} method, there is no application buffer to deliver into.

This implies that in order to avoid intermediate copies of the stream data, the buffer holding the
QUIC packet in which the stream data arrived cannot be reused to receive other QUIC packets until
the contained stream data are delivered to the application. Tracking which buffers can be reused for
receiving the following QUIC packets can be complicated because QUIC packets can carry multiple
\STREAM{} frames.

We believe that the implementational complexity of avoiding intermediate copies outweighs the
possible performance gain. For this reason, our pilot implementation will use a two-copy approach:
the first copy from the packet into an intermediate buffer, the second copy from the intermediate
buffer into the destination memory provided by the application.

\subsubsection{Packet Reordering and Data Deduplication}

Unreliability of the UDP protocol can cause QUIC packets to be reordered, lost or received multiple
times. Because of that, sections of the stream may be received multiple times, and the contents of
\STREAM{} frames can arbitrarily overlap.

Even though QUIC Flow control provides an upper limit on the data that needs to be buffered at any
given moment, allocating one large buffer may be a waste of memory, as the entire buffer might not
be needed at any given point in time. Therefore, our implementation uses a list of smaller buffers
and allocates only the necessary number of buffers needed to buffer currently received data. In
order to reduce the pressure on the \gls{gc}, buffers are reused using the \ArrayPoolOf{\Byte{}}
class to avoid their frequent allocation.

\subsubsection{Reading Data by the Application}

Because user code runs on a different thread from the internal \QuicConnection{} logic, access to
the buffered data must be synchronized. Our implementation uses the \ChannelOf{} class, which
provides an efficient implementation of the producer-consumer queue with support for asynchronous
operations using the \ValueTaskOf{} type.

Each time data are received for the stream, the implementation checks if there is a contiguous block
of memory that could be delivered to the application. For this reason, the implementation needs to
keep track of the parts of the stream which have already been received. If there is a new part of
the stream that can be delivered, a view into the relevant region of the buffer is queued into the
\ChannelOf{} using the \MemoryOf{\Byte{}} type.

\autoref{fig:03-receive-stream} illustrates how all parts of the \ReceiveStream{} work together. The
data from the incoming \STREAM{} frame are copied to the proper offset in intermediate buffers.
These buffers are rented from an \ArrayPoolOf{\Byte{}} as needed. A \MemoryOf{\Byte} instance
providing a view into the newly deliverable parts of the stream is then queued into a \ChannelOf{}
instance for delivery. The application thread retrieves the memory regions as needed and copies the
data into the buffer provided by the application. Once all data from an intermediate buffer are
delivered, the intermediate buffer is returned to the \ArrayPoolOf{\Byte{}} to be reused for
buffering future data.

\begin{myFigure}{fig:03-receive-stream}{Implementation of the receiving part of the stream}

  \resizebox{\linewidth}{!}{\input{img/03-receive-stream.pdf_tex}}

\end{myFigure}

\subsubsection{Flow Control Considerations}

QUIC Flow control limits for streams specify the maximum offset of data that an endpoint can send.
An endpoint needs to update these limits by sending a \MAXSTREAMDATA{} frame after the data are
delivered to the application. However, sending updated limits in every packet could waste space that
could be used by the application data. On the other hand, updating the limits too late could lead to
the other endpoint being blocked and signal the fact by sending a \STREAMDATABLOCKED{} frame. In
such a state, the endpoint cannot send more application data until it receives an update via
\MAXSTREAMDATA{} frame.

Our implementation will send a \MAXSTREAMDATA{} frame after the client uses more than half of the
limit provided since the last update. This should prevent updates being too often and at the same
time early enough that the sender does not run out of flow control credit.

\subsection{Sending Part of the Stream}\label{sec:03-send-stream}

The sending part of a QUIC stream must keep track of the state of all outbound data on the stream.
Any part of the stream can be lost during transmission, requiring its retransmission, and any
previously sent part of the stream can be acknowledged.

\subsubsection{Stream Data Buffering}

Similarly to buffering received data, we will analyze how many times the outgoing data need to be
copied before they are sent. The semantics of the \method{Write} and \method{WriteAsync} methods on
the \Stream{} class are such that when the method completes, the provided memory can be reused for
other purposes.

In this case, a single-copy implementation approach would require that the \method{Write} and
\method{WriteAsync} methods complete after the peer acknowledges that the data were received. In the
best-case scenario where no packet loss occurs, this approach implies a full round trip delay for
calling either of the methods. This approach would make the \QuicStream{} implementation
inconsistent with the other network-enabled \Stream{} implementations. Moreover, substantial time
would be spent waiting for acknowledgment instead of sending more data. In order to fully saturate
the available bandwidth, users of the \QuicStream{} would have to either provide very large data
buffers or use the \method{WriteAsync} method and overlap its execution by maintaining multiple
outstanding \ValueTask{}s.

In order to fit with the other \Stream{} implementations, our implementation will create an
intermediate copy of the data, and the \method{Write} and \method{WriteAsync} will complete once the
data is internally buffered. This allows applications to queue enough data to maximally utilize the
connection's bandwidth. In order to be memory-efficient, our implementation will, as in
\ReceiveStream{} implementation, use an \ArrayPoolOf{\Byte{}} to reduce the pressure on the garbage
collector.

\subsubsection{Acknowledgement, Loss, and Retransmission}

Each byte that in the stream can be, conceptually, in three different states:

\begin{itemize}

  \litem{Pending} The byte needs to be sent to the peer in some future \STREAM{} frame.

  \litem{In-flight} The byte has been sent, but it is uncertain if the containing packet was
received.

  \litem{Acknowledged} The packet containing the byte has been acknowledged by the other endpoint.
It is no longer necessary to buffer this byte.

\end{itemize}

The transitions between these states are straightforward. The data transition from pending to
in-flight by sending them in a \STREAM{} frame. When the packet is deemed lost by the loss detection
algorithm, the data transition back to the pending state, and if the packet has been acknowledged,
the data transition to the acknowledged state. The state of the individual parts of the stream can
be tracked by maintaining three sets of \textit{ranges} --- starts and ends of the blocks of data in
the same state.

\subsubsection{Writing Data by the Application}

Similarly to the receiving part of the stream, our implementation will use \ChannelOf{} type to
provide synchronization with the application thread. However, since \ChannelOf{} does not allow
random access to the provided data, a separate list of buffers is maintained for data which were
sent but not yet acknowledged.

\autoref{fig:03-send-stream} illustrates the process of writing data into the \SendStream{}. The
application data are copied to an internal buffer. This buffer comes from a shared
\ArrayPoolOf{\Byte{}}. When the buffer is full or the stream is flushed, the buffer is enqueued into
the synchronization \ChannelOf{}. The background thread then dequeues buffers from the \ChannelOf{}
and appends them into a list of retransmission buffers. Each time a new \STREAM{} frame for this
stream is written into a QUIC packet, the earliest not-yet-sent data are selected and copied into
the outgoing QUIC packet buffer. Once all data in a particular retransmission buffer are
acknowledged, the buffer can be removed and returned to the \ArrayPoolOf{\Byte}.

\begin{myFigure}{fig:03-send-stream}{Implementation of the sending part of the stream}

  \resizebox{\linewidth}{!}{\input{img/03-send-stream.pdf_tex}}

\end{myFigure}


\subsection{Abort/Dispose Model for Streams}

The \QuicStream{} class implements the \interface{IDisposable} interface which should provide
automatic closing of the stream when \QuicStream{} is used in a \keyword{using} statement or
\keyword{using} block. This implies resetting the writable part of the stream using \RESETSTREAM{}
frame and requesting reset of the readable part of the stream using \STOPSENDING{} frame.

However, both these frames require an application-level error code. QUIC specification does not
define any transport-level error codes for these stream operations and there is no way for the
application to specify which error code should be used. This has been established as a flaw of the
current API and is expected to change in future iterations design
iterations~\cite{dotnetGithubQuicShutdown}.

Leaving \interface{IDisposable} unimplemented would be misleading to users in the prototype
implementation. Therefore, our implementation will deviate from the specification by using error
code 0 regardless of its semantics in the application protocol, unless the stream has been
explicitly closed by the application using the \method{AbortRead} or \method{AbortWrite} methods.

Another area where the QUIC API is not well defined is the behavior of pending \method{AcceptStream}
calls on \QuicConnection{} when the connection is closed. In case the connection has been closed
gracefully from the application protocol's perspective, it would be better if this method did not
throw an exception but returned \keyword{null} reference. However, the semantics of the error codes
are defined by the application protocol and, therefore, the QUIC implementation cannot distinguish
between graceful or abortive connection close. Until the behavior of these methods in such scenarios
is better defined, our implementation will throw \class{QuicConnectionAborted\allowbreak{}Exception}
for all pending async calls.

\section{TLS Implementation}

TLS handshake forms an integral part of the QUIC connection establishment. Because correct TLS
implementation is crucial for ensuring the security of the resulting implementation, this thesis
should avoid implementing TLS algorithms by itself. Instead, it should reuse some existing and
well-tested implementation.

The novel way QUIC integrates with TLS requires a specific API from the TLS implementation. Below is
a non-exhaustive list of operations the TLS library's API must provide:

\begin{itemize}

  \item Querying the current state of the handshake

  \item Retrieving both the application-level secrets and secrets used to protect the handshake process

  \item Retrieving raw, unencrypted TLS messages to be sent to the other endpoint

  \item Obtaining the negotiated cipher

  \item Specifying protocols used for ALPN

  \item Specifying a custom TLS extension in order to exchange QUIC transport parameters

\end{itemize}

The \dotnet{} runtime libraries use different native libraries to provide TLS functionality on
different operating systems. On Windows, \dotnet{} uses \libname{Secure Channel}~\cite{Schannel}
(\libschannel{} for short) which is part of the Windows operating system. On Linux and macOS
systems, the \libopenssl{} library~\cite{OpenSSLWeb} is used.

\begin{description}

    \ditem{\libname{Secure Channel}} The \libschannel{} versions present in the latest Windows 10
builds support only TLS 1.2. However, future updates will also implement TLS 1.3. A preview of the
\libschannel{} with TLS 1.3 support can be obtained by installing an Insider build of Windows 10.
Because \libmsquic{} uses the \libschannel{} library when compiled for Windows, we assume that
\libschannel{} exposes the API necessary for a QUIC implementation.

    \ditem{\libopenssl{}} None of the mainstream versions of \libopenssl{} library expose necessary
API for integration into QUIC, and there are no plans to include such API in the next \libopenssl{}
3.0.0 release~\cite{OpensslBlogNoQuic}. However, developers at Akamai maintain a fork of
\libopenssl{} which adds the QUIC-enabling API~\cite{AkamaiOpensslGithub}. This modified version of
\libopenssl{} is used by \libmsquic{} (in Linux builds) and some other QUIC libraries like
\libname{quiche} from Cloudflare~\cite{quicheGithub}. Akamai's changes may be merged into
\libopenssl{} for the following 3.1.0 or later releases.

\end{description}

In conclusion, the APIs required for our QUIC implementation are currently only accessible in only
the preview versions of Windows 10. Relying solely on \libschannel{} for TLS 1.3 support would
severely impact cross-platform availability of our prototype implementation. We have, therefore,
decided to integrate with the modified \libopenssl{} library which supports all platforms supported
by \dotnet{}. This solution, however, has some drawbacks:

\begin{itemize}

  \item The modified \libopenssl{} binary must be present on the machine running our QUIC
implementation. This implies that the library must be compiled from source beforehand for the target
machine because pre-built binaries for the modified library are not available.

  \item Only a limited integration with X.509 certificates is possible because the
\class{X509Certificate} class implementation will be using a different binary --- \libname{CryptoAPI}
on Windows, unmodified \libopenssl{} on Linux and macOS\@.

\end{itemize}

These drawbacks are acceptable for the prototype implementation and will be eliminated once the
support for QUIC is released in mainstream versions of the \libopenssl{} and \libschannel{}
libraries.

\section{Packet Protection}\label{sec:03-packet-protection}

As described in \autoref{sec:02-packet-protection}, the packet encryption process consists of two
phases --- payload protection and header protection. The combined process requires the following
inputs:

\begin{itemize}

  \item Protection keys derived using the process explained in
\autoref{sec:02-encryption-key-derivation}

  \item Negotiated cipher

  \item Packet number (for encryption), or expected packet number (for decryption)

  \item The QUIC packet to encrypt or decrypt

\end{itemize}

The cipher is negotiated once and cannot be changed during the lifetime of the connection.
Protection keys can be changed only for the 1-RTT packets using the process of
\textit{\gls{key-update}} (see \autoref{sec:02-key-update}), which can be expected to be relatively
infrequent. The rest of the inputs change with every packet. Therefore, our implementation
encapsulates the packet protection implementation in a \class{CryptoSeal} class, which does not
depend on the rest of the QUIC implementation.

The process of receiving packets requires intermediate validation of the header fields. The
individual steps --- header protection and payload protection --- must be, therefore, exposed
separately. Also, the encryption and decryption should happen in-place to avoid unnecessary
allocations and copying of the packets.

Important consideration needs to be made for performing the actual encryption/decryption once all
inputs to \gls{aead} have been gathered. \dotnet{} does not contain an implementation of the CHACHA
family of ciphers. Fortunately, the \libopenssl{} library can be configured to not allow this cipher
to be negotiated and, therefore, our prototype can work without the CHACHA cipher support.

The other ciphers are based on the AES family of ciphers, which are supported by \dotnet{}. However,
individual classes implementing these ciphers do not share a common base class or interface.
Therefore, our implementation wraps the concrete AES implementations in classes derived from an
abstract \class{CryptoSealAlgorithm} class which defines a common interface required by
the \CryptoSeal{} class. \autoref{fig:03-crypto-seal} illustrates how the \CryptoSeal{} and
\class{CryptoSeal\allowbreak{}Algorithm} classes are connected.

\begin{myFigure}{fig:03-crypto-seal}{Relationship between CryptoSeal and CryptoSealAlgorithm
classes}

  \resizebox{\linewidth}{!}{\input{img/03-crypto-seal.pdf_tex}}

\end{myFigure}

The \gls{key-update} operation can be implemented by replacing the existing instance of the
\CryptoSeal{} class with a new one with the updated protection keys.

\section{Loss Detection and Loss Recovery}

In order to detect lost packets and retransmit any lost data, the \QuicConnection{} implementation
must keep track of which data have been sent in which packets and the timestamp when the packet was
sent for timeout detection. Our implementation will encapsulate this information and the loss
detection algorithm in a dedicated \RecoveryController{} class, which does not depend on the
\QuicConnection{} class. This way, its implementation can be unit tested separately from the rest of
the connection logic.

Another responsibility of the \RecoveryController{} class is maintaining the congestion window.
Although the specification defines only a single algorithm for congestion control based on TCP
NewReno~\autocite[Section~7]{draft-ietf-quic-recovery}, there are already experiments with other
algorithms like HyStart++ and CUBIC~\cite{cloudflareCubic}. The ability to support multiple such
algorithms could provide an opportunity for future experimentations. For that reason, the
\RecoveryController{} will use the \gls{strategy-pattern}~\cite{wiki:strategy-pattern} to allow
choosing the congestion control algorithm at runtime.

\section{Automated Testing}

It is necessary to implement automated tests that assert that the implementation conforms to the
QUIC protocol specification. Additionally, the public API makes use of existing concepts, namely the
\class{Stream} abstraction for QUIC streams, and thus it is necessary to ensure that \QuicStream{}
conforms to the expected \class{Stream} behavior.

The \dotnet{} runtime repository into which we wish to integrate our QUIC implementation uses the
\xUnit{} testing framework~\cite{xunit}. The \xUnit{} framework is one of the most mature testing
frameworks for \dotnet{}. Therefore, we will use it as well for writing tests for our
implementation.

\subsection{Testing the QUIC Protocol}

The QUIC protocol specification~\cite{draft-ietf-quic-transport} mostly specifies the protocol
behaviors in terms of what endpoint may or may not send in specific scenarios. This implies that the
correctness of the implementation as a whole can be tested by inspecting the contents of the
generated UDP datagrams.

Because our implementation separates QUIC connection logic from the socket IO, the unit tests can be
written against the internal \QuicConnection{} API\@. Exchange of QUIC packets between two
\QuicConnection{} instances can be simulated by the unit testing code. An advantage of using this
internal API is that the unit tests can be written as single-threaded --- without any background
processing thread which could introduce nondeterminism to the tests. However, in order to make the
tests completely deterministic, the testing code also must be able to control the exact timing of
events. In order to achieve that, our \QuicConnection{} implementation will not obtain the current
timestamp directly, e.g., by calling the \class{Stopwatch}\texttt{.\method{GetTimestamp}()} method.
Instead, the current timestamp will be provided as an argument to the internal API, either by the
\QuicConnectionContext{} class or the unit testing code.

\subsubsection{Testing Harness}\label{sec:03-testing-harness}

A large number of the unit tests will consist of inspecting contents of the QUIC packets exchanged
between a pair of \QuicConnection{} instances or checking the internal state of a connection after a
particular packet exchange. However, correctly implemented \QuicConnection{} will never violate the
protocol and, therefore, additional logic is required to test error conditions.

In order to test the behavior in erroneous conditions, the testing code needs to be able to either
compose an invalid packet to be sent or intercept a valid packet and modify it to elicit an error
response. However, doing this is difficult for two reasons:

\begin{itemize}

  \item All packets are encrypted and protected against modification. The testing code must first
retrieve the correct \CryptoSeal{} from the sender \QuicConnection{} instance and unprotect the
packet. After any modification, the encryption must be reapplied.

  \item Modifying a particular QUIC frame may change the encoded frame's size because of the
variable-length encoding used (see \autoref{sec:02-variable-length-encoding}). If the size of the
QUIC frame changes, then all following frames must be shifted, and the Length field in the packet
header must be adjusted.

\end{itemize}

Doing this manually would make unit testing function very verbose and make the code hard to
understand. To keep the tests concise and easy to understand, we will implement a \textit{testing
harness} which will provide the functionality mentioned above via a set of helper methods. These
methods will provide a declarative way to specify what the QUIC packet or datagram must contain and
register callbacks for packet modification to elicit error responses.

\autoref{lst:03-desired-unit-test} shows an example how the desired testing harness would be used in
conjunction with \xUnit{} testing framework to test if the first UDP datagram sent by the client has
the correct size\footnote{As part of the prevention against traffic amplification attacks, all UDP
datagrams containing an Initial packet sent by client must be larger than 1200
bytes~\autocite[Section~8.1]{draft-ietf-quic-transport}.}. The \method{GetDatagramToSend} method
will get the next UDP datagram to be sent and present it in a structured manner. Later, the
\method{ShouldHaveFrame<\class{TFrame}>} method will internally check if a frame represented by
\class{TFrame} type is present in the packet and will invoke the provided callback for further
assertions.

\begin{myListingCsharpNoPageBreak}{lst:03-desired-unit-test}{Example unit test inspecting QUIC packets contents.}{Fact, Assert, Datagram, InitialPacket, CryptoFrame, QuicConstants}{}
    [Fact]
    public void |ClientInitialDatagramHasInitialPacketWithCryptoFrame|()
    {
        var datagram = |GetDatagramToSend|(Client);
        Assert.|Equal|(
            QuicConstants.MinimumClientInitialDatagramSize,
            datagram.Size);

        var initial = Assert.|IsType|<InitialPacket>(
            Assert.|Single|(datagram.Packets));
        Assert.|Equal|(0, initial.PacketNumber);

        initial.|ShouldHaveFrame|<CryptoFrame>(crypto =>
        {
            Assert.|Equal|(0, crypto.Offset);
            Assert.|NotEmpty|(crypto.CryptoData);
        });
    }
\end{myListingCsharpNoPageBreak}

However, in \autoref{sec:03-data-representation}, we mentioned that the types representing QUIC
frames must be \keyword{ref struct}s in order to contain \SpanOf{\class{T}} instances, and that
\keyword{ref struct}s cannot be used as type arguments for generic classes or methods. This could be
overcome by using \MemoryOf{\class{T}} instead of \SpanOf{\class{T}}, but the use of structs for
frame types still poses a problem for modification of the QUIC frames by a callback.

Structs are normally passed by value; therefore, the callback would either have to accept the frame
by reference using the \keyword{ref} keyword or return the modified frame as the return value.
Neither of these solutions is perfect. Parameters types with \keyword{ref} modifiers cannot be
inferred for lambdas and need to be stated explicitly, and having to return the modified frame is
unintuitive for non-modifying callbacks. Furthermore, \keyword{struct}s representing QUIC frames
should be preferably made \keyword{readonly} to allow more compiler optimizations, so modifying the
frame would require creating a new instance of the frame, overwriting the old value.

For the above reasons, we decided to duplicate the types for frame representation using mutable
\keyword{class}es. This duplication will allow expressing the unit tests in a succinct, natural
declarative manner. This is illustrated in the test in \autoref{lst:03-unit-test-intercept} which
uses an \method{Intercept1RttFrame<\class{TFrame}>} method to intercept a frame of type
\class{TFrame} and modify it. In this case, we shift the range of acknowledged packets by one and,
by doing so, simulate acknowledging a packet that the server has not sent yet, which is a violation
of the protocol which should result in connection termination via a \CONNECTIONCLOSE{} frame.

\begin{myListingCsharpNoPageBreak}{lst:03-unit-test-intercept}{Example unit test simulating error conditions.}{Fact, Assert, InitialPacket, QuicError, ConnectionCloseFrame, AckFrame}{TransportErrorCode, FrameType}
    [Fact]
    public void |ConnectionCloseWhenAckingFuturePacket|()
    {
        // ... setup ommited for brevity

        |Intercept1RttFrame|<AckFrame>(Client, Server, ack =>
        {
            // ack one packet more than originally intended
            ack.LargestAcknowledged++;
        });

        |Get1RttToSend|(Server).|ShouldHaveFrame|<ConnectionCloseFrame>(f =>
        {
            Assert.|Equal|(TransportErrorCode.ProtocolViolation, f.ErrorCode);
            Assert.|Equal|(FrameType.Ack, f.ErrorFrameType);
            Assert.|Equal|(QuicError.InvalidAckRange, f.ReasonPhrase);
        });
    }
\end{myListingCsharpNoPageBreak}

\subsection{Testing the Public API}

The \dotnet{} runtime repository already contains a suite of tests used to test the
\libmsquic{}-based QUIC implementation. Another large separate suite of tests exists for ensuring
that all \class{Stream} implementation behave consistently. All these tests can be used to ensure
that our implementation of \QuicListener{}, \QuicConnection{}, and \QuicStream{} classes conforms to
the public API specification.

\section{Diagnostics}

QUIC is a very complex protocol, and bug investigation can be complicated. By stopping the
application on a breakpoint, the developer inadvertently changes the application's behavior.
Extended pauses for inspecting internal connection state make the implementation miss important
timeouts, possibly leading to connection being closed due to the idle timeout. This makes
interactive debugging of issues which span multiple roundtrips almost impossible.

There are two possible ways of gaining better insight into the connection's behavior --- externally
inspecting the packets sent over the network and producing verbose logs (called traces) by the
implementation.

\subsection{Inspecting the Sent Packets}

Inspecting the connection behavior by observing the packets sent over the network is a non-invasive
way of diagnosing issues in any networking protocol. An example of a tool that can be used for this
task is Wireshark~\cite{web:wireshark}, which also supports QUIC\@.

However, since QUIC is always encrypted, inspecting QUIC packets via Wireshark is not as
straightforward as for other network protocols like plain TCP\@. Implementations must leak the
encryption secrets used in the connection, e.g., into a log file, and these keys must be provided to
Wireshark to enable decrypting the packets. While this approach is relatively simple to implement,
it provides little concrete information about the internal state of the connection. The developer
must infer the internal connection state from his knowledge about the implementation and the
contents of the captured QUIC packets.

\subsection{Producing Traces from the Connection}

A better insight into the connection's behavior can be gained by emitting verbose logs that can be
later analyzed. However, such traces can consist of thousand lines of output which may be difficult
to read and reason about. Fortunately, there are tools which can visualize individual events in the
connection in a graphic format that is easier to understand. One of these tool is
\textit{qvis}~\cite{web:qvis} which is part of the \textit{quiclog} suite~\cite{githubquiclog}. The
\textit{qvis} tool consumes logs in a JSON format, which can be produced directly or generated from
implementation-specific log format or even from generated from network traffic captured using
Wireshark, provided that the encryption secrets used in the connection are known.

For our implementation, we have decided to keep the tracing implementation simple by directly
emitting traces in the JSON format which can be directly consumed by the \textit{qvis} tool without
any further conversions. However, serializing packet information into JSON format is likely to incur
noticeable overhead and, therfore, a better logging format should be considered for future
development.

\section{Integration into .NET Runtime Codebase}

The work of this thesis aims to be eventually mergeable into the \dotnet{} runtime codebase. As of
writing this thesis, there already exists a project for \textit{System.Net.Quic.dll}. Therefore, the
source code may be placed there directly without additional changes.

There is, however, the issue with the \libopenssl{} dependency. The \libopenssl{} library is written
in C and must be compiled for the target machine. Integrating the \libopenssl{} compilation into the
\dotnet{} runtime build process would impose additional requirements for the \dotnet{} runtime
compilation. The additional dependencies would complicate building the \dotnet{} runtime as part of
the continuous integration process. For this reason, it is necessary to provide also a mock TLS
implementation that would be used in automated tests as part of continuous integration.

The mock TLS implementation can also be used locally to avoid building the \libopenssl{} dependency.
The only limitation of the mock TLS implementation is that it does not allow interop with other QUIC
implementations.
