\chapter{Analysis}\label{chap:03-analysis}

\todo{this belongs to TLS section}

Although the managed QUIC implementation should be portable, the API necessary for integrating TLS
into QUIC is currently unavailable in the native libraries used by \dotnet{}. To avoid implementing
TLS 1.3 ourselves, this thesis will select a suitable TLS library and integrate it into the QUIC
implementation. Dependence on another library is expected to be temporary until the QUIC enabling
APIs are present in TLS libraries currently in use in \dotnet{}.

\begin{itemize}

    \item Design considerations
    \begin{itemize}

        \item Performance
        \begin{itemize}

            \item Reducing allocations
            \item Threading model
            \item Use of Exceptions

        \end{itemize}

        \item Robustness

        \item Testability

        \item Debugging --- quic-log qlog formats

    \end{itemize}

    \item Integration:
    \begin{itemize}

        \item Build as part of dotnet runtime
        \item Standalone build for easier evaluation

    \end{itemize}

    \item Selecting implemented subset
    \begin{itemize}

        \item What is necessary for the promised example to work?
        \item +: connection, handshake, flow control, recovery, necessary frames
        \item -: version negotiation, versioning,

    \end{itemize}

    \item Target API
    \begin{itemize}

        \item Description of methods on QUIC API classes + semantics
        \item QuicListener
        \item QuicConnection
        \item QuicStream

    \end{itemize}

    \item Socket utilization
    \begin{itemize}

        \item Supporting server socket multiplexing
        \item Future support for server preferred address
        \item Future support for connection migration
        \item Threading model
        \item Incoming packet handling
        \item User request handling
        \item Timeout events

    \end{itemize}

    \item Epoch handling
    \begin{itemize}
        \item Separation of epochs in protocol
    \end{itemize}

    \item Encryption
    \begin{itemize}

        \item Key derivation implementation
        \item Future updating 1-RTT protection keys (Key phase bit)

    \end{itemize}

    \item TLS implementation

    \begin{itemize}

        \item Necessary API
        \item Available choices:
        \begin{itemize}

            \item OpenSSL
            \item SChannel (insider build)
            \item Others
            \item Native library interop and implications (distribution etc)
            \item Support for SslAuthenticationOptions (API surface)

        \end{itemize}
    \end{itemize}

    \item Packet serialization/Deserialization
    \begin{itemize}

        \item QUIC Primitives encoding
        \item Packet types
        \item Frame types
        \item Performance considerations
        \item Allocation-free packet reading/writing

    \end{itemize}

    \item Stream implementation
    \begin{itemize}

        \item Separation of inbound and outbound part of the stream
        \item Inbound part

        \begin{itemize}

            \item Buffering
            \item out of order reception
            \item Reading data by user
            \item Flow control consideration

        \end{itemize}

        \item Outbound part
        \begin{itemize}

            \item Buffering
            \item Acknowledgement, retransmission
            \item Flow control consideration
        \end{itemize}

        \item Abort/Dispose model for streams
        \item Managing streams within connection
        \item Queue of streams to be updated

    \end{itemize}

    \item Flow control
    \begin{itemize}

        \item Connection level
        \item Stream level
        \item MaxStreams and API

    \end{itemize}

    \item Recovery/Robustness
    \begin{itemize}

        \item Detecting lost packets
        \item Detecting duplicate packets
        \item Tail loss probe
        \item Congestion window
        \item Interface for Congestion control algorithm

    \end{itemize}

\end{itemize}
