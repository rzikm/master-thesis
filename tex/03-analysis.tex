\chapter{Analysis}\label{chap:03-analysis}

In this chapter, we analyze the protocol and select the necessary subset needed to evaluate a pure
\dotnet{} implementation. Afterward, we design the architecture and outline the implementation of
its major parts. Lastly, we investigate the means for testing and debugging the implementation.

\section{Implemented Feature Subset Selection}\label{sec:03-feature-selection}

The decision which features of the QUIC protocol we implement is guided by the goals we set in
\autoref{sec:01-goals}. For feature selection, these goals can be rephrased into the following:

\begin{enumerate}

  \item Support basic data transport, enabling some experimentation with QUIC as the transport for
application layer protocols. \textit{(goals 1 and 2)}

  \item Enable performance measurements that are representative of the potential full QUIC
implementation. \textit{(goals 1 and 3)}

\end{enumerate}

The first goal requires full implementation of the multiplexed stream abstraction as defined by the
QUIC specification. It also requires implementing loss detection and recovery to ensure that no data
gets lost during the transport.

In order to get representative performance measurements, we should implement all performance
affecting features. The most important are packet protection and flow control because they influence
performance throughout the lifetime of the QUIC connection.

To summarize, the thesis should implement at least the following features:

\begin{itemize}

    \item Connection lifetime support (establishment, termination)

    \item Stream multiplexing

    \item Packet protection

    \item Loss detection and recovery

    \item Flow control

\end{itemize}

On the other hand, we can disregard many QUIC features that react to one-time events or do not
otherwise influence the implementation's performance. These features include:

\begin{itemize}

    \item Connection migration, and therefore multiple Connection IDs support

    \item Complex (token-based) address validation

    \item Network path MTU detection

    \item Version negotiation

    \item 1-RTT key updates

    \item Advanced security measures (see Section 21 in transport
specification~\cite{draft-ietf-quic-transport})

\end{itemize}

The rest of the features form a grey area that can be implemented fully, partially, or even not at
if convenient.

\section{Design Considerations}

Before we start the actual analysis, we will briefly outline the design principles used for the
actual design of the implementation and their rationale.

\subsection{Performance}

One of the critical factors in the decision between managed \dotnet{} implementation of QUIC or
using an external library like \libmsquic{} is performance. Therefore, the decisions made during the
implementation design should focus towards greater performance, possibly sacrificing maintainability
if the trade-off is justified.

As a general rule, the implementation will:

\begin{itemize}

    \litem{Avoid excessive heap allocations} Although heap allocation is cheap in \dotnet{} thanks
to the garbage collector, allocating many objects may lead to frequent garbage collections. Each
collection introduces a small stall into the program. Therefore, heap allocations on hot paths of
the code executions should be minimized.

    \litem{Prefer return codes over exceptions} Throwing an exception is an expensive operation, and
their frequent use would negatively impact the performance.

\end{itemize}

\subsection{Testability}

The second design aspect we would focus on is the testability of the implementation. Ideally, the
design would minimize the need for live debugging of the implementation. This is especially
important because stopping the implementation on a breakpoint will inevitably disrupt the
connection, possibly leading to termination because of a timeout.

The design intention is to allow writing automated tests that can inspect the packets sent by the
endpoint and verify that they are consistent with the behavior defined by the QUIC specification.

\section{Target \dotnet{} API}

The public API, which was designed by the \dotnet{} team for use by other developers, uses similar
concepts like TCP and UDP \dotnet{} networking APIs. There is a \QuicListener{} class intended to be
used by servers for listening for incoming connections like using the \class{TcpListener} class. The
actual connection is represented by the \QuicConnection{} class, which serves the same purpose as
the \class{TcpClient} class. The individual QUIC streams are exposed using the \QuicStream{} class,
which implements the \class{Stream} abstraction.

The API is expected to be used asynchronously using the \keyword{async}/\keyword{await} model. Most
methods return the \class{ValueTask} which should be \keyword{await}ed to efficiently wait until the
completion. The full list of methods with detailed description can be found later in
\autoref{sec:06-api}.

\section{High-Level Architecture}\label{sec:03-high-level-architecture}

The \QuicConnection{} implementation will require some sort of background processing
thread\footnote{To reduce verbosity, this text will be using the term \textit{thread} to mean both a
dedicated \class{Thread} instance or a \class{Task} running on a thread-pool thread.} to send
acknowledgments for incoming packets and to resend data after being determined lost due to a
timeout. Because the correctness of a multithreaded code is hard to test, our implementation of
\QuicConnection{} will not interface with the underlying \Socket{} instance directly. Instead, the
implementation will provide an internal interface for exchanging the datagrams to be sent/received.
The actual sending of these datagrams will be handled by a separate class named
\QuicSocketContext{}. By separating the socket I/O from the connection logic implementation, we can
write unit tests that inspect the contents of the sent datagrams and assert that their contents
conform to the protocol specification.

On the server's side, the \QuicSocketContext{} class will also handle any stateless packet
processing, such as sending Retry and Version Negotiation packet. It will be also responsible for
matching packets to the appropriate \QuicConnection{} instances and queueing new connections to
\QuicListener{} to be read by the application. \autoref{fig:03-architecture} illustrates the
relationship between \QuicSocketContext{}, \QuicConnection{} and \QuicListener{} classes.

\begin{myFigure}{fig:03-architecture}{High-level background processing architecture.}

\resizebox{0.8\linewidth}{!}{\input{img/03-architecture.pdf_tex}}

\end{myFigure}

\subsection{Servicing the Socket}

The \QuicSocketContext{} class will implement the necessary background processing management. Its
responsibilies are:

\begin{itemize}

  \item routing incoming datagrams from \Socket{} to the proper \QuicConnection{} instance;

  \item calling timeout handlers after a timeout set by the connection expires; and

  \item sending outgoing datagrams provided by the \QuicConnection{}.

\end{itemize}

For performance reasons, it may be better to process timeouts and send the datagrams using one
thread and process received datagrams using another thread. However, the cost of synchronization of
the shared state may outweigh the performance gain. Our prototype implementation will use a single
thread to service both types of events but will allow for the possibility of future experimentation
with separate threads for sending and receiving.

\subsubsection{Processing Multiple Connections in Parallel}

For client connections, the background processing provided by the \QuicSocketContext{} class needs
to service only a single connection. However, on the server's side, there can be multiple
connections receiving datagrams from the same socket, and therefore being served by a single
\QuicSocketContext{} instance. Servicing multiple \QuicConnection{} instances by only one thread
could limit the server's throughput.

In order to allow processing multiple QUIC connections in parallel, each \QuicConnection{} should
have a dedicated thread for background processing. Our implementation separates the per-connection
logic into separate \QuicConnectionContext{} class. The \QuicSocketContext{} class will be directly
responsible only for stateless packet processing like sending a Version Negotiation packet for
incoming packets with unsupported versions. Packets belonging to existing connections will be queued
for processing by the appropriate \QuicConnectionContext{} instance. A possible runtime structure of
the \QuicSocketContext{} servicing two separate connections is illustrated in
\autoref{fig:03-socket-context-architecture}. The two \QuicConnectionContext{} instances are
serviced by separate dedicated threads. The \QuicSocketContext{} itself does not need a dedicated
thread. Instead, its logic is performed in whatever thread that completes the pending asynchronous
socket receive call.

\begin{myFigure}{fig:03-socket-context-architecture}{Architecture of the server-side background
processing.}

\resizebox{0.95\linewidth}{!}{\input{img/03-server-socket-context.pdf_tex}}

\end{myFigure}

\subsubsection{Accessing the Socket Object}

Introduction of \QuicConnectionContext{} solved possible performance bottlenecks for servers, but it
introduces a complication in the \Socket{} access. Ideally, Each thread would call the
\method{ReceiveFromAsync} method on the \Socket{} with the known remote endpoint address. This naive
approach would work, considering that we plan to omit support for connection migration. However, as
mentioned in \autoref{sec:02-path-validation}, the endpoint can also change its address due to NAT
rebinding. In that case, the connection would be eventually terminated due to the idle timeout (see
\autoref{sec:02-idle-timeout}).

Therefore, our implementation will use a separate thread to receive all UDP datagrams and queue them
into a \ChannelOf{} instance of the appropriate \QuicConnectionContext{} based on the value of
the DCID field of the packet header.

\subsubsection{Future Support for Connection Migration}

Lastly, we need to review the architecture for possible support for connection migration. Thanks to
the \QuicConnectionContext{} not depending directly on a particular \Socket{} instance, it is
conceivable that a single \QuicConnectionContext{} instance could be part of two
\QuicSocketContext{}s --- one for the old endpoint address, and the other for the new one. Use of the
\ChannelOf{} class already ensures that the incoming datagrams could be queued concurrently by
both \QuicSocketContext{} instances. The only modification needed on the architecture level would be
allowing the \QuicConnectionContext{} to select the right \Socket{} from which the outgoing
datagrams should be sent.

\subsection{Public API Threading model}

The target API uses the \class{ValueTask} type designed for efficient asynchronous methods. The user
code will start an asynchronous operation that can be completed by a background thread servicing the
connection. This way, the user code cannot block the connection's background thread, which could
otherwise cause timeouts to be missed.

The \QuicConnection{} can be potentially used in a multithreaded environment, and the implementation
should allow concurrent usage of \QuicConnection{} and \QuicStream{} classes when it makes sense.
For example, it makes sense for an application code to process each \QuicStream{} in a different
thread. However, it does not make sense to concurrently write into one \QuicStream{} from two
threads without any synchronization. Our implementation will, therefore, provide the following
thread-safety guarantees for the API:

\begin{itemize}

  \item Individual streams can be used concurrently from different threads. However, each direction
of the stream (reading and writing) can be accessed only by one thread at a time and must be,
therefore, synchronized.

  \item Accepting/opening new streams on a connection can be done concurrently from multiple
threads.

  \item All other operations on \QuicConnection{} must be synchronized, including, e.g., starting
and aborting the connection.

\end{itemize}

\section{Packet Serialization/Deserialization}

Special care needs to be taken when implementing the QUIC packets' serialization to the wire format.
Inefficient data representation can negatively impact overall performance because processing
individual QUIC packets and frames will likely be a hot path in the implementation. The packet and
frame representation should be, therefore, carefully designed to avoid allocations.

The incoming packets can contain arbitrary encoding errors that should be handled efficiently and
gracefully, i.e., without throwing exceptions. Possible errors include values being outside of the
range of allowed values and incomplete or damaged packets.

\subsection{QUIC Packet and Frame Representation}\label{sec:03-data-representation}

Some packets contain many fields and, therefore, passing them around as individual variables would
make the implementation harder to maintain. The QUIC frames form coherent messages that should be
represented by individual \dotnet{} types. This gives us also an opportunity to keep the
serialization and deserialization code next to each other, making it easier to ensure that, e.g.,
the order of the serialized fields match in both methods.

Representing QUIC frames as individual classes would introduce many heap allocations for every
received QUIC packet. This thesis, therefore, will model QUIC frames and the QUIC packet headers as
value types. Also, the payload of some frames may consist of large blocks of memory. Examples
include \STREAM{} and \CRYPTO{} frames, which can fill the entire payload of the QUIC packet.
Duplicating this block of memory into a separate \ArrayOf{\Byte{}} would be another unnecessary memory
allocation. \dotnet{}~2.1 introduced the \SpanOf{\class{T}} type, which can be used to efficiently
reference arbitrary memory block, including memory allocated on the stack using the
\keyword{stackalloc} keyword.

However, the \SpanOf{\class{T}} type is a \keyword{ref struct} --- a special kind of value type which
can be stored only in local variables or inside another \keyword {ref struct}s. Limitations on the
usage of \keyword{ref struct}s also forbid their use as generic type arguments, e.g., for
\FuncOf{} and other generic delegates. This limitation is acceptable for \QuicConnection{}
implementation because the QUIC frames can be processed one after another right after being
deserialized from the QUIC packet.

\subsection{QuicReader and QuicWriter}

Both serialization and deserialization require maintaining the current position in the buffer read
from or written into. In order to simplify the serialization or deserialization code, our
implementation introduces \QuicReader{} and \QuicWriter{} classes as a primary means of reading and
writing QUIC primitives to memory. Although \QuicReader{} and \QuicWriter{} are reference types
allocated on the heap, their instances are expected to be cached by the class that uses them.

The \QuicReader{} and \QuicWriter{} classes also take care of converting the endianity of the data
between big-endian used by QUIC and the endianity used by the machine running the application. This
is done by forwarding the calls to respective methods on the \class{BinaryPrimitives} class.

\section{Stream Implementation}\label{sec:03-stream-implementation}

QUIC is a transport protocol, and, therefore, its entire purpose is transferring streams of data.
Since it is likely to be the hot path of the implementation, the internal handling of streams must
be efficient and avoid unnecessary copying of stream data blocks.

QUIC recognizes four types of streams. These streams can have a sending part, receiving part, or
both. The fact which endpoint initiated the stream controls only which flow control limits apply to
that stream. Otherwise, all streams are handled equally. As mentioned in
\autoref{sec:02-stream-types}, bidirectional streams can be implemented as two unidirectional
streams. Therefore, our implementation will divide the QUIC stream logic into \SendStream{} and
\ReceiveStream{} classes, which will handle the sending and receiving behavior of the stream.

\subsection{Receiving Part of the Stream}\label{sec:03-receive-stream}

The implementation of \ReceiveStream{} must buffer the received data in case the QUIC packets send
by the peer were lost or reordered. It also needs to track the amount of buffered memory and how
much data was delivered to the application in order to correctly update flow control limits for the
peer.

\subsubsection{Stream Data Buffering}

Ideally, the stream implementation would be structured so that the stream data were copied straight
from the decrypted packet to the memory provided by the application. This would be possible if the
API used an event-based model with callbacks for incoming data. The API, as currently designed,
utilizes a method-based model. If the application does not call the \method{ReadAsync} method, there
is no buffer to deliver into.

In order to avoid additional copies, the buffer holding the stream data cannot be reused to receive
another QUIC packet because the part of the buffer with the \STREAM{} frame must be kept unmodified
until delivered to the application. Instead, a different buffer must be obtained for receiving the
next packet. For such a solution to be memory efficient, it would require an elaborate memory
pooling scheme to avoid needlessly, essentially implementing a custom memory allocator over a block
of memory. The development and maintenance cost may greatly outweigh the performance improvement of
such a solution. Therefore, until the cost of such a solution is justified by performance
measurements, our implementation will use a two-copy approach: the first copy from the packet to an
intermediate buffer, the second copy from the intermediate buffer to the destination memory provided
by the application.

\todo{if we speak about doing some performance measurements, then they should be done and their
results presented in some following chapter}

\todo{forward reference to the measurements, see 2020-10-22: 40m}

\subsubsection{Packet Reordering and Data Deduplication}

Unreliability of the UDP protocol can cause QUIC packets to be reordered or lost. Because of that,
parts of the stream may be received multiple times, and the contents of \STREAM{} frames can
arbitrarily overlap.

Even though QUIC Flow control provides an upper limit on the data that needs to be buffered at any
given moment, allocating one large buffer may be inefficient, as the entire buffer might not be
needed at any given point in time. Therefore, our implementation uses multiple smaller buffers and
allocates only the necessary number of buffers that are needed to buffer currently received data. To
reduce pressure on the garbage collector, buffers that are no longer necessary are returned to a
shared instance of \ArrayPoolOf{\Byte{}} class to avoid their frequent allocation.

\subsubsection{Reading Data by Application Code}

Because user code runs on a different thread from the internal \QuicConnection{} logic, there needs
to be some synchronization. Our implementation uses the \ChannelOf{} class, which provides an
efficient implementation of the producer-consumer queue with support for asynchronous operations
using the \ValueTaskOf{} type. Each time an incoming \STREAM{} frame allows a part of the
stream to be delivered to the application, the relevant region of the buffer is queued into the
\ChannelOf{} using the \MemoryOf{\Byte{}} type.

\todo{more of the descriptions of the}

\todo{the image is hard to understand, it needs a more detailed description and a legend in the
image so that the reader does not have to}

\todo{consider organizing the layers vertically in the image}

\todo{even though I explain the buffering etc earlier, I need to describe the image}

\autoref{fig:03-receive-stream} Illustrates how all parts of the \ReceiveStream{} fit together. On
the left, data from the incoming \STREAM{} frame are temporarily stored in pooled reordering
buffers. Regions of memory containing deliverable data are then queued for delivery in the channel.
The application thread retrieves the memory regions and copies the data into the destination buffer.

\begin{myFigure}{fig:03-receive-stream}{Implementation of the receiving part of the stream}

  \resizebox{\linewidth}{!}{\input{img/03-receive-stream.pdf_tex}}

\end{myFigure}

\subsubsection{Flow Control Considerations}

QUIC Flow control limits for streams specify the maximum offset of data that an endpoint can send.
An endpoint needs to update these limits by sending a \MAXSTREAMDATA{} frame after the data are
delivered to the application. However, sending updated limits in every packet could waste space that
could be used by the application data. On the other hand, updating the limits too late could lead to
the other endpoint being blocked and signal the fact by sending a \STREAMDATABLOCKED{} frame. In
such a state, the endpoint is unable to send more application data until it receives an update via
\MAXSTREAMDATA{} frame.

Our implementation will send a \MAXSTREAMDATA{} frame after the client uses more than half of the
limit provided since the last update. This should prevent updates being too often and at the same
time early enough that the sender does not run out of flow control credit.

\subsection{Sending Part of the Stream}\label{sec:03-send-stream}

The sending part of a QUIC stream must keep track of state of all outbound data on the stream. Any
part of the stream can be lost during transmission and the has to be retransmitted, and any part of
the sent data can be acknowledged.

\subsubsection{Stream Data Buffering}

Similarly to buffering received data, we will analyze how many times the outgoing data need to be
copied before they are sent. The semantics of the \method{Write} and \method{WriteAsync} methods is
that when the method completes, the provided memory can be reused. \todo{see question from PJ on
2020-10-22 1h about multiple parallel ValueTasks and completing when the data are acknowledged} A
single-copy approach, in this case, would require that the \method{Write} and \method{WriteAsync}
methods complete only after the peer acknowledges that the contained data were received. In the
best-case scenario, that would take a full roundtrip, which would be very inefficient. Therefore our
implementation will, as with the receiving part of the stream, use an intermediate buffer for
storing the data.

Our implementation will also use a variable number of fixed-size buffers rented from a shared
\ArrayPoolOf{\Byte{}} instance. These buffers will be used to store data that cannot be discarded
yet because the other endpoint did not acknowledge their reception.

\subsubsection{Acknowledgement, Loss, and Retransmission}

Each byte that was written into the stream can be conceptually in three different states:

\begin{itemize}

  \litem{Pending} The byte needs to be sent to the peer in some future \STREAM{} frame.

  \litem{In-flight} The byte has been sent, but it is uncertain if the containing packet was
received.

  \litem{Acknowledged} The packet containing the byte has been acknowledged by the other endpoint.
It is no longer necessary to buffer this byte.

\end{itemize}

The transitions between these states are straightforward. The data transition from pending to
in-flight by sending them via \STREAM{} frame. When the packet is deemed lost by the loss detection
algorithm, the data transition back to pending, and if the packet has been acknowledged, the data
transition to acknowledged. The state of the individual parts of the stream can be tracked by
maintaining three sets of \textit{ranges} --- starts and ends of the blocks of data in the same
state.

\subsubsection{Writing Data by Application Code}

Again, our implementation will use \ChannelOf{} type to provide synchronization with the
application thread. However, since \ChannelOf{} does not allow random access to the provided
data, a separate list of buffers with yet unacknowledged data need to be maintained as well.

\autoref{fig:03-send-stream} illustrates the process of writing data into the \SendStream{}. The
application data are coming from the left, where they are written into a buffer and queued into the
synchronization channel. From the channel, the buffer is dequeued and placed in a list of
retransmission buffers, where it stays until the other endpoint acknowledges the contained data.

\todo{why can't I gather-send from the buffers? because I need to encrypt, potentially more than
once if the data gets lost}

\begin{myFigure}{fig:03-send-stream}{Implementation of the sending part of the stream}

  \resizebox{\linewidth}{!}{\input{img/03-send-stream.pdf_tex}}

\end{myFigure}


\subsection{Abort/Dispose Model for Streams}

The \QuicStream{} class implements the \interface{IDisposable} interface which should provide
automatic closing of the stream when \QuicStream{} is used in a \keyword{using} statement or
\keyword{using} block. This implies resetting the writable part of the stream using \RESETSTREAM{}
frame and requesting reset of the readable part of the stream using \STOPSENDING{} frame.

However, both these frames require an application-level error code. QUIC specification does not
define any transport-level error codes for these stream operations. This has been established as an
incorrect design and can be expected to change in future iterations of the API design.

Leaving \interface{IDisposable} unimplemented would be misleading to users in the prototype
implementation. Therefore, our implementation will use error code 0 regardless of its semantics in
the application protocol, unless the stream has been explicitly closed by the application using the
\method{AbortRead} or \method{AbortWrite} methods.

Another area where the QUIC API is not well defined is the behavior of pending \method{AcceptStream}
call on \QuicConnection{} when the connection is closed. In case the connection has been closed
gracefully from the application protocol's perspective, it would be better if this method did not
throw an exception but returned \keyword{null} reference. However, the semantics of the error codes
are, again, defined by the application protocol. Until the behavior of these methods in such
scenarios is better defined, our implementation will throw
\class{QuicConnectionAborted\allowbreak{}Exception} for all pending async calls.

\section{TLS Implementation}

TLS handshake forms an integral part of the QUIC connection establishment. Because correct TLS
implementation is crucial for ensuring the security of the resulting implementation, this thesis
should avoid implementing TLS algorithms by itself. Instead, it should reuse some existing and
well-tested implementation.

The novel way QUIC integrates with TLS requires a specific API from the TLS implementation. Below is
a non-exhaustive list of operations the TLS library's API must provide:

\begin{itemize}

  \item Current state of the handshake

  \item Temporary keys used to protect the handshake process

  \item Raw unencrypted TLS messages to be sent to the other endpoint

  \item Negotiated cipher

  \item Specifying protocols used for ALPN

  \item Specifying custom TLS extension to exchange QUIC transport parameters

\end{itemize}

The \dotnet{} runtime libraries use different native libraries to provide TLS functionality on
different operating systems. On Windows, \libname{Secure Channel}~\cite{Schannel} (\libschannel{}
for short), which is part of the Windows operating system. On Linux and macOS systems, the
\libopenssl{} library~\cite{OpenSSLWeb} is used.

\begin{description}

    \ditem{\libname{Secure Channel}} The \libschannel{} versions present in the latest Windows 10
builds support only TLS 1.2. However, future updates will also implement TLS 1.3. The \libschannel{}
version with TLS 1.3 support can be obtained by installing an insider preview build of Windows 10.

    The \libmsquic{} library uses \libschannel{} library when compiled for Windows. Therefore, we
assume that \libschannel{} exposes the necessary API for our managed QUIC implementation.

    \ditem{\libopenssl{}} None of the released versions of \libopenssl{} library expose necessary
API for integration with QUIC, and there are no plans to include such API in the next \libopenssl{}
3.0.0 release~\cite{OpensslBlogNoQuic}.

    However, developers at Akamai maintain a fork of \libopenssl{} which adds the QUIC-enabling
API~\cite{AkamaiOpensslGithub}. This modified version of \libopenssl{} is used by \libmsquic{} (on
Linux) and some other QUIC libraries like \libname{quiche} from Cloudflare~\cite{quicheGithub}. The
changes made by Akamai may be merged into \libopenssl{} into the following 3.1.0 or later releases.

\end{description}

In conclusion, the APIs required for our QUIC implementation are currently only accessible in only
the preview versions of Windows 10. Relying solely on \libschannel{} for TLS 1.3 support in our
prototype implementation would severely impact cross-platform availability.

The \libopenssl{} library and supports all platforms supported by \dotnet{}, and could therefore be
used to implement TLS integration into QUIC in a platform-compatible manner. This solution, however,
has some drawbacks:

\begin{itemize}

  \item Modified \libopenssl{} binary must be distributed with \dotnet{} runtime.

  \item Only limited integration with X.509 certificates is possible because the
\class{X509Certificate} class implementation will be using different binary --- \libname{CryptoAPI}
on Windows, unmodified \libopenssl{} on Linux and macOS.

\end{itemize}

These drawbacks are acceptable for the prototype implementation and will be eliminated once the
modified \libopenssl{} is replaced with \libschannel{} integration for Windows, and mainstream
version of \libopenssl{} once the support for QUIC is released. This thesis will, therefore,
integrate with the forked \libopenssl{} library from Akamai.

\section{Packet Protection}\label{sec:03-packet-protection}

As described in \autoref{sec:02-packet-protection}, the packet encryption consists of two phases ---
payload protection and header protection. The combined process requires the following inputs:

\begin{itemize}

  \item Keys derived from the protection secrets (as explained in
\autoref{sec:02-encryption-key-derivation})

  \item Negotiated cipher

  \item Packet number (for encryption), or expected packet number (for decryption)

  \item The QUIC packet to encrypt or decrypt

\end{itemize}

The cipher is negotiated once and cannot be changed during the lifetime of the connection. Keys for
the protection can be changed only for the 1-RTT packets using the process of \textit{key update}
(see \autoref{sec:02-key-update}), which can be expected to be relatively infrequent. The rest of
the inputs change with every packet. Therefore, our implementation encapsulates the packet
protection implementation in a \class{CryptoSeal} class, which does not depend on the rest of the
QUIC implementation.

The process of receiving packets requires intermediate validation of the header fields. The
individual steps --- header protection and payload protection --- must be therefore exposed
separately. Also, the protection should happen in-place to avoid unnecessary allocations and copying
of the packets.

Important consideration needs to be made for performing the actual encryption/decryption once all
inputs to the AEAD have been gathered. \dotnet{} does not contain an implementation of the CHACHA
family of ciphers. Because the implementation can work without it by instructing the TLS library not
to allow its negotiation, the prototype implementation of QUIC will not support this kind of cipher.
The other ciphers are based on the AES family of ciphers, which are supported by \dotnet{}, but
individual classes implementing these ciphers do not share a common interface. Therefore, our
implementation wraps the concrete AES implementations in classes derived from an abstract
\class{CryptoSealAlgorithm} class which defines a common interface required by \CryptoSeal{}.
\autoref{fig:03-crypto-seal} illustrates how the \CryptoSeal{} and \class{CryptoSealAlgorithm}
classes are connected.

\begin{myFigure}{fig:03-crypto-seal}{Relationship between CryptoSeal and CryptoSealAlgorithm
classes}

  \resizebox{\linewidth}{!}{\input{img/03-crypto-seal.pdf_tex}}

\end{myFigure}

The key update operation can be implemented by replacing the existing instance of the \CryptoSeal{}
class with the new one with the updated protection secret.

\section{Loss Detection and Loss Recovery}

In order to detect lost packets and retransmit any lost data, the \QuicConnection{} implementation
must keep track of which data have been sent in which packets and the timestamp when the packet was
sent for timeout detection. Our implementation will encapsulate this information together with the
loss detection algorithm in a dedicated \RecoveryController{} class, which does not depend on the
\QuicConnection{} class. This way, its implementation can be unit tested separately from the rest of
the connection logic.

Another responsibility of the \RecoveryController{} class is maintaining the congestion window.
Although the specification defines only a single algorithm for congestion control based on TCP
NewReno~\cite[Section~7]{draft-ietf-quic-recovery}, there are already experiments with other
algorithms like HyStart++ and CUBIC~\cite{cloudflareCubic}. The ability to support multiple such
algorithms could provide an opportunity for future experimentations. For that reason, the
\RecoveryController{} will use the \gls{strategy-pattern}~\cite{wiki:strategy-pattern} to allow
choosing the congestion control algorithm at runtime.

\section{Automated Testing}

It is necessary to implement automated tests that assert that the implementation conforms to the
QUIC protocol specification. Additionally, the public API makes use of existing concepts, namely
\class{Stream} abstraction for QUIC streams, and it is necessary to ensure that \QuicStream{}
conforms to the expected \class{Stream} behavior expectations.

The \dotnet{} runtime repository into which we wish to integrate our QUIC implementation uses the
\xUnit{} testing framework~\cite{xunit}. The \xUnit{} framework is one of the most mature testing
frameworks for \dotnet{}. Therefore, we will use it as well for writing tests for our
implementation.

\subsection{Testing the QUIC Protocol}

The QUIC protocol specification~\cite{draft-ietf-quic-transport} mostly specifies the protocol
behaviors in terms of what endpoint may or may not send in specific scenarios. This implies that the
correctness of the implementation as a whole can be tested by inspecting the contents of the
generated UDP datagrams. Because our implementation separates QUIC connection logic from socket IO,
the unit tests can be written against internal \QuicConnection{} API, exchanging buffers containing
sent and received UDP datagrams. This way, the background processing threads are avoided, and the
tests can be made single-threaded with manual time-stepping, making the tests deterministic.

\subsubsection{Testing Harness}\label{sec:03-testing-harness}

A large number of the tests will consist of inspecting contents of QUIC packets exchanged between a
pair of \QuicConnection{} instances or checking the internal state of a connection after a
particular packet exchange. However, correctly implemented \QuicConnection{} will never violate the
protocol and, therefore, additional logic is required to test error conditions.

In order to test the behavior in erroneous conditions, the testing code needs to be able to either
compose an invalid packet to be sent or intercept a valid packet and modify it to elicit an error
response. However, doing this is difficult for two reasons:

\begin{itemize}

  \item All packets are encrypted and protected against modification. The testing code must first
retrieve the correct \CryptoSeal{} from the sender \QuicConnection{} instance and unprotect the
packet. After any modification, the encryption must be reapplied.

  \item Modifying a particular QUIC frame may change the encoded frame's size because of the
variable-length encoding used. If the size changes, then all following frames must be shifted, and
the Length field in the packet header must be adjusted.

\end{itemize}

Doing this manually would make unit testing function very verbose and hard to understand. In order
to keep the tests concise and easy to understand, we will implement a \textit{testing harness} which
will provide the functionality mentioned above via a set of helper methods. These methods will
provide a declarative way to specify what the QUIC packet or datagram must contain and register
callbacks for packet modification to elicit error responses.

\autoref{lst:03-desired-unit-test} shows how the desired testing harness would be used in
conjunction with \xUnit{} testing framework to test if the first UDP datagram sent by the client has
the correct size. The \method{GetDatagramToSend} method will get the next datagram to be sent and
present it in a structured manner. Later, the \method{ShouldHaveFrame<\class{TFrame}>} method will
internally check if a frame represented by \class{TFrame} type is present in the packet and will
invoke the provided callback for further assertions.

\begin{myListingCsharp}{lst:03-desired-unit-test}{Example unit test inspecting QUIC packets contents.}{Fact, Assert, Datagram, InitialPacket, CryptoFrame, QuicConstants}{}
    [Fact]
    public void |ClientInitialDatagramHasInitialPacketWithCryptoFrame|()
    {
        var datagram = |GetDatagramToSend|(Client);
        Assert.|Equal|(
            QuicConstants.MinimumClientInitialDatagramSize,
            datagram.Size);

        var initial = Assert.|IsType|<InitialPacket>(
            Assert.|Single|(datagram.Packets));
        Assert.|Equal|(0, initial.PacketNumber);

        initial.|ShouldHaveFrame|<CryptoFrame>(crypto =>
        {
            Assert.|Equal|(0, crypto.Offset);
            Assert.|NotEmpty|(crypto.CryptoData);
        });
    }
\end{myListingCsharp}

However, in \autoref{sec:03-data-representation}, we mentioned that the types representing QUIC
frames must be \keyword{ref struct}s in order to contain \SpanOf{\class{T}} instances, and that
\keyword{ref struct}s cannot be used as type arguments for generic classes or methods. This could be
overcome by using \MemoryOf{\class{T}} instead of \SpanOf{\class{T}}, but the use of structs for frame
types still poses a problem for modification of the QUIC frames by a callback.

Structs are normally passed by value; therefore, the callback would either have to accept the frame
by reference using the \keyword{ref} keyword or return the modified frame as the return value.
Neither of these solutions is perfect. Parameters types with \keyword{ref} modifiers cannot be
inferred for lambdas and need to be stated explicitly, and having to return the modified frame is
unintuitive for non-modifying callbacks. Furthermore, \keyword{struct}s representing QUIC frames
should be preferably made \keyword{readonly} to allow more compiler optimizations, so modifying the
frame would require creating a new instance of the frame, overwriting the old value.

For the above reasons, we decided to duplicate the types for frame representation using mutable
\keyword{class}es. This duplication will allow expressing the unit tests in a succinct, natural
declarative manner. This is illustrated in the test in \autoref{lst:03-unit-test-intercept} which
uses a \method{Intercept1RttFrame<\class{TFrame}>} to intercept a frame of type \class{TFrame} and
modify it. In this case, we shift the range of acknowledged packets by one and by doing so, simulate
acknowledging a packet that the server has not sent yet, which is a protocol violation.

\begin{myListingCsharp}{lst:03-unit-test-intercept}{Example unit test simulating error conditions.}{Fact, Assert, InitialPacket, QuicError, ConnectionCloseFrame, AckFrame}{TransportErrorCode, FrameType}
    [Fact]
    public void |ConnectionCloseWhenAckingFuturePacket|()
    {
        // ... setup ommited for brevity

        |Intercept1RttFrame|<AckFrame>(Client, Server, ack =>
        {
            // ack one packet more than originally intended
            ack.LargestAcknowledged++;
        });

        |Get1RttToSend|(Server).|ShouldHaveFrame|<ConnectionCloseFrame>(f =>
        {
            Assert.|Equal|(TransportErrorCode.ProtocolViolation, f.ErrorCode);
            Assert.|Equal|(FrameType.Ack, f.ErrorFrameType);
            Assert.|Equal|(QuicError.InvalidAckRange, f.ReasonPhrase);
        });
    }
\end{myListingCsharp}

\subsection{Testing the Public API}

The \dotnet{} runtime repository already contains a suite of tests which are used to test the
\libmsquic{}-based QUIC implementation. Another large separate set of tests exists for ensuring that
all \class{Stream} implementation behave consistently. All these tests can be used to ensure that
our implementation of \QuicListener{}, \QuicConnection{} and \QuicStream{} conforms to the public
API specification.

\section{Diagnostics}

QUIC is a very complex protocol, and bug investigation can be complicated. By stopping the
application on a breakpoint, the developer inadvertently changes the implementation's behavior,
making interactive debugging of issues spanning multiple roundtrips almost impossible. There are two
possible ways of gaining better insight into the connection's behavior --- externally inspecting the
packets sent over the network and producing verbose logs (called traces) by the implementation.

\subsection{Inspecting the Sent Packets}

Inspecting the connection behavior by observing the packets sent over the network is a non-invasive
way of diagnosing issues in any networking protocol. An example of a tool which can be used for this
task is Wireshark~\cite{web:wireshark}, which also supports QUIC.

However, since QUIC is always encrypted, inspecting QUIC packets via Wireshark is not
straightforward as for other network protocols. Implementations must leak the encryption secrets
used in the connection, e.g., in a log file, and these keys must be provided to Wireshark to decrypt
the packets.

\subsection{Producing Traces from the Connection}

A better insight into the connection's behavior can be gained by emitting verbose logs that can be
later analyzed, but such logs may be difficult to read and reason about. Fortunately, some tools can
visualize individual events in the connection in a graphic format that is easy to understand. One of
these tool is \textit{qvis}~\cite{web:qvis} which is part of \textit{quiclog}
suite~\cite{githubquiclog}. The \textit{qvis} tool consumes logs in a JSON format, which can be
either produced directly or generated from implementation-specific log format or even from network
packet captures created by Wireshark, provided that encryption secrets are logged elsewhere for the
given connection.

Although JSON format produces larger log files, it can be directly consumed by the \textit{qvis}
tool without any further conversions. For this reason, we chose to use it in our implementation and
will consider more efficient logging options if producing the JSON format incurs too significant
overhead.

\section{Integration into .NET Runtime Codebase}

The work of this thesis aims to be eventually mergeable into the \dotnet{} runtime codebase. As of
writing this thesis, there already exists a project for \textit{System.Net.Quic.dll}. Therefore, the
source code may be placed there directly without additional changes.

There is, however, the issue with the \libopenssl{} dependency. The \libopenssl{} library is written
in C and must be compiled for the target machine. Integrating the \libopenssl{} compilation into the
\dotnet{} runtime build process would impose additional requirements for the \dotnet{} runtime
compilation. The additional dependencies would complicate building the \dotnet{} runtime as part of
continous integration process. For this reason, it is necessary to provide also a mock TLS
implementation that would be used in automated tests as part of continuous integration.

The mock TLS implementation can be also used locally to avoid building the \libopenssl{} dependency.
The only limitation of the mock TLS implementation is that it does not allow interop with other QUIC
implementations.
