\chapter{Analysis}\label{chap:03-analysis}

In this chapter, we analyze the protocol and select the necessary subset needed for evaluation of a
pure \dotnet{} implementation. Afterwards, we design the architecture and outline the implementation
\todo{this should be expanded once the chapter is written}

\section{Protocol Features Selection}

The set of features to be implemented is guided by the goals we set in this thesis goals in the
introduction chapter. For the purpose of feature selection, we can rephrase subset of the original
goals into the following:

\todo{the goals below are a convenient rephrasing and can't be mapped 1-to-1 to the goals stated in
  the introduction, is that OK?}

\begin{enumerate}

  \item Support basic data transport, enabling some experimentation with QUIC as the transport for
    application layer protocols.

  \item Enable performance measurements that are representative of the potential full QUIC
    implementation.

\end{enumerate}

The first goal definitely requires full implementation of the multiplexed stream abstraction as
defined by the QUIC specification. It requires also implementing loss detection and recovery to
ensure that no data gets lost during the transport.

In order to get representative performance measurements, all performance affecting aspects of the
protocol should be implemented. The most important are packet protection and flow control because
they influence performance throughout the lifetime of the QUIC connection.

To summarize, the thesis should implement at least the following features:

\begin{itemize}

    \item Connection lifetime support (establishment, termination)

    \item Stream multiplexing

    \item Packet protection

    \item Loss detection and recovery

    \item Flow control

\end{itemize}

On the other hand, many QUIC features that react to one-time events can be disregarded, because
there either is no need for them in the evaluation environment, or they do not have relevant
performance or functional implications. In particular, implementation of the following features can
be avoided:

\begin{itemize}

    \item Connection migration, and therefore multiple Connection IDs support

    \item Complex (token-based) address validation

    \item Network path MTU detection

    \item Version negotiation

    \item 1-RTT key updates

    \item Advanced security measures (see Section 21 in transport specification~\cite{draft-ietf-quic-transport})

\end{itemize}

The rest of the features form a grey area which can be implemented fully, partially, or even not at
if convenient.

\section{Design Considerations}

Before we start the actual analysis, we will briefly outline the design principles used for the
actual design of the implementation and their rationale.

\subsection{Performance}

One of the key factors in the decision between managed \dotnet{} implementation of QUIC or using
external library like \libmsquic{} is performance. Therefore, the decisions made during the
implementation design should focus towards greater performance, possibly sacrificing maintainability
if the trade-off is justified.

As a general rule, the implementation will:

\begin{itemize}

    \litem{Avoid heap allocations} Although heap allocation is considered cheap, frequent
    allocations put pressure on the \dotnet{} garbage collector, leading to unnecessary stalls.
    Therefore, the amount of heap allocation on hot paths of the code executions should be
    minimized.

    \litem{Prefer return codes over exceptions} Throwing an exception is an expensive operation, and
    their frequent use would have negative impact on the performance.

\end{itemize}

\subsection{Testability}

The second design aspect we would focus on is testability of the implementation. Ideally, the design
would minimize the need for live debugging of the implementation. This is especially important
because stopping the implementation on a breakpoint will inevitably disrupt the connection, possibly
leading to termination because of a timeout.

The design intention is to allow writing automated tests that are able to inspect the packets sent
by the endpoint, and verify that they are consistent with the behavior defined by the QUIC
specification.

\subsection{Robustness?}

\todo{probably can be removed, can be mentioned when parsing incoming packets}

\subsection{Debugging?}

\todo{consider qlog and quic-log formats}

\section{Target \dotnet{} API}

This section describes API that will be used to expose the QUIC implementation to other developers.
As mentioned in the introduction chapter, the current design is a work-in-progress and is subject to
change in the future. All of the mentioned classes are located in the \namespace{System.Net.Quic}
namespace.

\subsection{QuicListener Class}

The \class{QuicListener} class is the equivalent of the \class{TcpListener}. Servers use this
class to accept incoming QUIC connections.

\begin{description}

    \ditemctor{QuicListener}{QuicListenerOptions} Constructor.

    \ditemproperty{IPEndPoint}{ListenEndPoint}{\propget} The IP endpoint being listened to for new connection. Read-only.

    \ditemmethod{ValueTask<QuicConnection>}{AcceptConnectionAsync}{CancellationToken}
    Accepts a new incoming QUIC Connection.

    \ditemmethod{void}{Start}{} Starts listening.

    \ditemmethod{void}{Close}{} Stops listening and closes the listener. Does not close already accepted connections.

\end{description}

\subsection{QuicListenerOptions Class}

The \class{QuicListenerOptions} class holds all configuration used to construct new \class{QuicListener}s.

\begin{description}

    \ditemproperty{SslServerAuthenticationOptions}{ServerAuthenticationOptions}{\propgetset}
        SSL related options like certificate selection/validation callbacks, and supported protocols for ALPN\@.

    \ditemproperty{string}{CertificateFilePath}{\propgetset} Path to the X509 certificate used by the server.

    \ditemproperty{string}{CertificateKeyPath}{\propgetset} Path to the private key for the used X509 certificate.

    \ditemproperty{string}{CertificateKeyPath}{\propgetset} Path to the private key for the used X509 certificate.

    \ditemproperty{IPEndPoint}{ListenEndPoint}{\propgetset} The IP endpoint to listen on.

    \ditemproperty{int}{ListenBacklog}{\propgetset} Number of connection to be held waiting for acceptance by the application. Upon reaching this limit, further connections will be refused.

    \ditemproperty{long}{MaxBidirectionalStreams}{\propgetset} Limit on the number of bidirectional streams the client can open in an accepted connection.

    \ditemproperty{long}{MaxUnidirectionalStreams}{\propgetset} Limit on the number of unidirectional streams the client can open in an accepted connection.

    \ditemproperty{TimeSpan}{IdleTimeout}{\propgetset} The period of inactivity after which the connection will be closed via idle timeout.

\end{description}

\subsection{QuicConnection Class}

The \class{QuicConnection} class provides operation on the QUIC connection. Clients open new
connections by creating a new instance of this class and calling the \method{ConnectAsync} method.
Servers receive new connections using the \class{QuicListener} class.

\begin{description}

    \ditemctor{QuicConnection}{QuicClientConnectionOptions} Constructor. The newly created instance is not connected until the call to \method{ConnectAsync} method.

    \ditemproperty{bool}{Connected}{\propget} Indicates whether the \class{QuicConnection} is connected (the handshake has completed).

    \ditemproperty{IPEndPoint}{LocalEndPoint}{\propget} Local IP endpoint of the connection.

    \ditemproperty{IPEndPoint}{RemoteEndPoint}{\propget} Remote IP endpoint of the connection.

    \ditemmethod{ValueTask}{ConnectAsync}{CancellationToken} Connects to the remote endpoint.

    \ditemmethod{QuicStream}{OpenUnidirectionalStream}{} Opens a new unidirectional stream. Throws a \class{QuicException} if the stream cannot be opened.

    \ditemmethod{QuicStream}{OpenBidirectionalStream}{} Opens a new bidirectional stream. Throws a \class{QuicException} if the stream cannot be opened.

    \ditemmethod{ValueTask<QuicStream>}{AcceptStreamAsync}{Cancellationtoken} Accepts an incoming stream.

    \ditemmethod{ValueTask}{CloseAsync}{long, CancellationToken} Closes the connection with the specified given error code and terminates all active streams.

    \ditemmethod{long}{GetRemoteAvailableUnidirectionalStreamCount}{} Gets the maximum number of unidirectional streams that this endpoint can open.

    \ditemmethod{long}{GetRemoteAvailableBidirectionalStreamCount}{} Gets the maximum number of bidirectional streams that this endpoint can open.

\end{description}

\subsection{QuicClientConnectionOptions}

The \class{QuicClientConnectionOptions} is used by clients to configure new QUIC conections.

\begin{description}

    \ditemproperty{SslClientAuthenticationOptions}{ClientAuthenticationOptions}{\propgetset} Client authentication options to use when establishing the connection.

    \ditemproperty{IPEndPoint}{LocalEndPoint}{\propgetset} The local IP endpoint that will be bound to.

    \ditemproperty{IPEndPoint}{RemoteEndPoint}{\propgetset} The IP endpoint to connect to.

    \ditemproperty{long}{MaxBidirectionalStreams}{\propgetset} Limit on the number of bidirectional streams the server can open.

    \ditemproperty{long}{MaxUnidirectionalStreams}{\propgetset} Limit on the number of unidirectional streams the server can open.

    \ditemproperty{TimeSpan}{IdleTimeout}{\propgetset} The period of inactivity after which the connection will be closed via idle timeout.

\end{description}

\subsection{QuicStream Class}

The \class{QuicStream} class represents a single stream in a QUIC connection. This class inherits
from the abstract \class{Stream} class. The list below mentions only members specific for the \class{QuicStream} class, members from the \class{Stream} base class are omitted.

The class contains methods for both inbound and outbound streams. Invoking read methods on
write-only stream will cause an exception to be thrown and vice versa.

\begin{description}

    \ditemproperty{long}{StreamId}{\propget} The Stream ID\@.

    \ditemmethod{void}{AbortRead}{long} Aborts the receiving part of the stream with the provided error code.

    \ditemmethod{void}{AbortWrite}{long} Aborts the sending part of the stream with the provided error code.

    \ditemmethod{ValueTask}{WriteAsync}{*, CancellationToken} Multiple overloads of this method offer writing from various types of buffers: \class{ReadOnlyMemory<byte>}, \class{ReadOnlySequence<byte>}, and \class{ReadOnlyMemory\allowbreak<ReadOnlyMemory<byte>>}. The returned task completes when the provided data have been buffered internally and the buffers can be reused for other purposes.

    \ditemmethod{ValueTask}{WriteAsync}{*, bool, CancellationToken} Like the overload above, but also allows pecifying that the provided data are the last on the stream and that the stream should be gracefully closed.

    \ditemmethod{ValueTask}{ShutdownWriteCompleted}{CancellationToken} The returned task completes when the stream shutdown completes. Meaning that acknowledgement from the peer is received.

    \ditemmethod{ValueTask}{Shutdown}{} Closes the stream with error code 0. And blocks until shutdown completes.

\end{description}

\subsection{Exceptions}

The QUIC API can throw following exceptions:

\begin{description}

    \ditem{\ditemsrcsize\class{QuicException}} Base class for all thrown exceptions, used when a more specific exception is not available

    \ditem{\ditemsrcsize\class{QuicConnectionAbortedException}} Thrown when the connection is forcibly closed either by the transport or by the remote endpoint.

    \ditem{\ditemsrcsize\class{QuicStreamAbortedException}} Thrown when the stream was aborted by the remote endpoint.

    \ditem{\ditemsrcsize\class{QuicOperationAbortedException}} Thrown when the pending operation was aborted by the local endpoint.

\end{description}

\section{TLS Implementation}

TLS handshake forms an integral part of the QUIC connection establishment. Because correct TLS
implementation is crucial for ensuring the security of the resulting implementation, this thesis
should avoid implementing TLS by itself. Instead, it should reuse some existing and well-tested
implementation.

The novel way in which QUIC integrates with TLS requires specific functionality on the API of the
TLS implementation. Below is a nonexhaustive list of operations the TLS library's API must provide:

\begin{itemize}

  \item Current state of te handshake

  \item Temporary keys used to protect the handshake process

  \item Raw unencrypted TLS messages to be sent to the other endpoint

  \item Negotiated cipher

  \item Specifying protocols used for ALPN

  \item Specifying custom TLS extension to exchange QUIC transport parameters

\end{itemize}

\subsection{TLS 1.3 Support in \dotnet{}}

The \dotnet{} runtime libraries use different native libraries to provide TLS functionality on
different operating systems. On Windows, \libname{Secure Channel}~\cite{Schannel} (\libschannel{}
for short), which is part of the Windows operating system. On Linux and macOS systems, the
\libopenssl{} library~\cite{OpenSSLWeb} is used.

\begin{description}

    \ditem{\libname{Secure Channel}}
    The \libschannel{} versions present in the latest Windows 10 builds support only TLS 1.2. However,
    future near updates will implement also TLS 1.3. The \libschannel{} version with TLS 1.3 support can
    be obtained by installing an insider preview build of Windows 10.

    The \libmsquic{} library uses \libschannel{} library when compiled for Windows. Therefore, we assume
    that \libschannel{} exposes the necessary API for our managed QUIC implementation.

    \ditem{\libopenssl{}}
    None of the released versions of \libopenssl{} library expose necessary API for integration
    with QUIC, and there are no plans to include such API in the next \libopenssl{} 3.0.0
    release~\cite{OpensslBlogNoQuic}.

    However, developers at Akamai maintain a fork of \libopenssl{} which adds the QUIC-enabling
    API~\cite{AkamaiOpensslGithub}. This modified version of \libopenssl{} is used by \libmsquic{} (for Linux
    builds) and other QUIC libraries \todo{mention them by name + link?: quiche}. It is possible that
    the changes made by Akamai will be merged into \libopenssl{} for the 3.1.0 or later releases.

\end{description}

In conclusion, the APIs required for our QUIC implementation are currently only accessible in only
the preview versions of Windows 10. Relying on \libschannel{} for TLS 1.3 support in our prototype
implementation would severely impact cross-platform availability.

The \libopenssl{} library and supports all platforms supported by \dotnet{}, and could therefore be
used to implement TLS integration into QUIC in a platform-compatible manner. The only drawback of
using the modified \libopenssl{} in \dotnet{} libraries is that the compiled library has to be
distributed with \dotnet{} runtime to ensure that the modified library is present. This solution is
expected to be temporary only for the prototype QUIC implementation, and will be replaced with
\libschannel{} integration for Windows, and mainstream version of \libopenssl{} once the support for
QUIC is released.

\todo{Mention why schannel-based impl is needed: integration with certificate stores, etc.}

This thesis will therefore integrate with the forked \libopenssl{} library from Akamai. Because the
library is written in C, it has to be integrated to the build process to be build together with the
native code of the \dotnet{} runtime.

\section{High-Level Architecture}

In this section we present a high-level architecture design for the QUIC protocol implementation.
Implementation of the QUIC connection requires some background processing for handling timeouts and
acknowledgement.

\todo{separation of IO and Connection logic implies testability}

\subsection{Supporting server socket multiplexing}
\subsection{Future support for server preferred address}
\subsection{Future support for connection migration}
\subsection{Threading model}
\subsection{Incoming packet handling}
\subsection{User request handling}
\subsection{Timeout events}

\section{Integration:}

\todo{should be after TLS because of OpenSSL dependency}


\subsection{Build as part of dotnet runtime}
\subsection{Standalone build for easier evaluation}

\todo{Maybe remove altogether and move this to documentation}

\section{Epoch handling}
\subsection{Separation of epochs in protocol}

\section{Encryption}

\subsection{Key derivation implementation}
\subsection{Future updating 1-RTT protection keys (Key phase bit)}


\section{Packet serialization/Deserialization}

\subsection{QUIC Primitives encoding}
\subsection{Packet types}
\subsection{Frame types}
\subsection{Performance considerations}
\subsection{Allocation-free packet reading/writing}


\section{Stream implementation}

\subsection{Separation of inbound and outbound part of the stream}
\subsection{Inbound part}


\subsubsection{Buffering}
\subsubsection{out of order reception}
\subsubsection{Reading data by user}
\subsubsection{Flow control consideration}


\subsection{Outbound part}

\subsubsection{Buffering}
\subsubsection{Acknowledgement, retransmission}
\subsubsection{Flow control consideration}

\subsection{Abort/Dispose model for streams}
\subsection{Managing streams within connection}
\subsection{Queue of streams to be updated}


\section{Flow control}

\subsection{Connection level}
\subsection{Stream level}
\subsection{MaxStreams and API}


\section{Recovery/Robustness}

\subsection{Detecting lost packets}
\subsection{Detecting duplicate packets}
\subsection{Tail loss probe}
\subsection{Congestion window}
\subsection{Interface for Congestion control algorithm}
