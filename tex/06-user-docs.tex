\chapter{User Documentation}

This section provides guidance on how to obtain the build of the \dotnet{} runtime with managed QUIC
implementation, how to install it, and how to use the code to develop other applications.

\section{Getting Started}

This thesis provides a branch of the \dotnet{} runtime codebase with managed QUIC implementation.
Since our branch contains changes only in the \SystemNetQuicDll{}, the easiest way of composing a
fully working \dotnet{} distribution is obtaining a full SDK installation of the latest master
development version of \dotnet{}~6 and replacing the \SystemNetQuicDll{}. This section explains how
this can be done without affecting other \dotnet{} SDK installations present on the machine.

The managed implementation depends on a particular \libopenssl{} version in order to interoperate
with other QUIC implementations. This section also explains how to deploy a locally built
\libopenssl{} so that it is automatically used by user code. The process consists of following
steps which will be explained in following subsections:

\begin{enumerate}

  \item Build the \SystemNetQuicDll{} library from our code branch runtime.

  \item Compose a new local \dotnet{}~6 SDK installation with our \SystemNetQuicDll{}.

  \item \textit{(optional)} Compile a QUIC-supporting \libopenssl{} from source and deploy it.

  \item Configure the development environment to use the local installation when compiling the code.

\end{enumerate}

\subsection{Building the System.Net.Quic Library from Source}

The source code for the \dotnet{} runtime with managed QUIC implementation is part of this thesis'
attachments. \todo{path to the src once the attachments are finalized}. Alternatively, the code can
also be found on github \todo{link}. For the remainder of this section, all paths will be relative
to the \dotnet{} runtime repository directory.

The \dotnet{} runtime repository contains a descriptive guide on how to build the sources. The
necessary prerequisities are listed in files inside the \filename{docs/workflow/requirements/}
directory, separately for each operating system.

Once you have installed all prerequisities, you can build the entire \dotnet{} runtime using the
\filename{build.cmd} or \filename{build.sh} script:

\begin{myVerbatim}
> build.cmd -subset clr+libs -configuration release
\end{myVerbatim}

The above command will build the Common Language Runtime (CLR) and all libraries in Release
configuration. The artifacts are available in the \filename{artifacts/bin/System.Net.Quic}
directory. The important artifacts from this directory are:

\begin{itemize}

  \item \filename{ref/net6.0-Release/System.Net.Quic.dll}: The so-called reference assembly that specifies the public API of the library.

  \item \filename{net6.0-{OS}-Release/System.Net.Quic.dll}: Where \verb|{OS}| is the operating system platform you are running. This is the library with the actual implementation.

\end{itemize}

\subsection{Creating a Local Installation of \dotnet{}}\label{sec:06-local-dotnet}

Now we need to download the latest \dotnet{}~6 SDK. This zip archive containing this SDK can be
downloaded from a link listed in the official SDK installer GitHub
repository~\cite{dotnetSdkGithub}. Download the ``Master (6.0.x Runtime)'' build for your platform
and extract it to a convenient location. In the remainder of this guide, the directory containing
the extracted contents will be referred to as \filename{DOTNET_ROOT}.

The \SystemNetQuicDll{} produced in the previous subsection must be coppied to appropriate locations
in the \filename{DOTNET_ROOT}. The reference \SystemNetQuicDll{} assembly should be copied over the
existing one in the
\filename{DOTNET_ROOT/packs/Microsoft.NETCore.App.Ref/6.0.0-{version}/ref/net6.0/} directory, and
the implementation assembly to should be copied to the
\filename{DOTNET_ROOT/shared/Microsoft.NETCore.App/6.0.0-{version}/} directory, overwriting the
existing files.

\subsection{Adding the OpenSSL Library}

The implementation requires a QUIC-supporting \libopenssl{} library build from a development branch
maintained by Akamai~\cite{AkamaiOpensslGithub}. The appropriate source codes are attached
\todo{path}, or can be obtained from the github repository in branch \todo{branch name}.

Before building the source, check \filename{NOTES.{OS}} file in the repository and make sure you
have all prerequisities installed on your machine. After that, you can build the \libopenssl{}
library by running following command inside the repository. Note that for Windows OS, you must run
these commands using the \textit{Developer command prompt for Visual Studio} in order to have the necessary tools in
\texttt{PATH}.

\begin{myVerbatim}
# Windows
> perl Configure VC-WIN64A
> nmake

# Linux
> ./config
> make
\end{myVerbatim}

This will produce the \libname{libcrypto} and \libname{libssl} libraries in the \libopenssl{}
repository root. On windows, these libraries are named \texttt{libcrypto-1_1-x64.dll} and
\texttt{libssl-1_1-x64.dll}. These libraries are loaded by managed QUIC implementation at runtime
and, therefore, must be present in a location where the OS loader can find them. This can be achieved by putting the libraries in any following location:

\begin{itemize}

  \item Next to the compiled executable

  \item A directory listed in the \texttt{PATH} environment variable

  \item \textit{(preferred)} next to the \SystemNetQuicDll{} library in the \dotnet{} installation directory, i.e., \filename{DOTNET_ROOT/shared/Microsoft.NETCore.App/6.0.0-{version}/}.

\end{itemize}

\subsection{Configuring the Development Environment}\label{sec:06-env-vars}

Lastly, we need to configure the environment variables so that the \dotnet{} SDK installation created in previous step is used when building the user code. For this, following environment variables need to be defined correctly.

\begin{description}

        \ditem{\texttt{DOTNET_ROOT}} Path to the dotnet installation directory. This instructs the build process to use the SDK installed in this directory.

        \ditem{\texttt{DOTNET_MULTILEVEL_LOOKUP}} Set this to ``0''. This instructs the build process not to look for SDK installation in other places than \filename{DOTNET_ROOT}.

        \ditem{\texttt{PATH}} Prepend the \filename{DOTNET_ROOT} directory to the beginning of the \texttt{PATH} variable to make sure the \filename{dotnet} executable from the \filename{DOTNET_ROOT} is used over the system-wide installed one.
\end{description}

After configuring the variables check the output of the \verb|dotnet --info| command. Assuming
\texttt{DOTNET_ROOT} is \filename{C:\dotnet}, then the output should be similar to the
\autoref{lst:06-dotnet-info-output}. The list of installed \dotnet{} runtimes should contain
\texttt{Microsoft.NETCore.App} from the local \dotnet{}~6 SDK installation prepared in
\autoref{sec:06-local-dotnet}. Note that the listing contains version numbers valid at the time of
writing this text and the SDK installer would be updated since then to a newer version.

\todo{this has been shortened and greyed out}

\begin{myListingVerbatim}[Output of the \texttt{dotnet --info} command in configured environment]{lst:06-dotnet-info-output}{Output of the \texttt{dotnet --info} command in correctly configured environment. The unimportant portions of the output in grey has been left out brevity.}
> dotnet --info
&color{colorunimportant}.NET SDK (reflecting any global.json):
&color{colorunimportant} ...

&color{colorunimportant}Runtime Environment:
&color{colorunimportant} ...

&color{colorunimportant}Host (useful for support):
&color{colorunimportant} ...

.NET SDKs installed:
  6.0.100-alpha.1.20563.2 [C:\dotnet\sdk]

.NET runtimes installed:
  Microsoft.AspNetCore.App 6.0.0-alpha.1.20526.6 [C:\dotnet\shared\&textcolor{colorunimportant}{...}]
  Microsoft.NETCore.App 6.0.0-alpha.1.20560.10 [C:\dotnet\shared\&textcolor{colorunimportant}{...}]
  Microsoft.WindowsDesktop.App 6.0.0-alpha.1.20560.7 [C:\dotnet\shared\&textcolor{colorunimportant}{...}]

&color{colorunimportant}To install additional .NET runtimes or SDKs:
&color{colorunimportant}  https://aka.ms/dotnet-download
\end{myListingVerbatim}

With environment configured as described, you can now use the
\verb|dotnet build| command, or start Visual Studio or other IDE to develop programs using the
provided SDK installation.

\subsection{Creating a Sample \dotnet{}~6 Project}

The last step is creating a new project and configuring it to use \dotnet{}~6. This section
demonstrates how this can be done using the \verb|dotnet| command line tool. Assuming the
environment variables have been configured as specified in \autoref{sec:06-env-vars}, you can create
a new project using the following commands:

\begin{myVerbatim}
> mkdir hello-net6
> cd hello-net6
> dotnet new console
\end{myVerbatim}

These commands will create a \filename{hello-net6/hello-net6.csproj} file. Use of \dotnet{}~6
preview requires minor changes to the project file. Namely changing the \xmltag{TargetFramework}
property to \verb|net6.0|. The modified project file contents are listed in \autoref{lst:06-csproj-net6.0}.

\begin{myListingXml}{lst:06-csproj-net6.0}{Project file for \dotnet{}~6 console application project.}{Sdk}
<Project Sdk="Microsoft.NET.Sdk">

  <PropertyGroup>
    <OutputType>Exe</OutputType>
    <TargetFramework>net6.0</TargetFramework>
    <RootNamespace>hello_net6</RootNamespace>
  </PropertyGroup>

</Project>
\end{myListingXml}

Lastly, the NuGet package feed must be configured to use the correct source for the preview packages
for \dotnet{}~6. This can be done by creating a \filename{NuGet.Config} file in the project
directory with contents as listed in \autoref{lst:06-nuget-config}. The contents can be also copied
from the webpage from which the \dotnet{}~6 SDK was downloaded.

\begin{myListingXml}[basicstyle=\ttfamily\scriptsize]{lst:06-nuget-config}{NuGet configuration file for \dotnet{}~6 projects.}{key,value}
<configuration>
  <packageSources>
    <add key="dotnet6"
      value="https://pkgs.dev.azure.com/dnceng/public/_packaging/dotnet6/nuget/v3/index.json" />
  </packageSources>
</configuration>
\end{myListingXml}

To demonstrate that we are using the correct SDK, you can try compiling and running program from
\autoref{lst:06-hello-net6}.

\begin{myListingCsharp}{lst:06-hello-net6}{\csharp{} program for testing the SDK installation.}{Program,FileVersionInfo,Console,Path}{}
using System;
using System.Diagnostics;

namespace hello_net6._0
{
    class Program
    {
        static void |Main|(string[] args)
        {
            string assemblyPath = typeof(object).Assembly.Location;
            string assemblyDir = Path.|GetDirectoryName|(assemblyPath);
            var info = FileVersionInfo.|GetVersionInfo|(assemblyPath);
            Console.|WriteLine|($"Hello from .NET {info.ProductVersion}");
            Console.|WriteLine|($"Runtime location: {assemblyDir}");
        }
    }
}
\end{myListingCsharp}

Running the program should produce similar output to the following:

\begin{myVerbatim}
> dotnet run
Hello from .NET 6.0.0-alpha.1.20560.10+72b7d236ad634c2280c73499ebfc2b594995ec06
Runtime location: C:\dotnet\shared\Microsoft.NETCore.App\6.0.0-alpha.1.20560.10
\end{myVerbatim}

\section{Using QUIC in \dotnet{}}

This section is a walkthrough on how to use QUIC in \dotnet{}.

\todo{a most trivial echo example}

\section{An example application}

\todo{deeper explanation of the example throughput testing application that was used in Evaluation}

\section{QUIC API Reference}\label{sec:06-api}

This section describes the API that was designed by the \dotnet{} development team to expose QUIC to
other developers. This thesis will use this API to allow swapping underlying implementation between
this thesis implementation and the \libmsquic{}-based implementation. As mentioned in the
introduction chapter, the current design is a work-in-progress and is subject to change in the
future. All of the mentioned classes are located in the \namespace{System.Net.Quic} namespace.

\subsection{QuicListener Class}

The \class{QuicListener} class is the equivalent of the \class{TcpListener}. Servers use this
class to accept incoming QUIC connections.

\begin{description}

    \ditemctor{QuicListener}{\QuicListenerOptions{}} Constructor.

    \ditemproperty{IPEndPoint}{ListenEndPoint}{\propget} The IP endpoint being listened to for new connection. Read-only.

    \ditemmethod[]{\ValueTaskOf{\QuicConnection{}}}{AcceptConnectionAsync}{\CancellationToken{}}
    Accepts a new incoming QUIC Connection.

    \ditemmethod[\keyword]{void}{Start}{} Starts listening.

    \ditemmethod[\keyword]{void}{Close}{} Stops listening and closes the listener. Does not close already accepted connections.

\end{description}

\subsection{QuicListenerOptions Class}

The \class{QuicListenerOptions} class holds all configuration used to construct new \class{QuicListener} instances.

\begin{description}

    \ditemproperty{SslServerAuthenticationOptions}{\ServerAuthenticationOptions{}}{\propgetset}
        SSL related options like certificate selection/validation callbacks, and supported protocols for ALPN\@.

    \ditemproperty[\keyword]{string}{CertificateFilePath}{\propgetset} Path to the X509 certificate used by the server.

    \ditemproperty[\keyword]{string}{CertificateKeyPath}{\propgetset} Path to the private key for the used X509 certificate.

    \ditemproperty[\keyword]{string}{CertificateKeyPath}{\propgetset} Path to the private key for the used X509 certificate.

    \ditemproperty{IPEndPoint}{ListenEndPoint}{\propgetset} The IP endpoint to listen on.

    \ditemproperty[\keyword]{int}{ListenBacklog}{\propgetset} Number of connection to be held waiting for acceptance by the application. Upon reaching this limit, further connections will be refused.

    \ditemproperty[\keyword]{long}{MaxBidirectionalStreams}{\propgetset} Limit on the number of bidirectional streams the client can open in an accepted connection.

    \ditemproperty[\keyword]{long}{MaxUnidirectionalStreams}{\propgetset} Limit on the number of unidirectional streams the client can open in an accepted connection.

    \ditemproperty{TimeSpan}{IdleTimeout}{\propgetset} The period of inactivity after which the connection will be closed via idle timeout.

\end{description}

\subsection{QuicConnection Class}

The \QuicConnection{} class provides operation on the QUIC connection. Clients open new
connections by creating a new instance of this class and calling the \method{ConnectAsync} method.
Servers receive new connections using the \class{QuicListener} class.

\begin{description}

    \ditemctor{QuicConnection}{\QuicClientConnectionOptions{}} Constructor. The newly created instance is not connected until the call to \method{ConnectAsync} method.

    \ditemproperty[\keyword]{bool}{Connected}{\propget} Indicates whether the \QuicConnection{} is connected (the handshake has completed).

    \ditemproperty{IPEndPoint}{LocalEndPoint}{\propget} Local IP endpoint of the connection.

    \ditemproperty{IPEndPoint}{RemoteEndPoint}{\propget} Remote IP endpoint of the connection.

    \ditemmethod{ValueTask}{ConnectAsync}{\CancellationToken{}} Connects to the remote endpoint.

    \ditemmethod{QuicStream}{OpenUnidirectionalStream}{} Opens a new unidirectional stream. Throws a \class{QuicException} if the stream cannot be opened.

    \ditemmethod{QuicStream}{OpenBidirectionalStream}{} Opens a new bidirectional stream. Throws a \class{QuicException} if the stream cannot be opened.

    \ditemmethod[]{\ValueTaskOf{\QuicStream{}}}{AcceptStreamAsync}{\CancellationToken{}} Accepts an incoming stream.

    \ditemmethod{ValueTask}{CloseAsync}{\Long{}, \CancellationToken{}} Closes the connection with the specified given error code and terminates all active streams.

    \ditemmethod[\keyword]{long}{GetRemoteAvailableUnidirectionalStreamCount}{} Gets the maximum number of unidirectional streams that this endpoint can open.

    \ditemmethod[\keyword]{long}{GetRemoteAvailableBidirectionalStreamCount}{} Gets the maximum number of bidirectional streams that this endpoint can open.

\end{description}

\subsection{QuicClientConnectionOptions}

The \class{QuicClientConnectionOptions} is used by clients to configure new QUIC conections.

\begin{description}

    \ditemproperty{SslClientAuthenticationOptions}{ClientAuthenticationOptions}{\propgetset} Client authentication options to use when establishing the connection.

    \ditemproperty{IPEndPoint}{LocalEndPoint}{\propgetset} The local IP endpoint that will be bound to.

    \ditemproperty{IPEndPoint}{RemoteEndPoint}{\propgetset} The IP endpoint to connect to.

    \ditemproperty[\keyword]{long}{MaxBidirectionalStreams}{\propgetset} Limit on the number of bidirectional streams the server can open.

    \ditemproperty[\keyword]{long}{MaxUnidirectionalStreams}{\propgetset} Limit on the number of unidirectional streams the server can open.

    \ditemproperty{TimeSpan}{IdleTimeout}{\propgetset} The period of inactivity after which the connection will be closed via idle timeout.

\end{description}

\subsection{QuicStream Class}

The \QuicStream{} class represents a single stream in a QUIC connection and derives from the
abstract \class{Stream} class. The \class{Stream} class is a bidirectional stream abstraction and
since not all QUIC streams are bidirectional, user should check if the specific \QuicStream{}
instance supports supports the operation by inspecting the \method{CanRead} and \method{CanWrite}
properties. Invoking read methods on write-only (i.e. unidirectional sending) stream will cause an
exception to be thrown and vice versa.

The list below mentions the members specific for the \class{QuicStream} class and some important
members inherited from the \class{Stream} class.

\begin{description}

    \ditemproperty[\keyword]{long}{StreamId}{\propget} The Stream ID\@.

    \ditemproperty[\keyword]{bool}{CanRead}{\propget} Returns \keyword{true} if the stream supports reading.

    \ditemproperty[\keyword]{bool}{CanWrite}{\propget} Returns \keyword{true} if the stream supports reading.

    \ditemmethod[\keyword]{void}{AbortRead}{\Long} Aborts the receiving part of the stream with the provided error code.

    \ditemmethod[\keyword]{void}{AbortWrite}{\Long} Aborts the sending part of the stream with the provided error code.

    \ditemmethod[\keyword]{int}{Read}{\SpanOf{\byte{}}} Reads the content
    of the stream into the provided buffer, blocks if no data is available. Returns 0 only when there will be no more data in the stream.

    \ditemmethod[]{\ValueTaskOf{\keyword{int}}}{ReadAsync}{\MemoryOf{\byte{}}, \CancellationToken{}} Reads the content
    of the stream into provided buffer, blocks until some data is available. Returns 0 only when there will be no more data in the stream.

    \ditemmethod[\keyword]{void}{Write}{\SpanOf{\byte}} Writes the content
    of the provided buffer into the stream, returns when the data have been buffered internally.

    \ditemmethodWithComment{ValueTask}{WriteAsync}{*, \CancellationToken{}}{multiple overloads} Multiple overloads of this method offer writing from various types of buffers: \ReadOnlyMemoryOf{\byte{}}, \ReadOnlySequenceOf{\byte{}}, and \genericClass{ReadOnlyMemory\allowbreak{}}{\ReadOnlyMemoryOf{\byte{}}}. The last one can be used to perform scatter/gather IO. The returned task completes when the provided data have been buffered internally and the buffers can be reused for other purposes.

    \ditemmethodWithComment{ValueTask}{WriteAsync}{*, \bool{}, \CancellationToken{}}{multiple overloads} Like the methods above, but also allow specifying that the provided data are the last on the stream and that the stream should be gracefully closed.

    \ditemmethod{ValueTask}{ShutdownWriteCompleted}{\CancellationToken{}} The returned task completes when the stream shutdown completes. Meaning that acknowledgement from the peer is received.

    \ditemmethod{ValueTask}{Shutdown}{} Closes the stream with error code 0. And blocks until shutdown completes.

\end{description}

\subsection{Exceptions}

The QUIC API can throw following exceptions:

\begin{description}

    \ditem{\ditemsrcsize\class{QuicException}} Base class for all thrown exceptions, used when a more specific exception is not available

    \ditem{\ditemsrcsize\class{QuicConnectionAbortedException}} Thrown when the connection is forcibly closed either by the transport or by the remote endpoint.

    \ditem{\ditemsrcsize\class{QuicStreamAbortedException}} Thrown when the stream was aborted by the remote endpoint.

    \ditem{\ditemsrcsize\class{QuicOperationAbortedException}} Thrown when the pending operation was aborted by the local endpoint.

\end{description}
