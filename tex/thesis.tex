%%% The main file. It contains definitions of basic parameters and includes all other parts.

%% Settings for single-side (simplex) printing
% Margins: left 40mm, right 25mm, top and bottom 25mm
% (but beware, LaTeX adds 1in implicitly)
% \documentclass[12pt,a4paper]{report}
% \setlength\textwidth{145mm}
% \setlength\textheight{247mm}
% \setlength\oddsidemargin{15mm}
% \setlength\evensidemargin{15mm}
% \setlength\topmargin{0mm}
% \setlength\headsep{0mm}
% \setlength\headheight{0mm}
% \openright makes the following text appear on a right-hand page
%\let\openright=\clearpage

%% Settings for two-sided (duplex) printing
\documentclass[12pt,a4paper,twoside,openright]{report}
\setlength\textwidth{145mm}
\setlength\textheight{247mm}
\setlength\oddsidemargin{14.2mm}
\setlength\evensidemargin{0mm}
\setlength\topmargin{0mm}
\setlength\headsep{0mm}
\setlength\headheight{0mm}
\let\openright=\cleardoublepage

%% Generate PDF/A-2u
\usepackage[a-2u]{pdfx}

%% Fix anoying bug when using biblatex together with pdfx
%% https://tex.stackexchange.com/questions/564990/error-after-miktex-reinstall-text-line-contains-an-invalid-character
\catcode30=12

%% Character encoding: usually latin2, cp1250 or utf8:
\usepackage[utf8]{inputenc}

%% Prefer Latin Modern fonts
\usepackage{lmodern}

%% set language
\usepackage[english]{babel}
\usepackage{csquotes}

%% Further useful packages (included in most LaTeX distributions)
\usepackage{amsmath}        % extensions for typesetting of math
\usepackage{amsfonts}       % math fonts
\usepackage{amsthm}         % theorems, definitions, etc.
\usepackage{bbding}         % various symbols (squares, asterisks, scissors, ...)
\usepackage{bm}             % boldface symbols (\bm)
\usepackage{graphicx}       % embedding of pictures
\usepackage{fancyvrb}       % improved verbatim environment
%\usepackage[numbers]{natbib}         % citation
\usepackage[backend=biber,sorting=none]{biblatex}         % citation
\usepackage[nottoc]{tocbibind} % makes sure that bibliography and the lists
			    % of figures/tables are included in the table
			    % of contents
\usepackage{dcolumn}        % improved alignment of table columns
\usepackage{booktabs}       % improved horizontal lines in tables
\usepackage{paralist}       % improved enumerate and itemize
\usepackage{xcolor}         % typesetting in color
\usepackage{booktabs}       % better horizontal lines for tables
\usepackage[binary-units]{siunitx}        % thousands separators for large numbers in text and units typesetting
\usepackage[labelfont=bf]{caption} % use bold captions lables (Figure X.X:)
\usepackage{subcaption}     % subfigures
\usepackage{tikz}           % for drawing (checkmark etc)
\usepackage{bytefield}      % packet structure
\usepackage{float}       % better positioning of figures
\usepackage{newfloat}       % better positioning of figures
\usepackage{filecontents}   % for auto creating of .xmpdata file
\usepackage{listings}       % code listings
\usepackage{xparse}         % for serious parsing magic
\usepackage[strings]{underscore}     % underscores which do not screw up hyphenation
\usepackage[toc,nopostdot,indexonlyfirst]{glossaries}     % lists of terms and abbreviations
\usepackage{dirtree}        % directory listings with comments
\usepackage{mdframed}       % fancy frames
\usepackage{epstopdf}       % render output from gnuplot to epslatex terminal
\usepackage{enumitem}       % modifying list styles
\usepackage{letterspace}    % adjusting spacing between letters

\addbibresource{bibliography.bib}

%% suppress warnings for inkscape-generated figures
%% https://tex.stackexchange.com/questions/76273/multiple-pdfs-with-page-group-included-in-a-single-page-warning
\pdfsuppresswarningpagegroup=1

%%% Graphics path (for some reason needed on windows)
\graphicspath{{img/}}

%%% Make listings part of the table of contents
\renewcommand{\lstlistoflistings}{\begingroup
\tocfile{List of Listings}{lol}
\endgroup}

\setlist[description]{style=nextline}

%%% Basic information on the thesis

% Thesis title in English (exactly as in the formal assignment)
\def\ThesisTitle{QUIC Protocol Implementation for .NET}

% Author of the thesis
\def\ThesisAuthor{Bc. Radek Zikmund}

% Year when the thesis is submitted
\def\YearSubmitted{2020}

% Name of the department or institute, where the work was officially assigned
% (according to the Organizational Structure of MFF UK in English,
% or a full name of a department outside MFF)
\def\Department{Department of Distributed and Dependable Systems}

% Is it a department (katedra), or an institute (ústav)?
\def\DeptType{Department}

% Thesis supervisor: name, surname and titles
\def\Supervisor{Mgr. Pavel Ježek, Ph.D.}

% Supervisor's department (again according to Organizational structure of MFF)
\def\SupervisorsDepartment{Department of Distributed and Dependable Systems}

% Study programme and specialization
\def\StudyProgramme{Computer Science}
\def\StudyBranch{Software and Data Engineering}

% An optional dedication: you can thank whomever you wish (your supervisor,
% consultant, a person who lent the software, etc.)
\def\Dedication{%
Dedication.
}

% Abstract (recommended length around 80-200 words; this is not a copy of your thesis assignment!)
\edef\Abstract{%
TODO: do the abstract last%
}

% 3 to 5 keywords (recommended), each enclosed in curly braces
\def\Keywords{%
{QUIC} {.NET} {network protocol}
}

%% The hyperref package for clickable links in PDF and also for storing
%% metadata to PDF (including the table of contents).
%% Most settings are pre-set by the pdfx package.
\hypersetup{unicode}
\hypersetup{breaklinks=true}

%% make \autoref use section instead of subsection and subsubsection
\let\subsectionautorefname\sectionautorefname
\let\subsubsectionautorefname\sectionautorefname

%% generate thesis.xmpdata file
%% see https://tex.stackexchange.com/questions/534035/macros-commands-inside-a-filecontents-environment-does-not-expand
%% for the \filecontentsspecials magic

\def\filecontentsspecials#1#2#3{
  \global\let\ltxspecials\dospecials
  \gdef\dospecials{\ltxspecials
    \catcode`#1=0
    \catcode`#2=1
    \catcode`#3=2
    \global\let\dospecials\ltxspecials
  }
}
\filecontentsspecials|[]
\begin{filecontents}[overwrite]{\jobname.xmpdata}
\Author{|ThesisAuthor}
\Title{|ThesisTitle}
\Keywords{QUIC\sep .NET\sep network protocol}
\Subject{|Abstract}
\Publisher{Charles University}
\end{filecontents}

%% also generate abstract.txt so that it can be included from the abstract.tex file
\filecontentsspecials|[]
\begin{filecontents}[overwrite]{abstract.txt}
|Abstract
\end{filecontents}

% Definitions of macros (see description inside)
%%% This file contains definitions of various useful macros and environments %%%
%%% Please add more macros here instead of cluttering other files with them. %%%

%%% Minor tweaks of style

% These macros employ a little dirty trick to convince LaTeX to typeset
% chapter headings sanely, without lots of empty space above them.
% Feel free to ignore.
\makeatletter
\def\@makechapterhead#1{
  {\parindent \z@ \raggedright \normalfont
   \Huge\bfseries \thechapter. #1
   \par\nobreak
   \vskip 20\p@
}}
\def\@makeschapterhead#1{
  {\parindent \z@ \raggedright \normalfont
   \Huge\bfseries #1
   \par\nobreak
   \vskip 20\p@
}}
\makeatother

% This macro defines a chapter, which is not numbered, but is included
% in the table of contents.
\def\chapwithtoc#1{
\chapter*{#1}
\addcontentsline{toc}{chapter}{#1}
}

% Draw black "slugs" whenever a line overflows, so that we can spot it easily.
\overfullrule=1mm



%%% Functional foreach construct
%%% (https://stackoverflow.com/questions/2402354/split-comma-separated-parameters-in-latex)

\makeatletter

% #1 - Function to call on each comma-separated item in #3
% #2 - Parameter to pass to function in #1 as first parameter
% #3 - Comma-separated list of items to pass as second parameter to function #1
\def\foreach#1#2#3{%
  \@test@foreach{#1}{#2}#3,\@end@token%
}

% Internal helper function - Eats one input
\def\@swallow#1{}

% Internal helper function - Checks the next character after #1 and #2 and
% continues loop iteration if \@end@token is not found
\def\@test@foreach#1#2{%
  \@ifnextchar\@end@token%
    {\@swallow}%
    {\@foreach{#1}{#2}}%
}

% Internal helper function - Calls #1{#2}{#3} and recurses
% The magic of splitting the third parameter occurs in the pattern matching of the \def
\def\@foreach#1#2#3,#4\@end@token{%
  #1{#2}{#3}%
  \@test@foreach{#1}{#2}#4\@end@token%
}

%%% foreach usage:
% Example-function used in foreach, which takes two params and builds hrefs
%\def\makehref#1#2{\href{#1/#2}{#2}}

% Using foreach by passing #1=function, #2=constant parameter, #3=comma-separated list
%\foreach{\makehref}{http://stackoverflow.com}{2409851,2408268}

% Will in effect do
%\href{http://stackoverflow.com/2409851}{2409851}\href{http://stackoverflow.com/2408268}{2408268}

\makeatother



%%% Macros for definitions, theorems, claims, examples, ... (requires amsthm package)

\theoremstyle{plain}
\newtheorem{thm}{Theorem}
\newtheorem{lemma}[thm]{Lemma}
\newtheorem{claim}[thm]{Claim}

\theoremstyle{plain}
\newtheorem{defn}{Definition}

\theoremstyle{remark}
\newtheorem*{cor}{Corollary}
\newtheorem*{rem}{Remark}
\newtheorem*{example}{Example}

%%% An environment for proofs

\newenvironment{myproof}{
  \par\medskip\noindent
  \textit{Proof}.
}{
\newline
\rightline{$\qedsymbol$}
}

%%% An environment for typesetting of program code and input/output
%%% of programs. (Requires the fancyvrb package -- fancy verbatim.)

\DefineVerbatimEnvironment{code}{Verbatim}{fontsize=\small, frame=single}

%%% The field of all real and natural numbers
\newcommand{\R}{\mathbb{R}}
\newcommand{\N}{\mathbb{N}}

%%% Useful operators for statistics and probability
\DeclareMathOperator{\pr}{\textsf{P}}
\DeclareMathOperator{\E}{\textsf{E}\,}
\DeclareMathOperator{\var}{\textrm{var}}
\DeclareMathOperator{\sd}{\textrm{sd}}

%%% Transposition of a vector/matrix
\newcommand{\T}[1]{#1^\top}

%%% Various math goodies
\newcommand{\goto}{\rightarrow}
\newcommand{\gotop}{\stackrel{P}{\longrightarrow}}
\newcommand{\maon}[1]{o(n^{#1})}
\newcommand{\abs}[1]{\left|{#1}\right|}
\newcommand{\dint}{\int_0^\tau\!\!\int_0^\tau}
\newcommand{\isqr}[1]{\frac{1}{\sqrt{#1}}}

%%% Various table goodies
\newcommand{\pulrad}[1]{\raisebox{1.5ex}[0pt]{#1}}
\newcommand{\mc}[1]{\multicolumn{1}{c}{#1}}

% Outputs red TODOs in the document. Requires \usepackage{color}.
%
% Usage: \todo{Document the TODO command.}
%
% Comment out second line to disable.
\newcommand{\todo}[1]{}
\renewcommand{\todo}[1]{{\color{red} TODO: {#1}}}

% the symbol of .NET
\newcommand{\dotnet}{.NET\@}

% prettier C# typesetting
\newcommand{\csharp}{%
  {\settoheight{\dimen0}{C}C\kern-.05em \resizebox{!}{\dimen0}{\raisebox{\depth}{\#}}}}

% special items for itemize and enumerate environments which have a label
\newcommand\litem[1]{\item{\textit{#1}}:}

% special item for description lists to make the description go on the next line
\newcommand\ditem[1]{\item[#1] \leavevmode \\}
\newcommand\ditemWithComment[2]{\item[#1] \hfill \textit{(#2)} \leavevmode \\}

% macro for typesetting library names
\newcommand\libname[1]{\texttt{\itshape#1}}

% macros for library names used throughout the thesis
\newcommand{\libmsquic}{\libname{MsQuic}}
\newcommand{\libcurl}{\libname{libcurl}}
\newcommand{\libopenssl}{\libname{OpenSSL}}
\newcommand{\libschannel}{\libname{Schannel}}

% environment setting common attributes of figures
\newcommand{\figureArgs}{} % dummy macro to be redefined, used to make caption go under the figure
\newenvironment{myFigure}[2]{% [label, caption]
  \renewcommand{\figureArgs}{\caption{#2}\label{#1}}%
  \figure[ht]{}
  \centering
}{ %
  \figureArgs{}
  \endfigure{}
}

% macros for typesetting various code elements inside th text, TODO: add colors

\definecolor{codecolorclass}{rgb}{0, 0.4, 0}
\definecolor{codecolorkeyword}{rgb}{0, 0, 0.5}
\definecolor{codecolornamespace}{gray}{0}

\newcommand{\class}[1]{{\color{codecolorclass}\texttt{#1}}}
\newcommand{\keyword}[1]{\color{codecolorkeyword}\texttt{#1}}
\newcommand{\namespace}[1]{\color{codecolornamespace}\texttt{#1}}
\newcommand{\method}[1]{\textit{\texttt{#1}}}

% macros for typesetting description lists with C# class members

\newcommand{\propgetset}{\{ \keyword{get}; \keyword{set}; \}}
\newcommand{\propget}{\{ \keyword{get}; \}}
\newcommand{\propset}{\{ \keyword{set}; \}}

\newcommand{\ditemsrcsize}{\footnotesize}


\newcommand{\ditemmethodseparator}{}

% method to apply in \foreach macro
\def\makeditemmethodarg#1#2{\ditemmethodseparator{}\class{#2}%
\renewcommand{\ditemmethodseparator}{, }%set proper separator for the next item
}

\newcommand{\ditemmethodmakearglist}[1]{% arglist
\renewcommand{\ditemmethodseparator}{}%clear the separator for the first item
\foreach{\makeditemmethodarg}{}{#1}%
}

\newcommand{\ditemmethod}[3]% type, name, args (types only)
  {\ditem{\ditemsrcsize\texttt{\class{#1} \method{#2}(\ditemmethodmakearglist{#3})}}}

\newcommand{\ditemctor}[2]% name, args
  {\ditem{\ditemsrcsize\texttt{\method{#1}(\ditemmethodmakearglist{#2})}}}

\newcommand{\ditemproperty}[3]% type, name, get/set
  {\ditem{\ditemsrcsize\texttt{\class{#1} \method{#2} #3}}}

% environment setting common attributes of tables
\newenvironment{myTable}[5][\normalsize]{% [fontsize, label, caption, columnspec, header]
  \table[ht]
  #1
  \caption{#3}\label{#2}% for tables, the caption goes to the top
  \centering
  \tabular{#4}
  \toprule
  #5 \\
  \midrule
}{ %
  \bottomrule
  \endtabular{}
  \endtable{}
}

% checkmark
\def\checkmark{\tikz\fill[scale=0.4](0,.35) -- (.25,0) -- (1,.7) -- (.25,.15) -- cycle;}


\makeglossaries

\renewcommand{\glsnamefont}[1]{\capitalisewords{#1}}

% we use glossary even for abbreviations, so that we can include a description
\newcommand{\newdefinedabbreviation}[4]{
    \newglossaryentry{#1}
    {
        text={#2},
        long={#3},
        name={#3 (#2)},
        first={#3 (#2)},
        firstplural={\glsentrylong{#1}\glspluralsuffix (\glsentryname{#1}\glspluralsuffix )},
        description={#4}
    }
}

\newglossaryentry{managed-code}
{
  name=managed code,
  description={Code written in one of the \dotnet{} languages running on the \dotnet{} virtual machine.}
}

\newglossaryentry{native-code}
{
  name=native code,
  description={Code written in ahead-of-time compiled language. This code runs directly on the target CPU.}
}

\newglossaryentry{network-path}
{
    name=network path,
    description={An imaginary path between two network addresses. Each end of a network address consists of a pair of local address and port number.}
}

\newglossaryentry{head-of-line-blocking}
{
    name=head-of-line blocking,
    description={Performance limiting phenomenon caused by a line of packets being held up by the first packet}
}

\newglossaryentry{quic-packet}
{
    name=QUIC packet,
    description={A complete processable unit of QUIC that can be encapsulated in a UDP datagram.
Multiple QUIC packets can be encapsulated in a single UDP datagram.}
}

\newglossaryentry{out-of-order-packet}
{
  name=out-of-order packet,
  description={A packet that does not arrive directly after the packet that was
  sent before it.  A packet can arrive out of order if it is delayed, if earlier packets are
  lost or delayed, or if the sender intentionally skips a packet number.}
}

\newglossaryentry{cid}
{
  name=Connection ID,
  description={An opaque identifier that is used to identify a QUIC
  connection at an endpoint.  Each endpoint sets a value for its
  peer to include in packets sent towards the endpoint.}
}

\newglossaryentry{ack-eliciting-packet}
{
  name=ack-eliciting packet,
  description={A QUIC packet which contains at least one frame which is not \PADDING{},  \ACK{}, or \CONNECTIONCLOSE{}}
}

\newglossaryentry{strategy-pattern}
{
  name=strategy pattern,
  description={A behavioral design pattern which allows selecting an internal implementation (algorithm) at runtime.}
}

\newglossaryentry{traffic-amplification-attack}
{
  name=traffic amplification attack,
  description={A type of distributed denial-of-service (DDoS) attack in which attacker sends a small amount of data to a server and the server responds with larger amount of data to the target. An example of such attack is initiating new connections with a falsified source IP address which make the server flood the victim with handshake attempts.}
}

\newglossaryentry{path-validation}
{
  name=path validation,
  description={A process during which QUIC endpoint validates that it's peer is reachable via a particular \gls{network-path}. Consists of an exchange of \PATHCHALLENGE{} and \PATHRESPONSE{} frames.}
}

\newglossaryentry{packet-pacing}
{
  name=packet pacing,
  description={A mechanism which evens out microbursts of packets in order to prevent network congestion. An ideal network pacer spreads entire congestion window worth of traffic evenly over the round trip time period.}
}

\newglossaryentry{micro-bursting}
{
  name=micro-bursting,
  description={A performance limiting phenomenon in which packets arrive in short rapid bursts. These bursts may cause overflow in the receiving buffers and cause the receiver to discard incoming packets.}
}

\newglossaryentry{abstract-factory}
{
  name=abstract factory pattern,
  description={A design pattern encapsulating creation of families of related classes. A typical example are factories for creating GUI widgets, where different factories can create various types of widgets backed by a particular GUI rendering library.}
}

\newdefinedabbreviation{alpn}{ALPN}{Application-Layer Protocol Negotiation}{A TLS extension that allows the application layer to negotiate which application protocol will be in the connection. The negotiation is done as part of the TLS handshake and avoids additional round trips. ALPN is used, e.g., for HTTP version negotiation in HTTPS connections.}

\newdefinedabbreviation{sni}{SNI}{Server Name Indication}{A TLS extension that allows the client to specify the hostname it is attempting to connect to at the start of the handshake process. This allows using different security configurations for each different website hosted on the server.}

\newdefinedabbreviation{0rtt}{0-RTT}{Zero round trip time resumption}{TLS mode of operation allowing clients to send application data in the very first but possibly exposing the server to reply attacks.}

\newdefinedabbreviation{scid}{SCID}{Source Connection ID}{Connection ID used by the sender of the QUIC packet.}

\newdefinedabbreviation{dcid}{DCID}{Destination Connection ID}{Connection ID used by the receiver of the QUIC packet.}

\newdefinedabbreviation{aead}{AEAD}{Authenticated encryption with associated data}{Form of encryption which simultaneously assure the confidentiality and authenticity of data.}


% Title page and various mandatory informational pages
\begin{document}
\include{title}

%%% A page with automatically generated table of contents of the master thesis

\tableofcontents

%%% Each chapter is kept in a separate file
\chapter{Introduction}

The internet as we know it today heavily relies on the use of the HTTP protocol. Not only is it used
by web browsers to download and otherwise interact with websites, it also serves as a transport
medium for use with REST web APIs or technologies such as gRPC and GraphQL.\todo{References?}

Currently, the latest published version of the protocol is HTTP/2 \todo{Maybe some reference?} from
2015. However, there is already a draft version of its successor HTTP/3. HTTP/2 greatly improved the
efficiency and loading times of web pages by introducing improvements such as request multiplexing
over single TCP connection, header compression and the server-push feature. HTTP/3 improves on
HTTP/2 by replacing the TCP transport layer by new UDP-based protocol named QUIC.

There are several resources comparing the performance between HTTP/3 and HTTP/2. In 2015 the
experimental implementation by the Chromium team showed a 3\% improvement in mean page load time and
30\% less rebuffers when when watching videos
\todo{https://blog.chromium.org/2015/04/a-quic-update-on-googles-experimental.html}. Cloudfare
launched HTTP/3 support in April 2020 and has measured 12.4\% improvement in \textit{time to first
byte} metric. \todo{https://blog.cloudflare.com/http-3-vs-http-2/}. The main improvements of HTTP/3
+ QUIC over HTTP/2 + TCP can be summarized in following points:

\begin{itemize}
  \item \textit{Absence of head-of-line blocking} ---
    HTTP/2 uses multiplexing over a single TCP connection to execute multiple requests in parallel.
    However, due to TCP's guaranteed in-order delivery, losing a packet from one request delays
    execution of other requests, because the lost packet needs to be retransmitted first. QUIC
    implements stream multiplexing over UDP transport. QUIC delivers streams independently, so that in
    most cases packet loss affecting one stream does not affect other streams.

  \item \textit{Always encrypted} ---
    QUIC transport protocol is always encrypted using TLS 1.3 in order to provide secure-by-default
    transport.

  \item \textit{Faster connection establishment} ---
    QUIC interleaves TCP-like three-way-handshake with TLS handshake, requiring fewer round-trips to
    establish connections. QUIC also supports TLS 1.3 Zero Round Trip Time Resumption (0-RTT)
    \todo{Explanation: https://blog.cloudflare.com/introducing-0-rtt/}, allowing it to send user data
    with the very first packet.
    \todo{maybe include the image from one of the sources}

\end{itemize}

Although QUIC is being designed together with HTTP/3, it is well separated and suitable to be used
as a transport layer for other application-level protocols.

Even though the QUIC specification is not yet finalized, there are already several implementations
of the draft standards. These implementations are used to research and improve the protocol before
the final version of the specification.

\todo{Remove this paragraph?}
The purpose of this thesis is to implement support for the QUIC protocol for \.NET ecosystem.

\section{Support for QUIC in \.NET}

There are long-term plans to provide full QUIC and HTTP/3 support in \.NET.\ While HTTP/3 is going to
be used transparently by the \texttt{HttpClient}, QUIC implementation will provide public facing API
for programmers to use, most likely located in \texttt{System.Net.Quic} namespace.

There has already been some work done on HTTP/3 and QUIC support. However, once it became clear that
the final specification for HTTP/3 and QUIC will not be ready soon enough for it to be implemented
for the next \.NET 5 release, the QUIC protocol API has been made internal and further work has been
postponed.

The existing internal QUIC implementation relies on the C msquic library \todo{link}. Although
msquic is built for high-performance, depending on a native library from \.NET brings additional
development and maintenance cost. Furthermore, the performance gained from use of native code may be
outweighed by the cost of interop between native and managed code.

Implementing QUIC in managed C\# inside the \.NET runtime libraries has potential to bring following
benefits:

\begin{itemize}
  \item \textit{Experimentability/maintainability} ---
    Native dependencies require knowledge and experience in another programming language and making
    sure appropriate changes are applied to both sides of the interop boundary. Removing the need
    for interop boundary greatly simplifies making changes to the implementation.

  \item \textit{Portability} ---
    There have been issues in the past where native dependencies on Linux behave differently on
    different distributions. Single managed implementation prevents such issues.

  \item \textit{Performance} ---
    Native library interop code may require pinning of allocated objects, which may affect
    performance of the \.NET garbage collector. Also, interop with libraries with different
    threading model may lead to more performance problems. \todo{mention libcurl as an example? same
    for portability}
\end{itemize}

\section{Goals of this Thesis}

The goal of this thesis is to provide a partial C\# QUIC implementation which can be used as a
drop-in replacement for the current msquic based implementation present in the \.NET runtime
repository. Because this work has potential to be merged into the \.NET runtime codebase, the
implementation will be developed in a branch of \todo{fork of?} the official \.NET runtime repository \todo{link?}.

\todo{not sure if this paragraph belongs there.}
In \.NET, TLS is not implemented in managed code, but rather delegated to native libraries
like OpenSSL and SChannel. These implementations do not expose API needed by the QUIC protocol. In
order to avoid implementing entirety of TLS 1.3 as well as the QUIC protocol, this thesis will
select a suitable TLS implementation to be integrated in the implementation. \todo{do we expect to
  change the TLS backend in the future?}

\begin{enumerate}
  \item \textit{}
\end{enumerate}

\section*{Copy of the goals in the thesis assignment}

\todo{Remove this section}

The goal of this thesis is to provide a partial C\# implementation of the QUIC protocol for \.NET. The
implementation should replace the current experimental QUIC support based on the msquic native
library. Thesis implementation should target improving portability, maintainability, and simplifying
experiments with QUICs congestion control algorithms. While TLS is an integral part of the QUIC
protocol, the goal of this thesis is not to reimplement TLS but to choose an appropriate existing
implementation and incorporate it into the solution.

The resulting pilot implementation of the protocol should support at least reliably sending data
through a controlled testing environment between a demo client and server applications on at least
one platform supported by \.NET Core. However, the implementation should be extensible to cover the
entire QUIC specification. The student should closely cooperate with the Microsoft development team
to ensure eventual mergeability into the master \.NET Core branch.

\chapter{QUIC Protocol}\label{chap:02-quic}

This chapter is intended as a summary of the QUIC protocol specification and aims to provide
sufficient background for the design of our implementation. This text is based on version 29 of the
draft specification documents from June 2020, more specifically on the documents describing the core
transport protocol~\cite{draft-ietf-quic-transport}, TLS integration~\cite{draft-ietf-quic-tls}, and
congestion control mechanism~\cite{draft-ietf-quic-recovery}. Readers familiar with these documents
may skip this chapter.

We will start this chapter by first providing a high-level overview of QUIC and then providing a
more detailed description of the protocol's individual parts.

\section{Introduction}

QUIC protocol provides reliable and secure transport of multiple streams of data over a single
connection. QUIC provides following services:

\begin{itemize}

  \item Stream multiplexing

  \item Stream and connection-level flow control

  \item Low-latency connection establishment

  \item Connection migration and resilience to NAT rebinding

\end{itemize}

QUIC is implemented on top of UDP, which provides only unreliable transfer of datagrams. Therefore,
in addition to stream multiplexing, QUIC also implements loss recovery, congestion control,
transport security, and other features known from TCP or TLS protocols.

\subsection{Basic Concepts}

Throughout this chapter, and later when analysing the QUIC protocol specification, this text will
use several terms in a very specific meaning. This meaning is the same as defined in the
main specification document~\cite[Section~1.2]{draft-ietf-quic-transport}.

\todo{again, following list is almost verbatim from RFC}

\begin{description}

  \ditem{QUIC packet}  A complete processable unit of QUIC that can be
    encapsulated in a UDP datagram.  Multiple QUIC packets can be
    encapsulated in a single UDP datagram.

  \ditem{Out-of-order packet}  A packet that does not arrive directly after the packet that was
    sent before it.  A packet can arrive out of order if it is delayed, if earlier packets are
    lost or delayed, or if the sender intentionally skips a packet number.

   \ditem{Endpoint}  An entity that can participate in a QUIC connection by
      generating, receiving, and processing QUIC packets.  There are
      only two types of endpoint in QUIC: client and server.

   \ditem{Client}  The endpoint initiating a QUIC connection.

   \ditem{Server}  The endpoint accepting incoming QUIC connections.

  \ditem{Address}  When used without qualification, the tuple of IP version,
    IP address, UDP protocol, and UDP port number that represents one
    end of a network path.

  \ditem{Connection ID}  An opaque identifier that is used to identify a QUIC
    connection at an endpoint.  Each endpoint sets a value for its
    peer to include in packets sent towards the endpoint.

  \ditem{Stream}  A unidirectional or bidirectional channel of ordered bytes
    within a QUIC connection.  A QUIC connection can carry multiple
    simultaneous streams.

  \ditem{Application}  An entity that uses QUIC to send and receive data.

\end{description}

\subsection{Notational Conventions}

Numeric constants in hexadecimal always use network byte order (big-endian) and are prefixed by 0x.
In case of constants of non-conventional integer sizes, such as 2-bit integers used in packet
fields, the notation will omit unnecessary leading zeros. For example, \binary{FE} represents
decimal number 254 and \binary{2} represents decimal number 2.

This chapter contains graphical diagrams of several elements such as QUIC packets. In those
schematics, individual fields include length information as follows:

\begin{itemize}

  \litem{X (A)} Indicates that X is A bits long.

  \litem{X (i)} Indicates that X is encoded using the QUIC's variable-length encoding. This encoding is described in \autoref{sec:02-variable-length-encoding}.

  \litem{X (A..B)} Indicates that X can be any length from A to B, either of these limits can be omitted to indicate minimum size of zero, or no upper size limit, respectively.

  \litem{X (A) = C} Indicates that X has a fixed value C.

  \litem{X (A) ...} Indicates that X has is repeated zero or more times and each instance is A bits long.

\end{itemize}

\subsection{Wire Encoding}\label{sec:02-wire-encoding}

The process of encoding QUIC packets to be sent via the network is optimized for size. All values
sent over the network are encoded in big endian order. There are two nontrivial encodings used:
variable-length integer encoding and packet number encoding.

\subsubsection{Variable-Length Integer Encoding}\label{sec:02-variable-length-encoding}

Almost all numeric values stored in QUIC packets are encoded using a \textit{variable-length integer
  encoding}. This encoding uses the first byte's two most significant bits to encode whether the
value is encoded as 1, 2, 4, or 8-byte integer. This encoding supports only positive numbers. The
ranges available for individual encoding lengths are listed in \autoref{tab:02-quic-varint-length}.

\begin{myTable} {tab:02-quic-varint-length} {Variable-length integer encoding lengths}
  {ccr}
  {Most significant bits & Encoding length (B) & Maximum value}
  \binary{0}             & 1                   & \num{63}         \\
  \binary{1}             & 2                   & \num{16383}      \\
  \binary{2}             & 4                   & \num{1073741824} \\
  \binary{3}             & 8                   & $2^{62}-1$       \\
\end{myTable}

In order to optimize the size of QUIC packets, the implementations are encouraged to always choose
the shortest encoding necessary to represent the given number.

\subsubsection{Packet Number Encoding}\label{sec:02-packet-number-encoding}

\todo{check this for legibility}

Packets in QUIC are sequentially numbered and, therefore, packet numbers received by an endpoint
form a mostly ascending sequence with occasional reordering or gaps due to network unreliability.
This causes the packet number of a newly received packet to be close to that of the previously
received packet. QUIC leverages this fact by sending only the lower bytes of the packet number. The
number of bytes sent is variable between one to four bytes and is always chosen so that the receiver
can reconstruct the original packet number using the previously received packet numbers and the
bytes from the arriving packet.

When determining how many bytes of the packet number need to be included in the packet, the sender
uses the value of the highest acknowledged packet number. The encoding length must use enough bytes
to be able to represent at least twice the difference between the current packet number and the
highest acknowledged number.

As an example, consider following situation: as sender prepares to send packet \binary{4417}, its
highest acknowledged packet number is \binary{43a0}. The difference between the two numbers is
\binary{77} which is still small enough to require only the least significant byte (\binary{17}) of
the packet number to be sent in the packet. Suppose that by the time the packet arrives at the
receiver, its highest received packet number is \binary{43f5}. Based on the encoded packet number
length, the deconstructed packet number must be from the range \binary{4375}--\binary{4475}. The only
number from this range which matches the least significant byte sent in the packet is \binary{4417}
which is the correct packet number.

\section{QUIC Packets}

In QUIC connection, endpoints exchange \textit{QUIC packets} enclosed in UDP datagrams. A single UDP
datagram can contain multiple packets, although in most cases they contain only one. Packing
multiple QUIC packets into a single UDP datagram is called \textit{coalescing}. A single QUIC packet
cannot span multiple UDP datagrams. QUIC uses a total of six different packet types:

\todo{The order of the packet here now matches the subsubsections later, but I think different order here might be better}

\begin{enumerate}

  \litem{Version Negotiation} sent by the server when the client tries to establish a connection
  with QUIC version that is not supported by the server;

  \litem{Retry} optionally used by servers for Client Address Validation \todo{forward reference
  address validation} when establishing new connections;

  \litem{Initial} used to initiate a new connection and exchange the initial information;

  \litem{Handshake} used during the connection handshake;

  \litem{0-RTT} when TLS 1.3 0-RTT mode of operation is enabled, 0-RTT packets carry \textit{early
  data} --- application data that is sent before the TLS handshake is complete in order to reduce
  latency; and

  \litem{1-RTT} main packet type used throughout the lifetime of QUIC connection. 1-RTT and 0-RTT
  packets are the only packet types which carry application data.

\end{enumerate}

Version Negotiation and Retry packets are sent as one-time responses in special scenarios and are
not individually numbered. QUIC is designed in a way that lets the server send these packets without
maintaining any state for the connection. These packets are not used in an established connection.

Packets of the other types --- Initial, Handshake, 0-RTT and 1-RTT --- are individually numbered. The
packet numbers do not form a single sequence, as is the case with TCP packets. Instead, QUIC
organizes these packets into three separate \textit{packet number spaces}, and uses a separate
sequence of numbers for each packet number space. These packet number spaces are:

\begin{enumerate}

  \litem{Initial} Used for initiating new connections and exchanging initial cryptographic
  information. Contains only Initial packets.

  \litem{Handshake} Used during the connection handshake process. Contains only Handshake packets.

  \litem{Application} Used throughout the lifetime of the connection to transfer application data.
  This packet space contains both 0-RTT and 1-RTT packets.

\end{enumerate}

In addition to being numbered separately, packets from each packet number space are processed
completely independently of the other packet number spaces. For example, Initial packets can be
acknowledged only by another Initial packet. In their payload, the Initial, Handshake, 0-RTT and
1-RTT packets carry \textit{QUIC Frames} which are low level protocol messages carrying e.g.
acknowledgements and parts of streams sent by the application.

Each packet type has a header and a payload. QUIC tries to optimize the packet encoding in order to
maximize the amount of application data that can be sent in a single UDP datagram. Therefore, two
types of packet headers exist. 1-RTT packets use a \textit{short header} and all other packet types
use the \textit{long header}. The short header contains only a subset of fields present in the long
header and omits data which are relevant only during connection establishment, like protocol version
identifier.

The type of the header is specified by the most significant bit of the first byte in the packet ---
the \textit{Header Form} bit. The second most significant bit --- the \textit{Fixed Bit} --- is always
set to 1 in valid QUIC packets. Following sections describe the structure of the rest of the packet
for individual packet types.

\newcommand{\longFieldHeight}{1}

\subsection{Long Packet Header}

Long headers contain information that is necessary for the connection establishment. It includes
both the Source and Destination Connection IDs. It also contains \textit{Version} and \textit{Packet
  Type} fields. A few leftover bits of the header are reserved for type-specific information. The
structure of the long header is illustrated in \autoref{fig:02-long-header}.

\begin{myFigure}{fig:02-long-header}{Long packet header structure}

  \begin{bytefield}[bitwidth=2.5em]{8}
    \bitheader{0-7} \\
    \bitbox{1}{H=1} & \bitbox{1}{F=1} & \bitbox{2}{T (2)} & \bitbox{4}{Type Specific Bits (4)} \\
    \wordbox[tlrb]{\longFieldHeight}{Version (32)} \\
    \wordbox[tlrb]{1}{Destination Connection ID Length (8)} \\
    \wordbox[tlrb]{\longFieldHeight}{Destination Connection ID (..) } \\
    \wordbox[tlrb]{1}{Source Connection ID Length (8)} \\
    \wordbox[tlrb]{\longFieldHeight}{Source Connection ID (..) } \\
  \end{bytefield}

  H = Header Form, F = Fixed Bit, T = Packet Type

\end{myFigure}

The semantics of the individual fields of the long header is as follows:

\begin{description}

    \ditem{Header Form Bit} used to distinguish between packets with so called \textit{long header}
    and \textit{short header} format. In case of the long header packets. the form bit is set to \binary{1}.

    \ditem{Fixed Bit} A bit that is always set to \binary{1} in valid QUIC packets.

    \ditem{Packet Type} Discriminator of the packet type. the possible values are listed in \autoref{tab:02-packet-type}.

\begin{myTable}{tab:02-packet-type}{Values of the Packet Type field in long packet header}
  {cc}
  {Packet Type & Value}
  Initial      & \binary{0} \\
  0-RTT        & \binary{1} \\
  Handshake    & \binary{2} \\
  Retry        & \binary{3} \\
\end{myTable}

    \ditem{Version} Indicates which version of QUIC is in use. Location of this field will be
    the same across all QUIC versions However, the structure of the rest of the packet may be different
    in future versions of QUIC.

    \ditem{Destination Connection ID Length} Length of the Destination Connection ID field.

    \ditem{Destination Connection ID}  The Connection ID issued by the recipient of the packet.

    \ditem{Source Connection ID Length}  Length of the Source Connection ID field.

    \ditem{Source Connection ID} The Connection ID issued by the sender of the packet.

    \ditem{Reserved Bits} Bits reserved for use in the future QUIC versions. In the initial QUIC
    version, these bits must be set to 0.

\end{description}

\subsection{Version Negotiation Packet}\label{sec:02-version-negotiation-packet}

The Version Negotiation packet is sent by the server when it receives a long header packet
requesting a version of QUIC unsupported by the server. As an exception to other long header
packets, Version Negotiation packet is not discriminated by a specific value in the Packet Type
field in the header, but by a special value \binary{00000000} in the Version field.

After the header, the packet contains a list of supported versions, each one listed as 32-bit
integer in big-endian. Also, since future QUIC versions may allow larger Connection IDs than 20
bytes, a valid Version Negotiation packet can contain up to 255-byte Connection IDs. The structure
of the Version Negotiation packet is illustrated in \autoref{fig:02-version-negotiation-packet}.

\begin{myFigure}{fig:02-version-negotiation-packet}{Version Negotiation packet structure}

  \begin{bytefield}[bitwidth=2.5em]{8}
    \bitheader{0-7} \\
    \begin{rightwordgroup}{long \\ header}
      \bitbox{1}{H=1} & \colorbitbox{bytefieldunused}{7}{Unused} \\
      \wordbox[tlrb]{\longFieldHeight}{Version (32) = \binary{00000000}} \\
      \wordbox[tlrb]{1}{Destination Connection ID Length (8)} \\
      \wordbox[tlrb]{\longFieldHeight}{Destination Connection ID (0..2040) } \\
      \wordbox[tlrb]{1}{Source Connection ID Length (8)} \\
      \wordbox[tlrb]{\longFieldHeight}{Source Connection ID (0..2040) }
    \end{rightwordgroup} \\
    \wordbox[tlrb]{\longFieldHeight}{Supported Version 1 (32)} \\
    \wordbox[tlrb]{\longFieldHeight}{Supported Version 2 (32)} \\
    \wordbox[tlrb]{1}{...} \\
  \end{bytefield}

  H = Header Form

\end{myFigure}

The identifier of the initial QUIC version is \binary{00000001}. There are also special identifiers
for draft versions of the protocol valid only until the QUIC specification is finalized.

\subsection{Retry Packet}\label{sec:02-retry-packet}

Retry packets are distinguished by value \binary{3} in the \textit{Packet Type} field of the long
header. Retry packets are sent by a server as part of the optional \textit{Address Validation}
mechanism used to protect against traffic amplification attack. \todo{forward ref to address
  validation?} The structure of the packet is illustrated in \autoref{fig:02-retry-packet}.

\begin{myFigure}{fig:02-retry-packet}{Retry packet structure}

  \begin{bytefield}[bitwidth=2.5em]{8}
    \bitheader{0-7} \\
    \begin{rightwordgroup}{long \\ header}
      \bitbox{1}{H=1} & \bitbox{1}{F=1} & \bitbox{2}{T=\binary{3}} & \colorbitbox{bytefieldunused}{4}{Unused} \\
      \wordbox[tlrb]{\longFieldHeight}{Version (32)} \\
      \wordbox[tlrb]{1}{Destination Connection ID Length (8)} \\
      \wordbox[tlrb]{\longFieldHeight}{Destination Connection ID (0..160) } \\
      \wordbox[tlrb]{1}{Source Connection ID Length (8)} \\
      \wordbox[tlrb]{\longFieldHeight}{Source Connection ID (0..160) }
    \end{rightwordgroup} \\
    \wordbox[tlrb]{\longFieldHeight}{Retry Token (..)} \\
    \wordbox[tlrb]{\longFieldHeight}{Retry Integrity Tag (128)} \\
  \end{bytefield}

  H = Header Form, F = Fixed Bit, T = Packet Type

\end{myFigure}

The semantics of the fields specific to the Retry packet are:

\begin{description}

    \ditem{Retry Token} Contains an opaque token generated by the server. This token must be
    echoed back in an Initial packet during next connection attempt. The contents of the token must be
    difficult to guess by the attacker and verifyable in a stateless manner (i.e. without saving it in
    memory for future comparisons).

    \ditem{Retry Integrity Tag} Tag used to check the integrity of the packet. For the details on
    how the ingegrity tag is calculated, refer to the TLS integration
    specification~\cite[Section~5.8]{draft-ietf-quic-tls}.

\end{description}

\subsection{Initial, Handshake and 0-RTT Packets}

Initial, Handshake and 0-RTT packets are almost identical in structure. All three types use the
\textit{Type-Specific Bits} from the long header to store \textit{Reserved Bits} and the
\textit{Packet Number Length}. After the long packet header, these packets contain the
\textit{Length} field containing the packet length, \textit{Packet Number} field and the actual
payload consisting of QUIC frames. The only exception to this structure is the Initial packet, which
contains two additional fields just after the long header: \textit{Token Length} and \textit{Token},
which are used to carry the Retry Token from the Retry packet in case Address Validation is
requested by the server. The structure of these three packets is illustrated in
\autoref{fig:02-initial-handshake-0rtt-packets}.

\begin{myFigure}{fig:02-initial-handshake-0rtt-packets}{Structure of the Initial, Handshake, and 0-RTT packets}

  \begin{bytefield}[bitwidth=2.5em]{8}
    \bitheader{0-7} \\
    \begin{rightwordgroup}{long \\ header}
      \bitbox{1}{H=1} & \bitbox{1}{F=1} & \bitbox{2}{T (2)} & \bitbox{2}{R=\binary{0}} & \bitbox{2}{L (2)} \\
      \wordbox[tlrb]{\longFieldHeight}{Version (32)} \\
      \wordbox[tlrb]{1}{Destination Connection ID Length (8)} \\
      \wordbox[tlrb]{\longFieldHeight}{Destination Connection ID (0..160) } \\
      \wordbox[tlrb]{1}{Source Connection ID Length (8)} \\
      \wordbox[tlrb]{\longFieldHeight}{Source Connection ID (0..160) }
    \end{rightwordgroup} \\
    \begin{leftwordgroup}{Initial \\ only}
      \colorbitbox{bytefieldunused}{8}{Token Length (i)} \\
      \colorwordbox{bytefieldunused}{tlrb}{\longFieldHeight}{Token (..)}
    \end{leftwordgroup} \\
    \wordbox[tlrb]{1}{Length (i)} \\
    \wordbox[tlrb]{1}{Packet Number (8..32)} \\
    \wordbox[tlrb]{\longFieldHeight}{Packet Payload (..)} \\
    \wordbox[tlrb]{\longFieldHeight}{Integrity Tag (16)} \\
  \end{bytefield}

  H = Header Form, F = Fixed Bit, T = Packet Type,\\
  R = Reserved Bits, L = Packet Number Length

\end{myFigure}

The semantics of the new fields in these packets are:

\begin{description}

    \ditemWithComment{Token Length}{Initial only} Length of the Token field.

    \ditemWithComment{Token}{Initial only} Contains an opaque token if the server provided
    one in the Retry packet as part of Address Validation.

    \ditem{Reserved Bits} Bits reserved for use in the future QUIC versions. In the initial QUIC
    version, these bits must be set to 0.

    \ditem{Packet Number Length} The length of the encoding used for the packet number.

    \ditem{Length} The length of the remainder of the packet. This includes the Packet Number,
    the payload of the packet, and --- for encrypted packets --- the AEAD integrity tag.

    \ditem{Packet Number} The sequence number of this packet in the respective packet number space.
    This field uses a special encoding which is described in \autoref{sec:02-packet-number-encoding}.

    \ditem{Packet Payload} Serialized sequence of QUIC frames.

    \ditem{Integrity Tag} Opaque checksum produced by the AEAD cipher during packet encryption.
    Packet encryption is described in detail in \autoref{sec:02-packet-protection}.

\end{description}

\subsection{1-RTT Packet}

1-RTT packets are the only packet that uses the short header to make more space for application data
in the UDP datagram. Besides the Header Form, Fixed, Reserved Bits and Packet Number Length fields
which have the same meaning as in the long header, short header contains \textit{Spin Bit} and
\textit{Key Phase Bit} fields.

1-RTT packets can be sent only after the connection has been successfully established. This implies
that connection IDs used for the connection by both endpoints are already known, and there is no
need to repeat the Source Connection ID or specify the length of the Destination Connection ID.
1-RTT packets also lack the Length field. It is assumed that 1-RTT packets fill the rest of the UDP
datagram. Therefore, after the short header, only the Packet Number field and the QUIC frame payload
follows. The structure of the 1-RTT frame is illustrated in \autoref{fig:02-1rtt-packet}.

\begin{myFigure}{fig:02-1rtt-packet}{1-RTT packet structure}

  \begin{bytefield}[bitwidth=2.5em]{8}
    \bitheader{0-7} \\
    \begin{rightwordgroup}{short \\ header}
      \bitbox{1}{H=0} & \bitbox{1}{F=1} & \bitbox{1}{S} & \bitbox{2}{R=\binary{0}} & \bitbox{1}{K} & \bitbox{2}{L (2)} \\
      \wordbox[tlrb]{\longFieldHeight}{Destination Connection ID (0..160) }
    \end{rightwordgroup} \\
    \wordbox[tlrb]{1}{Packet Number (8..32)} \\
    \wordbox[tlrb]{\longFieldHeight}{Packet Payload (..)} \\
    \wordbox[tlrb]{\longFieldHeight}{Integrity Tag (16)} \\
  \end{bytefield}

  H = Header Form, F = Fixed Bit, S = Spin Bit, \\
  R = Reserved Bits, K = Key Phase Bit, L = Packet Number Length

\end{myFigure}

The semantics of the fields specific to 1-RTT packets are:

\begin{description}

    \ditem{Spin Bit} A bit used for an optional QUIC feature which allows on-path nodes to measure
    connection latency by observing changes in this bit. \todo{describe the SPIN bit feature?}

    \ditem{Key Phase Bit} A bit used to communicate that the keys used for the packet encryption
    need to be updated. Updating the keys is necessary after using them to encrypt certain number of
    packets to provide stronger protection. The mechanism of Key Update is described in
    \autoref{sec:02-key-update}.

\end{description}

\section{QUIC Frames}\label{sec:02-quic-frames}

QUIC frames are low-level QUIC protocol messages carried in the payload of Initial, Handshake, 0-RTT
and 1-RTT packets. Examples of these frames include, e.g., \ACK{} frames carrying acknowledgments
for received packets, \STREAM{} frames carrying the application data, and \CRYPTO{} frames carrying
data for the TLS handshake.

During the lifetime of the connection, all QUIC packets have to be acknowledged by sending ACK in
another packet in the same packet number space. However, not all packets have to be acknowledged
immediately, e.g., acknowledging packets containing only ACK would cause an endless flood of ACK
packets. Instead, the ACK frame is sent later together with a more urgent data. Frame types that
require immediate acknowledgments are called \textit{ack-eliciting frames}, and the packets with at
least one such frame are called \textit{ack-eliciting packets}.

Because packets in differnet packet number spaces offer different level of confidentiality, not all
frames can be sent in any packet type. For example, application data in \STREAM{} frames cannot be
sent in Initial and Handshake to avoid compromising security. \autoref{tab:02-frame-types} lists all
frame types, whether they are ack-eliciting and in which packets they can be sent.

\begin{myTable}[\small] {tab:02-frame-types} {QUIC frame types}
  {l@{\hskip -0.1in}ccccc}
  {                     &               & \multicolumn{4}{c}{Allowed in packet type} \\ \cmidrule(lr){3-6}
    Frame type          & Ack-eliciting & Initial & Handshake & 0-RTT & 1-RTT}
  \PADDING{}            &               & \checkmark{}     & \checkmark{}       & \checkmark{}   & \checkmark{}          \\
  \PING{}               & \checkmark{}           & \checkmark{}     & \checkmark{}       & \checkmark{}   & \checkmark{}          \\
  \ACK{}                &               & \checkmark{}     & \checkmark{}       &       & \checkmark{}          \\
  \RESETSTREAM{}        & \checkmark{}           &         &           & \checkmark{}   & \checkmark{}          \\
  \STOPSENDING{}        & \checkmark{}           &         &           & \checkmark{}   & \checkmark{}          \\
  \CRYPTO{}             & \checkmark{}           & \checkmark{}     & \checkmark{}       &       & \checkmark{}          \\
  \NEWTOKEN{}           & \checkmark{}           &         &           &       & \checkmark{}          \\
  \STREAM{}             & \checkmark{}           &         &           & \checkmark{}   & \checkmark{}          \\
  \MAXDATA{}            & \checkmark{}           &         &           & \checkmark{}   & \checkmark{}          \\
  \MAXSTREAMDATA{}      & \checkmark{}           &         &           & \checkmark{}   & \checkmark{}          \\
  \MAXSTREAMS{}         & \checkmark{}           &         &           & \checkmark{}   & \checkmark{}          \\
  \DATABLOCKED{}        & \checkmark{}           &         &           & \checkmark{}   & \checkmark{}          \\
  \STREAMDATABLOCKED{}  & \checkmark{}           &         &           & \checkmark{}   & \checkmark{}          \\
  \STREAMSBLOCKED{}     & \checkmark{}           &         &           & \checkmark{}   & \checkmark{}          \\
  \NEWCONNECTIONID{}    & \checkmark{}           &         &           & \checkmark{}   & \checkmark{}          \\
  \RETIRECONNECTIONID{} & \checkmark{}           &         &           & \checkmark{}   & \checkmark{}          \\
  \PATHCHALLENGE{}      & \checkmark{}           &         &           & \checkmark{}   & \checkmark{}          \\
  \PATHRESPONSE{}       & \checkmark{}           &         &           & \checkmark{}   & \checkmark{}          \\
  \CONNECTIONCLOSE{}    &               & \checkmark{}     & \checkmark{}       & \checkmark{}   & \checkmark{}          \\
  \HANDSHAKEDONE{}      & \checkmark{}           &         &           &       & \checkmark{}          \\
\end{myTable}

\todo{Short description of each packet type and what it is used for?}


\section{QUIC Connection}

\todo{expand this part} This section describes the details of connection lifetime.

\subsection{Connection ID}

Traditional network protocols use the combination of remote endpoint IP address and port to identify
the connection. QUIC, on the other hand, uses a dedicated Connection ID identifier. This in essence
enables migrating the connection to different network paths and interfaces, e.g. from cellular data
to a Wi-Fi network, because the connection identity does not depend on the peer's IP address.

Connection IDs are opaque byte sequences between 8 to 20 bytes in length. Each endpoint in a QUIC
connection independently selects Connection IDs it will use to identify the QUIC connection. These
Connection IDs are then used to populate the Source Connection ID and Destination Connection ID of
the QUIC packets.

The first pair of Connection IDs is exchanged during connection establishment. Additional Connection
IDs can be issued independently by each endpoint during the lifetime of the connection. These
additional Connection IDs are primarily used when migrating the connection to a new network path.
QUIC requires that the same Connection ID be used only on one network path. When migrating a
connection, the endpoints must start using different Connection ID for sending packets to prevent
correlation of the network traffic by external observers.

Connection IDs can be also retired by an endpoint. By retiring a Connection ID, the endpoint
communicates that it will no longer use the Connection ID, and the other endpoint should drop any
incoming packets that use it. Retiring a Connection ID serves as a request to the peer to issue a
new Connection ID as a replacement.

Alternatively, endpoints can use a zero-length Connection ID. In that case, the connection identity
is tied to the IP address and port of the remote endpoint. Using zero-length Connection ID saves
space in the sent datagrams, but imposes several limitations on the connection. For example, if an
endpoint uses a zero-length Connection ID, it cannot issue additional connection IDs and, therefore,
it cannot migrate a connection to a new local address.

\subsection{Matching Packets to Connections}

When a packet arrives to an endpoint, it needs to be associated with an existing connection, or ---
for servers --- potentially initiate a new connection. If packet contains a non-zero-length Connection
ID in the DCID field, the Connection ID is used to find an existing connection. If the packet uses a zero-length conneciton id, then the local and remote addresses determine the target connection.

\autoref{fig:02-connection-multiplexing} illustrates how server endpoint processes incoming packets
from client connections. Server maintains a table mapping between DCIDs corresponding connections,
and dispatches the incoming packets accordingly. If a packet cannot be associated with an existing
connection, then it may be a new connection attempt, otherwise the packet is discarded.

\begin{myFigure}{fig:02-connection-multiplexing}{Multiple QUIC connections on the same machine port}

\input{img/02-socket-multiplexing.pdf_tex}

\todo{Check if the image is more clear now}

\end{myFigure}

In case the packet cannot be associated with an existing connections, client endpoints simply ingore
the packet. Server, on the other hand, check the packet type and version of the protocol it
requires. For valid Initial packets with supported verions, server proceeds with the handshake as
described in the previous section. For invalid Initial packets, the server responds with an Initial
packet containing a \CONNECTIONCLOSE{} frame signaling the refusal of the connection.

In case the client packet requests an unsupported QUIC version, server replies with a Version
Negotiation packet (described in \autoref{sec:02-version-negotiation-packet}). After receiving such
a packet, the client can try again using one of the supported versions. Figure
\autoref{fig:02-version-negotiation-flow} illustrates such an exchange. In the figure, the client
tries to establish connection using an unsupported version denoted by X. Server then replies with
Version Negotiation packet listing versions 1,2, and 3. Client the tries to establish connection
again using the version 1.

\begin{myFigure}{fig:02-version-negotiation-flow}{Example Version Negotiation packet exchange}

\resizebox{\linewidth}{!}{\input{img/02-version-negotiation-flow.pdf_tex}}

\end{myFigure}

Lastly, the server may choose to send a Stateless Reset (see later in
\autoref{sec:02-stateless-reset}) for any other packet that cannot be matched to an existing
connection.

\subsection{QUIC Transport Parameters}\label{sec:02-transport-parameters}

QUIC has several options for parameterizing the connection. These are called \textit{Transport
  Parameters} and they essentially represent constraints for the other endpoint. Each endpoint sets
the transport parameters for the other endpoint. QUIC leverages the extensibility of the TLS
protocol to exchange transport parameters during the connection handshake.

\todo{should I mention them by name? e.g. I mention max\_idle\_timeout later}
Example transport parameters are:

\begin{itemize}

  \item Initial flow control limits

  \item Whether connection migration is allowed

  \item Maximum delay before sending an acknowledgment for ack-eliciting packets

  \item Maximum idle timeout before the connection is silently closed

  \item Maximum size of a UDP packet the endpoint is willing to receive

\end{itemize}

Many of the parameters have a default value, which is used when the given transport parameter is not
sent. Other transport parameters are mandatory. The exhaustive list of the transport parameters can be found in the core transport specification~\cite[Section~7.4]{draft-ietf-quic-transport}.

\subsection{Connection Establishment}

In order to initiate a new connection, clients sends an Initial packet to the server, which
initiates the handshake process. After the handshake completes, both peers have derived protection
keys necessary to send and receive 1-RTT packets with application data.

An example handshake flow is illustrated in \autoref{fig:02-example-handshake-flow}. The figure
shows client and server endpoints exchanging UDP datagrams with QUIC packets. For illustration, we
list the contents of the individual QUIC frames. The figure also lists contents of the \CRYPTO{}
frames sent. However, these are only for illustrative purposes because QUIC does not interpret the
\CRYPTO{} frames' contents.

\begin{myFigure} {fig:02-example-handshake-flow} {Example QUIC handshake flow}

\resizebox{\linewidth}{!}{\input{img/02-handshake-flow.pdf_tex}}

\end{myFigure}

In its first datagram, the client sends an Initial packet with initial information consisting of a
single \CRYPTO{}(CH) frame. This frame contains the \textit{Client Hello} TLS message.

The server replies with a datagram containing three coalesced QUIC packets. The first is an Initial
packet that acknowledges the client's Initial packet using the \ACK{}(0) frame, and a \CRYPTO{}(SH)
frame with a \textit{Server Hello} message. The contents of Client Hello and Server Hello messages
are used by TLS to derive Handshake protection keys. The server advances the TLS handshake further
by sending another \CRYPTO{} frame in the Handshake packet. The server also has enough information
to derive also the 1-RTT keys, so it can also start sending data on Stream 1 using the \STREAM{}(1,
``...'') frame in a 1-RTT packet.

Because the server's Initial packet contained an ack-eliciting \CRYPTO{}(SH) frame, the client needs
to acknowledge it by sending an Initial frame with an \ACK{}(0) frame. The client can posesses both
Client and Server Hello mesages and can derive the Handshake keys, enabling him to process the
server's Handshake packet. The Handshake packet needs to be separately acknowledged by another
\ACK{}(0) frame, and a reply from the TLS layer must be sent using another \CRYPTO{} frame. From the
information in the server's Handshake packet's \CRYPTO{} frame, the client derives 1-RTT protection
keys and processes the server's 1-RTT packet. In addition to sending an \ACK{}(0) frame for the
server's 1-RTT packet, the client can now start sending application data on stream 0 using a
\STREAM{}(0, ``...'') frame.

After server detects that Handshake has successfully completed, it sends a \HANDSHAKEDONE{} frame to
communicate the fact to the client.

\todo{retry? or do it as part of Security section?}

\subsection{Connection Termination}

QUIC connection can be terminated in three ways:

\begin{itemize}

  \item Idle timeout

  \item Immediate close

  \item Stateless reset

\end{itemize}

\subsubsection{Idle Timeout}\label{sec:02-idle-timeout}

If idle timeout is enabled, the endpoint silently closes the connection if it does not receive a
packet from the peer for a specified time period. Each peer may advertise a timeout period using the
\MaxIdleTimeout{} transport parameter, but the effective value is the minimum of the two
values.

In order to prevent timeouts, endpoints can send a \PING{} or another ack-eliciting frame to test
the liveness of the connection. However, sending \PING{} frames should be initiated by the
application protocol, not QUIC implementation, to prevent unnecessary network traffic.

\subsubsection{Immediate Close}

An immediate close can be initiated both by QUIC implementation and by the application protocol.
Either endpoint can initiate an immediate close by sending a \CONNECTIONCLOSE{} frame. By sending a
\CONNECTIONCLOSE{} frame, the peer enters a \textit{closing state}, in which it includes the
\CONNECTIONCLOSE{} frame in all packets it sends in reply to incoming packets. The closing state is
received also if the endpoint receives a \CONNECTIONCLOSE{} frame from the peer. In that case the
endpoint also echoes the \CONNECTIONCLOSE{} frame back to the other endpoint.

The closing state lasts until the endpoint is sure the other endpoint is also in the closing state ---
e.g., until it also receives \CONNECTIONCLOSE{} --- or until a \textit{closing timeout} expires. The
closing timeout period is calculated from the current estimate of the round-trip time of the
connection.

The \CONNECTIONCLOSE{} frame carries an error code and, optionally, a human-readable error phrase.
When initiated by QUIC, the error codes semantics defined by the QUIC specification. However, when
initiated by the application protocol, the semantics of all possible error code values sent in the
frame are defined by the application protocol itself. This implies that the implementations must
require an error code when closing the connection and do not provide a default error code value.

\subsubsection{Stateless Reset}\label{sec:02-stateless-reset}

A stateless reset is an option of last resort for an endpoint that does not have access to the state
of a connection, possibly resulting from a crash or outage. An endpoint may send a stateless reset
in response to receiving a packet that it cannot associate with an active connection.

In such cases, the endpoint sends a specially crafted packet that ends with a Stateless Reset Token
associated with his Connection ID\@. The Stateless Reset Token requirements are quite complex, and
we encourage readers to read the full specification if they are interested in details~\cite[Section~10.4]{draft-ietf-quic-transport}.

\subsection{Connection Migration}

A novel feature of QUIC is the ability to migrate connections to a different network path. This is
enabled by using a dedicated Connection ID, instead of using the endpoint's address to identify the
connection. In the initial QUIC versio, only client endpoints can migrate the connection to a
different address.

Prior to migrating a connection, the endpoint can optionally check the reachability of the other
endpoint using the process called \textit{path validation}. Path validation consists of exchanging
\textit{probing packets}, and is described in \autoref{sec:02-path-validation}.

If the path is validated, the endpoint can migrate the connection by starting to send packets from
the mew local address. After server receives the first non-probing packet from the new client's
address, it starts sending all future packets to that new address.

In order to prevent network traffic being correlated by the outside observers, QUIC requires each
Connection ID to be used for only one combination of local and remote endpoint addresses. Therefore,
when connection is migrated, both endpoints must switch to using different Connection IDs.

The connection migration process is illustrated in \autoref{fig:02-connection-migration-flow}.
Client and servers use Connection IDs \textbf{C1} and \textbf{S1}, respectively. Client first probes
the reachability of the server with a probing packet containing a \PATHCHALLENGE{} frame. Because
this packet is sent from a different local address, it uses a different Connection ID (\textbf{S2})
that was issued previously by the server. Likewise, server uses different Connection ID \textbf{C2}
to reply with a packet containing a \PATHRESPONSE{} frame, confirming the reachability from the
clients new local address. After that, client migrates the connection by sending all packets via the
new local address. After receiving next non-probing packet from the new address, server switches to
the new client's address as well. Connection IDs \textbf{C1} and \textbf{S1} are no longer used in
the rest of the connection and can be retired later.

\begin{myFigure}{fig:02-connection-migration-flow}{Flow of packets during connection migration}

\resizebox{\linewidth}{!}{\input{img/02-connection-migration-flow.pdf_tex}}

\end{myFigure}

\section{QUIC Streams}

QUIC can transport multiple streams of data in a single connection. Each stream is identified by its
\textit{Stream ID} and is processed independently of the other streams. Each QUIC packet can carry
data for one or more QUIC streams. \autoref{fig:02-stream-multiplexing} illustrates how QUIC may
pack two streams into frames such that those streams are transported in parallel.

\begin{myFigure}{fig:02-stream-multiplexing}{Stream multiplexing in QUIC}

  \input{img/02-stream-multiplexing.pdf_tex}

\end{myFigure}

\subsection{Streams Types}

Streams transported by QUIC can be either unidirectional or bidirectional. Unidirectional streams
carry data from the initiator to its peer, and bidirectional streams carry data in both directions.
Both client and server can open new streams. QUIC recognizes four types of streams, and the type of
the stream is encoded in the two least significant bits of the Stream ID. The stream types and their
associated encoding is summarized in \autoref{tab:02-stream-id-type-map}.

\begin{myTable} {tab:02-stream-id-type-map} {Mapping of QUIC Stream types to Stream ID bits}
  {cc}
  {Stream type                     & Least significant bits}
  Client-Initiated, Bidirectional  & \binary{0} \\
  Server-Initiated, Bidirectional  & \binary{1} \\
  Client-Initiated, Unidirectional & \binary{2} \\
  Server-Initiated, Unidirectional & \binary{3} \\
\end{myTable}

Bidirectional streams can be viewed as the combination of two unidirectional streams. After opening
the stream, each direction of the stream behaves as separate inbound and outbound unidirectional
streams. This implies that the sending and receiving part of the stream can be closed independently
of each other.

\subsection{Stream Lifetime}

Opening a stream does no require any special action. Streams are opened simply by sending the first
\STREAM{} frame carrying data for that stream. However, streams of a particular type can be opened
only in ascending order of their Stream IDs. For example, stream with ID 2 must be opened before
opening stream 6. Sending data for higher numbered streams will automatically open all lower
numbered streams of the same stream type.

Streams can be closed either gracefully or abortively. Graceful stream close is signaled by a
\textit{Fin} bit in the \STREAM{} frame, signaling that data carried by this packet are the last
part of the stream. The stream is gracefully closed once all stream data is confirmed received by
the other endpoint.

Abortive stream close is achieved using the \RESETSTREAM{} frame and, therefore, this action is also
reffered to as \textit{resetting the stream}. Streams can be reset only by the sender. Receiver can
request abortive stream close by sending a \STOPSENDING{} frame if it no longer wishes to receive
data on that stream. Both \RESETSTREAM{} and \STOPSENDING{} frames carry an application-level error
code, that is then read by the application using QUIC.

After the stream is closed, it's Stream ID cannot be reused. Instead, the next available Stream ID
must be used. QUIC uses the variable-length integer encoding (see
\autoref{sec:02-variable-length-encoding}) and, therefore, there is no shortage of available stream
IDs, which range from 0 to $2^{62}-1$.

\subsection{Required Operations on Streams}

The implementation should provide the following operations on sending part of the stream:

\begin{itemize}

  \item write data;

  \item end the stream by specifying that all data has been written; and

  \item terminate the stream with an application-level error code.

\end{itemize}

On receiving part of the stream, application protocols must to be able to:

\begin{itemize}

  \item read data; and

  \item abort reading with an application-level error code.

\end{itemize}

\section{Flow Control}

QUIC aims to be a general-purpose transport protocol to be used over a potentially untrusted
network, and as such, it needs to protect endpoints from malicious peers. To prevent malicious
senders from exhausting all available memory on the receiver by sending large amounts of data, or
fast senders from overwhelming slow receivers, QUIC employs a credit-based flow control scheme.

All QUIC streams are flow controlled both individually and together as an aggregate. Each endpoint
also controls the number of streams the other peer is allowed to open. All flow control limits are
communicated to the peer using three types of frames:

\begin{itemize}

\litem{\MAXSTREAMDATA{}} maximum offset of data sent on a stream with specified ID\@.

\litem{\MAXDATA{}} maximum sum of all offsets of data sent on all streams.

\litem{\MAXSTREAMS{}} maximum number of opened streams of a particular stream type.

\end{itemize}

Endpoints can only increase the flow control limits. Their peers must ignore any attempts to
decrease the flow control limits to ensure consistency when two consecutive QUIC packets with flow
control updates are reordered during transit. In case the peer violates any of the control flow
limits mentioned above, the QUIC implementation must immediately terminate the connection.

\section{Loss Detection and Recovery}

Because UDP is an unreliable transport protocol, QUIC must implement measures to recover from packet
loss. The packet loss detection is implemented similarly to TPC --- each endpoint sends
acknowledgments for each received packet. However, an essential difference from TCP is that QUIC
endpoints do not retransmit entire lost packets using the same packet number. Instead, each QUIC
frame in the original packet is updated and sent in some future packet, or dropped altogether if the
information contained in the frame is no longer relevant.

The acknowledgement is communicated using the \ACK{} frame which contains ranges of received packet
numbers that the endpoint acknowledges. A packet number can sent multiple times (in \ACK{} frames in
multiple packets) until the endpoint can determine that the other endpoint received the
acknowledgement.

For packets containing an \textit{ack-eliciting} frame (see \autoref{sec:02-quic-frames}),
acknowledgements must be sent before the delay specified by the \MaxAckDelay{} transport
parameter. For other packets, such as packets containing only \ACK{} or \PADDING{} frames, the
acknowledgement can be delayed until an ack-eliciting packet is received.

\autoref{fig:02-packet-loss-example} illustrates the loss detection and retransmission process in
action. When server receives the acknowledgment for packet 3, but not for packet 2 sent earlier, it
infers that packet 2 never reached the receiver and retransmits the payload \textbf{B} in packet 4.
The server does not have to retransmit the \ACK{} for packet 1 because it was already sent in packet
3 which was already acknowledged by the client. Therefore, packet 4 includes ACK only for packet 2.

\begin{myFigure}{fig:02-packet-loss-example}{Loss detection and retransmission example}

\input{img/02-retransmission-example.pdf_tex}

\end{myFigure}

The exact criteria for a packet to be deemed lost by a QUIC endpoint are following:

\begin{enumerate}

  \item The packet was not acknowledged.

  \item A packet which was sent later has been acknowledged.

  \item Either the packet has been sent long enough in the past, or its packet number is
sufficiently smaller than the highest acknowledged packet number.

\end{enumerate}

The third condition is dependent on the values of particular constants. The specification recommends
that the packet is considered lost if the gap between its packet number and the highest acknowledged
packet number is at least 3, or if it was sent for longer than $9/8$ times the estimate of the
current round-trip time.

The first condition mentioned above requires receipt of a packet from the peer to declare any packet
as lost. However, the loss of the last packet in a sequence could go undetected because there is no
following packet that can be acknowledged. To avoid possible deadlocks in such scenarios, QUIC
endpoint sends up to two ack-eliciting \textit{probe packets} if it does not receive a packet from a
peer in a period called \textit{probe timeout} (PTO for short). The PTO duration doubles each time
probe packets are sent until either a reply is received or the connection is terminated due to idle
timeout (see \autoref{sec:02-idle-timeout})

Similarly to TCP, QUIC also uses congestion control to manage the \textit{congestion window} --- the
amount of data that can be in-flight. The selection of the congestion control algorithm is left on
the implementation. As an example, The QUIC specification document for loss detection and recovery~\cite[Section~7]{draft-ietf-quic-recovery} describes a congestion control algorithm similar to TCP NewReno~\cite{rfc6582} algorithm.

\section{Security}

This section describes the mechanisms used to ensure security of the protocol. Besides encrypting
all packets sent throughout the lifetime of the connection, QUIC uses additional mechanisms to
ensure that the servers using the protocol are resistant to denial-of-service and other
cyber-attacks.

\subsection{TLS Integration}

Instead of designing a new handshake protocol, QUIC offloads the encryption negotiation to TLS
protocol (more precisely, TLS version 1.3). The low-level messages used in TLS, such as
\textit{Server Hello} and \textit{Client Hello} are transported by QUIC inside \CRYPTO{} frames and
passed to a TLS implementation on the other side. This way, QUIC is able to offer the same
confidentiality level as conventional TLS connections.

The TLS protocol is extensible, among the standard extensions which are also used by QUIC are
Application Level Protocol Negotiation~\cite{rfc7301} (ALPN for short) and Server Name
Indication~\cite{rfc6066} (SNI for short).

Application Level Protocol Negotiation is used when multiple application protocols are supported on
the same TCP or UDP port. ALPN allows the application layer to negotiate --- as part of the TLS
handshake --- which application protocol will be used in the established connection.

Server Name Indication is used by clients to specify the hostname of the server to which they are
connecting. When multiple websites are hosted on the same IP address and port, SNI allows the server
to customize the security configuration for each hosted website. During connection establishment,
proper security configuration, such as the SSL certificate to be used, can be selected based on the
value of the hostname.

A custom TLS extension is used by QUIC to exchange transport parameters during the handshake. The
transport parameters are described in \autoref{sec:02-transport-parameters}.

\subsection{Packet Protection}\label{sec:02-packet-protection}

All QUIC packets of type Initial, Handshake, 1-RTT and 0-RTT are encrypted to ensure integrity and
confidentiality of the transmitted data. Negotiation of the cryptographic ciphers and the encryption
keys is handled by the TLS handshake. This section focuses on how the negotiated encryption is
applied to QUIC packets.

\subsubsection{Authenticated Encryption with Associated Data}

QUIC uses type of encryption called Authenticated encryption with associated data~\cite{rfc5116}
(AEAD for short). This type of encryption ensures both confidentiality and authenticity of the
encrypted data. In addition to encrypted data --- called \textit{ciphertext} --- AEAD encryption
outputs also an authentication tag which is used to check the integrity of the payload during
decryption. The encryption can be authenticated by supplying additional authentication data (AAD for
short) which are not encrypted, but influence the authentication tag and, therefore, must be
supplied also during decryption. As an additional protection, AEAD also accepts a \textit{nonce}
parameter, which is an additional input which is supposed to be unique for each encrypted packet.

The programming interface for AEAD provides following operations:

\begin{itemize}

  \item Encryption:

  \begin{itemize}

    \item input: plaintext, key, nonce, AAD (optional)

    \item output: ciphertext, authentication tag

  \end{itemize}

  \item Decryption:

  \begin{itemize}

    \item input: ciphertext, key, nonce, authentication tag, AAD (if provided during encryption)

    \item output: plaintext or error if the authentication tag does not match the supplied
      ciphertext and AAD

  \end{itemize}

\end{itemize}

\subsubsection{Deriving QUIC Protection Keys}\label{sec:02-encryption-key-derivation}

QUIC derives multiple distinct keys from the secrets negotiated by TLS handshake. The derivation
uses the HKDF-Expand-Label function~\cite{rfc5869} to derive keys of desired length. The keys are
derived as follows:

\begin{equation*}
  \begin{split}
  key & = \operatorname{HKDF-Expand-Label}(secret, \texttt{"quic key"}, \texttt{""}, 32) \\
  iv  & = \operatorname{HKDF-Expand-Label}(secret, \texttt{"quic iv"}, \texttt{""}, 12)  \\
  hp  & = \operatorname{HKDF-Expand-Label}(secret, \texttt{"quic hp"}, \texttt{""}, 32)  \\
  \end{split}
\end{equation*}

The following subsections describe how the $key$, $iv$, and $hp$ keys are used in the actual process
of packet encryption.

\subsubsection{Packet Protection Procedure}

When actually encrypting the packets, QUIC first encrypts the packet payload using the AEAD cipher
negotiated by the TLS implementation. QUIC specification allows use of all AEAD ciphers allowed in
TLS 1.3. These ciphers are:

\begin{itemize}

  \item TLS\_AES\_128\_GCM\_SHA256

  \item TLS\_AES\_256\_GCM\_SHA384

  \item TLS\_CHACHA20\_POLY1305\_SHA256

  \item TLS\_AES\_128\_CCM\_SHA256

  \item TLS\_AES\_128\_CCM\_8\_SHA256

\end{itemize}


The parameters for AEAD for packet payload protection are:

\begin{itemize}

    \litem{key} the $key$ derived in previous section

    \litem{nonce} the $iv$ derived in previous section, last 8 bytes XORed with the packet number

    \litem{AAD} the contents of the packet header

    \litem{plaintext} the payload of the packet.

\end{itemize}

The produced authentication tag is appended to the protected payload and is included in the payload
size.

QUIC also protects the header of the packet. The header protection mechanism is more complex than
that of the payload. The process requires calculating the \textit{header protection mask}, which is
then applied using XOR to parts of the packet header. The details of the header protection mask
calculation depend on the negotiated cipher, but it always uses the $hp$ key and a 16-byte sample of
the encrypted payload.

The parts of the packets protected by the payload encryption and header
protection mechanisms are illustrated in \autoref{fig:02-protected-fields}.


\begin{myFigure}{fig:02-protected-fields}{Fields protected by payload encryption and header protection}

  \newcommand{\legendsquare}[1]{%
    \textcolor{#1}{\rule{0.7em}{0.7em}}%
  }

  \definecolor{hp}{rgb}{0.7, 1, 0.7}
  \definecolor{pp}{rgb}{0.7, 0.7, 1}

  \legendsquare{hp}~Header protection \hspace{1cm} \legendsquare{pp}~Payload encryption

  \vspace{5mm}

  \begin{mySubFigure}{\textwidth}{fig:02-protected-fields-long}{Long header packets}

    \begin{bytefield}[bitwidth=2.5em]{8}
      \bitheader{0-7} \\
      \begin{rightwordgroup}{long \\ header}
        \bitbox{1}{H=1} & \bitbox{1}{F=1} & \bitbox{2}{T (2)} & \colorbitbox{hp}{2}{R=\binary{0}} & \colorbitbox{hp}{2}{L (2)} \\
        \wordbox[tlrb]{\longFieldHeight}{Version (32)} \\
        \wordbox[tlrb]{1}{Destination Connection ID Length (8)} \\
        \wordbox[tlrb]{\longFieldHeight}{Destination Connection ID (0..160) } \\
        \wordbox[tlrb]{1}{Source Connection ID Length (8)} \\
        \wordbox[tlrb]{\longFieldHeight}{Source Connection ID (0..160) }
      \end{rightwordgroup} \\
      \begin{leftwordgroup}{Initial \\ only}
        \colorbitbox{bytefieldunused}{8}{Token Length (i)} \\
        \colorwordbox{bytefieldunused}{tlrb}{\longFieldHeight}{Token (..)}
      \end{leftwordgroup} \\
      \wordbox[tlrb]{1}{Length (i)} \\
      \colorwordbox{hp}{ltrb}{\longFieldHeight}{Packet Number (8..32)} \\
      \colorwordbox{pp}{tlrb}{\longFieldHeight}{Packet Payload (..)} \\
      \wordbox[tlrb]{\longFieldHeight}{Integrity Tag (16)} \\
    \end{bytefield}

  \end{mySubFigure}

  \vspace{5mm}

  \begin{mySubFigure}{\textwidth}{fig:02-protected-fields-short}{Short header packets}

    % this one has to be off-center to make the image aligned with the top one
    \hspace{1.8cm}\begin{bytefield}[bitwidth=2.5em]{8}
      \bitheader{0-7} \\
      \begin{rightwordgroup}{short \\ header}
        \bitbox{1}{H=0} & \bitbox{1}{F=1} & \bitbox{1}{S} & \colorbitbox{hp}{2}{R=\binary{0}} & \colorbitbox{hp}{1}{K} & \colorbitbox{hp}{2}{L (2)} \\
        \wordbox[tlrb]{\longFieldHeight}{Destination Connection ID (0..160) }
      \end{rightwordgroup} \\
      \colorwordbox{hp}{ltrb}{\longFieldHeight}{Packet Number (8..32)} \\
      \colorwordbox{pp}{tlrb}{\longFieldHeight}{Packet Payload (..)} \\
      \wordbox[tlrb]{\longFieldHeight}{Integrity Tag (16)} \\
    \end{bytefield}

  \end{mySubFigure}

\end{myFigure}

\subsubsection{Updating 1-RTT Protection Keys}\label{sec:02-key-update}

Some AEAD functions have limits for how many packets can be encrypted using the same keys. Analysis
in the specification document shows that updating the protection keys after sending at most $2^{23}$
packets provides the same confidentiality level as a standard TLS
connection~\cite[Appendix~B]{draft-ietf-quic-tls}.

The key update is signalled by flipping a \textit{Key Phase} bit in the 1-RTT packet header. And
after observing the change in the \textit{Key Phase}, an endpoint derives new secret and $key$ and
$iv$ keys using the HKDF-Expand-Label function:

\begin{equation*}
  \begin{split}
  secret_{n+1} & = \operatorname{HKDF-Expand-Label}(secret_{n}, \texttt{"quic ku"}, \texttt{""}, 32) \\
  key_{n+1} & = \operatorname{HKDF-Expand-Label}(secret_{n+1}, \texttt{"quic key"}, \texttt{""}, 32) \\
  iv_{n+1}  & = \operatorname{HKDF-Expand-Label}(secret_{n+1}, \texttt{"quic iv"}, \texttt{""}, 12)  \\
  \end{split}
\end{equation*}

Because the \textit{Key Phase} bit is protected by header protection, the $hp$ key must remain
unchanged to ensure that the other endpoint can correctly remove the header protection.

\subsection{Client Address Validation}\label{sec:02-address-validation}

After receiving the first Initial packet from a new client, server can request address validation by
sending a Retry packet (see \autoref{sec:02-retry-packet}). The Retry packet carries a token which
must be echoed back to the server in all following Initial packets. As long as an attacker cannot
generate a valid token for its address and the client is able to return that token, this exchange
proves to the server that the client has received the token.

\autoref{fig:02-client-address-validation-flow} illustrates the use of Retry packet to valididate
client address during connection establishment. By default, clients do not fill the Token field of
the Initial packet. The server rejects the initial connection attempt and issues a Retry Token
(denoted ``ABCD'' in the figure). Client then tries again with another Initial with the provided
token and the server proceeds with the usual handshake.

\begin{myFigure}{fig:02-client-address-validatin-flow}{Client address validation using a Retry packet.}

\resizebox{\linewidth}{!}{\input{img/02-client-address-validation-flow.pdf_tex}}

\end{myFigure}

\subsection{Path Validation}\label{sec:02-path-validation}

QUIC is layered on top of UDP which is a connection-less protocol. This means that changes in
endpoint address can also happen without active migration on the endpoint's part, e.g., because of
NAT rebinding along the network path.

The endpoint address may also be spoofed by the other endpoint in an attempt to perform
\textit{traffic amplification attack} against the spoofed address. For this reason, the amount of
data sent to the new endpoint address must be limited until path validation determines that the
address belongs to the endpoint. This path validation is performed by sending a \textit{probing
  packet} containing a \PATHCHALLENGE{} frame with an unpredictable token. The other endpoint must
echo the token back in a \PATHRESPONSE{} frame. After that the new address is considered validated
and the sending rate restrictions are lifted.

\chapter{Analysis}\label{chap:03-analysis}

In this chapter, we analyze the protocol and select the necessary subset needed for evaluation of a
pure \dotnet{} implementation. Afterwards, we design the architecture and outline the implementation
\todo{this should be expanded once the chapter is written}

\section{Protocol Features Selection}

The set of features to be implemented is guided by the goals we set in this thesis goals in the
introduction chapter. For the purpose of feature selection, we can rephrase subset of the original
goals into the following:

\todo{the goals below are a convenient rephrasing and can't be mapped 1-to-1 to the goals stated in
  the introduction, is that OK?}

\begin{enumerate}

  \item Support basic data transport, enabling some experimentation with QUIC as the transport for
    application layer protocols.

  \item Enable performance measurements that are representative of the potential full QUIC
    implementation.

\end{enumerate}

The first goal definitely requires full implementation of the multiplexed stream abstraction as
defined by the QUIC specification. It requires also implementing loss detection and recovery to
ensure that no data gets lost during the transport.

In order to get representative performance measurements, all performance affecting aspects of the
protocol should be implemented. The most important are packet protection and flow control because
they influence performance throughout the lifetime of the QUIC connection.

To summarize, the thesis should implement at least the following features:

\begin{itemize}

    \item Connection lifetime support (establishment, termination)

    \item Stream multiplexing

    \item Packet protection

    \item Loss detection and recovery

    \item Flow control

\end{itemize}

On the other hand, many QUIC features that react to one-time events can be disregarded, because
there either is no need for them in the evaluation environment, or they do not have relevant
performance or functional implications. In particular, implementation of the following features can
be avoided:

\begin{itemize}

    \item Connection migration, and therefore multiple Connection IDs support

    \item Complex (token-based) address validation

    \item Network path MTU detection

    \item Version negotiation

    \item 1-RTT key updates

    \item Advanced security measures (see Section 21 in transport specification~\cite{draft-ietf-quic-transport})

\end{itemize}

The rest of the features form a grey area which can be implemented fully, partially, or even not at
if convenient.

\section{Design Considerations}

Before we start the actual analysis, we will briefly outline the design principles used for the
actual design of the implementation and their rationale.

\subsection{Performance}

One of the key factors in the decision between managed \dotnet{} implementation of QUIC or using
external library like \libmsquic{} is performance. Therefore, the decisions made during the
implementation design should focus towards greater performance, possibly sacrificing maintainability
if the trade-off is justified.

As a general rule, the implementation will:

\begin{itemize}

    \litem{Avoid heap allocations} Although heap allocation is considered cheap, frequent
    allocations put pressure on the \dotnet{} garbage collector, leading to unnecessary stalls.
    Therefore, the amount of heap allocation on hot paths of the code executions should be
    minimized.

    \litem{Prefer return codes over exceptions} Throwing an exception is an expensive operation, and
    their frequent use would have negative impact on the performance.

\end{itemize}

\subsection{Testability}

The second design aspect we would focus on is testability of the implementation. Ideally, the design
would minimize the need for live debugging of the implementation. This is especially important
because stopping the implementation on a breakpoint will inevitably disrupt the connection, possibly
leading to termination because of a timeout.

The design intention is to allow writing automated tests that are able to inspect the packets sent
by the endpoint, and verify that they are consistent with the behavior defined by the QUIC
specification.

\subsection{Robustness?}

\todo{probably can be removed, can be mentioned when parsing incoming packets}

\subsection{Debugging?}

\todo{consider qlog and quic-log formats}

\section{Target \dotnet{} API}

This section describes API that will be used to expose the QUIC implementation to other developers.
As mentioned in the introduction chapter, the current design is a work-in-progress and is subject to
change in the future. All of the mentioned classes are located in the \namespace{System.Net.Quic}
namespace.

\subsection{QuicListener Class}

The \class{QuicListener} class is the equivalent of the \class{TcpListener}. Servers use this
class to accept incoming QUIC connections.

\begin{description}

    \ditemctor{QuicListener}{QuicListenerOptions} Constructor.

    \ditemproperty{IPEndPoint}{ListenEndPoint}{\propget} The IP endpoint being listened to for new connection. Read-only.

    \ditemmethod{ValueTask<QuicConnection>}{AcceptConnectionAsync}{CancellationToken}
    Accepts a new incoming QUIC Connection.

    \ditemmethod{void}{Start}{} Starts listening.

    \ditemmethod{void}{Close}{} Stops listening and closes the listener. Does not close already accepted connections.

\end{description}

\subsection{QuicListenerOptions Class}

The \class{QuicListenerOptions} class holds all configuration used to construct new \class{QuicListener}s.

\begin{description}

    \ditemproperty{SslServerAuthenticationOptions}{ServerAuthenticationOptions}{\propgetset}
        SSL related options like certificate selection/validation callbacks, and supported protocols for ALPN\@.

    \ditemproperty{string}{CertificateFilePath}{\propgetset} Path to the X509 certificate used by the server.

    \ditemproperty{string}{CertificateKeyPath}{\propgetset} Path to the private key for the used X509 certificate.

    \ditemproperty{string}{CertificateKeyPath}{\propgetset} Path to the private key for the used X509 certificate.

    \ditemproperty{IPEndPoint}{ListenEndPoint}{\propgetset} The IP endpoint to listen on.

    \ditemproperty{int}{ListenBacklog}{\propgetset} Number of connection to be held waiting for acceptance by the application. Upon reaching this limit, further connections will be refused.

    \ditemproperty{long}{MaxBidirectionalStreams}{\propgetset} Limit on the number of bidirectional streams the client can open in an accepted connection.

    \ditemproperty{long}{MaxUnidirectionalStreams}{\propgetset} Limit on the number of unidirectional streams the client can open in an accepted connection.

    \ditemproperty{TimeSpan}{IdleTimeout}{\propgetset} The period of inactivity after which the connection will be closed via idle timeout.

\end{description}

\subsection{QuicConnection Class}

The \QuicConnection{} class provides operation on the QUIC connection. Clients open new
connections by creating a new instance of this class and calling the \method{ConnectAsync} method.
Servers receive new connections using the \class{QuicListener} class.

\begin{description}

    \ditemctor{QuicConnection}{QuicClientConnectionOptions} Constructor. The newly created instance is not connected until the call to \method{ConnectAsync} method.

    \ditemproperty{bool}{Connected}{\propget} Indicates whether the \QuicConnection{} is connected (the handshake has completed).

    \ditemproperty{IPEndPoint}{LocalEndPoint}{\propget} Local IP endpoint of the connection.

    \ditemproperty{IPEndPoint}{RemoteEndPoint}{\propget} Remote IP endpoint of the connection.

    \ditemmethod{ValueTask}{ConnectAsync}{CancellationToken} Connects to the remote endpoint.

    \ditemmethod{QuicStream}{OpenUnidirectionalStream}{} Opens a new unidirectional stream. Throws a \class{QuicException} if the stream cannot be opened.

    \ditemmethod{QuicStream}{OpenBidirectionalStream}{} Opens a new bidirectional stream. Throws a \class{QuicException} if the stream cannot be opened.

    \ditemmethod{ValueTask<QuicStream>}{AcceptStreamAsync}{Cancellationtoken} Accepts an incoming stream.

    \ditemmethod{ValueTask}{CloseAsync}{long, CancellationToken} Closes the connection with the specified given error code and terminates all active streams.

    \ditemmethod{long}{GetRemoteAvailableUnidirectionalStreamCount}{} Gets the maximum number of unidirectional streams that this endpoint can open.

    \ditemmethod{long}{GetRemoteAvailableBidirectionalStreamCount}{} Gets the maximum number of bidirectional streams that this endpoint can open.

\end{description}

\subsection{QuicClientConnectionOptions}

The \class{QuicClientConnectionOptions} is used by clients to configure new QUIC conections.

\begin{description}

    \ditemproperty{SslClientAuthenticationOptions}{ClientAuthenticationOptions}{\propgetset} Client authentication options to use when establishing the connection.

    \ditemproperty{IPEndPoint}{LocalEndPoint}{\propgetset} The local IP endpoint that will be bound to.

    \ditemproperty{IPEndPoint}{RemoteEndPoint}{\propgetset} The IP endpoint to connect to.

    \ditemproperty{long}{MaxBidirectionalStreams}{\propgetset} Limit on the number of bidirectional streams the server can open.

    \ditemproperty{long}{MaxUnidirectionalStreams}{\propgetset} Limit on the number of unidirectional streams the server can open.

    \ditemproperty{TimeSpan}{IdleTimeout}{\propgetset} The period of inactivity after which the connection will be closed via idle timeout.

\end{description}

\subsection{QuicStream Class}

The \class{QuicStream} class represents a single stream in a QUIC connection. This class inherits
from the abstract \class{Stream} class. The list below mentions only members specific for the \class{QuicStream} class, members from the \class{Stream} base class are omitted.

The class contains methods for both inbound and outbound streams. Invoking read methods on
write-only stream will cause an exception to be thrown and vice versa.

\begin{description}

    \ditemproperty{long}{StreamId}{\propget} The Stream ID\@.

    \ditemmethod{void}{AbortRead}{long} Aborts the receiving part of the stream with the provided error code.

    \ditemmethod{void}{AbortWrite}{long} Aborts the sending part of the stream with the provided error code.

    \ditemmethodWithComment{ValueTask}{WriteAsync}{*, CancellationToken}{multiple overloads} Multiple overloads of this method offer writing from various types of buffers: \class{ReadOnlyMemory<byte>}, \class{ReadOnlySequence<byte>}, and \class{ReadOnlyMemory\allowbreak<ReadOnlyMemory<byte>>}. The returned task completes when the provided data have been buffered internally and the buffers can be reused for other purposes.

    \ditemmethodWithComment{ValueTask}{WriteAsync}{*, bool, CancellationToken}{multiple overloads} Like the methods above, but also allow specifying that the provided data are the last on the stream and that the stream should be gracefully closed.

    \ditemmethod{ValueTask}{ShutdownWriteCompleted}{CancellationToken} The returned task completes when the stream shutdown completes. Meaning that acknowledgement from the peer is received.

    \ditemmethod{ValueTask}{Shutdown}{} Closes the stream with error code 0. And blocks until shutdown completes.

\end{description}

\subsection{Exceptions}

The QUIC API can throw following exceptions:

\begin{description}

    \ditem{\ditemsrcsize\class{QuicException}} Base class for all thrown exceptions, used when a more specific exception is not available

    \ditem{\ditemsrcsize\class{QuicConnectionAbortedException}} Thrown when the connection is forcibly closed either by the transport or by the remote endpoint.

    \ditem{\ditemsrcsize\class{QuicStreamAbortedException}} Thrown when the stream was aborted by the remote endpoint.

    \ditem{\ditemsrcsize\class{QuicOperationAbortedException}} Thrown when the pending operation was aborted by the local endpoint.

\end{description}

\section{TLS Implementation}

TLS handshake forms an integral part of the QUIC connection establishment. Because correct TLS
implementation is crucial for ensuring the security of the resulting implementation, this thesis
should avoid implementing TLS by itself. Instead, it should reuse some existing and well-tested
implementation.

The novel way in which QUIC integrates with TLS requires specific functionality on the API of the
TLS implementation. Below is a nonexhaustive list of operations the TLS library's API must provide:

\begin{itemize}

  \item Current state of te handshake

  \item Temporary keys used to protect the handshake process

  \item Raw unencrypted TLS messages to be sent to the other endpoint

  \item Negotiated cipher

  \item Specifying protocols used for ALPN

  \item Specifying custom TLS extension to exchange QUIC transport parameters

\end{itemize}

\subsection{TLS 1.3 Support in \dotnet{}}

The \dotnet{} runtime libraries use different native libraries to provide TLS functionality on
different operating systems. On Windows, \libname{Secure Channel}~\cite{Schannel} (\libschannel{}
for short), which is part of the Windows operating system. On Linux and macOS systems, the
\libopenssl{} library~\cite{OpenSSLWeb} is used.

\begin{description}

    \ditem{\libname{Secure Channel}} The \libschannel{} versions present in the latest Windows 10
    builds support only TLS 1.2.  However, future near updates will implement also TLS 1.3. The
    \libschannel{} version with TLS 1.3 support can be obtained by installing an insider preview
    build of Windows 10.

    The \libmsquic{} library uses \libschannel{} library when compiled for Windows. Therefore, we
    assume that \libschannel{} exposes the necessary API for our managed QUIC implementation.

    \ditem{\libopenssl{}}
    None of the released versions of \libopenssl{} library expose necessary API for integration
    with QUIC, and there are no plans to include such API in the next \libopenssl{} 3.0.0
    release~\cite{OpensslBlogNoQuic}.

    However, developers at Akamai maintain a fork of \libopenssl{} which adds the QUIC-enabling
    API~\cite{AkamaiOpensslGithub}. This modified version of \libopenssl{} is used by \libmsquic{}
    (on Linux) and other QUIC libraries \todo{mention them by name + link?: quiche}. It is possible
    that the changes made by Akamai will be merged into \libopenssl{} for the 3.1.0 or later
    releases.

\end{description}

In conclusion, the APIs required for our QUIC implementation are currently only accessible in only
the preview versions of Windows 10. Relying solely on \libschannel{} for TLS 1.3 support in our
prototype implementation would severely impact cross-platform availability.

The \libopenssl{} library and supports all platforms supported by \dotnet{}, and could therefore be
used to implement TLS integration into QUIC in a platform-compatible manner. This solution, however,
has some drawbacks:


\begin{itemize}

  \item Modified \libopenssl{} binary must be distributed with \dotnet{} runtime.

  \item Only limited integration with X.509 certificates is possible because the
    \class{X509Certificate} class implementation will be using different binary ---
    \libname{CryptoAPI} on Windows, unmodified \libopenssl{} on Linux and macOS.

\end{itemize}

These drawbacks are acceptable for the prototype implementation, and will be eliminated once the
modified \libopenssl{} is replaced with \libschannel{} integration for Windows, and mainstream
version of \libopenssl{} once the support for QUIC is released.

This thesis will therefore integrate with the forked \libopenssl{} library from Akamai. Because the
library is written in C, it has to be integrated to the build process to be build together with the
native code of the \dotnet{} runtime.

\section{Threading model}

Implementation of the QUIC connection requires some background processing for handling timeouts and
reacting to incoming packets. Together with user threads, there will be more than one
thread\footnote{To reduce verbosity, this text will be using term thread to mean both code executing on a
  dedicated \class{Thread} or code running on the thread-pool using the \class{Task} type and
  \dotnet{} async/await model.} accessing the internal state of the connection.

\subsection{Public API}

The target API is designed to use the \class{ValueTask} type designed for efficient asynchronous
method implementation. Using this type, user code will start an asynchronous operation, that can be
completed by a background thread servicing the connection. This way the user code cannot block the
execution of the connection's background thread which could otherwise cause timeouts to be missed.

The implementation should make possible to use the \QuicConnection{} from multiple threads when
it makes sense. For example, it can be useful to be able to process individual QUIC streams in
parallel. Therefore, when designing the implementation we will assume following threading model for
the API:

\begin{itemize}

  \item Individual streams can be accessed concurrently. However, a single stream can be
    accessed only by one thread at a time.

  \item Accepting/opening new streams on a connection can be done concurrently from multiple threads.

  \item All other operations on \QuicConnection{} must be synchronized. This includes, e.g.,
    starting and aborting the connection.

\end{itemize}

\subsection{Internal Connection State}

To reduce the need for internal synchronization as much as possible in the \QuicConnection{}
implementation, our implementation will allow only a single thread to access the internal connection
state. The cost of synchronization of multi-threaded access would likely outweigh the benefits of
parallelism.

\section{High-Level Architecture}

Server implementation will need to be able to use a single IP address and port for conducting
multiple QUIC connections. Also, during connection migration on clients. The connection will need to
accept datagrams from multiple sockets at the same time. Although the prototype implementation will
omit the connection migration feature, the architecture should allow it's implementation in the
future.

Instead of interacting with the underlying \class{Socket} instance from the \QuicConnection{} class
directly, our implementation will introduce a new \QuicSocketContext{} class to perform the socket
I/O and background processing. \autoref{fig:03-architecture} shows how the \QuicSocketContext{}
class fits into the architecture.

\begin{myFigure}{fig:03-architecture}{High-level background processing architecture.}

  \input{img/03-architecture.pdf_tex}

\end{myFigure}

If necessary, the implementation of \QuicSocketContext{} class can be different for server and
client endpoints. For example, server implementation may need to utilize multiple background
processing threads to be able to handle large number of connections without degrading performance.

\todo{mention future support for connection migration?}

Separation of I/O from \QuicConnection{} outlined in previous subsection also makes the
implementation more testable. The interface between the \QuicSocketContext{} and \QuicConnection{}
will consists of passing buffers with received UDP datagrams and datagrams to be sent.

Therefore, automated tests can inspect the contents of the sent datagrams and inject specially
crafted datagrams to provoke behavior to be tested. The testing code can also manually step time to
make the tests deterministic.

\section{QuicSocketContext Implementation}

The connection background processing will need to process the following events:

\begin{itemize}

  \item An incoming UDP datagram has arrived on the socket.

  \item User has performed some action that potentially requires sending UDP datagrams to the other
    endpoint. This includes e.g. writing data to a stream or aborting a stream.

  \item A timeout has expired, this includes e.g. timeout for loss detection.

\end{itemize}

\subsection{Incoming packet handling}

\subsection{User request handling}
\subsection{Timeout events}


\section{Epoch handling}
\subsection{Separation of epochs in protocol}

\section{Packet Protection}

As described in \autoref{sec:packet-protection}, the packet encryption consists of two phases ---
payload protection and header protection. The combined process requires following inputs:

\begin{itemize}

  \item Keys derived from the protection secrets (as explained in
  \autoref{sec:02-encryption-key-derivation})

  \item Negotiated cipher

  \item Packet number (for encryption), or expected packet number (for decryption)

  \item The QUIC packet to encrypt or decrypt

\end{itemize}

The cipher is negotiated once and cannot be changed during the lifetime of the connection. Keys for
the protection can be changed only for the 1-RTT packets using the process of \textit{key update}
(see \autoref{sec:02-key-update}), which can be expected to be relatively infrequent. The rest of
the inputs change with every packet. Therefore, our implementation encapsulates the packet
protection implementation in a \class{CryptoSeal} class, which does not depend on the rest of the
QUIC implementation.

The process of receiving packets requires intermediate validation of the header fields. The
individual steps --- header protection and payload protection --- must be therefore exposed separately.
Also, the protection should happen in-place to avoid unnecessary allocations and copying of the
packets.

An important consideration needs to be made for performing the actual encryption/decryption once all
inputs to the AEAD have been gathered. \dotnet{} does not contain an implementation of the CHACHA
family of ciphers. Because the implementation can work without it by instructing TLS library to not
allow its negotiation, the prototype implementation of QUIC will not support this kind of cipher.
The other ciphers are based on the AES family of ciphers, which are supported by \dotnet{}, but
individual classes implementing these ciphers do not share a common interface. Our implementation
therefore wraps the concrete AES implementations in classes derived from an abstract
\class{CryptoSealAlgorithm} class defining a common interface required by \CryptoSeal{}.
\autoref{fig:03-crypto-seal} illustrates how the \CryptoSeal{} and \class{CryptoSealAlgorithm}
classes are connected.

\begin{myFigure}{fig:03-crypto-seal}{Relationship between CryptoSeal and CryptoSealAlgorithm classes}

  \resizebox{\linewidth}{!}{\input{img/03-crypto-seal.pdf_tex}}

\end{myFigure}

Key update can be implemented by replacing the existing instance of the
\CryptoSeal{} class with the new one with the updated protection secret.

\section{Packet Serialization/Deserialization}

Special care needs to be taken when implementing serialization of the QUIC packets to the wire
format because inefficient implementation can have very negative impact on the overall performance.
Because significant amount of time is spent parsing and processing individual QUIC frames. The
packet and frame representation should be carefully designed to avoid allocations on the hot path.

The incoming packets can contain arbitrary encoding errors which should be handled without the use
of exceptions for above-mentioned performance reasons. Therefore, all possible errors encountered
during packet deserialization must be handled gracefully. This includes values being outside of
range of allowed values and incomplete or damaged packets.

\subsection{QUIC Packet and Frame Representation}

Some packets contain a large number of fields and, therefore, passing them around as individual
variables would make the implementation unnecessarily verbose. Logically, the QUIC frames form
coherent messages that should be represented by individual \dotnet{} types.

Representing QUIC frames as instances of individual classes would introduce a lot of allocation for
every received QUIC packet. This thesis therefore models QUIC frames headers of QUIC packets as value types.

Payload of some frames may consist of large blocks of memory. Examples include \STREAM{} and
\CRYPTO{} frames, which can fill the entire payload of the QUIC packet. Duplicating this block of
memory would be another unnecessary memory allocation. Recent versions of \dotnet{} \todo{which
  one?} introduced the \class{Span<T>} type which can be used to efficiently reference arbitrary
memory block, including memory allocated on stack using the \keyword{stackalloc} keyword.

However, the \class{Span<T>} type is a \keyword{ref struct} --- a special kind of value type which can
be stored only on program stack, or inside other \keyword{ref struct}s. One of the consequences of
using \keyword{ref struct}s to represent QUIC frames is that it is impossible to create a collection
containing all frames from the QUIC packet. All frames must be processed right after they are
parsed. This limitation would greatly complicate writing tests for the \class{QuicConnection}
behavior. Fortunately, this limitation can be overcome by also creating reference type versions of
the frame types to be used in the testing framework. Even though this introduces duplication to the
codebase, we believe that having allocation-free packet serialization justifies this decision.

\subsection{QuicReader and QuicWriter}

Both serialization and deserialization requires maintaining the current position in the buffer
containing the memory to deserialized. To simplify this, our implementation introduces \QuicReader{}
and \QuicWriter{} classes as a primary means to reading and writing QUIC primitives to memory.
Although \QuicReader{} and \QuicWriter{} are reference types allocated on the heap, their instances
are expected to be cached by the class that uses them.

\todo{I feel big-endianity should be mentioned, but I can't find a good place for it} Like many
other networking protocols, QUIC uses \textit{network order} of bytes, also called
\textit{big-endian} order.

\subsection{QUIC Primitives Encoding}

There are two primitives used in QUIC packet encoding that require nontrivial encoding. The first
one being variable-length integer encoding, the other one being packet number encoding. Both of
these encodings are described in \autoref{sec:wire-encoding}.

\todo{this feels like there should be more in this subsection}

\section{Stream Implementation}

QUIC is a transport protocol and therefore its entire purpose is transferring streams of data. Since
it is likely to be the hot path of the implementation, the internal handling of streams must be
efficient and avoid unnecessary copying of blocks of stream data.

QUIC recognizes four types of streams. These streams can have sending part, receiving part, or both.
The fact which endpoint initiated the stream controls only which flow control limits apply to that
stream. Otherwise all streams are handled equally.

\subsection{Receiving Part of the Stream}

The implementation of the receiving part of the stream must be able to buffer the received data in
case the QUIC packets send by the peer were lost or reordered. It also needs to track the amount of
buffered memory and how much data wad delivered to the application in order to correctly update flow
control limits for the peer.

\subsubsection{Stream Data Buffering}

Ideally, the stream implementation would be structured in a way that the stream data were copied
straight from the decrypted packet to the memory provided by the application. This would be possible
if the API used an event-based model with callbacks for incoming data. The API, as currently
designed, utilizes method-based model. If the application does not call the \method{ReadAsync}
method, there is no buffer to deliver to.

In order to avoid additional copies, the buffer holding the stream data cannot be reused to receive
another QUIC packet, because the part of the buffer with the \STREAM{} frame must be kept unmodified
until delivered to the application. Instead, different buffer must be obtained for receiving the
next packet. For such a solution to be memory efficient, it would require a complex memory pooling
scheme to avoid needlessly, essentially implementing a custom memory allocator over a block of
memory. The performance improvement of such a solution may be greatly outweighed by the development
and maintenance costs. Therefore, until the cost of such a solution is justified by performance
measurements, our implementation will use a two-copy approach: the first copy from the packet to
internal stream buffers, the second copy from said buffers to the destination memory provided by the
application.

\subsubsection{Packet Reordering and Data Deduplication}

Unreliability of the UDP protocol can cause QUIC packets to be reordered or lost. Because of that,
parts of the stream may be received multiple times, and the contents of \STREAM{} frames can
arbitrarily overlap. It is therefore necessary to keep track which parts of the stream have been
received and buffer only the newly received segments of the stream.



The \STREAM{} frames can

\subsubsection{Reading data by user}
\subsubsection{Flow control consideration}

\subsection{Outbound part}

\subsubsection{Buffering}
\subsubsection{Acknowledgement, retransmission}
\subsubsection{Flow control consideration}

\subsection{Abort/Dispose model for streams}
\subsection{Managing streams within connection}
\subsection{Queue of streams to be updated}


\section{Flow control}

\subsection{Connection level}
\subsection{Stream level}
\subsection{MaxStreams and API}


\section{Recovery/Robustness}

\subsection{Detecting lost packets}
\subsection{Detecting duplicate packets}
\subsection{Tail loss probe}
\subsection{Congestion window}
\subsection{Interface for Congestion control algorithm}

\section{Integration:}

\todo{should be after TLS because of OpenSSL dependency}

\subsection{Build as part of dotnet runtime}
\subsection{Standalone build for easier evaluation}

\todo{Maybe remove altogether and move this to documentation}

\chapter{Developer Documentation}

The Managed QUIC implementation developed in this thesis is contained in the
\filename{System.Net.Quic} project in a fork of the official \dotnet{} runtime repository. The
source code is attached in \todo{path to the src in attachments.}

The source code for the \filename{System.Net.Quic} project is located inside the
\filename{src/dotnet-runtime/src/libraries/System.Net.Quic/} directory. The directory listing in
\autoref{lst:05-system-net-quic-structure} shows the structure of the project directory. The figure
also emphasises the \filename{Managed} and \filename{UnitTests} directories which contain the code
developed as part of this thesis. These directories are the main focus of this chapter.

\newcommand{\mydtcomment}[1]{\DTcomment{#1}}

% use smaller width to make the tree more compact
\DTsetlength{0.2em}{0.6em}{0.2em}{0.4pt}{1.6pt}
\renewcommand{\DTstylecomment}[1]{{\footnotesize\rmfamily #1}}
\renewcommand{\DTstyle}[1]{{\footnotesize\texttt{#1}}}
\definecolor{dtemphcolor}{rgb}{0,0,0.75}
\newcommand{\dtemph}[1]{\textcolor{dtemphcolor}{\emph{#1}}}

\begin{myListing}[Directory structure of the System.Net.Quic project.]{lst:05-system-net-quic-structure}{Directory structure of the System.Net.Quic project. Emphasised items contain the implementation developed in this thesis.}
\dirtree{%
  .1 {src/dotnet-runtime/src/libraries/System.Net.Quic/}.
  .2 {ref}\mydtcomment{Refererence assembly code}.
  .2 {src}\mydtcomment{Main library source code}.
  .3 {Resources}.
  .4 {Strings.resx}\mydtcomment{Definition of localizable strings like exception messages.}.
  .3 {System}.
  .4 {Net}.
  .5 {Quic}.
  .6 {Implementations}\mydtcomment{Root directory for all QUIC implementations}.
  .7 {\dtemph{Managed}}\mydtcomment{\dtemph{This thesis' implementation sources}}.
  .7 {MsQuic}\mydtcomment{\libmsquic{}-based implementation sources}.
  .7 {Mock}\mydtcomment{Mock implementation used only in tests}.
  .6 {Interop}\mydtcomment{Imports from native libraries}.
  .3 {System.Net.Quic.csproj}.
  .2 {tests}\mydtcomment{Library tests source code}.
  .3 {certs}\mydtcomment{X.509 certificates used in tests}.
  .4 {cert.crt}\mydtcomment{Public certificate file}.
  .4 {cert.key}\mydtcomment{Private key file}.
  .3 {FunctionalTests}\mydtcomment{Tests against the public API}.
  .4 {System.Net.Quic.Functional.Tests.csproj}.
  .3 {\dtemph{UnitTests}}\mydtcomment{\dtemph{Managed implementation unit tests}}.
  .4 {\dtemph{System.Net.Quic.Unit.Tests.csproj}}.
  .2 {Directory.Build.props}.
  .2 {System.Net.Quic.sln}.
}
\end{myListing}

\todo{sections for: development (environment setup, setup for debugging)? running tests?}

\section{QUIC Implementation Providers}

The \filename{System.Net.Quic} project contains multiple implementations of the QUIC protocol. This
is not achieved via polymorphism of the QUIC API classes, but rather using indirection to
\textit{implementation providers}. Each \QuicListener{}, \QuicConnection{}, and \QuicStream{}
instance contains a reference to a \QuicListenerProvider{}, \QuicConnectionProvider{}, or
\QuicStreamProvider{} instance, respectively. Implementations of each providers are provided by each
QUIC implementation in the \filename{System.Net.Quic} library. Figure
\autoref{fig:05-quic-impl-providers} illustrates this indirection layer using a class diagram.

\begin{myFigure}{fig:05-quic-impl-providers}{Implementation providers for the QUIC API classes}

  \todo{class diagram showing inheritance + grouping of classes + access modifiers}

\end{myFigure}

Creation of new \QuicListener{} and \QuicConnection{} instances is implemented using the
\gls{abstract-factory}~\cite{wiki:abstract-factory-pattern}. There are multiple implementations of
\QuicImplementationProvider{} class, one for each QUIC implementation. Instances of
\QuicImplementationProvider{} class are available as static properties of the
\class{QuicImplementationProviders} static class. There are five such providers:

\begin{itemize}

  \litem[]{\texttt{Managed}} Managed implementation with TLS backed by \libopenssl{} fork with QUIC enabling API.

  \litem[]{\texttt{ManagedMockTls}} Managed implementation with mock TLS, which does not depend on external TLS implementation, but cannot interoperate with other QUIC implementations.

  \litem[]{\texttt{MsQuic}} QUIC implementation backed by \libmsquic{} native library.

  \litem[]{\texttt{Mock}} A mock QUIC implementation for use in tests.

  \litem[]{\texttt{Default}} The default implementation. Same as \texttt{Managed}, but can be redirected to other implementations by setting \texttt{DOTNETQUIC_PROVIDER} environment variable to the desired provider name.

\end{itemize}

The \QuicListener{} and \QuicConnection{} classes have a constructor overload which accepts an
instance of the \QuicImplementationProvider{} to be used. This way, the QUIC implementation can be
selected during runtime. This also allows reusing a suite of functional tests for all
implementations by simply changing the implementation provider.

\section{Managed QUIC Implementation}

The source code for the managed implementation developed in this thesis is located under the
\filename{System.Net.Quic/src/System/Net/Quic/Implementations/Managed/} subdirectory.
\autoref{lst:05-managed-quic-structure} outlines the directory structure of the implementation.

\begin{myListing}{lst:05-managed-quic-structure}{Directory structure of the managed QUIC implementation}
\dirtree{%
  .1 {System.Net.Quic/src/System/Net/Quic/Implementations/Managed}.
  .2 {Internal}\mydtcomment{Internal code of the implementation}.
  .3 {Crypto}\mydtcomment{Cryptographic facilities}.
  .3 {Frames}\mydtcomment{Definition of QUIC frames}.
  .3 {Headers}\mydtcomment{Definition of QUIC packet headers}.
  .3 {Packets}\mydtcomment{QUIC packet number spaces handling}.
  .3 {Parsing}\mydtcomment{Parsing of QUIC primitives}.
  .3 {Recovery}\mydtcomment{Loss detection and recovery}.
  .3 {Sockets}\mydtcomment{Servicing socket IO}.
  .3 {Streams}\mydtcomment{Stream buffering}.
  .3 {Tls}\mydtcomment{TLS integration}.
  .4 {Mock}\mydtcomment{Mock TLS implementation}.
  .4 {OpenSsl}\mydtcomment{OpenSSL TLS integration}.
  .3 {Tracing}\mydtcomment{Tracing and logging facilities}.
  .2 {ManagedQuicConnection.cs}\mydtcomment{Implementation of public API}.
  .2 {ManagedQuicConnection.Frames.cs}\mydtcomment{Processing of QUIC frames}.
  .2 {ManagedQuicConnection.Packets.cs}\mydtcomment{Processing of QUIC packets}.
  .2 {ManagedQuicConnection.Recovery.cs}\mydtcomment{Handling packet loss}.
  .2 {ManagedQuicConnection.Stream.cs}\mydtcomment{Stream management}.
  .2 {ManagedQuicImplementationProvider.cs}\mydtcomment{Abstract factory}.
  .2 {ManagedQuicListener.cs}\mydtcomment{Implementation of public API}.
  .2 {ManagedQuicStream.cs}\mydtcomment{Implementation of public API}.
}
\end{myListing}

The implementation is exposed using the \ManagedQuicListener{}, \ManagedQuicConnection{},
\ManagedQuicStream{} implementation provider classes and the \ManagedQuicImplementationProvider{}
factory. The source code for these classes can be found in the root directory of the
implementation.

\section{Tests Implementation}

\chapter{User Documentation}\label{chap:06-user-docs}

This section provides guidance on how to obtain the build of the \dotnet{} runtime with managed QUIC
implementation, how to install it, and how to use the code to develop other applications.

\section{Getting Started}

This thesis provides a branch of the \dotnet{} runtime codebase with managed QUIC implementation.
Since our branch contains changes only in the \SystemNetQuicDll{}, the easiest way of composing a
fully working \dotnet{} distribution is obtaining a full SDK installation of the latest master
development version of \dotnet{}~6 and replacing the \SystemNetQuicDll{}. This section explains how
to do this without affecting other \dotnet{} SDK installations present on the machine.

The managed implementation depends on a particular \libopenssl{} version to interoperate with other
QUIC implementations. This section also explains how to deploy a locally built \libopenssl{} to be
automatically used by user code.

The complete setup process consists of the following steps, which we explained in greater detail in
the subsequent subsections:

\begin{enumerate}

  \item Build the \SystemNetQuicDll{} library from our branch of the \dotnet{} runtime sources.

  \item Compose a new local \dotnet{}~6 SDK installation with locally built \SystemNetQuicDll{}.

  \item \textit{(optional)} Compile a QUIC-supporting \libopenssl{} from source and deploy it.

  \item Configure the development environment to use the new \dotnet{} installation when compiling
and running user applications.

\end{enumerate}

\subsection{Building the System.Net.Quic Library from Source}\label{sec:06-build-runtime}

The source code for the \dotnet{} runtime with managed QUIC implementation is part of this thesis'
attachments in the \filename{src/dotnet-runtime/} directory. The latest version of the source code
can also be found on the thesis author's GitHub~\cite{githubRzikmRuntimelab}. For the remainder of
this section, all paths will be relative to the \dotnet{} runtime repository directory.

The \dotnet{} runtime repository contains a descriptive guide on how to build the sources. The
necessary prerequisites are listed in files inside the \filename{docs/workflow/requirements/}
directory, separately for each operating system. Once all necessary prerequisities are installed,
the entire \dotnet{} runtime can be built using the \filename{build.cmd} batch file (on Windows) or
\filename{build.sh} script (on Linux). The arguments are the same for both operating systems.

\begin{myVerbatim}
> ./build.cmd -subset clr+libs -configuration release
\end{myVerbatim}

The above command will build the Common Language Runtime (CLR) and all libraries in Release
configuration. The artifacts are available in the \filename{artifacts/bin/System.Net.Quic}
directory. The important artifacts from this directory are:

\begin{itemize}

  \item \filename{ref/net6.0-Release/System.Net.Quic.dll}: The so-called reference
assembly~\cite{ReferenceAssemblyDocs} that specifies the public API of the library.

  \item \filename{net6.0-{OS}-Release/System.Net.Quic.dll}: Where \verb|{OS}| is the identifier for
the operating system running on the machine. This is the \dotnet{} assembly with the actual QUIC
implementation.

\end{itemize}

\subsection{Creating a Local Installation of \dotnet{}}\label{sec:06localdotnet}

Now we need to download the latest \dotnet{}~6 SDK\@. A zip archive containing the SDK can be
downloaded from a link listed in the official SDK installer GitHub
repository~\cite{dotnetSdkGithub}. Download the ``Master (6.0.x Runtime)'' build for your platform
and extract it to a convenient location. In the remainder of this guide, the directory containing
the extracted contents will be referred to as \filename{DOTNET_ROOT}.

The \SystemNetQuicDll{} produced in the previous subsection must be copied to appropriate locations
in the \filename{DOTNET_ROOT}. The reference \SystemNetQuicDll{} assembly should be copied over the
existing one in the
\filename{DOTNET_ROOT/packs/Microsoft.NETCore.App.Ref/6.0.0-{version}/ref/net6.0/} directory, and
the implementation assembly to should be copied to the
\filename{DOTNET_ROOT/shared/Microsoft.NETCore.App/6.0.0-{version}/} directory, overwriting the
existing files.

\subsection{Adding the OpenSSL Library}\label{sec:06openssl}

The implementation requires a QUIC-supporting \libopenssl{} library build from a development branch
maintained by Akamai. The appropriate source codes can be found in the \filename{extern/openssl}
directory in the thesis attachments. The source code is also available online on Akamai's
GitHub~\cite{AkamaiOpensslGithub}. The implementation has been developed and tested with the
\texttt{OpenSSL_1_1_1g-quic} branch of the code, but other QUIC-enabled branches of \libopenssl{}
version 1.1.1 should work as well.

Before building the \libopenssl{} library form source, check the \filename{NOTES.{OS}} file in the
repository and make sure all prerequisites are installed on the machine. After that, the
\libopenssl{} library can be built by running the following command inside the repository. Note that
for Windows OS, you must run these commands using the \textit{Developer command prompt for Visual
  Studio} in order to have the necessary tools in \texttt{PATH}.

\begin{myVerbatim}
# Windows
> perl Configure VC-WIN64A
> nmake

# Linux
> ./config
> make
\end{myVerbatim}

This will produce the \libname{libcrypto} and \libname{libssl} libraries in the \libopenssl{}
repository root. On windows, these libraries are named \texttt{libcrypto-1_1-x64.dll} and
\texttt{libssl-1_1-x64.dll}. These libraries are loaded by the managed QUIC implementation during
runtime and, therefore, must be present in a location where the OS loader can find them. This can be
achieved by putting the libraries in any of the following locations:

\begin{itemize}

  \item Next to the compiled program executable.

  \item A directory listed in the \texttt{PATH} environment variable

  \item \textit{(preferred)} next to the \SystemNetQuicDll{} library in the \dotnet{} installation
directory, i.e., \filename{DOTNET_ROOT/shared/Microsoft.NETCore.App/6.0.0-{version}/}.

\end{itemize}

Note that if there is already a different version of \libopenssl{} installed on the system, it is
necessary to ensure that the system loads the correct \libopenssl{} version. This is different for
each operating system:

\begin{itemize}

  \item On Windows, this can be ensured by placing the DLL files in the same directory as the
program executable. The entire library search process, including the order of directories searched,
is described in Windows documentation~\cite{windowsDllSearch} in detail.

  \item On Linux, this can be achieved by defining the \texttt{LD_LIBRARY_PATH} environment variable
to the directory containing the \libopenssl{} libraries. Additional information about loading of
dynamic libraries on Linux can be found in the manual pages for \texttt{ld.so}. These manual pages
are also available online~\cite{linuxDllSearch}.

\end{itemize}

\subsection{Configuring the Development Environment}\label{sec:06-env-vars}

Lastly, we need to configure the environment variables so that the \dotnet{} SDK installation
created in the previous step is used when building the user code. For this, the following
environment variables need to be defined correctly.

\begin{description}

        \ditem{\texttt{DOTNET_ROOT}} Path to the \dotnet{} installation directory. This instructs
the build process to use the SDK installed in this directory. Use the path to the local \dotnet{}~6
SDK installation created in \autoref{sec:06localdotnet}.

        \ditem{\texttt{DOTNET_MULTILEVEL_LOOKUP}} Set this to ``0''. This instructs the build
process not to look for SDK installation in other places than \filename{DOTNET_ROOT}.

        \ditem{\texttt{PATH}} Prepend the \filename{DOTNET_ROOT} directory to the beginning of the
\texttt{PATH} variable to make sure the \filename{dotnet} executable from the \filename{DOTNET_ROOT}
is used over the system-wide installed one.
\end{description}

After configuring the variables, check the output of the \verb|dotnet --info| command. Assuming
\texttt{DOTNET_ROOT} is \filename{C:\dotnet\\}, then the output should be similar to the
\autoref{lst:06-dotnet-info-output}. The list of installed \dotnet{} runtimes should contain
\texttt{Microsoft.NETCore.App} from the local \dotnet{}~6 SDK installation prepared in
\autoref{sec:06-local-dotnet} (check the path inside the brackets). Note that the listing contains
version numbers of the latest \dotnet{}~6 SDK at the time of writing this text, and the SDK
installer would be updated since then to a newer version.

\begin{myListingVerbatim}[Output of the \texttt{dotnet --info} command in configured environment]{lst:06-dotnet-info-output}{Output of the \texttt{dotnet --info} command in correctly configured environment. The unimportant portions of the output in grey has been left out brevity}
> dotnet --info
&color{colorunimportant}.NET SDK (reflecting any global.json):
&color{colorunimportant} ...

&color{colorunimportant}Runtime Environment:
&color{colorunimportant} ...

&color{colorunimportant}Host (useful for support):
&color{colorunimportant} ...

.NET SDKs installed:
  6.0.100-alpha.1.20563.2 [&textbf{C:\dotnet\sdk}]

.NET runtimes installed:
  Microsoft.AspNetCore.App 6.0.0-alpha.1.20526.6 [&textbf{C:\dotnet\shared\}&textcolor{colorunimportant}{...}]
  Microsoft.NETCore.App 6.0.0-alpha.1.20560.10 [&textbf{C:\dotnet\shared\}&textcolor{colorunimportant}{...}]
  Microsoft.WindowsDesktop.App 6.0.0-alpha.1.20560.7 [&textbf{C:\dotnet\shared\}&textcolor{colorunimportant}{...}]

&color{colorunimportant}To install additional .NET runtimes or SDKs:
&color{colorunimportant}  https://aka.ms/dotnet-download
\end{myListingVerbatim}

With the environment configured as described, \dotnet{} applications can be compiled against the
\dotnet{}~6 SDK either using the \verb|dotnet build| command-line command, or using Visual Studio or
any other IDE\@.

\subsection{Creating a Sample \dotnet{}~6 Project}

The last step is creating a new project and configuring it to use \dotnet{}~6. This section
demonstrates how this can be done using the \verb|dotnet| command-line tool. Assuming the
environment variables have been configured as specified in \autoref{sec:06-env-vars}, you can create
a new project using the following commands:

\begin{myVerbatim}
> mkdir hello-net6
> cd hello-net6
> dotnet new console
\end{myVerbatim}

These commands will create a \filename{hello-net6/hello-net6.csproj} file. Use of \dotnet{}~6
preview requires minor changes to the project file. Namely changing the \xmltag{TargetFramework}
property to \verb|net6.0|. The modified project file contents are listed in
\autoref{lst:06-csproj-net6.0}.

\begin{myListingXml}{lst:06-csproj-net6.0}{Project file for \dotnet{}~6 console application project.}{Sdk}
<Project Sdk="Microsoft.NET.Sdk">

  <PropertyGroup>
    <OutputType>Exe</OutputType>
    <TargetFramework>net6.0</TargetFramework>
    <RootNamespace>hello_net6</RootNamespace>
  </PropertyGroup>

</Project>
\end{myListingXml}

Lastly, the NuGet package feed must be configured to use the correct package source for the preview
packages for \dotnet{}~6. This can be done by creating a \filename{NuGet.Config} file in the project
directory with contents as listed in \autoref{lst:06-nuget-config}. The contents can also be copied
from the webpage from which the \dotnet{}~6 SDK was downloaded~\cite{dotnetSdkGithub}.

\begin{myListingXml}[basicstyle=\ttfamily\scriptsize]{lst:06-nuget-config}{NuGet configuration file for \dotnet{}~6 projects.}{key,value}
<configuration>
  <packageSources>
    <add key="dotnet6"
      value="https://pkgs.dev.azure.com/dnceng/public/_packaging/dotnet6/nuget/v3/index.json" />
  </packageSources>
</configuration>
\end{myListingXml}

To check that the development environment is configured correctly, the program in
\autoref{lst:06-hello-net6} can be used.

\begin{myListingCsharp}{lst:06-hello-net6}{\csharp{} program for testing the SDK installation.}{Program,FileVersionInfo,Console,Path,QuicImplementationProviders}{}
using System;
using System.Diagnostics;
using System.IO;
using System.Net.Quic;

namespace hello_net6._0
{
    class Program
    {
        static void |Main|(string[] args)
        {
            string assemblyPath = typeof(object).Assembly.Location;
            string assemblyDir = Path.|GetDirectoryName|(assemblyPath);
            var info = FileVersionInfo.|GetVersionInfo|(assemblyPath);
            Console.|WriteLine|($"Hello from .NET {info.ProductVersion}");
            Console.|WriteLine|($"Runtime location: {assemblyDir}");
            Console.|WriteLine|($"QUIC: {QuicImplementationProviders.Default}");
        }
    }
}
\end{myListingCsharp}

Running the program should produce output similar to the following:

\noindent\begin{minipage}{\textwidth}
\begin{myVerbatim}
> dotnet run
Hello from .NET 6.0.0-alpha.1.20560.10+72b7d236ad634c2280c73499ebfc2b594995ec06
Runtime location: C:\dotnet\shared\Microsoft.NETCore.App\6.0.0-alpha.1.20560.10
QUIC: System.Net.Quic.Implementations.Managed.ManagedQuicImplementationProvider
\end{myVerbatim}
\end{minipage}\medskip

If the output lists a different runtime location, verify that the environment variables have been
set correctly. If the QUIC provider is different, it means that the \SystemNetQuicDll{} with our
managed QUIC implementation was not copied to the correct directory.

\subsection{Deploying \dotnet{} Applications}

The applications built against the preview \dotnet{}~6 SDK will run only if the environment is
configured according to \autoref{sec:06-env-vars}. In order to run the applications outside the
configured environment, it is necessary to build them as
\textit{self-contained}~\cite{SelfContainedPublishDocs}. Self-contained builds of \dotnet{}
applications contain a copy of the \dotnet{} runtime and other necessary binaries. Self-contained
applications can be built using following command:

\begin{myVerbatim}
> dotnet publish --self-contained --runtime <RID>
\end{myVerbatim}

Where \texttt{<RID>} is the runtime identifier for which to publish. Commonly used values are
\texttt{win-x64} and \texttt{linux-x64}. Full list of supported runtime identifiers can be found in
the official \dotnet{} documentation~\cite{dotnetRIDs}.

Unfortunately, the packaged application contains the unmodified \SystemNetQuicDll{} without managed
QUIC support. Therefore, the last step is manually overwriting the \SystemNetQuicDll{} with the one
locally built from the \dotnet{} runtime sources in \autoref{sec:06-build-runtime}.

\section{Simple Echo Server using QUIC}\label{sec:06tutorial}

This section is a walkthrough on how to use QUIC in \dotnet{}. In this section, we will create a
trivial echo server. When a new connection is established, clients will open a single bidirectional
stream and send arbitrary data over it. The server will then echo the data back to the client using
the same stream. For simplicity, we will use a single \dotnet{} program to represent both client and
server and conduct the connection over the loopback network interface. We will also omit error
checking from this example for brevity. The complete source code for this example can be found in
the \filename{src/supplementary/samples/Echo/} directory in the thesis attachments.

\subsection{Echo Server Implementation}

The use of TLS 1.3 for encryption is mandatory for QUIC\@. This requires providing an X.509
certificate on the server's side. The current QUIC API requires that the certificate and the private
key be saved in separate files in the PEM format. Certificate files that can be used in this example
are provided in the attachments. The public certificate file is at \filename{certs/cert.crt} and the
private key is at \filename{certs/cert.key}. Alternatively, a new certificate can be created using
the \texttt{openssl} command-line utility. In case \libopenssl{} was not previously installed on the
machine, you can use the binary which was built locally together with the libraries in
\autoref{sec:06openssl}. The locally built \texttt{openssl} binary can be located under the
\filename{apps/} subdirectory of the \libopenssl{} repository.

\begin{myVerbatim}
> openssl req -x509 -newkey rsa -keyout key.pem -out cert.pem -days 365 -nodes
\end{myVerbatim}

In order to accept incoming QUIC connections, we need to create an instance of \QuicListener{}.
\autoref{lst:06-echo-quic-listener} shows how the \QuicListener{} can be created and provided with:

\begin{itemize}

  \item listening endpoint;

  \item identifier of the application-layer protocol to be used\footnote{QUIC is not intended to be
        used standalone, but as a transport protocol for other application-layer protocols. Since
        servers can support multiple versions of the application protocol, QUIC uses the \gls{alpn}
        extension to TLS to negotiate the application-layer protocol as part of the QUIC connection
        handshake.}, even though we are not implementing any standard protocol, we still need to
        provide one; for our example, we chose to use \texttt{"echo"} as the \gls{alpn} identifier;
        and

  \item paths to the X.509 certificate and private key file.

\end{itemize}

The \QuicListener{} class implements \interface{IDisposable} and, therefore, we can also use the
\keyword{using} statement to make sure the \QuicListener{} is closed when the method returns.

\begin{myListingCsharp}{lst:06-echo-quic-listener}{Creating and starting a new \QuicListener{} for the echo server}{Task,QuicListener,QuicListenerOptions,SslServerAuthenticationOptions,List,SslApplicationProtocol,IPEndPoint,CancellationToken}{}
public static async Task<int> |RunServer|(IPEndPoint listenEp,
    string certificateFile, string privateKeyFile, CancellationToken token)
{
    using QuicListener listener = new QuicListener(new QuicListenerOptions
    {
        ListenEndPoint = listenEp,
        CertificateFilePath = certificateFile,
        PrivateKeyFilePath = privateKeyFile,
        ServerAuthenticationOptions = new SslServerAuthenticationOptions
        {
            ApplicationProtocols = new List<SslApplicationProtocol>
            {
                new SslApplicationProtocol("echo")
            }
        }
    });

    // ...
}
\end{myListingCsharp}

We can then use the \method{AcceptConnectionAsync} method to wait for new incoming connections
asynchronously. \autoref{lst:06-echo-server-accept} shows how to accept new connections and process
them asynchronously in a separate \class{Task} so that multiple connections can be served
concurrently.

\begin{myListingCsharp}{lst:06-echo-server-accept}{Accepting new connections on \QuicListener{}}{List,Task,QuicConnection}{}
    // QuicListener must be started before accepting connections.
    listener.|Start|();

    // tasks that need to be awaited when trying to exit gracefully
    List<Task> tasks = new List<Task>();

    try
    {
        QuicConnection conn;
        while ((conn = await listener.|AcceptConnectionAsync|(token)) != null)
        {
            // copy the connection into a variable with narrower scope which
            // can be safely captured inside the lambda function
            QuicConnection captured = conn;
            var task = Task.|Run|(
                () => |HandleServerConnection|(captured, token));
            tasks.|Add|(task);
        }
    }
    finally
    {
        // wait until all connections are closed
        await Task.|WhenAll|(tasks);
    }
\end{myListingCsharp}

\autoref{lst:06-echo-server-loop} Shows the implementation of \method{HandleServerConnection} which
does the actual echoing of the incoming data. The \QuicStream{} is accepted using the
\method{AcceptStreamAsync} method on the \QuicConnection{} class and, because it is a bidirectional
stream, we can use it to send the data back to the client. Lastly, once all data is sent, we
gracefully close the connection using the \method{CloseAsync} method.

\begin{myListingCsharp}{lst:06-echo-server-loop}{Echo server reading and writing data to \QuicStream{}}{Task,QuicConnection,CancellationToken,QuicStream}{}
public static async Task |HandleServerConnection|(QuicConnection connection,
    CancellationToken token)
{
    try
    {
        QuicStream stream = await connection.|AcceptStreamAsync|(token);

        byte[] buffer = new byte[4 * 1024];

        int read;
        while ((read = await stream.|ReadAsync|(buffer, token)) > 0)
        {
            await stream.|WriteAsync|(buffer, 0, read, token);
            await stream.|FlushAsync|(token);
        }
    }
    finally
    {
        // gracefully close the connection with 0 error code
        await connection.|CloseAsync|(0);
    }
}
\end{myListingCsharp}

\subsection{Echo Client Implementation}

The client implementation is more straightforward than that of the server.
\autoref{lst:06-echo-client-connection} shows how to create a client \QuicConnection{} using the
using the server endpoint address and the \gls{alpn} identifier. The connection can be then
established by calling the \method{ConnectAsync} method.

\begin{myListingCsharpNoPageBreak}{lst:06-echo-client-connection}{Creating a client \QuicConnection{} instance}{Task,IPEndPoint,CancellationToken,QuicConnection,QuicClientConnectionOptions,SslClientAuthenticationOptions,List,SslApplicationProtocol}{}
public static async Task<int> |RunClient|(IPEndPoint serverEp,
    CancellationToken token)
{
    using var client = new QuicConnection(new QuicClientConnectionOptions
    {
        RemoteEndPoint = serverEp,
        ClientAuthenticationOptions = new SslClientAuthenticationOptions
        {
            ApplicationProtocols = new List<SslApplicationProtocol>
            {
                new SslApplicationProtocol("echo")
            }
        }
    });

    await client.|ConnectAsync|(token);

    // ...
}
\end{myListingCsharpNoPageBreak}

Once the connection is established, the client opens a bidirectional \QuicStream{} using the
\method{OpenBidirectionalStream} method. The \QuicStream{} can be used like any other \Stream{}
instance. \autoref{lst:06-echo-client-loop} shows the rest of the echo client implementation.

\begin{myListingCsharp}{lst:06-echo-client-loop}{Sending standard input via \QuicStream{}}{QuicStream,Task,Encoding,Console}{}
    try
    {
        await using QuicStream stream = client.|OpenBidirectionalStream|();

        // spawn a reader task to not let server be flow-control blocked
        _ = Task.|Run|(async () =>
        {
            byte[] arr = new byte[4 * 1024];
            int read;
            while ((read = await stream.|ReadAsync|(arr, token)) > 0)
            {
                string s = Encoding.ASCII.|GetString|(arr, 0, read);
                Console.|WriteLine|($"Received: {s}");
            }
        });

        string line;
        while ((line = Console.|ReadLine|()) != null)
        {
            // convert into ASCII byte array before sending
            byte[] bytes = Encoding.ASCII.|GetBytes|(line);
            await stream.|WriteAsync|(bytes, token);
            // flush the stream to send the data immediately
            await stream.|FlushAsync|();
        }

        // once all stdin is written, close the stream
        stream.|Shutdown|();

        // wait until the server receives all data
        await stream.|ShutdownWriteCompleted|(token);
    }
    finally
    {
        // gracefully close the connection with 0 error code
        await client.|CloseAsync|(0, token);
    }
\end{myListingCsharp}

\subsection{A More Complex Example Application}

In this thesis, we developed a more complex version of the echo server example from the previous
section for benchmarking purposes. The source code for this benchmarking application can be found in
the \filename{src/supplementary/benchmark/ThroughputTests/} directory of the thesis attachments. We
will provide more information about the application in \autoref{sec:04-benchmark-app}.

\section{QUIC API Reference}\label{sec:06-api}

This section describes the API designed by the \dotnet{} development team to expose QUIC to other
developers. As mentioned in the introduction chapter, the current design is a work-in-progress and
is subject to change in the future. All of the mentioned classes are located in the
\namespace{System.Net.Quic} namespace.

\subsection{QuicListener Class}

The \class{QuicListener} class is the equivalent of the \class{TcpListener} for TCP connections.
Servers use this class to accept incoming QUIC connections.

\begin{description}

    \ditemctor{QuicListener}{\QuicListenerOptions{}} Constructor.

    \ditemproperty{IPEndPoint}{ListenEndPoint}{\propget} The IP endpoint being listened to for new
connection. Read-only.

    \ditemmethod[]{\ValueTaskOf{\QuicConnection{}}}{AcceptConnectionAsync}{\CancellationToken{}}
Accepts a new incoming QUIC Connection.

    \ditemmethod[\keyword]{void}{Start}{} Starts listening.

    \ditemmethod[\keyword]{void}{Close}{} Stops listening and closes the listener. Does not close
already accepted connections.

\end{description}

\subsection{QuicListenerOptions Class}

The \class{QuicListenerOptions} class holds all configuration used to construct new
\class{QuicListener} instances.

\begin{description}

    \ditemproperty{SslServerAuthenticationOptions}{\ServerAuthenticationOptions{}}{\propgetset} SSL
related options like certificate selection/validation callbacks, and supported protocols for ALPN\@.

    \ditemproperty[\keyword]{string}{CertificateFilePath}{\propgetset} Path to the X.509 certificate
used by the server.

    \ditemproperty[\keyword]{string}{CertificateKeyPath}{\propgetset} Path to the private key for
the used X.509 certificate.

    \ditemproperty{IPEndPoint}{ListenEndPoint}{\propgetset} The IP endpoint to listen on.

    \ditemproperty[\keyword]{int}{ListenBacklog}{\propgetset} Number of connection to be held
waiting for acceptance by the application. Upon reaching this limit, further connections will be
refused.

    \ditemproperty[\keyword]{long}{MaxBidirectionalStreams}{\propgetset} Limit on the number of
bidirectional streams the client can open in an accepted connection.

    \ditemproperty[\keyword]{long}{MaxUnidirectionalStreams}{\propgetset} Limit on the number of
unidirectional streams the client can open in an accepted connection.

    \ditemproperty{TimeSpan}{IdleTimeout}{\propgetset} The period of inactivity after which the
connection will be closed via idle timeout.

\end{description}

\subsection{QuicConnection Class}

The \QuicConnection{} class represents the QUIC connection itself. Clients open new connections by
creating a new instance of this class and calling the \method{Connect\allowbreak{}Async} method. Servers receive
new connections using the \class{QuicListener} class.

\begin{description}

    \ditemctor{QuicConnection}{\QuicClientConnectionOptions{}} Constructor. The newly created
instance must be explicitly connected using the \method{ConnectAsync} method.

    \ditemproperty[\keyword]{bool}{Connected}{\propget} Indicates whether the \QuicConnection{} is
connected (the handshake has completed).

    \ditemproperty{IPEndPoint}{LocalEndPoint}{\propget} Local IP endpoint of the connection.

    \ditemproperty{IPEndPoint}{RemoteEndPoint}{\propget} Remote IP endpoint of the connection.

    \ditemmethod{ValueTask}{ConnectAsync}{\CancellationToken{}} Connects to the remote endpoint.

    \ditemmethod{QuicStream}{OpenUnidirectionalStream}{} Opens a new unidirectional stream. Throws a
\class{QuicException} if the stream cannot be opened.

    \ditemmethod{QuicStream}{OpenBidirectionalStream}{} Opens a new bidirectional stream. Throws a
\class{QuicException} if the stream cannot be opened.

    \ditemmethod[]{\ValueTaskOf{\QuicStream{}}}{AcceptStreamAsync}{\CancellationToken{}} Accepts an
incoming stream.

    \ditemmethod{ValueTask}{CloseAsync}{\Long{}, \CancellationToken{}} Closes the connection with
the specified given error code and terminates all active streams.

    \ditemmethod[\keyword]{long}{GetRemoteAvailableUnidirectionalStreamCount}{} Gets the maximum
number of unidirectional streams that this endpoint can open.

    \ditemmethod[\keyword]{long}{GetRemoteAvailableBidirectionalStreamCount}{} Gets the maximum
number of bidirectional streams that this endpoint can open.

\end{description}

\subsection{QuicClientConnectionOptions}

The \class{QuicClientConnectionOptions} is used by clients to configure new QUIC conections.

\begin{description}

    \ditemproperty{SslClientAuthenticationOptions}{ClientAuthenticationOptions}{\propgetset} Client
authentication options to use when establishing the connection.

    \ditemproperty{IPEndPoint}{LocalEndPoint}{\propgetset} The local IP endpoint that will be bound
to.

    \ditemproperty{IPEndPoint}{RemoteEndPoint}{\propgetset} The IP endpoint to connect to.

    \ditemproperty[\keyword]{long}{MaxBidirectionalStreams}{\propgetset} Limit on the number of
bidirectional streams the server can open.

    \ditemproperty[\keyword]{long}{MaxUnidirectionalStreams}{\propgetset} Limit on the number of
unidirectional streams the server can open.

    \ditemproperty{TimeSpan}{IdleTimeout}{\propgetset} The period of inactivity after which the
connection will be closed via idle timeout.

\end{description}

\subsection{QuicStream Class}

The \QuicStream{} class represents a single stream in a QUIC connection and derives from the
abstract \class{Stream} class. The \class{Stream} class is a bidirectional stream abstraction and
since not all QUIC streams are bidirectional, user should check if the specific \QuicStream{}
instance supports supports the operation by inspecting the \method{CanRead} and \method{CanWrite}
properties. Invoking write methods on read-only --- more specifically, incoming unidirectional ---
stream will cause an \class{InvalidOperationException} to be thrown and vice versa.

The list below mentions the members specific for the \class{QuicStream} class and some important
members inherited from the \class{Stream} class.

\begin{description}

    \ditemproperty[\keyword]{long}{StreamId}{\propget} The Stream ID\@.

    \ditemproperty[\keyword]{bool}{CanRead}{\propget} Returns \keyword{true} if the stream supports
reading.

    \ditemproperty[\keyword]{bool}{CanWrite}{\propget} Returns \keyword{true} if the stream supports
reading.

    \ditemmethod[\keyword]{void}{AbortRead}{\Long} Aborts the receiving part of the stream with the
provided error code.

    \ditemmethod[\keyword]{void}{AbortWrite}{\Long} Aborts the sending part of the stream with the
provided error code.

    \ditemmethod[\keyword]{int}{Read}{\SpanOf{\Byte{}}} Reads the content of the stream into the
provided buffer, blocks if no data is available. Returns 0 only when there will be no more data in
the stream.

    \ditemmethod[]{\ValueTaskOf{\keyword{int}}}{ReadAsync}{\MemoryOf{\Byte{}}, \CancellationToken{}}
Reads the content of the stream into provided buffer, blocks until some data is available. Returns 0
only when there will be no more data in the stream.

    \ditemmethod[\keyword]{void}{Write}{\SpanOf{\Byte{}}} Writes the content of the provided buffer
into the stream, returns when the data have been buffered internally.

    \ditemmethodWithComment{ValueTask}{WriteAsync}{*, \CancellationToken{}}{multiple overloads}
Multiple overloads of this method offer writing from various types of buffers:
\ReadOnlyMemoryOf{\Byte{}}, \ReadOnlySequenceOf{\Byte{}}, and
\genericClass{ReadOnlyMemory\allowbreak{}}{\ReadOnlyMemoryOf{\Byte{}}}. The last one can be used to
perform \textit{Vectored I/O}~\cite{wiki:vectored-io}\@. The returned task completes when the
provided data have been buffered internally and the buffers can be reused for other purposes.

    \ditemmethodWithComment{ValueTask}{WriteAsync}{*, \bool{}, \CancellationToken{}}{multiple
overloads} Like the methods above, but also allow specifying that the provided data are the last on
the stream and that the stream should be gracefully closed.

    \ditemmethod{ValueTask}{ShutdownWriteCompleted}{\CancellationToken{}} The returned task
completes when the stream shutdown completes. Meaning that acknowledgment from the peer is received.

    \ditemmethod{ValueTask}{Shutdown}{} Gracefully closes the writing direction of the stream,
indicating that no more data will be sent.

\end{description}

\subsection{Exceptions}

The QUIC API can throw the following exceptions:

\begin{description}

    \ditem{\ditemsrcsize\class{QuicException}} Base class for all thrown exceptions, used when a
more specific exception is not available

    \ditem{\ditemsrcsize\class{QuicConnectionAbortedException}} Thrown when the connection is
forcibly closed either by the transport or by the remote endpoint.

    \ditem{\ditemsrcsize\class{QuicStreamAbortedException}} Thrown when the stream was aborted by
the remote endpoint.

    \ditem{\ditemsrcsize\class{QuicOperationAbortedException}} Thrown when the pending operation was
aborted by the local endpoint.

\end{description}

\chapter{Evaluation}

\todo{what we want to measure}

\todo{how we will measure it (high-level structure of the application)}

\todo{measurements}

\todo{conclusion}

\chapter*{Conclusion}
\addcontentsline{toc}{chapter}{Conclusion}

To conclude our thesis, we will revisit the goals we set in the introduction chapter in
\autoref{sec:01-goals}.

\begin{enumerate}

  \item \textit{Select a sufficient subset of QUIC specification needed to support the most basic data
        transfer and implement it inside \dotnet{} runtime codebase.}

        In \autoref{sec:03-feature-selection}, we analyzed parts of the QUIC protocol and selected
        the necessary parts for our prototype implementation. All features selected in this section
        were implemented to the extent that the implementation can successfully and reliably
        transfer data over network.

  \item \textit{Allow switching between the new managed implementation and the existing \libmsquic{}-based one.}

        Our implementation integrates into the pre-existing implementation indirection layer, which
        allows explicitly selecting the QUIC implementation provider for newly created
        \QuicListener{} and \QuicConnection{} instances. Additionally, the default provider can be
        influenced by the \texttt{DOTNETQUIC_PROVIDER} environment variable.

  \item \textit{Evaluate the managed QUIC implementation by using it to implement a simple client-server
echo application.}

        In \autoref{sec:06-tutorial}, we provided directions on how to implement a simple echo
        server and client. Additionally, we implemented a benchmarking application for use in
        performance measurements in \autoref{chap:04-evaluation}.

  \item \textit{(optional) Try to compare the performance of the new implementation with the
previous \libmsquic{}-based one and with TCP+TLS-based \class{SslStream}.}

       In \autoref{sec:04-perf-results}, we presented the results of our performance measurements.

       \todo{summarize the results}

\end{enumerate}

At the time of writing this conclusion, our managed QUIC implementation has caught attention of the
\dotnet{} development team, and this implementation will be added to the list of experiments in the
runtimelab repository~\cite{runtimelabGithub} in the \texttt{feature/ManagedQuic} branch.

The next big release --- \dotnet{}~6 --- will ship with production-ready QUIC implementation. In early
2021, a decision will be made whether this QUIC support will be based on \libmsquic{} or our QUIC
implementation. However, even if the more mature \libmsquic{} implementation is chosen for the
\dotnet{}~6 release. Our implementatino will be considered as a managed replacement for subsequent
\dotnet{} releases.

\subsection*{Future Work}

The prototype QUIC implementation developed in this thesis will require large amount of work before
becoming production-ready. Some parts of the QUIC specification were left unimplemented, other parts
were simplified in order to fit into the scope of this master thesis. However, the core part of the
implementation should provide a solid foundation on which a fully conformant QUIC implementation
could be built. Following list outlines the next development steps for the implementation.

\begin{itemize}

  \item \textit{Update the implementation to match the latest QUIC specification.} QUIC
        specification drafts evolved both during the course of implementation and during writing of
        this thesis. At the time of writing this conclusion, the 33rd version of the QUIC
        specification draft is awaiting on a last-call before it becomes a valid RFC document. The
        implementation presented in this document is based on draft version 26. Updating the
        implementation should be, therefore, the first future goal.

  \item \textit{Implement missing parts of the protocol.} This thesis implements only a subset of
        the QUIC specification. Many features like connection migration, stateless reset and path
        validation have not been implemented.

  \item \textit{Performance improvements for scalability.} The performance measurements done in
        \autoref{sec:04-multi-stream-perf} show that our implementation does not scale very well in
        face of large amounts of parallel connections. Possible improvements to the backend
        processing architecture include allowing parallel sending and receiving on a single
        connection, and using a single thread to process multiple connections, like done in
        \libmsquic{}.

  \item \textit{More realistic performance measurements.} This thesis performed all performance
        measurements over the loopback network interface and, therefore, do not represent the
        behavior on a real network.

  \item \textit{Interoperability tests with other QUIC implementations.} This thesis has done only a
        very superficial interoperability test with \libmsquic{}. There is an open source QUIC
        Interop Test Runner~\cite{QuicInteropRunner} which repeatedly tests compatibility between
        latest versions of popular QUIC implementations.

\end{itemize}


%%% Bibliography
%%% Bibliography (literature used as a source)
%%%

\renewcommand{\bibname}{Bibliography}

%% fix overfull hboxes in biblatex bibliography list
%% https://tex.stackexchange.com/questions/171999/overfull-hbox-in-biblatex
\emergencystretch=2em

\printbibliography[heading=bibintoc]


%%% Figures used in the thesis (consider if this is needed)
\listoffigures

%%% Tables used in the thesis (consider if this is needed)
%%% In mathematical theses, it could be better to move the list of tables to the beginning of the thesis.
\listoftables

%%% Code listings used in the thesis (consider if this is needed)
\lstlistoflistings

%%% Attachments to the master thesis, if any. Each attachment must be
%%% referred to at least once from the text of the thesis. Attachments
%%% are numbered.
%%%
%%% The printed version should preferably contain attachments, which can be
%%% read (additional tables and charts, supplementary text, examples of
%%% program output, etc.). The electronic version is more suited for attachments
%%% which will likely be used in an electronic form rather than read (program
%%% source code, data files, interactive charts, etc.). Electronic attachments
%%% should be uploaded to SIS and optionally also included in the thesis on a~CD/DVD.
%%% Allowed file formats are specified in provision of the rector no. 72/2017.
\appendix
\chapter{Attachments}

Contents of the attachment archive:

\dirtree{%
  .1 {/}.
  .2 {certs}\mydtcomment{Certificates for use in tests and attached samples}.
  .3 {cert.crt}\mydtcomment{Public certificate file}.
  .3 {cert.key}\mydtcomment{Private key file}.
  .3 {cert.pfx}\mydtcomment{Certificate and private key combined in PFX format}.
  .2 {extern}\mydtcomment{Third-party source code}.
  .3 {openssl}\mydtcomment{Copy of the QUIC-enabled fork of \libopenssl{} by Akamai}.
  .2 {src}\mydtcomment{Root for all source code}.
  .3 {dotnet-runtime}\mydtcomment{\dotnet{} runtime fork with managed QUIC implementation}.
  .3 {supplementary}\mydtcomment{Root of additional \dotnet{} projects}.
  .4 {benchmark}\mydtcomment{Root of all benchmark projects}.
  .5 {ThroughputTests}\mydtcomment{Benchmarking application used in \autoref{chap:04-evaluation}}.
  .6 {run.ps1}\mydtcomment{Script used to automate running the benchmarking application}.
  .4 {samples}\mydtcomment{Root of all sample projects}.
  .5 {Echo}\mydtcomment{Simple echo server and client implementation from \autoref{chap:06-user-docs}}.
  .5 {hello-net6.0}\mydtcomment{Hello world application for testing \dotnet{}~6 installation}.
  .4 {ManagedQuic.sln}\mydtcomment{\dotnet{} Solution file for all projects in subdirectory}.
  .2 {tex}\mydtcomment{\LaTeX{} source code for this thesis}.
  .3 {measurements}\mydtcomment{Raw CSV data with measurements presented in \autoref{chap:04-evaluation}}.
  .2 {README.md}\mydtcomment{File describing the contents of the archive}.
}


%%% Abbreviations used in the thesis, if any, including their explanation
%%% In mathematical theses, it could be better to move the list of abbreviations to the beginning of the thesis.
\printglossary

\openright
\end{document}
