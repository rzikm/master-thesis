%%% A template for a simple PDF/A file like a stand-alone abstract of the thesis.

\documentclass[12pt]{report}

\usepackage[czech]{babel}
\usepackage[a4paper, hmargin=1in, vmargin=1in]{geometry}
\usepackage[a-2u]{pdfx}
\usepackage[utf8]{inputenc}
\usepackage[T1]{fontenc}
\usepackage{lmodern}
\usepackage{textcomp}
\usepackage{tikz}           % for drawing (checkmark etc)

\newcommand{\csharp}{%
  {\settoheight{\dimen0}{C}C\kern-.05em \resizebox{!}{\dimen0}{\raisebox{\depth}{\#}}}}
\newcommand{\dotnet}{.NET\@}

\begin{document}

QUIC je general-purpose síťový protokol transportní vrstvy, který byl navržen jako náhrada TCP a TLS
pro HTTP/3. QUIC je postaven nad UDP a poskytuje vždy zašifrované spojení schopné paraleního přenosu
vícero proudů dat. V porovnání s TCP, QUIC slibuje nižší dobu odezvy, větší flexibilitu congestion
control a řešení head-of-line blocking problému, který se vyskytuje v multiplexovaných HTTP/2
spojeních.

Nejnovější verze \dotnet{} --- \dotnet{}~5 --- byla vydána s experimentální podporou pro QUIC, která je
založena na knihovně MsQuic napsané v jazyce C. Nicméně, při implementování nových feature ve
standartních knihovnách v \dotnet{} jsou preferovány vlastní implementace přímo v \dotnet{} místo
přidávání závislostí na nativních knihovnách, protože implementace v \dotnet{} jsou snáze
udržovatelné a v některých případech dokonce i více výkonné. Tato práce zkoumá proveditelnost
implementace QUIC protokolu v jazyce \csharp{} jakožto budoucí náhrady stávajícího řešení pro vydání
v \dotnet{}~6 nebo pozdější verzi.

Výsledkem této práce je fork oficiálního repozitáře \dotnet{} runtime s částečnou implementací QUIC
protokolu v jazyce \csharp{}. Tato práce implementovala podmnožinu QUIC specifikace dostačující pro
základní zhodnocení výkonu. Jako součást práce jsme provedli měření propustnosti a odezvy naší
implementace a předchozí implementace založené na knihovně MsQuic a porovnali jsme je s výkonem TCP
ve dvou prostředích: LAN síť a simulovaná mobilní síť. Přestože měření ukázalo že naše implementace
je pomalejší než ta založená na MsQuic, identifikovali jsme hlavní faktory limitující výkon naší
implementace a navrhli jsme směr pro její budoucí vývoj.

\end{document}
