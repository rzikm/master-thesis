\chapter*{Conclusion} \addcontentsline{toc}{chapter}{Conclusion}

To conclude our thesis, we will revisit the goals we set in the introduction chapter in
\autoref{sec:01-goals}.

\begin{enumerate}

  \item \textit{Select a sufficient subset of QUIC specification needed to support the most basic
data transfer and implement it inside \dotnet{} runtime codebase.}

        In \autoref{sec:03-feature-selection}, we analyzed parts of the QUIC protocol and selected
the necessary parts for our prototype implementation. All features selected in this section were
implemented to the extent that the implementation can successfully and reliably transfer data over
the network.

  \item \textit{Allow switching between the new managed implementation and the existing
\libmsquic{}-based one.}

        Our implementation integrates into the pre-existing implementation indirection layer, which
allows explicitly selecting the QUIC implementation provider for newly created \QuicListener{} and
\QuicConnection{} instances. Additionally, the default provider can be selected using the
\texttt{DOTNETQUIC_PROVIDER} environment variable.

  \item \textit{Evaluate the managed QUIC implementation by using it to implement a simple
client-server echo application.}

        In \autoref{sec:06tutorial}, we provided directions on implementing a simple echo server and
client. Additionally, we implemented a benchmarking application for use in performance measurements
in \autoref{chap:04-evaluation}.

  \item \textit{(optional) Compare the performance of the new implementation with the previous
\libmsquic{}-based one and with TCP+TLS-based \class{SslStream}.}

        In \autoref{sec:04-perf-results}, we presented the results of our performance measurements.
Our QUIC implementation outperforms the \libmsquic{}-based one in a small number of scenarios.
However, in most cases and especially under heavy load, the \libmsquic{}-based QUIC implementation
performs better both in terms of throughput and latency. However, neither of the QUIC
implementations performed better in our tests than the combination of TCP+TLS\@.

\end{enumerate}

At the time of writing this conclusion, our managed QUIC implementation has caught the attention of
the \dotnet{} development team, and this implementation will be added to the list of experiments in
the runtimelab repository~\cite{runtimelabGithub} in the \texttt{feature/ManagedQuic} branch.

The next big release --- \dotnet{}~6 --- will ship with production-ready QUIC implementation. In early
2021, a decision will be made whether this QUIC support will be based on \libmsquic{} or our QUIC
implementation. However, even if the more mature \libmsquic{} implementation is chosen for the
\dotnet{}~6 release, our implementation will be considered as a managed replacement for subsequent
\dotnet{} releases.

\subsection*{Future Work}

The prototype QUIC implementation developed in this thesis will require more development and
substantial performance tuning before becoming production-ready. Some parts of the QUIC
specification were left unimplemented, and other parts were simplified to fit into the scope of this
master thesis. However, the core part of the implementation should provide a solid foundation upon
which a fully conformant QUIC implementation could be built. The following list outlines the next
development steps for the implementation.

\begin{itemize}

  \item \textit{Update the implementation to match the latest QUIC specification.} QUIC
specification drafts evolved both during implementation and writing of the text of this thesis. At
the time of writing this conclusion, the 33rd version of the QUIC specification draft is awaiting a
last-call before it becomes a valid RFC document. The implementation presented in this document is
based on draft version 27. Updating the implementation should be, therefore, the first future goal.

  \item \textit{Implement missing parts of the protocol.} This thesis implements only a subset of
the QUIC specification. Many features like connection migration, stateless reset, and path
validation have been omitted from the prototype but should be implemented to make the implementation
fully conform to the QUIC specification.

  \item \textit{Performance improvements for scalability.} The performance measurements done in
\autoref{sec:04-multi-stream-perf} show that our implementation does not scale very well in the face
of large amounts of parallel connections. Possible improvements to the backend processing
architecture include allowing parallel sending and receiving on a single connection and using a
single thread to process multiple connections like done in \libmsquic{}.

  \item \textit{Experiment with congestion control algorithms.} This thesis implemented only the
NewReno~\autocite[Section~7]{draft-ietf-quic-recovery} congestion control algorithm described in the
QUIC protocol specification. However, other algorithms such as CUBIC or
HyStart++~\cite{cloudflareCubic} have shown better performance in some scenarios.

  \item \textit{More realistic performance measurements.} This thesis performed measurements using
\SI[per-mode=symbol]{1}{\giga\bit\per\second} Ethernet connection in LAN network. Some
implementations easily saturated such a network, and we had to extrapolate from results measured
over the software-based loopback network interface. It would be better to confirm the measurements
over \SI[per-mode=symbol]{10}{\giga\bit\per\second} connection.

  \item \textit{Interoperability tests with other QUIC implementations.} This thesis has done only a
straightforward interoperability test with \libmsquic{}. There is an open-source QUIC Interop Test
Runner~\cite{QuicInteropRunner} which repeatedly tests compatibility between the latest versions of
popular QUIC implementations.

\end{itemize}
