\chapter*{Conclusion}
\addcontentsline{toc}{chapter}{Conclusion}

To conclude our thesis, we will revisit the goals we set in the introduction chapter in
\autoref{sec:01-goals}.

\begin{enumerate}

  \item \textit{Select a sufficient subset of QUIC specification needed to support the most basic data
        transfer and implement it inside \dotnet{} runtime codebase.}

        In \autoref{sec:03-feature-selection}, we analyzed parts of the QUIC protocol and selected
        the necessary parts for our prototype implementation. All features selected in this section
        were implemented to the extent that the implementation can successfully and reliably
        transfer data over network.

  \item \textit{Allow switching between the new managed implementation and the existing \libmsquic{}-based one.}

        Our implementation integrates into the pre-existing implementation indirection layer, which
        allows explicitly selecting the QUIC implementation provider for newly created
        \QuicListener{} and \QuicConnection{} instances. Additionally, the default provider can be
        influenced by the \texttt{DOTNETQUIC_PROVIDER} environment variable.

  \item \textit{Evaluate the managed QUIC implementation by using it to implement a simple client-server
echo application.}

        In \autoref{sec:06-tutorial}, we provided directions on how to implement a simple echo
        server and client. Additionally, we implemented a benchmarking application for use in
        performance measurements in \autoref{chap:04-evaluation}.

  \item \textit{(optional) Compare the performance of the new implementation with the
previous \libmsquic{}-based one and with TCP+TLS-based \class{SslStream}.}

       In \autoref{sec:04-perf-results}, we presented the results of our performance measurements.

       \todo{summarize the results}

\end{enumerate}

At the time of writing this conclusion, our managed QUIC implementation has caught attention of the
\dotnet{} development team, and this implementation will be added to the list of experiments in the
runtimelab repository~\cite{runtimelabGithub} in the \texttt{feature/ManagedQuic} branch.

The next big release --- \dotnet{}~6 --- will ship with production-ready QUIC implementation. In early
2021, a decision will be made whether this QUIC support will be based on \libmsquic{} or our QUIC
implementation. However, even if the more mature \libmsquic{} implementation is chosen for the
\dotnet{}~6 release. Our implementatino will be considered as a managed replacement for subsequent
\dotnet{} releases.

\subsection*{Future Work}

The prototype QUIC implementation developed in this thesis will require large amount of work before
becoming production-ready. Some parts of the QUIC specification were left unimplemented, other parts
were simplified in order to fit into the scope of this master thesis. However, the core part of the
implementation should provide a solid foundation on which a fully conformant QUIC implementation
could be built. Following list outlines the next development steps for the implementation.

\begin{itemize}

  \item \textit{Update the implementation to match the latest QUIC specification.} QUIC
        specification drafts evolved both during the course of implementation and during writing of
        this thesis. At the time of writing this conclusion, the 33rd version of the QUIC
        specification draft is awaiting on a last-call before it becomes a valid RFC document. The
        implementation presented in this document is based on draft version 26. Updating the
        implementation should be, therefore, the first future goal.

  \item \textit{Implement missing parts of the protocol.} This thesis implements only a subset of
        the QUIC specification. Many features like connection migration, stateless reset and path
        validation have not been implemented.

  \item \textit{Performance improvements for scalability.} The performance measurements done in
        \autoref{sec:04-multi-stream-perf} show that our implementation does not scale very well in
        face of large amounts of parallel connections. Possible improvements to the backend
        processing architecture include allowing parallel sending and receiving on a single
        connection, and using a single thread to process multiple connections, like done in
        \libmsquic{}.

  \item \textit{More realistic performance measurements.} This thesis performed all performance
        measurements over the loopback network interface and, therefore, do not represent the
        behavior on a real network.

  \item \textit{Interoperability tests with other QUIC implementations.} This thesis has done only a
        very superficial interoperability test with \libmsquic{}. There is an open source QUIC
        Interop Test Runner~\cite{QuicInteropRunner} which repeatedly tests compatibility between
        latest versions of popular QUIC implementations.

\end{itemize}
