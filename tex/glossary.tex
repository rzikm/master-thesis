\makeglossaries

\renewcommand{\glsnamefont}[1]{\capitalisewords{#1}}

% we use glossary even for abbreviations, so that we can include a description
\newcommand{\newdefinedabbreviation}[4]{
    \newglossaryentry{#1}
    {
        text={#2},
        long={#3},
        name={#3 (#2)},
        first={#3 (#2)},
        firstplural={\glsentrylong{#1}\glspluralsuffix (\glsentryname{#1}\glspluralsuffix )},
        description={#4}
    }
}

\newglossaryentry{managed-code}
{
  name=managed code,
  description={Code written in one of the \dotnet{} languages running on the \dotnet{} virtual machine.}
}

\newglossaryentry{native-code}
{
  name=native code,
  description={Code written in ahead-of-time compiled language. This code runs directly on the target CPU.}
}

\newglossaryentry{network-path}
{
    name=network path,
    description={An imaginary path between two network addresses. Each end of a network address consists of a pair of local address and port number.}
}

\newglossaryentry{head-of-line-blocking}
{
    name=head-of-line blocking,
    description={Performance limiting phenomenon caused by a line of packets being held up by the first packet}
}

\newglossaryentry{quic-packet}
{
    name=QUIC packet,
    description={A complete processable unit of QUIC that can be encapsulated in a UDP datagram.
Multiple QUIC packets can be encapsulated in a single UDP datagram.}
}

\newglossaryentry{out-of-order-packet}
{
  name=out-of-order packet,
  description={A packet that does not arrive directly after the packet that was
  sent before it.  A packet can arrive out of order if it is delayed, if earlier packets are
  lost or delayed, or if the sender intentionally skips a packet number.}
}

\newglossaryentry{cid}
{
  name=Connection ID,
  description={An opaque identifier that is used to identify a QUIC
  connection at an endpoint.  Each endpoint sets a value for its
  peer to include in packets sent towards the endpoint.}
}

\newglossaryentry{ack-eliciting-packet}
{
  name=ack-eliciting packet,
  description={A QUIC packet which contains at least one frame which is not \PADDING{},  \ACK{}, or \CONNECTIONCLOSE{}}
}

\newglossaryentry{strategy-pattern}
{
  name=strategy pattern,
  description={A behavioral design pattern which allows selecting an internal implementation (algorithm) at runtime.}
}

\newglossaryentry{traffic-amplification-attack}
{
  name=traffic amplification attack,
  description={A type of distributed denial-of-service (DDoS) attack in which attacker sends a small amount of data to a server and the server responds with larger amount of data to the target. An example of such attack is initiating new connections with a falsified source IP address which make the server flood the victim with handshake attempts.}
}

\newglossaryentry{path-validation}
{
  name=path validation,
  description={A process during which QUIC endpoint validates that it's peer is reachable via a particular \gls{network-path}. Consists of an exchange of \PATHCHALLENGE{} and \PATHRESPONSE{} frames.}
}

\newglossaryentry{packet-pacing}
{
  name=packet pacing,
  description={A mechanism which evens out microbursts of packets in order to prevent network congestion. An ideal network pacer spreads entire congestion window worth of traffic evenly over the round trip time period.}
}

\newglossaryentry{micro-bursting}
{
  name=micro-bursting,
  description={A performance limiting phenomenon in which packets arrive in short rapid bursts. These bursts may cause overflow in the receiving buffers and cause the receiver to discard incoming packets.}
}

\newglossaryentry{abstract-factory}
{
  name=abstract factory pattern,
  description={A design pattern encapsulating creation of families of related classes. A typical example are factories for creating GUI widgets, where different factories can create various types of widgets backed by a particular GUI rendering library.}
}

\newdefinedabbreviation{alpn}{ALPN}{Application-Layer Protocol Negotiation}{A TLS extension that allows the application layer to negotiate which application protocol will be in the connection. The negotiation is done as part of the TLS handshake and avoids additional round trips. ALPN is used, e.g., for HTTP version negotiation in HTTPS connections.}

\newdefinedabbreviation{sni}{SNI}{Server Name Indication}{A TLS extension that allows the client to specify the hostname it is attempting to connect to at the start of the handshake process. This allows using different security configurations for each different website hosted on the server.}

\newdefinedabbreviation{0rtt}{0-RTT}{Zero round trip time resumption}{TLS mode of operation allowing clients to send application data in the very first but possibly exposing the server to reply attacks.}

\newdefinedabbreviation{scid}{SCID}{Source Connection ID}{Connection ID used by the sender of the QUIC packet.}

\newdefinedabbreviation{dcid}{DCID}{Destination Connection ID}{Connection ID used by the receiver of the QUIC packet.}

\newdefinedabbreviation{aead}{AEAD}{Authenticated encryption with associated data}{Form of encryption which simultaneously assure the confidentiality and authenticity of data.}
