\makeglossaries

\newglossaryentry{managed-code}
{
  name=managed code,
  description={Code written in one of the \dotnet{} languages running on the \dotnet{} virtual machine.}
}

\newglossaryentry{native-code}
{
  name=native code,
  description={Code written in ahead-of-time compiled language. This code runs directly on the target CPU.}
}

\newglossaryentry{network-path}
{
    name=network path,
    description={An imaginary path between two network addresses. Each end of a network address consists of a pair of local address and port number.}
}

\newglossaryentry{head-of-line-blocking}
{
    name=head-of-line blocking,
    description={Performance limiting phenomenon caused by a line of packets being held up by the first packet}
}

\newglossaryentry{quic-packet}
{
    name=QUIC packet,
    description={A complete processable unit of QUIC that can be
  encapsulated in a UDP datagram.  Multiple QUIC packets can be
  encapsulated in a single UDP datagram.}
}

\newglossaryentry{out-of-order-packet}
{
  name=out-of-order pakcet,
  description={A packet that does not arrive directly after the packet that was
  sent before it.  A packet can arrive out of order if it is delayed, if earlier packets are
  lost or delayed, or if the sender intentionally skips a packet number.}
}

\newglossaryentry{cid}
{
  name=Connection ID,
  description={An opaque identifier that is used to identify a QUIC
  connection at an endpoint.  Each endpoint sets a value for its
  peer to include in packets sent towards the endpoint.}
}

\newglossaryentry{ack-eliciting-packet}
{
  name=ack-eliciting packet,
  description={A QUIC packet which contains at least one frame which is not \PADDING{},  \ACK{}, or \CONNECTIONCLOSE{}}
}

\newacronym{0rtt}{0-RTT}{Zero Round Trip Time Resumption}
\newacronym{scid}{SCID}{Source Connection ID}
\newacronym{dcid}{DCID}{Destination Connection ID}
