\chapter{Developer Documentation}

The Managed QUIC implementation developed in this thesis is contained in a fork of the official
\dotnet{} runtime repository. The source code of this fork is attached in
\filename{src/dotnet-runtime/} directory in the thesis attachments.

The documentation inside the \dotnet{} runtime repository contains detailed workflow instructions
necessary for the development of the \dotnet{} runtime. These instructions list the necessary
prerequisites and explain how to build the product and run unit tests. The workflow instructions'
top-level file is located at \filename{docs/workflow/README.md}.

This chapter will focus on the \filename{System.Net.Quic} library, which contains the QUIC protocol
implementation. The source code for this library is located inside the
\filename{src/libraries/System.Net.Quic/} directory inside the \dotnet{} runtime codebase. The
directory listing in \autoref{lst:05-system-net-quic-structure} shows the structure of the
\filename{System.Net.Quic} project directory. The listing also emphasizes the \filename{Managed} and
\filename{UnitTests} directories, which contain the code developed as part of this thesis. These
directories are the main focus of this chapter.

\newcommand{\mydtcomment}[1]{\DTcomment{#1}}
% use smaller width to make the tree more compact
\DTsetlength{0.2em}{0.6em}{0.2em}{0.4pt}{1.6pt}
\renewcommand{\DTstylecomment}[1]{{\footnotesize\rmfamily #1}}
\renewcommand{\DTstyle}[1]{{\footnotesize\texttt{#1}}}
\definecolor{dtemphcolor}{rgb}{0,0,0.75}
\newcommand{\dtemph}[1]{\textcolor{dtemphcolor}{\emph{#1}}}

\begin{myListing}[Directory structure of the System.Net.Quic project.]{lst:05-system-net-quic-structure}{Directory structure of the System.Net.Quic project. Emphasised items contain the implementation developed in this thesis.}
\dirtree{%
  .1 {src/dotnet-runtime/src/libraries/System.Net.Quic/}.
  .2 {ref}\mydtcomment{Refererence assembly code}.
  .2 {src}\mydtcomment{Main library source code}.
  .3 {Resources}.
  .4 {Strings.resx}\mydtcomment{Definition of localizable strings like exception messages.}.
  .3 {System}.
  .4 {Net}.
  .5 {Quic}.
  .6 {Implementations}\mydtcomment{Root directory for all QUIC implementations}.
  .7 {\dtemph{Managed}}\mydtcomment{\dtemph{This thesis' implementation sources}}.
  .7 {MsQuic}\mydtcomment{\libmsquic{}-based implementation sources}.
  .7 {Mock}\mydtcomment{Mock implementation used only in tests}.
  .6 {Interop}\mydtcomment{Imports from native libraries}.
  .3 {System.Net.Quic.csproj}.
  .2 {tests}\mydtcomment{Library tests source code}.
  .3 {certs}\mydtcomment{X.509 certificates used in tests}.
  .4 {cert.crt}\mydtcomment{Public certificate file}.
  .4 {cert.key}\mydtcomment{Private key file}.
  .3 {FunctionalTests}\mydtcomment{Tests against the public API}.
  .4 {System.Net.Quic.Functional.Tests.csproj}.
  .3 {\dtemph{UnitTests}}\mydtcomment{\dtemph{Managed implementation unit tests}}.
  .4 {\dtemph{System.Net.Quic.Unit.Tests.csproj}}.
  .2 {Directory.Build.props}.
  .2 {System.Net.Quic.sln}.
}
\end{myListing}

\section{QUIC Implementation Providers}

The \filename{System.Net.Quic} project internally contains an infrastructure for switching between
multiple implementations of the QUIC protocol. The API classes themselves are \keyword{sealed} but
they delegate all methods to polymorphic \textit{implementation providers}. Each \QuicListener{},
\QuicConnection{}, and \QuicStream{} instance contains a reference to a \QuicListenerProvider{},
\QuicConnectionProvider{}, or \QuicStreamProvider{} instance, respectively. Implementations of each
provider are provided by each QUIC implementation in the \filename{System.Net.Quic} library.
\autoref{fig:05-quic-impl-providers} illustrates this indirection layer using a class diagram.

\begin{myFigure}{fig:05-quic-impl-providers}{Implementation providers for the QUIC API classes}

\resizebox{0.90\textwidth}{!}{\input{img/05-quic-implementation-providers.pdf_tex}}

\end{myFigure}

The infrastructure implements the \gls{abstract-factory}~\cite{wiki:abstract-factory-pattern}. New
instances of \QuicListenerProvider{} and \QuicConnectionProvider{} are created by concrete
implementations of the \QuicImplementationProvider{} class. The singleton instances of
\QuicImplementationProvider{} class implementations are exposed as static properties on the
\class{QuicImplementationProviders} static class.

There were two pre-existing \QuicImplementationProvider{}s in the \filename{System.Net.Quic}
project. Implementation of these two providers is not the primary focus of this text and, therefore,
this text will not provide further details on these providers:

\begin{itemize}

  \litem[]{\texttt{MsQuic}} QUIC implementation backed by \libmsquic{} native library

  \litem[]{\texttt{Mock}} A mock QUIC implementation for use in tests

\end{itemize}

As part of this thesis, we implemented the following two new providers:

\begin{itemize}

  \litem[]{\texttt{Managed}} Managed implementation with TLS backed by \libopenssl{} fork with QUIC
enabling API\@.

  \litem[]{\texttt{ManagedMockTls}} Managed implementation with mock TLS implementation, which does
not depend on external libraries, but cannot interoperate with other QUIC implementations.

\end{itemize}

Additionally, there is the \texttt{Default} provider. This provider can be influenced by setting the
\texttt{DOTNETQUIC_PROVIDER} environment variable to the desired provider name. If the environment
variable is not set, then the \texttt{Managed} provider is used.

The \QuicListener{} and \QuicConnection{} classes have a constructor overload which accepts an
instance of the \QuicImplementationProvider{} to be used. This way, the QUIC implementation can be
selected during runtime. This also allows reusing a suite of functional tests for all
implementations by simply changing the implementation provider.

\section{Managed QUIC Implementation Overview}

The source code for the managed implementation developed in this thesis is located under the
\filename{System.Net.Quic/src/System/Net/Quic/Implementations/Managed/} subdirectory.
\autoref{lst:05-managed-quic-structure} outlines the directory structure of the implementation.

\begin{myListing}{lst:05-managed-quic-structure}{Directory structure of the managed QUIC implementation}
\dirtree{%
  .1 {System.Net.Quic/src/System/Net/Quic/Implementations/Managed}.
  .2 {Internal}\mydtcomment{Internal code of the implementation}.
  .3 {Crypto}\mydtcomment{Cryptographic facilities}.
  .3 {Frames}\mydtcomment{Definition of QUIC frames}.
  .3 {Headers}\mydtcomment{Definition of QUIC packet headers}.
  .3 {Packets}\mydtcomment{QUIC packet number spaces handling}.
  .3 {Parsing}\mydtcomment{Parsing of QUIC primitives}.
  .3 {Recovery}\mydtcomment{Loss detection and recovery}.
  .3 {Sockets}\mydtcomment{Servicing socket IO}.
  .3 {Streams}\mydtcomment{Stream buffering}.
  .3 {Tls}\mydtcomment{TLS integration}.
  .4 {Mock}\mydtcomment{Mock TLS implementation}.
  .4 {OpenSsl}\mydtcomment{OpenSSL TLS integration}.
  .3 {Tracing}\mydtcomment{Tracing and logging facilities}.
  .2 {ManagedQuicConnection.cs}\mydtcomment{Implementation of public API}.
  .2 {ManagedQuicConnection.Frames.cs}\mydtcomment{Processing of QUIC frames}.
  .2 {ManagedQuicConnection.Packets.cs}\mydtcomment{Processing of QUIC packets}.
  .2 {ManagedQuicConnection.Recovery.cs}\mydtcomment{Handling packet loss}.
  .2 {ManagedQuicConnection.Stream.cs}\mydtcomment{Stream management}.
  .2 {ManagedQuicImplementationProvider.cs}\mydtcomment{Abstract factory implementation}.
  .2 {ManagedQuicListener.cs}\mydtcomment{Implementation of public API}.
  .2 {ManagedQuicStream.cs}\mydtcomment{Implementation of public API}.
}
\end{myListing}

The implementation is exposed using the \ManagedQuicListener{}, \ManagedQuicConnection{},
\ManagedQuicStream{} implementation provider classes and the \ManagedQuicImplementationProvider{}
factory. The source code for these classes can be found in the root directory of the implementation.

The high-level architecture has been described in \autoref{sec:03-high-level-architecture}. This and
following sections will provide further implementation details. The class diagram in
\autoref{fig:05-high-level-class-diagram} shows the releationship between the key architecture
classes which were also mentioned in \autoref{fig:03-architecture} and
\autoref{fig:03-socket-context-architecture}. These classes are:

\begin{itemize}

  \litem[]{\ManagedQuicConnection{}} Implementation provider for \QuicConnection{}. Implements
stateful connection logic.

  \litem[]{\ManagedQuicListener{}} Implementation provider for \QuicListener{}. Maintains a queue of
incoming connections to be accepted by the application.

  \litem[]{\QuicConnectionContext{}} Class hosting the background thread for servicing a single
\ManagedQuicConnection{}, including timeout expiration and sending or receiving UDP datagrams with
QUIC packets.

  \litem[]{\QuicSocketContext{}} Abstract class handling basic sending and receiving of UDP
datagrams, base class for \QuicClientSocketContext{} and \QuicServerSocketContext{}.

  \litem[]{\QuicServerSocketContext{}} Implements server-side stateless UDP datagram processing and
dispatch of incoming UDP datagrams to appropriate \QuicConnectionContext{} instance.

  \litem[]{\QuicClientSocketContext{}} Implements client-side \QuicSocketContext{} behavior, passing
all packets to a single single \QuicConnectionContext{}.

\end{itemize}

\begin{myFigure}{fig:05-high-level-class-diagram}{Relationship between key classes of managed QUIC
implementation}

\resizebox{\linewidth}{!}{\input{img/05-high-level-class-diagram.pdf_tex}}

\end{myFigure}

\section{Supporting Data Structures}

The QUIC implementation requires a few specialized data structures for internal implementation.

\subsection{RangeSet}

Efficient representation of ranges of acknowledged packet numbers and maintaining information about
which parts of the QUIC stream have been sent or acknowledged require a data structure capable of
efficiently performing set operations on ranges of integers. The \RangeSet{} class represents such a
set.

It is expected that the number of ranges in one \RangeSet{} instance will be relatively small.
Therefore, our implementation uses a \ListOf{\class{T}} instance to store individual ranges which
are sorted in ascending order. The ranges themselves are represented by
\class{RangeSet.\allowbreak{}Range} struct which contains the first and last element represented by
the range.

\subsection{PacketNumberWindow}

The \PacketNumberWindow{} class is used to test if QUIC packet with the given packet number has
already been received. Internally, it contains two 64-bit \keyword{ulong} fields: a bitmask marking
the received packet numbers and the offset of the window. All packet numbers above the window are
considered not received, and all packet numbers below the window are considered received. Packet
numbers inside the window are considered received if the corresponding bit is set to 1. When a new
packet number is received, the window is shifted, if necessary, and the corresponding bit is set.
This allows tracking 64 consecutive packet numbers at any given moment with very low overhead.

\autoref{fig:05-packet-number-window} illustrates the concept with an smaller 8-bit window.
\autoref{fig:05a} shows the state of the \PacketNumberWindow{} after receiving packets 0--5. When
packet 11 is received, the window is shifted by 4 and the highest bit --- which now corresponds to the
packet number 11 --- is set to 1, as shown in \autoref{fig:05b}. \autoref{fig:05c} shows the state of
the window when packet 15 is received next. The window needs to be shifted by another 4 bits, moving
unreceived packet numbers 6 and 7 outside the window (marked by red color in the figure).

\newlength{\origframesep}
\setlength{\fboxsep}{0.12em}

\begin{myFigure}{fig:05-packet-number-window}{Maintaining window of received packet numbers}

  \begin{mySubfigure}{\textwidth}{fig:05a}{State after receiving packets 0--5}
    offset = 0\hspace{1cm}...\textls[250]{\texttt{\textcolor{colorunimportant}{00000000000000}\fbox{00111111}}}
  \end{mySubfigure}

  \medskip

  \begin{mySubfigure}{\textwidth}{fig:05b}{State after receiving packet 11}
    offset = 4\hspace{1cm}...\textls[250]{\texttt{\textcolor{colorunimportant}{0000000000}\fbox{10000011}\textcolor{colorunimportant}{1111}}}
  \end{mySubfigure}

  \medskip

  \begin{mySubfigure}{\textwidth}{fig:05c}{State after receiving packet 15}
    offset = 8\hspace{1cm}...\textls[250]{\texttt{\textcolor{colorunimportant}{000000}\fbox{10001000}\textcolor{red}{11}\textcolor{colorunimportant}{111111}}}
  \end{mySubfigure}

\end{myFigure}

\setlength{\fboxsep}{\origframesep}

The packet numbers marked in red in \autoref{fig:05c} demonstrate that the data structure may
falsely label some old packet numbers as already received. In such a case, the packets are simply
discarded by the \ManagedQuicConnection{} implementation. This may be acceptable for the following
reasons:

\begin{itemize}

  \item The discarded packet will not be acknowledged, and the data will, therefore, be eventually
retransmitted in some other packet. The only possible harm to the connection will come in the form
of degraded performance because the other endpoint will interpret it like a packet loss and will
reduce his congestion window.

  \item For a packet $N$ to be falsely discarded in this way with a 64-bit packet window, it must
have been delayed long enough for packet $N+64$ or newer to be received first.

  \item Even if we used an exact method to track received packet numbers and correctly received
packet $N$, it is almost sure that it already was or will be marked as lost on the sender's side.
This is because, by that point, an acknowledgment had already been sent for packet $N+3$ or higher,
which will cause the packet to be considered lost due to the packet reordering
threshold\footnote{Receiving an acknowledgment for packet $N$ will mark all packets
$N - kReorderingThreshold$ or lower as lost. The QUIC specification recommends value 3 for the
$kReorderingThreshold$ constant. The QUIC loss detection algorithm has been described in greater
detail in \autoref{sec:02-loss-detection}}.

\end{itemize}

Returning to the situation illustrated in \autoref{fig:05-packet-number-window}. When packet 11 is
received (\autoref{fig:05b}), an acknowledgment is immediately sent. Once the sender receives the
acknowledgement --- which may be before or after the situation from \autoref{fig:05c} --- it will mark
packets 6 and 7 as lost and resend the data in some future packets.

\section{ManagedQuicConnection Implementation}

The \ManagedQuicConnection{} class implements the stateful QUIC connection logic, which makes up
most of the managed QUIC implementation. The source code of the \ManagedQuicConnection{} class is
separated by area into multiple files, as outlined previously in
\autoref{lst:05-managed-quic-structure}:

\begin{itemize}

  \litem{Public API} Implementation of the public API methods inherited from the
\QuicConnectionProvider{} class.

  \litem{Packets} Processing individual QUIC packets, applying and removing the packet protection,
generating packets to be sent.

  \litem{Frames} Processing individual QUIC frames and generating frames for outgoing packets.

  \litem{Stream} Management of created QUIC streams and flow control limits.

  \litem{Recovery} Tracking of sent QUIC packets, handling of acknowledgments and packet loss.
Congestion window management.

\end{itemize}

The following subsections describe the major parts of the implementation.

\subsection{Integration with Socket Management}

The \ManagedQuicConnection{} implementation is separated from socket IO management. Also, to allow
for deterministic unit testing, the implementation does not maintain an internal timer that would
automatically invoke some logic on expiration. Instead, all interactions with a \Socket{} are driven
externally by a \QuicConnectionContext{} class that maintains a background processing thread for
handling timeouts and sending and receiving QUIC packets. The interface used by
\QuicConnectionContext{} consists of following members on \ManagedQuicConnection{}:

\begin{description}

  \ditemproperty[\enum]{QuicConnectionState}{ConnectionState}{\propget} The current state of the
connection. Used to detect transitions to, e.g., connected state or closed state.

  \ditemmethod[\keyword]{void}{SendData}{\QuicWriter{}, \keyword{out} \class{EndPoint},
\class{QuicSocketContext.SendContext}} Allows the connection to write a UDP datagram into
the \QuicWriter{} instance and specify the \class{EndPoint} to which the UDP datagram should be
sent. The \class{SendContext} instance contains additional data like the current timestamp.

  \ditemmethod[\keyword]{void}{ReceiveData}{\QuicReader{}, \class{EndPoint},
\class{QuicSocketContext.RecvContext}} Processes a datagram from the provided \QuicReader{}
instance. The \class{Recv\allowbreak{}Context} instance contains additional data like current
timestamp.

  \ditemmethod[\keyword]{long}{GetNextTimerTimestamp}{} Retrieves the timestamp when the next
internal timer of expires. Examples of such timers are loss detection timer, draining timer before
closing the connection, or pacing timer, which evens out the outbound packet flow. When the timer
expires, the \ManagedQuicConnection{} instance may have more data to send.

\end{description}

The value of \method{GetNextTimerTimestamp} is then used to suspend the background processing thread
in order not to consume CPU resources. The background processing thread waits until either the timer
expires or until a new QUIC packet arrives. However, some application code actions like writing data
to stream require interrupting the wait. This is achieved by
calling the \texttt{\QuicConnectionContext{}.\method{WakeUp}()} method from the
\ManagedQuicConnection{}.

\subsection{Managing Packet Number Spaces}

The \ManagedQuicConnection{} class maintains an internal array of three \PacketNumberSpace{}
instances which encapsulate all state relevant for individual packet number spaces. The data
maintained in the \PacketNumberSpace{} include:

\begin{itemize}

  \item next packet number to be sent,

  \item \PacketNumberWindow{} of received packet numbers,

  \item largest received packet number and timestamp when the packet was received,

  \item \RangeSet{} of received packet numbers that are not yet acknowledged,

  \item whether an \gls{ack-eliciting-packet} was received and an \ACK{} needs to be sent,

  \item \CryptoSeal{} instances for protecting and unprotecting QUIC packets, and

  \item \SendStream{} and \ReceiveStream{} for buffering cryptographic data from TLS to be sent in
\CRYPTO{} frames.

\end{itemize}

The data in \PacketNumberSpace{}s is updated by the \ManagedQuicConnection{} each time a QUIC packet
is sent or received.

\subsection{Packet Loss Detection, Recovery and Congestion Control}\label{sec:05-recovery}

The implementation of loss detection and recovery are delegated to \RecoveryController{} class to
allow easier testing. Each sent packet is represented by an instance of \SentPacket{} which contains
information about the packet which is relevant to the loss detection and recovery algorithms, such
as packet number, the timestamp when the packet was sent, packet number ranges acknowledged by the
packet, size of the packet, and list of data ranges sent in \STREAM{} frames.

Similarly to \PacketNumberSpace{} class used to maintain connection-wide state for each packet
number space, the \RecoveryController{} maintains an array of
\texttt{\RecoveryController{}.\PacketNumberSpace{}} instances which contain about each packet number
space that are relevant only for recovery purposes. These include:

\begin{itemize}

  \item largest packet number acknowledged by the peer,

  \item timestap of last \gls{ack-eliciting-packet} sent,

  \item \SentPacket{}s awaiting acknowlegement,

  \item \SentPacket{}s that are newly acknowledged, and

  \item \SentPacket{}s that are newly considered lost,

\end{itemize}

Additionally, the \RecoveryController{} maintains some data that is shared across all packet number
spaces. These are mostly data relevant for congestion control algorithm and
\textit{\gls{packet-pacing}}. Packet-pacing is a mechanism that evens out outgoing packets to
prevent \textit{\gls{micro-bursting}} --- a phenomenon in which packets arrive in short rapid bursts
which may overflow the receiver and cause packet loss. The data managed by \RecoveryController{}
itself include:

\begin{itemize}

  \item estimates of the current round trip time,

  \item timestamp of the next packet loss event,

  \item timestamp when the last UDP datagram was sent,

  \item size of the last sent UDP datagram,

  \item number of bytes currently in-flight,

  \item current size of the congestion window, and

  \item \ICongestionController{} instance implementing the selected algorithm for congestion
control.

\end{itemize}

The main interface methods exposed to the \ManagedQuicConnection{} implementation are:

\begin{description}

      \ditemproperty[\keyword]{long}{LossRecoveryTimer}{\propget} Timestamp when the next packet
loss will occur unless an \ACK{} from peer is received.

      \ditemmethod[\keyword]{void}{OnLossDetectionTimeout}{} Performs loss detection and populates
collections of \SentPacket{} instances on appropriate \PacketNumberSpace{} instance with packets
which are now considered lost.

      \ditemmethod[\keyword]{void}{OnPacketSent}{\enum{PacketSpace} space, \SentPacket{} packet}
Registers the \SentPacket{} instance as sent and tracks it in loss detection algorithm.

      \ditemmethod[\keyword]{void}{OnAckReceived}{\enum{PacketSpace} space, \RangeSet{}
acknowledged} Acknowledges packet numbers from provided \RangeSet{} and moves appropriate
\SentPacket{} instances the collection of newly acknowledged packets.

      \ditemmethod[\keyword]{int}{GetSendingAllowance}{\Long{} timestamp} Gets the maximum size of a
UDP datagram that the pacer will allow to be sent at the given timestamp.

        \ditemmethod[\keyword]{long}{GetPacingTimerForNextFullPacket}{} Gets timestamp when the
pacer will allow sending next QUIC packet of maximum size\footnote{The maximum size of an outgoing
QUIC packet depends on multiple factors. Most significantly, it must be small enough to avoid
fragmentation of the UDP datagram by lower network layers. The other endpoint can also set an upper
limit on the UDP datagram size it is willing to receive. This limit is provided using the
\MaxUdpPayloadSize{} transport parameter during the connection handshake.}.

\end{description}

\subsection{QUIC Stream Management}

The management of QUIC streams is delegated to \StreamCollection{} class. The \StreamCollection{}
tracks already created streams by their type and checks that neither endpoint exceeds the maximum
number of created streams. It also implements efficient lookup of \ManagedQuicStream{} instances by
their Stream IDs and tracks queues of streams with QUIC frames to be sent in the following QUIC
packets.

The \StreamCollection{} class maintains two independent queues of \ManagedQuicStream{} instances:

\begin{itemize}

  \litem{Flushable} Streams that have data to send within the flow control limits on that stream.

  \litem{Updateable} Streams for which some other than the \STREAM{} frame needs to be sent. This
includes updating flow control limits or aborting the stream.

\end{itemize}

The reasoning for a separate queue for flushable streams is that when composing a packet, \STREAM{}
frames should be the last frames written into the QUIC packet and fill all remaining space in the
datagram. By processing the updateable queue first, the implementation ensures that updates for all
streams --- such as flow control limits --- are sent as soon as possible.

\subsection{Packet Encryption}

Applying and removing packet protection is delegated to \CryptoSeal{} class, as described in
\autoref{sec:03-packet-protection}. The \CryptoSeal{} class implements only the logic independent of
the specific \gls{aead} cipher used. The steps which are specific to each \gls{aead} cipher, such as
the actual in-place encryption and decryption and calculating the header protection mask, are
delegeated to an implementation of \CryptoSealAlgorithm{} abstract class. Each supported \gls{aead}
cipher has its own \CryptoSealAlgorithm{} implementation.

The interface exposed by \CryptoSeal{} to the \ManagedQuicConnection{} consists of following
methods:

\begin{description}

    \ditemmethod[\keyword]{void}{ProtectPacket}{\SpanOf{\Byte{}} packet, \Int{} pnOffset, \Long{}
pn} Applies packet payload protection and writes \gls{aead} integrity tag at the end of the packet.

    \ditemmethod[\keyword]{void}{ProtectHeader}{\SpanOf{\Byte{}} packet, \Int{} pnOffset} Applies
header protection.

    \ditemmethod[\keyword]{void}{UnprotectHeader}{\SpanOf{\Byte{}} packet, \Int{} pnOffset} Removes
header protection.

    \ditemmethod[\keyword]{bool}{UnprotectPacket}{\SpanOf{\Byte{}} packet, \Int{} pnOffset, \Long{}
expectedPn} Attempts to remove the payload protection, returns \keyword{true} on success.

\end{description}

The meaning of the method arguments has been left out for brevity, but their purpose should be
evident from their usage in source code.

\subsection{Incoming QUIC Packet Processing}

The majority of the \ManagedQuicConnection{} implementation is focused on processing QUIC packets
and the QUIC frames they contain. Processing of a QUIC packet can end with one of three possible
results which are represented by the \ProcessPacketResult{} enum:

\begin{itemize}

  \litem[]{\texttt{Ok}} Packet was processed without errors.

  \litem[]{\texttt{DropPacket}} Packet should be discarded without informing the peer.

  \litem[]{\texttt{Error}} Received packet violates the protocol and the connection should be closed
with an error code.

\end{itemize}

The process of receiving the UDP datagram with QUIC packets begins in the
\texttt{\ManagedQuicConnection{}.\method{ReceiveData}} method. The individual QUIC packets are
processed independently one after another using the following steps:

\begin{enumerate}

  \item detect the packet type,

  \item remove packet protection,

  \item parse and validate the packet header fields,

  \item check if the packet with same packet number has already been received,

  \item register the packet for future acknowledgment, and

  \item parse and process all contained QUIC frames.

\end{enumerate}

Because the parsed QUIC frames are represented using \keyword{ref struct}s (as explained in
\autoref{sec:03-data-representation}), QUIC frames contained in the QUIC packet are parsed and
immediately processed one by one, independently of the other QUIC frames. The code processing the
individual frame types is organized into separate methods --- one for each frame type --- for better
maintainability.

\subsection{Generating Outgoing QUIC Packets}

The logic which generates outgoing UDP datagrams starts in the
\texttt{\ManagedQuicConnection{}.\method{SendData}} method. When generating outgoing packets, the
implementation must first determine whether it has any data to send and, if so, in which QUIC packet
type it should be sent. This logic is implemented in the
\texttt{\ManagedQuicConnection{}.\method{GetWriteLevel}} method. Once the packet type to be sent is
known, the generation of the actual QUIC packet consists of the following steps:

\begin{enumerate}

  \item determine the maximum size of the packet that can be sent,

  \item compose the packet header,

  \item write QUIC frames into the packet payload up to the available packet size,

  \item add padding to the packet if necessary\footnote{Header protection (described in
\autoref{sec:02-packet-protection}) uses bytes 5 to 20 from the packet payload as a sample for
generating the header protection mask. The packet payload includes the packet number, QUIC frames
and \SI{16}{\byte} \gls{aead} integrity tag. This implies that packet number and QUIC frames
together must be at least \SI{4}{\byte} long.},

  \item apply packet protection, and

  \item add the packet to be tracked by the \RecoveryController{}.

\end{enumerate}

The order in which the QUIC frames are generated is based on the frames' relative importance to make
sure that the important packets fit into the packet and are not unnecessarily postponed. Most
importantly, \ACK{} frames are written first to avoid delaying acknowledgments, and \STREAM{} frames
are written last to use up the remainder of the available space.

\section{TLS Integration}

The TLS handshake related logic of the \ManagedQuicConnection{} class is delegated to an
implementation of \ITls{} interface. The \ITls{} interface defines methods needed by
\ManagedQuicConnection{}. These methods include:

\begin{description}

  \ditemmethod[\keyword]{bool}{TryAdvanceHandshake}{} Advances the TLS handshake by calling methods
on the \ManagedQuicConnection{} (listed later).

  \ditemmethod[\keyword]{void}{OnHandshakeDataReceived}{\enum{EncryptionLevel} level,
\ReadOnlySpanOf{\Byte{}} data} Provides the TLS implementation with data received from the peer via
\CRYPTO{} frames.

  \ditemmethod[\enum]{TlsCipherSuite}{GetNegotiatedCipher}{} Gets the identifier of the \gls{aead}
cipher that was negotiated during the TLS handshake.

  \ditemmethod{TransportParameters}{GetPeerTransportParameters}{} Returns a \TransportParameters{}
instance which contains the transport parameters set by the peer for this connection.

  \ditemproperty[\keyword]{bool}{IsHandshakeComplete}{\propget} Returns true if the TLS handshake is
considered complete by the TLS implementation.

\end{description}

Implementations of \ITls{} are expected to maintain a reference to the \ManagedQuicConnection{} and
call following functions from the \method{TryAdvanceHandshake} method when appropriate:

\begin{description}

  \ditemmethod[\keyword]{void}{SetEncryptionSecrets}{\enum{EncryptionLevel} level,
\ReadOnlySpanOf{\Byte{}} read, \\ \indent{} \ReadOnlySpanOf{\Byte{}} write} Provides the
\ManagedQuicConnection{} with read and write secrets negotiated for the specified encryption level.
The encryption level refers to one of the four encrypted QUIC packet: \texttt{Initial},
\texttt{Handshake}, \texttt{Application} (1-RTT) and \texttt{EarlyData} (0-RTT).

  \ditemmethod[\keyword]{void}{AddHandshakeData}{\enum{EncryptionLevel} level,
\ReadOnlySpanOf{\Byte{}} data} Adds TLS handshake data to be sent to the peer via \CRYPTO{} frames.
The encryption level specifies which QUIC packet should be used to send the \CRYPTO{} frame. This
method can be called multiple times.

  \ditemmethod[\keyword]{void}{FlushHandshakeData}{} Called after \method{AddHandshakeData} to
inform \ManagedQuicConnection{} that the handshake data should be sent.

  \ditemmethod[\keyword]{void}{SendTlsAlert}{\enum{EncryptionLevel} level, \Int{} alertCode} When
TLS implementation encounters an error, this function is used to provide the TLS alert code which is
then used to construct a \CONNECTIONCLOSE{} frame for terminating the connection.

\end{description}

There are two \ITls{} interface implementations:

\begin{itemize}

  \litem[]{\OpenSslTls{}} Backed by modified \libopenssl{} with QUIC-enabling API maintained by
Akamai~\cite{AkamaiOpensslGithub}. This implementation can be used to interoperate with other QUIC
implementations and is used by the \texttt{\QuicImplementationProviders{}\allowbreak{}.Managed}
provider.

  \litem[]{\MockTls{}} Intended to be used when runing functional tests in the continuous
integration environment where the modified \libopenssl{} library is not available. This
implementation cannot be used to interoperate with other QUIC implementations. \MockTls{} is used by
the \texttt{\QuicImplementationProviders{}\allowbreak{}.ManagedMockTls} provider.

\end{itemize}

The integration with TLS uses the \gls{abstract-factory} to create new instances of \ITls{} when
needed. The \ManagedQuicImplementationProvider{} instance keeps a reference to \QuicTlsProvider{}
which is used to create new \ITls{} instances for \ManagedQuicConnection{}s. The concrete factory
classes are realized by the \OpenSslQuicTlsProvider{} and \MockQuicTlsProvider{} classes.

Following subsections describe the two \ITls{} implementations in greater detail.

\subsection{OpenSslTls Implementation}

The \OpenSslTls{} class is a managed wrapper around the \SSL{} class from \libopenssl{} library
written in C. As per the established practice in the \dotnet{} runtime repository, the managed QUIC
implementation separates the definitions of the \keyword{extern} methods from the \libopenssl{}
library into a separate \class{Interop.OpenSslQuic} class.

The primary classes used from the \libopenssl{} library are \SSLCTX{} which is a top level object
maintaining global SSL/TLS configuration, and \SSL{} which represents one SSL/TLS session. The
\OpenSslTls{} implementation uses one global \SSLCTX{} object, and one \SSL{} object for each
\OpenSslTls{} instance.

The main part of the QUIC-enabling API exposed by the \libopenssl{} library consists of
\method{SSL_set_quic_method} function which registers callbacks from the TLS handshake state
machine. \autoref{lst:05-c-quic-method} lists the definition of the \libopenssl{} QUIC callback
functions (written in C).

\begin{myListingC}{lst:05-c-quic-method}{Callback definitions in OpenSSL library source code}{ssl_quic_method_st,SSL_QUIC_METHOD,SSL,SSL_CTX,uint8_t,size_t}{OSSL_ENCRYPTION_LEVEL}
struct ssl_quic_method_st {
    int (*set_encryption_secrets)(SSL *ssl, OSSL_ENCRYPTION_LEVEL level,
                                  const uint8_t *read_secret,
                                  const uint8_t *write_secret,
                                  size_t secret_len);
    int (*add_handshake_data)(SSL *ssl, OSSL_ENCRYPTION_LEVEL level,
                              const uint8_t *data, size_t len);
    int (*flush_flight)(SSL *ssl);
    int (*send_alert)(SSL *ssl, enum OSSL_ENCRYPTION_LEVEL level,
                      uint8_t alert);
};
typedef struct ssl_quic_method_st SSL_QUIC_METHOD;

int |SSL_CTX_set_quic_method|(SSL_CTX *ctx, const SSL_QUIC_METHOD *quic_method);
\end{myListingC}

\autoref{lst:05-csharp-quic-method} lists the definition of the \class{QuicMethodCallbacks} defined
by the interop layer of the managed QUIC implementation which mirrors the \class{ssl_quic_method_st}
on the managed \dotnet{} side. The \class{QuicMethodCallbacks} class uses the \csharp{} function
pointers feature introduced in \dotnet{}~5 to represent unmanaged pointers to \csharp{} methods
\footnote{More information about \csharp{} function pointers and other native code interop
  improvements done in \dotnet{}~5 can be found on Microsoft dev
  blog~\cite{dotnet5interopimprovements}.}.

\begin{myListingCsharpNoPageBreak}{lst:05-csharp-quic-method}{[\csharp{} mirror of the \class{ssl_quic_method_st} C struct]\csharp{} mirror of the \class{ssl_quic_method_st} C struct. The comments above the fields list C-like definition of the function pointer type.}{DllImport,IntPtr, UIntPtr, StructLayout, QuicMethodCallbacks,Libraries}{OpenSslEncryptionLevel,LayoutKind}
[DllImport(Libraries.Ssl, EntryPoint = "SSL_set_quic_method")]
internal static extern unsafe int |SslSetQuicMethod|(IntPtr ssl,
    QuicMethodCallbacks* methods);

[StructLayout(LayoutKind.Sequential)]
internal unsafe struct QuicMethodCallbacks
{
    // int (*)(IntPtr ssl, OpenSslEncryptionLevel level, byte* readSecret,
    //         byte* writeSecret, UIntPtr secretLen)
    internal delegate* unmanaged[Cdecl]<IntPtr, OpenSslEncryptionLevel,
        byte*, byte*, UIntPtr, int> SetEncryptionSecrets;

    // int (*)(IntPtr ssl, OpenSslEncryptionLevel level, byte* data,
    //         UIntPtr len)
    internal delegate* unmanaged[Cdecl]<IntPtr, OpenSslEncryptionLevel,
        byte*, UIntPtr, int> AddHandshakeData;

    // int (*)(IntPtr ssl)
    internal delegate* unmanaged[Cdecl]<IntPtr, int> FlushFlight;

    // int (*)(IntPtr ssl, OpenSslEncryptionLevel level, byte alert)
    internal delegate* unmanaged[Cdecl]<IntPtr, OpenSslEncryptionLevel, byte,
        int> SendAlert;
}
\end{myListingCsharpNoPageBreak}

The pointer to \class{QuicMethodCallbacks} structure passed to the \method{SSL_set_quic_method}
function must be valid throughout the lifetime of the \SSLCTX{} object. Therefore, the \OpenSslTls{}
class allocates the \class{QuicMethodCallbacks} instance into an unmanaged memory region using the
\texttt{\class{Marshal}.\method{AllocHGlobal}()} method and stores the resulting pointer in a
\keyword{private static} field.

In order to be able to create unmanaged function pointers to \csharp{} methods, the actual methods
must be annotated using the \class{UnmanagedCallersOnlyAttribute} specifying that the caller will
use the \texttt{cdecl} calling convention.

The last piece of the \libopenssl{} callback integration is calling apropriate methods on the
\ManagedQuicConnection{} class from the callbacks. Because the callbacks are converted to unmanaged
function pointers, they must be \keyword{static} methods. Their implementation must, therefore,
translate the \texttt{\SSL{}*} pointer to the original \OpenSslTls{} instance. This is done by
allocating a \class{GCHandle} for the \OpenSslTls{} instance and storing the unmanaged pointer to
the handle as user data inside the \SSL{} instance.

\autoref{fig:05-openssl-callback} illustrates the process of calling the \method{AddHandshakeData}
callback as an example. The \ManagedQuicConnection{} instance tries to advance the TLS handshake by
calling \method{TryAdvanceHandshake} which in turn calls the \method{SSL_do_handshake} native method
in the \libopenssl{} library. This invokes the next step in the internal state automaton and as a
result, the \libopenssl{} calls the static \texttt{\OpenSslTls{}.\method{AddHandshakeData}} method
which was registered as the \texttt{add_handshake_data} callback. This function retrieves the
\class{GCHandle} from the \SSL{} instance passed in the callback and invokes the
\method{AddHandshakeData} method on the original \ManagedQuicConnection{} instance.

\begin{myFigure}{fig:05-openssl-callback}{Callback integration with \libopenssl{} library}

  \resizebox{\linewidth}{!}{\input{img/05-openssl-callback.pdf_tex}}

\end{myFigure}

\subsection{MockTls Implementation}

The \MockTls{} implementation is used for running tests without the dependency on the \libopenssl{}
library and is not intended for use in production environment. The implementation imitates the
\OpenSslTls{} behavior during a successful TLS handshake and exchanges randomly generated
secrets.

\section{ManagedQuicListener Implementation}

The \ManagedQuicListener{} class is only a simple wrapper around the \QuicServerSocketContext{}
instance and a \ChannelOf{\ManagedQuicConnection{}} of newly accepted connections. The
\QuicServerSocketContext{} implementation observes changes in the
\texttt{\ManagedQuicConnection{}.ConnectionState} property and inserts the newly established
connections into the \class{Channel}. The application later retrieves these connections using the
\texttt{\ManagedQuicListener{}.\method{AcceptConnection\allowbreak{}Async()}} method.

When the \ManagedQuicListener{} is closed, it is necessary to keep the currently established
connections alive. Disposing the \ManagedQuicListener{}, therefore, means only that the
\QuicServerSocketContext{} instance will not accept new connections. The \QuicServerSocketContext{}
class tracks active connections and is disposed when the last \ManagedQuicConnection{} is closed.

\section{ManagedQuicStream Implementation}

As described in \autoref{sec:03-stream-implementation}, the behavior of sending and receiving parts
of a QUIC stream is implemented by the \ReceiveStream{} and \SendStream{} classes. The
\ManagedQuicStream{} class only checks the validity of arguments passed to the public API and
informs the \ManagedQuicConnection{} about the amount of data read/written to update flow control
limits.

\subsection{ReceiveStream Implementation}

Buffering and control flow considerations for the \ReceiveStream{} class have been analyzed in
detail in \autoref{sec:03-receive-stream}. The \ReceiveStream{} class exposes the following
interface to be used by the \ManagedQuicStream{} class from the application thread:

\begin{description}

  \ditemproperty[\texttt]{\Long{}?}{Error}{\propget} If the sender aborted the stream, this property
contains the application-level error code to be reported to the application.

  \ditemmethod[\keyword]{void}{RequestAbort}{\Long{} errorCode} Requests that the stream is aborted
by the sender. Results in sending a \STOPSENDING{} frame with the specified application-level error
code.

  \ditemmethod[\keyword]{int}{Deliver}{\SpanOf{\Byte{}} destination} Copies application data from
the stream into the provided \SpanOf{\Byte{}}. Blocks until at least some data are available.

  \ditemmethod[]{\ValueTaskOf{\Int{}}}{DeliverAsync}{\MemoryOf{\Byte{}} destination,
\CancellationToken{}} Asynchronous version of the \method{Deliver} method.

\end{description}

From the internal background processing thread, the main methods called from the
\ManagedQuicConnection{} are:

\begin{description}

  \ditemmethod[\keyword]{void}{OnResetStream}{\Long{} errorCode} Called when \RESETSTREAM{} frame
for this stream was received.

  \ditemmethod[\keyword]{void}{Receive}{\Long{} offset, \ReadOnlySpanOf{\Byte{}} data, \bool{} fin}
Called when \STREAM{} frame has been received.

\end{description}

\subsection{SendStream Implementation}

Buffering and control flow considerations for the \SendStream{} class have been analyzed in detail
in \autoref{sec:03-send-stream}. The \SendStream{} class exposes the following interface to be used
by the \ManagedQuicStream{} class from the application thread:

\begin{description}

  \ditemmethod[\keyword]{int}{Enqueue}{\ReadOnlySpanOf{\Byte{}} data} Adds the provided application
data into the stream. If the internal buffering capacity is full, this method blocks until the data
can be buffered.

  \ditemmethod[]{\ValueTaskOf{\Int}}{EnqueueAsync}{\ReadOnlyMemoryOf{\Byte{}} data} Asynchronous
version of the \method{Enqueue} method.

  \ditemmethod[\keyword]{void}{MarkEndOfData}{} Marks the stream as finished. No more data can be
written into the stream after calling this function. The \texttt{FIN} bit will be set in the
appropriate \STREAM{} frame.

  \ditemmethod[\keyword]{void}{RequestAbort}{\Long{} errorCode} Requests that the stream be aborted
with specified errorCode. This method is also called when \STOPSENDING{} frame was received for this
stream.

\end{description}

From the internal background processing thread, the main methods called from the
\ManagedQuicConnection{} are:

\begin{description}

    \ditemmethod[\texttt]{(\Long{} offset, \Long{} count)}{GetNextSendableRange}{} Returns the next
consecutive range of data that is in \textit{pending} state (see \autoref{sec:03-send-stream}). The
next data to be sent in STREAM{} frame will be from this range.

    \ditemmethod[\keyword]{void}{CheckOut}{\SpanOf{\Byte{}} destination} Copies the data from the
range returned from \method{GetNextSendableRange} into the destination buffer. The copied data is
transitoned into \textit{in-flight} state.

    \ditemmethod[\keyword]{void}{OnLost}{\Long{} offset, \Long{} count} Marks the range of data as
lost, transitioning them back into \textit{pending} state.

    \ditemmethod[\keyword]{void}{OnAck}{\Long{} offset, \Long{} count, \bool{} fin} Marks the range
of data as \textit{acknowleged}. Allowing the underlying buffers to be reused.

\end{description}

\subsection{Tracing and Diagnostics}

The managed QUIC implementation can produce verbose logs (called traces) to provide an insight into
the implementation's behavior. By default, no trace is generated. Generating a trace can be enabled
by defining the \texttt{DOTNETQUIC_TRACE} to one of the following values:

\begin{description}

    \ditem{\texttt{console}} Produces logs to the console output. This type of tracing produces a
large amount of console logs, which are hard to reason about. However, the console logs are the
fastest way of checking if any packets are sent/received or for diagnosing refused connections.

\ditem{\texttt{qlog}} Produces traces which can be visualized by the \textit{qvis}~\cite{web:qvis}
tool. The \textit{qvis} tool provides deep insight into the connection, such as the timeline of
sent/received packets, bytes in flight, size of the congestion window, and latency throughout the
lifetime of the connection. The traces are written into the current working directory into a
separate file for each connection. Note that for very busy connections, the trace files can grow
into hundreds of megabytes in a few seconds.

\end{description}

Both types of tracing incur a noticeable overhead and should not be used in the production
environment.

\section{Tests Implementation}

The QUIC implementation is covered by an extensive suite of unit tests and functional tests. The
unit tests are located in the \filename{System.Net.Quic/tests/UnitTests} directory and focus on the
correctness of the individual parts of the managed QUIC implementation. The unit tests also check
that the QUIC packets sent by the implementation conform to the QUIC protocol specification.

The functional tests are located in the
\filename{System.Net.Quic/tests/Functional}\allowbreak{}\filename{Tests} directory and test
high-level functionality like being able to send and receive data, and that the public API behavior
conforms to the public API specification.

\subsection{Unit Tests}

The most important part of the unit tests are tests that inspect the contents of the individual QUIC
packets sent between the server and client. In order to simplify writing these tests, the
\filename{System.Net.Quic.Unit.Tests} project implements a testing harness, which provides helper
methods for inspecting and modifying QUIC packets, as discussed in \autoref{sec:03-testing-harness}.
The test harness functionality is provided as a \ManualTransmissionQuicTestBase{} class intended to
be used as a base class for unit test classes. This class provides:

\begin{itemize}

  \item prepared connected \ManagedQuicConnection{} instances;

  \item automatic logging of sent QUIC packets to the test output;

  \item manual time stepping to make tests deterministic; and

  \item interception callbacks for inspecting and possibly modifying QUIC packets in-flight;

\end{itemize}

Because the \ManagedQuicConnection{} implementation uses \keyword{ref struct}s to represent
individual QUIC frames. The testing harness must duplicate the types for QUIC frames and QUIC
packets as classes. The classes used to represent QUIC packets and QUIC frames in test harness
derive from \class{FrameBase} and \class{PacketBase} abstract classes, which provide a common type
for aggregating QUIC packets and QUIC frames in collections.

The classes representing QUIC frames and QUIC packets are mutable, which allows modification by the
testing code to elicit a particular error response. \autoref{fig:05-test-harness-intercept}
illustrates the process. First, the \ManagedQuicConnection{} produces the UDP datagram containing
the QUIC packets to be sent. The testing harness then parses the datagram into mutable
representation. The parsed packets are then passed into the inspection callback provided by the test
method, which can perform any assertions and modifications. The QUIC packet is then serialized back
into binary representation and provided to the receiver \ManagedQuicConnection{} instance.

\begin{myFigure}{fig:05-test-harness-intercept}{Intercepting a QUIC packet by the testing harness}

  \resizebox{\textwidth}{!}{\input{img/05-test-harness-intercept.pdf_tex}}

\end{myFigure}

\subsection{Functional Tests}

The \filename{System.Net.Quic} project contains a large suite of functional tests against the public
QUIC API\@. The tests are structured in a way that allows them to be run against all QUIC
implementation providers in the library. Also, the \QuicStream{} class is tested using
\textit{stream conformance tests} --- a suite of tests run for each implementation of the \Stream{}
class to ensure consistent behavior of all implementations.

In order to allow running tests against different implementation providers, the test methods are
defined in \QuicListenerTests{}, \QuicConnectionTests{} and \QuicStreamTests{} generic classes,
which share a common base class \QuicTestBase{}. The type argument for all four of these classes is
constrained to be an implementation of \interface{IQuicImplProviderFactory} interface which allows
creating \QuicListener{} and \QuicConnection{} instances with desired implementation provider.

The \xUnit{} framework does not run tests from generic classes by itself. Instead, the generic test
suite must be ``instantiated'' by declaring a non-generic class deriving from the generic type. For
example, \class{QuicStreamTests_Managed\allowbreak{}Provider} derives from the
\genericClass{QuicStreamTests}{\class{ManagedProvider\allowbreak{}Factory}} class which makes
\xUnit{} run the tests from \QuicStreamTests{} class against the
\texttt{\QuicImplementationProviders{}.Managed} provider. These specific test classes are also
annotated by \class{ConditionalClassAttribute} which instructs \xUnit{} to run the tests only if the
given provider is supported in the current environment.

A similar mechanism is used to run stream conformance tests against all QUIC implementation
providers.
