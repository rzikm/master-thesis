\chapter{Developer Documentation}

The Managed QUIC implementation developed in this thesis is contained in the
\filename{System.Net.Quic} project in a fork of the official \dotnet{} runtime repository. The
source code is attached in \todo{path to the src in attachments.}

The source code for the \filename{System.Net.Quic} project is located inside the
\filename{src/dotnet-runtime/src/libraries/System.Net.Quic/} directory. The directory listing in
\autoref{lst:05-system-net-quic-structure} shows the structure of the project directory. The figure
also emphasises the \filename{Managed} and \filename{UnitTests} directories which contain the code
developed as part of this thesis. These directories are the main focus of this chapter.

\newcommand{\mydtcomment}[1]{\DTcomment{#1}}

% use smaller width to make the tree more compact
\DTsetlength{0.2em}{0.6em}{0.2em}{0.4pt}{1.6pt}
\renewcommand{\DTstylecomment}[1]{{\footnotesize\rmfamily #1}}
\renewcommand{\DTstyle}[1]{{\footnotesize\texttt{#1}}}
\definecolor{dtemphcolor}{rgb}{0,0,0.75}
\newcommand{\dtemph}[1]{\textcolor{dtemphcolor}{\emph{#1}}}

\begin{myListing}[Directory structure of the System.Net.Quic project.]{lst:05-system-net-quic-structure}{Directory structure of the System.Net.Quic project. Emphasised items contain the implementation developed in this thesis.}
\dirtree{%
  .1 {src/dotnet-runtime/src/libraries/System.Net.Quic/}.
  .2 {ref}\mydtcomment{Refererence assembly code}.
  .2 {src}\mydtcomment{Main library source code}.
  .3 {Resources}.
  .4 {Strings.resx}\mydtcomment{Definition of localizable strings like exception messages.}.
  .3 {System}.
  .4 {Net}.
  .5 {Quic}.
  .6 {Implementations}\mydtcomment{Root directory for all QUIC implementations}.
  .7 {\dtemph{Managed}}\mydtcomment{\dtemph{This thesis' implementation sources}}.
  .7 {MsQuic}\mydtcomment{\libmsquic{}-based implementation sources}.
  .7 {Mock}\mydtcomment{Mock implementation used only in tests}.
  .6 {Interop}\mydtcomment{Imports from native libraries}.
  .3 {System.Net.Quic.csproj}.
  .2 {tests}\mydtcomment{Library tests source code}.
  .3 {certs}\mydtcomment{X.509 certificates used in tests}.
  .4 {cert.crt}\mydtcomment{Public certificate file}.
  .4 {cert.key}\mydtcomment{Private key file}.
  .3 {FunctionalTests}\mydtcomment{Tests against the public API}.
  .4 {System.Net.Quic.Functional.Tests.csproj}.
  .3 {\dtemph{UnitTests}}\mydtcomment{\dtemph{Managed implementation unit tests}}.
  .4 {\dtemph{System.Net.Quic.Unit.Tests.csproj}}.
  .2 {Directory.Build.props}.
  .2 {System.Net.Quic.sln}.
}
\end{myListing}

\todo{sections for: development (environment setup, setup for debugging)? running tests?}

\section{QUIC Implementation Providers}

The \filename{System.Net.Quic} project contains multiple implementations of the QUIC protocol. This
is not achieved via polymorphism of the QUIC API classes, but rather using indirection to
\textit{implementation providers}. Each \QuicListener{}, \QuicConnection{}, and \QuicStream{}
instance contains a reference to a \QuicListenerProvider{}, \QuicConnectionProvider{}, or
\QuicStreamProvider{} instance, respectively. Implementations of each providers are provided by each
QUIC implementation in the \filename{System.Net.Quic} library. Figure
\autoref{fig:05-quic-impl-providers} illustrates this indirection layer using a class diagram.

\begin{myFigure}{fig:05-quic-impl-providers}{Implementation providers for the QUIC API classes}

  \todo{class diagram showing inheritance + grouping of classes + access modifiers}

\end{myFigure}

Creation of new \QuicListener{} and \QuicConnection{} instances is implemented using the
\gls{abstract-factory}~\cite{wiki:abstract-factory-pattern}. There are multiple implementations of
\QuicImplementationProvider{} class, one for each QUIC implementation. Instances of
\QuicImplementationProvider{} class are available as static properties of the
\class{QuicImplementationProviders} static class. There are five such providers:

\begin{itemize}

  \litem[]{\texttt{Managed}} Managed implementation with TLS backed by \libopenssl{} fork with QUIC enabling API.

  \litem[]{\texttt{ManagedMockTls}} Managed implementation with mock TLS, which does not depend on external TLS implementation, but cannot interoperate with other QUIC implementations.

  \litem[]{\texttt{MsQuic}} QUIC implementation backed by \libmsquic{} native library.

  \litem[]{\texttt{Mock}} A mock QUIC implementation for use in tests.

  \litem[]{\texttt{Default}} The default implementation. Same as \texttt{Managed}, but can be redirected to other implementations by setting \texttt{DOTNETQUIC_PROVIDER} environment variable to the desired provider name.

\end{itemize}

The \QuicListener{} and \QuicConnection{} classes have a constructor overload which accepts an
instance of the \QuicImplementationProvider{} to be used. This way, the QUIC implementation can be
selected during runtime. This also allows reusing a suite of functional tests for all
implementations by simply changing the implementation provider.

\section{Managed QUIC Implementation}

The source code for the managed implementation developed in this thesis is located under the
\filename{System.Net.Quic/src/System/Net/Quic/Implementations/Managed/} subdirectory.
\autoref{lst:05-managed-quic-structure} outlines the directory structure of the implementation.

\begin{myListing}{lst:05-managed-quic-structure}{Directory structure of the managed QUIC implementation}
\dirtree{%
  .1 {System.Net.Quic/src/System/Net/Quic/Implementations/Managed}.
  .2 {Internal}\mydtcomment{Internal code of the implementation}.
  .3 {Crypto}\mydtcomment{Cryptographic facilities}.
  .3 {Frames}\mydtcomment{Definition of QUIC frames}.
  .3 {Headers}\mydtcomment{Definition of QUIC packet headers}.
  .3 {Packets}\mydtcomment{QUIC packet number spaces handling}.
  .3 {Parsing}\mydtcomment{Parsing of QUIC primitives}.
  .3 {Recovery}\mydtcomment{Loss detection and recovery}.
  .3 {Sockets}\mydtcomment{Servicing socket IO}.
  .3 {Streams}\mydtcomment{Stream buffering}.
  .3 {Tls}\mydtcomment{TLS integration}.
  .4 {Mock}\mydtcomment{Mock TLS implementation}.
  .4 {OpenSsl}\mydtcomment{OpenSSL TLS integration}.
  .3 {Tracing}\mydtcomment{Tracing and logging facilities}.
  .2 {ManagedQuicConnection.cs}\mydtcomment{Implementation of public API}.
  .2 {ManagedQuicConnection.Frames.cs}\mydtcomment{Processing of QUIC frames}.
  .2 {ManagedQuicConnection.Packets.cs}\mydtcomment{Processing of QUIC packets}.
  .2 {ManagedQuicConnection.Recovery.cs}\mydtcomment{Handling packet loss}.
  .2 {ManagedQuicConnection.Stream.cs}\mydtcomment{Stream management}.
  .2 {ManagedQuicImplementationProvider.cs}\mydtcomment{Abstract factory}.
  .2 {ManagedQuicListener.cs}\mydtcomment{Implementation of public API}.
  .2 {ManagedQuicStream.cs}\mydtcomment{Implementation of public API}.
}
\end{myListing}

The implementation is exposed using the \ManagedQuicListener{}, \ManagedQuicConnection{},
\ManagedQuicStream{} implementation provider classes and the \ManagedQuicImplementationProvider{}
factory. The source code for these classes can be found in the root directory of the
implementation.

\section{Tests Implementation}
