\chapter{Evaluation}

In this chapter we evaluate the performance of the managed QUIC implementation developed as part of
this thesis and compare it to the existing \libmsquic{}-based implementation. We will also compare
both QUIC implementations to \class{SslStream} which uses a combination of TCP and TLS. Due to lack
of resources for proper end-to-end testing, we will run the performance tests locally on a single
machine. In order to evaluate the implementations in a non-ideal network environment, we will use
the Clumsy~\cite{clumsy} utility to simulate network issues such as lag and packet loss.

There are two performance characteristics which are relevant for networking protocol implementations:

\begin{itemize}

  \litem{Throughput} The rate at which the application data is sent. This metric is important mainly
for servers. High throughput indicates a capacity of a server to process large amounts of requests.

  \litem{Round Trip Time Latency} The delay between sending a request and receiving a response. This
metric is important mainly for clients. Rather than measuring average latency, a certain percentile
is measured. For example, if 99th percentile of latency is 10ms indicates that for 99\% requests,
the latency is at most 10ms.

\end{itemize}

\subsubsection{Evaluation Environment}

In order to obtain realistic measurements, our measurements will


\subsection{}

\todo{interoperability?}
\todo{performance?}

\todo{what we want to measure}

\todo{how we will measure it (high-level structure of the application)}

\todo{measurements}

\todo{conclusion}
