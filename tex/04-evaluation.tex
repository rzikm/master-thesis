\chapter{Evaluation}\label{chap:04-evaluation}

In this chapter, we evaluate the performance of the managed QUIC implementation developed in this
thesis and compare it to the existing \libmsquic{}-based implementation. We will also compare both
managed and \libmsquic{}-based QUIC implementations to \class{SslStream} which uses a combination of
TCP and TLS\@. In our evaluation, we will focus on two performance characteristics relevant for
network communication:

\begin{itemize}

        \litem{Throughput} The rate at which the application data are sent. High total throughput is
        especially important for servers when handling large amount of parallel connections.

        \litem{Round Trip Time Latency} The delay between sending a request and receiving a
        response. This metric is important mainly for clients, where lower latencies mean faster
        responses from the server. Rather than measuring the average latency, it is more common to
        measure, e.g., 99th percentile of latency which indicates minimum expected latencies in the
        slowest 1\% replies. This metric is important in the context of webpages and communication
        with browsers, especially for websites for which web browsers generate large number (even
        hundreds) of HTTP requests for a single page access. In such cases, the probability of at
        least one of the requests being slower than the 99th percentile raises significantly and,
        therefore, high value of 99th percentile would negatively affect the total page load time
        for significant number of users~\cite{Treat2015}.

\end{itemize}

\section{Evaluation Environment}

We have evaluated our implementation in two different environments. Our primary evaluation
environment consists of two blade servers on the same LAN network. The relevant software and
hardware parameters of the servers are:

\begin{itemize}

        \litem{CPU} Intel\textsuperscript{\textregistered{}} Xeon\textsuperscript{\textregistered{}}
        E3-1270 v6 @ \SI{3.80}{\giga\hertz} (\SI{4.20}{\giga\hertz} turbo boost, 4 cores, 2 threads
        per core)

        \litem{RAM} \SI{32}{\giga\byte} (\SI{2400}{\mega\hertz})

        \litem{Network} \SI[per-mode=symbol]{1}{\giga\bit\per\second} Ethernet connection.
        \todo{it's a shame we don't have 10Gbit}

        \litem{OS} Fedora 32

\end{itemize}

Unfortunately, the two servers are connected only by \SI[per-mode=symbol]{1}{\giga\bit\per\second}
Ethernet, which for TCP a theorethical upper limit for throughput at
\SI[per-mode=symbol]{117.5}{\mega\byte\per\second}, which was easily reached in our measurements.
Therefore, when appropriate, we also evaluated our implementation using a loopback network interface
on a desktop workstation PC with following parameters:

\begin{itemize}

        \litem{CPU} Intel\textsuperscript{\textregistered{}} Core\textsuperscript{\textregistered{}}
        i7-8700 @ \SI{3.20}{\giga\hertz} (\SI{4.60}{\giga\hertz} turbo boost, 6 cores, 2 threads per
        core)

        \litem{RAM} \SI{32}{\giga\byte} (\SI{2400}{\mega\hertz})

        \litem{OS} Windows 10 version 20H2\footnote{Update to Windows 10 20H2 includes significant
        TCP and UDP performance improvements~\cite{theregister20h2}. Measurements taken on older
        versions of Windows yield significantly different results.}

\end{itemize}

In both environments, the testing application was run in a a 64-bit process on preview
\dotnet{}~6.0.0 host (CoreCLR 6.0.20.55508, CoreFX 6.0.20.255508).

\subsection{Drawbacks of Measuring Loopback Performance}\label{sec:04-localhost-mtu}

Performing measurements over the loopback interface removes any upper limit on bandwidth because it
is implemented purely in software in the operating system. It also minimizes the transport latency
and guarantees zero packet loss. The software-based implementation of the loopback interface allows
optimizations which are not possible on real networks. As an example, on Windows operating system,
loopback connections can send \SI{64}{\kibi\byte} IP packets --- which is the maximum size allowed by
the IP protocol --- without any link-layer fragmentation. On the other hand, Ethernet connections can
carry only up to \SI{1500}{\byte} IP datagrams without fragmentation. The greater size of datagrams
on loopback interface reduces overhead for cases when TCP is used for inter-process communication on
the same machine.

Both managed and \libmsquic{}-based QUIC implementations send IP datagrams which are at most
\SI{1500}{\byte} even on loopback, because of the hardcoded limits in the codebase. In order to make
the loopback comparison more fair, we leveraged the ability to limit the maximum outbound IP packet
size to \SI{1500}{\byte} on Windows sockets API\@. From \csharp{}, this can be achieved using
following statement:

\lstsetcsharpsettings{}
\begin{lstlisting}[numbers=none,enums={SocketOptionLevel,SocketOptionName}]
socket.|SetSocketOption|(SocketOptionLevel.IP,
    (SocketOptionName)76 /* IP_USER_MTU */, 1500 /* size */);
\end{lstlisting}

However, the \texttt{IP_USER_MTU} option is only available on Windows 10 version 20H2 or newer. On
other platforms, the \method{SetSocketOption} call above would throw an exception. In order to keep
cross-platform compatibility, the attached source code does not contain this statement. However, it
was used to obtain results presented in this chapter.

\subsection{Running with MsQuic Support}

\libmsquic{}-based QUIC implementation does not work in the default installation of \dotnet{}
because \dotnet{} does not distribute the \libmsquic{} binary. Programs which want to use the
\libmsquic{}-based QUIC implementation should reference the
\texttt{System.Net\allowbreak{}.Experimental.MsQuic} NuGet
package~\cite{SystemNetQuicExperimentalMsquic} which provides the necessary \libmsquic{} binary.
However, the Windows \libmsquic{} binary distributed in the NuGet package requires a Windows Insider
Preview build of Windows because it depends on newer \libschannel{} version for TLS implementation
which is not available in mainstream Windows version.

Unfortunately, we cannot install a preview build of Windows on the workstation we will use for
running the tests. Fortunately, in order to support Linux, \libmsquic{} can also use the same
modified \libopenssl{} version for the TLS implementation as our managed QUIC implementation.
Eventhough building \libmsquic{} with \libopenssl{} for Windows is not officially supported, we were
able to produce such a build with small changes in the \libmsquic{} build configuration. Therefore,
the \libmsquic{} library used in experiments in this chapter has been compiled locally using source
code from commit \texttt{dc2a6cf0dd12e27} from the official \libmsquic{}
repository~\cite{msquicGithub} and configured to use the same \libopenssl{} library as the managed
QUIC implementation. It should be noted that our custom Windows build of \libmsquic{} may show
slightly different performance characteristics than the official build for Windows which uses the
\libschannel{} library.

On Linux, we could use the above mentioned
\texttt{System.Net\allowbreak{}.Experimental.\allowbreak{}MsQuic} NuGet package, but that would
complicate the build process because the package would have to be referenced only on Linux builds.
We have decided, therefore, to compile the Linux build of \libmsquic{} locally as well. Like the
Windows build, the Linux build of \libmsquic{} used in our tests was compiled from commit
\texttt{dc2a6cf0dd12e27}, but did not require changes to build configuration, as it uses the
modified \libopenssl{} by default.

All measurements of the \libmsquic{}-based QUIC support for \dotnet{} include any overhead
introduced by the interop layer bridging the native library API to \dotnet{} QUIC API\@. The
measured results, therefore, do not represent raw \libmsquic{} library performance, but the
performance observed by \dotnet{} applications which use the \libmsquic{}-based implementation
provider which can be validly compared to the performance of the our managed QUIC implementation.

% \subsection{Simulating Non-Ideal Network}

% \todo{this is relevant only if we use the desktop results as well, otherwise}

% In order to measure how implementations behave in presence of lag and packet loss, we used the
% Clumsy~\cite{clumsy} utility to simulate network problems. Clumsy is based on the
% WinDivert~\cite{WinDivert} library for capturing packets from the Windows network stack.
% \autoref{fig:04-clumsy-architecture} illustrates how Clumsy intercepts network traffic. The
% \texttt{WinDivert.sys} kernel driver intercepts incoming packets and checks them against a set of
% rules provided by Clumsy. Packets matching a rule are passed to Clumsy. Clumsy internally simulates
% network issues and reinjects packets into the Windows network stack as necessary.

% \begin{myFigure}{fig:04-clumsy-architecture}{Use of Clumsy to simulate network issues}

%   \resizebox{0.8\linewidth}{!}{\input{img/04-clumsy-architecture.pdf_tex}}

% \end{myFigure}

% However, the implementation of Clumsy uses only a single thread to process packets. This means that
% Clumsy becomes a bottleneck when processing large network load. \todo{mentioned that we could not
%   find anything better than clumsy} Since we were unable to find a better tool which would work on
% Windows, we will use Clumsy only in tests with a single connection where the single-threaded
% implementation of Clumsy has the least effect on the measurements.

\subsection{Benchmarking Application}\label{sec:04-benchmark-app}

The actual throughput and latency measurements are done using a dedicated benchmarking \dotnet{}
application whose source code can be found in the
\filename{src/supplementary/benchmark/ThroughputTests} directory in the attachments of this thesis.
The application implements a trivial echo server and clients which exchange messages of specified
size. The first parameter to the application selects one of the following modes:

\begin{description}

  \ditem{\texttt{server}} Starts only the server-side part of the application.

  \ditem{\texttt{client}} Starts only the client-side part of the application. In this mode, the
application spawns multiple clients and reports the measurement results on the standard output.

  \ditem{\texttt{inproc}} Starts both the server and client in the same process. All network
communication is done over the loopback network interface.

\end{description}

The benchmarking application accepts multiple parameters. List of the most important follows

\begin{description}

    \ditem{\texttt{-e, --endpoint}} The endpoint on which to listen (server) or to which to connect (client). Not applicable for \texttt{inproc} mode.

    \ditem{\texttt{--certificate-file}} Path to the X.509 certificate file.

    \ditem{\texttt{--key-file}} Path to the X.509 certificate private key file.

    \ditem{\texttt{-t, --tcp}} Use TCP instead of QUIC\@.

    \ditem{\texttt{-c, --connections}} The number of connetions to create.

    \ditem{\texttt{-s, --streams}} The number of streams to create in each connection. Applicable only when QUIC is used.

    \ditem{\texttt{-m, --message-size}} The size of sent messages in bytes.

    \ditem{\texttt{-w, --warmup-time}} Time before starting to take measurements.

    \ditem{\texttt{-d, --test-duration}} Time after which the measurement should stop.

    \ditem{\texttt{-i, --reporting-interval}} Delay between reports of intermediate measurements.

    \ditem{\texttt{-n --no-wait}} Whether clients wait for reply before sending another message.

\end{description}

The full list of parameters can be found in the \filename{Program.cs} file, or by running the
program with the \texttt{--help} parameter. By default, the implementation uses managed QUIC
implementation. Switching to \libmsquic{}-based implementation can be achieved by defining the
\texttt{DOTNETQUIC_PROVIDER} environment variable to \texttt{msquic}.

The client mode of the application switches between two behaviors. By default, clients do not wait
for reply from the server before sending another message. This lets us measure the total throughput
of the system. When using the \texttt{-n} switch, clients wait until a response from the server is
received and measure the latency as delay between sending the message and receiving the reply.

\section{Measurement Results}\label{sec:04-perf-results}

In the following subsections, we present the results of the individual performance experiments. The
throughput and latencies were measured as follows:

\begin{itemize}

  \litem{Throughput} Throughput of the entire server. This is the total amount of application data
  echoed back to clients across all connections.

  \litem{Latency} The delay between client writing the message to the \Stream{} and reading back the
full reply. In our tests, we will measure the 99th percentile of latency. The latency measurements
will be taken with the use of \texttt{-n} flag in the benchmark application.

\end{itemize}

We will perform three kinds of performance comparisions:

\begin{itemize}

        \litem{Multiple Parallel Streams Performance} These tests compare how the managed and
        \libmsquic{}-based QUIC implementations scale with increasing server load. These tests will
        utilize the stream multiplexing of QUIC to send messages using multiple parallel QUIC
        streams.

        \litem{Single Stream Performance} These tests use only a single stream in a connection.
        These tests should allow us to compare the performance between QUIC implementations and
        TCP+TLS-based \SslStream{}.

        \litem{Performance in Simulated Cellular Network} In these tests, we will try to approximate
        the characteristics of a real-world cellular network by increasing lag and packet loss in
        the LAN environment.

\end{itemize}

All experiments in this section, client and server parts of the benchmarking application were run in
separate processes. Each test case had a \SI{5}{\second} warm-up time and collected data for another
\SI{15}{\second}. These intervals were long enough to produce stable measurements across multiple
test runs.

In the benchmarking application, clients and server exchange messages of size specified by the
\texttt{-m} parameter. In our tests, we will use mainly two message sizes: \SI{256}{\byte} and
\SI{4096}{\byte}. The \SI{256}{\byte} message are small enough to fit into a single QUIC packet,
while the bigger \SI{4096}{\byte} messages are guaranteed to be split across multiple QUIC packets.
Our expectation is that increasing the message size should increase the latency because more packets
need to be send to send both the message and the reply. Also, increasing the message should increase
throughput because there are fewer calls for queueing the same amount of data into the stream.

\subsection{Multiple Parallel Streams Performance}\label{sec:04-multi-stream-perf}

The first set of tests compared the managed and \libmsquic{}-based QUIC implementations. These tests
were run with increasingly larger messages and with greater number of parallel connections and
streams to see how the two implementations scale.

\autoref{fig:04-lab-multi-stream-throughput} shows the measured throughput.
\autoref{fig:04-lab-multi-stream-throughput-a} shows the base-line performance when \SI{256}{\byte}
messages are sent using a single stream per connection. The other figures show measurements after
increasing message size to \SI{4096}{\byte} (\autoref{fig:04-lab-multi-stream-throughput-b}),
increasing number of streams to 32 (\autoref{fig:04-lab-multi-stream-throughput-c}), or both
(\autoref{fig:04-lab-multi-stream-throughput-d}). An immediate observation can be made that the
managed implementation produced very similar pattern in all four test runs, with the highest
throughput at 4 parallel connections and then decreasing. On the other hand, the \libmsquic{}-based
implementation maintains almost the same throughput regardless of the number of connections, but
increases throughput both when increasing the message size and number of streams in a connection.

\begin{myFigure}{fig:04-lab-multi-stream-throughput}{Multiple stream QUIC throughput measurements (LAN)}
\begin{mySubfigure}{0.49\linewidth}{fig:04-lab-multi-stream-throughput-a}{1 Stream, \SI{256}{\byte} Messages}
\footnotesize
\input{plots/04-lab-multi-stream-throughput-a.tex}
\end{mySubfigure}
\begin{mySubfigure}{0.49\linewidth}{fig:04-lab-multi-stream-throughput-b}{1 Stream, \SI{4096}{\byte} Messages}
\footnotesize
\input{plots/04-lab-multi-stream-throughput-b.tex}
\end{mySubfigure}

\begin{mySubfigure}{0.49\linewidth}{fig:04-lab-multi-stream-throughput-c}{32 Stream, \SI{256}{\byte} Messages}
\footnotesize
\input{plots/04-lab-multi-stream-throughput-c.tex}
\end{mySubfigure}
\begin{mySubfigure}{0.49\linewidth}{fig:04-lab-multi-stream-throughput-d}{32 Stream, \SI{4096}{\byte} Messages}
\footnotesize
\input{plots/04-lab-multi-stream-throughput-d.tex}
\end{mySubfigure}
\end{myFigure}

In case with \SI{4096}{\byte} messages, the \libmsquic{}-based implementation consistently saturates
the \SI{1}{\giga\bit} network connection between the two blade servers. To give more perspective on
the relative performance between the two implementation, \autoref{fig:04-multi-stream-throughput}
shows results of the two test runs on the Windows workstation over the loopback interface. Depending
on the number of connections, the throughput of the \libmsquic{}-based implementation was up to four
times greater than our implementation. Results of runs with \SI{256}{\byte} messages were left out
for brevity because they did not differ from the ones in LAN environment.

\begin{myFigure}{fig:04-multi-stream-throughput}{Multiple stream QUIC throughput measurements (Loopback)}
\begin{mySubfigure}{0.49\linewidth}{fig:04-multi-stream-throughput-b}{1 Stream, \SI{4096}{\byte} Messages}
\footnotesize
\input{plots/04-multi-stream-throughput-b.tex}
\end{mySubfigure}
\begin{mySubfigure}{0.49\linewidth}{fig:04-multi-stream-throughput-d}{32 Stream, \SI{4096}{\byte} Messages}
\footnotesize
\input{plots/04-multi-stream-throughput-d.tex}
\end{mySubfigure}
\end{myFigure}

The peak of the throughput of our managed QUIC implementation at four connections can be explained
by the fact that we are spawning a long-running background task for each connection. Four
connections lead to a number of parallel \class{Task}s that were enough to completely utilize the
entire CPU\@.

As for the decline of the throughput of managed QUIC implementation when 16 or more connections were
used, profiling showed that substantial amount of time was spent in the garbage collection. At 256
connections, up to 50\% of the total CPU time was spent in GC, divided into relatively large pauses
of \SI{20}{\milli\second} or more. The pauses introduced by GC led to QUIC packets being considered
lost which in turn led to collapse of the congestion window in the QUIC connection, further reducing
the rate at which data were sent. This points to the fact that eventhough we carefully avoided
needless allocations in our implementation, there are still enough sources of allocation left to
degrade the performance of a server under heavy load.

An example of a large source of allocations in the managed QUIC implementation is pooling the
buffers for sending or receiving QUIC packets. Even though the implementation uses
\ArrayPoolOf{\Byte}\texttt{.Shared} to reuse allocated buffers, this does not work well with large
number of connections. This is because \ArrayPoolOf{\Byte}\texttt{.Shared} retains only a small
number of \ArrayOf{\Byte} instances for pooling and lets GC collect the rest\footnote{In
  \dotnet{}~5, the implementation of \ArrayPoolOf{\Byte}\texttt{.Shared} maintains a separate pool
  for each CPU core. Each per-CPU core pool organizes pooled arrays into 17 buckets with array sizes
  ranging from \SI{16}{\byte} to \SI{1}{\mebi\byte}. Each bucket retains up to 8 array instances of
  similar size.}. With large number of connections, the buffers are often rented and returned in
large bursts, leading to a lot of large arrays being discarded upon return to the
\ArrayPoolOf{\Byte}\texttt{.Shared} and subsequently allocated anew.

Another source of allocations are \Socket{}\texttt{.\method{SendTo}} and
\Socket{}\texttt{.\method{ReceiveFrom}} methods which allocate of a new \class{EndPoint} instance
for each call. In order to remove this particular source of allocations, an allocation-free API for
sending or receiving UDP datagrams needs to be designed and implemented on the \Socket{} class.

\libmsquic{}-based implementation, shows similar throughput regardless of the number of parallel
connections. Instead, it greatly increases with message size and only slightly with the number of
streams in the connections. When we tried sending messages larger than \SI{4096}{\byte}, the total
throughput did not increase significantly anymore. Upon closer inspection of the \libmsquic{}
network traffic using Wireshark, it turns out that in tests with \SI{256}{\byte} messages many of
the sent packets were very small, possibly containing only \ACK{} frames. Thus, the total available
network bandwidth was largely unutilized. When \SI{4096}{\byte} messages are used, more space in
QUIC packets is utilized which increases the throughput. Our investigation, however, did not uncover
why the total throughput of \libmsquic{} does not increase with the number of connections in the
tests with \SI{256}{\byte} messages.

\autoref{fig:04-lab-multi-stream-latency} shows the latencies measured in the LAN environment for
the same four test cases as above. In all tests, the \libmsquic{}-based implementation exhibits a
lower 99th percentile of latency, especially in test cases with high number of connections. The same
measurement on Windows loopback did not yield significantly different results.

\begin{myFigure}{fig:04-lab-multi-stream-latency}{Multiple stream QUIC latency measurements (LAN)}
\begin{mySubfigure}{0.49\linewidth}{fig:04-lab-multi-stream-latency-a}{1 Stream, \SI{256}{\byte} Messages}
\footnotesize
\input{plots/04-lab-multi-stream-latency-a.tex}
\end{mySubfigure}
\begin{mySubfigure}{0.49\linewidth}{fig:04-lab-multi-stream-latency-b}{1 Stream, \SI{4096}{\byte} Messages}
\footnotesize
\input{plots/04-lab-multi-stream-latency-b.tex}
\end{mySubfigure}

\begin{mySubfigure}{0.49\linewidth}{fig:04-lab-multi-stream-latency-c}{32 Stream, \SI{256}{\byte} Messages}
\footnotesize
\input{plots/04-lab-multi-stream-latency-c.tex}
\end{mySubfigure}
\begin{mySubfigure}{0.49\linewidth}{fig:04-lab-multi-stream-latency-d}{32 Stream, \SI{4096}{\byte} Messages}
\footnotesize
\input{plots/04-lab-multi-stream-latency-d.tex}
\end{mySubfigure}
\end{myFigure}

When measuring latency, the benchmarking aplication waits for the server to respond before sending
another message. However, with large number of parallel connections and streams, the total network
traffic becomes similar to that produced when measuring throughput. This was the case especially in
cases where we see significant differences between latencies of the two implementations. The change
in latency can be, therefore, attributed to the frequent and long pauses introduced by the GC which
we have described above. On the other hand, in the test runs where the managed implmentation is
close to that of \libmsquic{}, the GC activity was under 5\% and did not cause long frequent pauses.

By increasing the message size, we increased the probability that the application message will be
affected by the packet loss introduced by the GC pauses. This happens because the message is spread
over multiple QUIC packets and loss of any of those packets will delay the entire message until the
missing part is retransmitted. This explains why increasing the message size drastically affects the
latency with large number of parallel connections.

Increasing the number of streams itself also increases the amount of network traffic, making our
implementation susceptible to the increased GC pauses. However, after certain threshold, increasing
the number of streams does not increase the throughput of the implementation anymore. Instead, it
leads to increase in latency, because it still increases the amount of outstanding parallel
requests.

Lastly, we would like to remind the reader that we were measuring the 99th percentile of the latency
which takes into account only the slowest 1\% of requests. To give more perspective on the latency
distribution on the two implementations, \autoref{fig:04-lab-multi-stream-latency-median} shows the
median measurements of the latency. For brevity, we include only the measurements for the cases with
\SI{4096}{\byte} messages. The median measurements of latencies of our implementation are similar or
lower than that of the \libmsquic{}-based one.

\begin{myFigure}{fig:04-lab-multi-stream-latency-median}{Multiple stream QUIC latency median measurements (LAN)}
\begin{mySubfigure}{0.49\linewidth}{fig:04-lab-multi-stream-latency-median-a}{1 Stream, \SI{4096}{\byte} Messages}
\footnotesize
\input{plots/04-lab-multi-stream-latency-median-a.tex}
\end{mySubfigure}
\begin{mySubfigure}{0.49\linewidth}{fig:04-lab-multi-stream-latency-median-b}{32 Stream, \SI{4096}{\byte} Messages}
\footnotesize
\input{plots/04-lab-multi-stream-latency-median-b.tex}
\end{mySubfigure}
\end{myFigure}

We believe that this difference in the latency distributions between the two implementation is due
to their different architectures. \libmsquic{} spawns several worker threads (depending on the
number of CPU cores) which process events for individual connections from an event queue. The event
queue ensures that the connections are serviced evenly and in stable intervals. When the load on the
server increases, the extra latency is spread evenly across all active connections. This is further
supported by the fact that the median values for the latency of \libmsquic{} are only slightly lower
than its 99th percentile from \autoref{fig:04-lab-multi-stream-latency}.

Our implementation, on the other hand, relies solely on the \dotnet{} \class{Task} scheduler which
schedules all background tasks independently and without any priority information. This allows for
greater variance in the latency of our QUIC implementation. In the future, the architecture of our
implementation should be improved to reduce the variance and, therefore, the 99th percentile of the
latency.

In summary, our QUIC implementation outperforms the \libmsquic{}-based one in scenarios where small
messages are exchanged between client and server using a small number of streams. However, our
implementation does not scale as well as the \libmsquic-based one. Large number of parallel
connections and streams increases pressure on the GC which in turn introduces frequent and large
pauses in the application. Lastly, the latency distribution of our implementation is much less
uniform than that of the \libmsquic{}-based one because of the way background work is scheduled on
the \dotnet{} thread-pool.

\subsection{Single Stream Performance}

In the second test, we compared the single-stream performance of managed QUIC implementations to the
performance of combination of TCP and TLS available via the \TcpClient{} and \SslStream{} \dotnet{}
classes. We will also include the measurements for \libmsquic{} for completeness.

\autoref{fig:04-lab-single-stream-throughput} shows the results of the throughput measurements taken
between the two blade servers connected by \SI{1}{\giga\bit} Ethernet. In both \SI{256}{\byte}
messages (\autoref{fig:04-lab-single-stream-throughput-a}) and \SI{4096}{\byte} messages
(\autoref{fig:04-lab-single-stream-throughput-b}) scenario, TCP implementation outperforms both QUIC
implementations. Also, out of the three implementations, our QUIC implementation is the only one
which is not bottlenecked by the \SI{1}{\giga\bit} LAN\@.

\begin{myFigure}{fig:04-lab-single-stream-throughput}{Single stream throughput measurements (LAN)}
\begin{mySubfigure}{0.49\linewidth}{fig:04-lab-single-stream-throughput-a}{\SI{256}{\byte} Messages}
\footnotesize
\input{plots/04-lab-single-stream-throughput-a.tex}
\end{mySubfigure}
\begin{mySubfigure}{0.49\linewidth}{fig:04-lab-single-stream-throughput-b}{\SI{4096}{\byte} Messages}
\footnotesize
\input{plots/04-lab-single-stream-throughput-b.tex}
\end{mySubfigure}
\end{myFigure}

Because the throughput of some implementations in the previous figure was limited by the
\SI{1}{\giga\bit} network, \autoref{fig:04-single-stream-throughput} shows the results measured over
the loopback interface on the Windows workstation. We can see clearly that TCP still provides
greater throughput than the \libmsquic{}-based implementation\footnote{Without limiting the loopback
  IP packet size by setting the \texttt{IP_USER_MTU} option to 1500 (see
  \autoref{sec:04-localhost-mtu}), the TCP throughput with \SI{4096}{\byte} messages would reach up
  to \SI[per-mode=symbol]{900}{\mebi\byte\per\second}.}

\begin{myFigure}{fig:04-single-stream-throughput}{Single stream throughput measurements (Loopback)}
\begin{mySubfigure}{0.49\linewidth}{fig:04-single-stream-throughput-a}{\SI{256}{\byte} Messages}
\footnotesize
\input{plots/04-single-stream-throughput-a.tex}
\end{mySubfigure}
\begin{mySubfigure}{0.49\linewidth}{fig:04-single-stream-throughput-b}{\SI{4096}{\byte} Messages}
\footnotesize
\input{plots/04-single-stream-throughput-b.tex}
\end{mySubfigure}
\end{myFigure}

By comparing the numbers in \autoref{fig:04-single-stream-throughput-b} to that from
\autoref{fig:04-multi-stream-throughput-d}, we see that the performance of TCP is also greater than
that of \libmsquic{}-based implementation when using 32 parallel streams per connection streams.

\autoref{fig:04-lab-single-stream-latency} shows the latency measurement results in the LAN
environment. These results show that TCP exhibits substantially lower latencies than either QUIC
implementation. Very similar results were obtained also on the Windows workstation.

\begin{myFigure}{fig:04-lab-single-stream-latency}{Single stream latency measurements (LAN)}
\begin{mySubfigure}{0.49\linewidth}{fig:04-lab-single-stream-latency-a}{\SI{256}{\byte} Messages}
\footnotesize
\input{plots/04-lab-single-stream-latency-a.tex}
\end{mySubfigure}
\begin{mySubfigure}{0.49\linewidth}{fig:04-lab-single-stream-latency-b}{\SI{4096}{\byte} Messages}
\footnotesize
\input{plots/04-lab-single-stream-latency-b.tex}
\end{mySubfigure}
\end{myFigure}

It is worth noticing that in \autoref{fig:04-lab-single-stream-latency-a}, the latency of managed QUIC
implementation is very low until it reaches a certain threshold and then increases rapidly as a
consequence of increased GC pauses.

The results presented in this section suggest that neither QUIC implementation can compete with the
\SslStream{} in neither local LAN networks or over the loopback interface (even with the MTU
reduction we explained in \autoref{sec:04-localhost-mtu}). It is possible, however unlikely, that
the performance over real \SI[per-mode=symbol]{10}{\giga\bit\per\second} network will yield
different results.

\subsection{Performance in Simulated Cellular Network}

In the last test, we compared the resilience of QUIC and TCP+TLS implementations in a simulated
network with delay and packet loss. When choosing the values of delay and packet loss parameters, we
tried to approximate the behavior of a 4G cellular broadband network. We performed a trivial
connection speed test of local 4G network using a smartphone and the
\href{https://speedtest.net}{\texttt{speedtest.net}} testing web application. This test reported a
latency of \SI{25}{\milli\second}. We have used this sample as the latency in the simulated network
in our tests.

As for the packet loss in 4G network, in 2012, \citeauthor{measuring4G} have measured
characteristics of 4G networks of major US cellular network carriers and measured packet loss
percentages from 0.004\% to 0.1\%~\cite{measuring4G}. Eventhough the cellular network technology may
have evolved since then, we were unable to find more recent measurements. Therefore, we will use
those two packet loss values in our test parameters.

It should be noted that our approximation of cellular network is not perfect. The latency in a real
network is not constant, but has a random distribution and the \SI{25}{\milli\second} we measured
was only the mean value. Similarly, packet loss does not affect each packet independently. Instead,
packets are often lost in bursts. There are multiple models for modeling more realitic packet loss,
such as the Gilbert-Elliot model~\cite{wiki:burst-error}. However, we were unable to find enough
data to compute appropriate parameters for such models.

In order to simulate the 4G network, we used the \textit{Traffic Control} capabilities of the Linux
kernel. The necessary configuration can be set using the \texttt{tc} utility. For example, the
following command was issued on both machines to configure \SI{25}{\milli\second} lag and 0.004\%
packet loss:

\begin{myVerbatim}
tc qdisc change dev eth0 root netem delay 12ms loss 0.004%
\end{myVerbatim}

Note that the set delay is half of the desired latency, because the traffic shaping done by
\texttt{tc} applies only to outbound traffic.

\autoref{fig:04-lab-loss-throughput} shows the measured throughput for 0.004\% packet loss
(\autoref{fig:04-lab-loss-throughput-a}) and 0.1\% packet loss
(\autoref{fig:04-lab-loss-throughput-b}). As expected, The increase in packet loss leads to decrease
in throughput. In presence of only low packet loss, the TCP implementation exhibits an order of
magnitude greater throughput. However, the throughput of the TCP protocol decreased in greater rate
than that of the two QUIC implementation. The throughput of the two QUIC implementation is
comparable in all tests.

\begin{myFigure}{fig:04-lab-loss-throughput}{Single stream throughput measurements in simulated network}
\begin{mySubfigure}{0.49\linewidth}{fig:04-lab-loss-throughput-a}{\SI{25}{\milli\second} latency, 0.004\% loss}
\footnotesize
\input{plots/04-lab-loss-throughput-a.tex}
\end{mySubfigure}
\begin{mySubfigure}{0.49\linewidth}{fig:04-lab-loss-throughput-b}{\SI{25}{\milli\second} latency, 0.1\% loss}
\footnotesize
\input{plots/04-lab-loss-throughput-b.tex}
\end{mySubfigure}
\end{myFigure}

Upon closer inspection, it turns out that the managed QUIC implementation is actually bottlenecked
by the stream buffering implementation. In \autoref{sec:03-send-stream}, we explained the design
behind the sending part of the stream, namely the \SendStream{} class. The implementation uses up to
8 buffers to store application data before blocking in the \method{Write} or \method{WriteAsync}
method until some data are acknowledged. Each of these buffers is \SI{16}{\kibi\byte}. The
application data must be buffered until they are acknowledged which takes at least one roundtrip,
i.e., \SI{25}{\milli\second}. This implies that the theoretical maximum throughput on a network with
\SI{25}{\milli\second} latency of our implementation is:
$8 \cdot 16 \cdot 1024 \div 0.025 = 5 \si[per-mode=symbol]{\mebi\byte\per\second} $. This hypothesis can be
easily tested by doubling the delay on the simulated network.
\autoref{fig:04-lab-loss-throughput-long} shows measurements with \SI{50}{\milli\second} latencies
(\autoref{fig:04-lab-loss-throughput-a-50}) and \SI{100}{\milli\second} latencies
(\autoref{fig:04-lab-loss-throughput-a-100}). In these variations of the test we see that the
throughput of the managed implementation halves everytime the latency is doubled.

\begin{myFigure}{fig:04-lab-loss-throughput-long}{Single stream throughput measurements in simulated network with long delays}
\begin{mySubfigure}{0.49\linewidth}{fig:04-lab-loss-throughput-a-50}{\SI{50}{\milli\second} latency, 0.004\% loss}
\footnotesize
\input{plots/04-lab-loss-throughput-a-50.tex}
\end{mySubfigure}
\begin{mySubfigure}{0.49\linewidth}{fig:04-lab-loss-throughput-a-100}{\SI{100}{\milli\second} latency, 0.004\% loss}
\footnotesize
\input{plots/04-lab-loss-throughput-a-100.tex}
\end{mySubfigure}
\end{myFigure}

This performance limitation could be fixed by simply increasing the number of buffers used to buffer
application data. However, large number of such buffers would lead to poor memory utilization.
Instead, the amount of data buffered should be changed dynamically based on information such as the
size of the congestion window or the flow control limit advertised by the other endpoint. We leave
the implementation of such dynamic buffering for the future work.

\autoref{fig:04-lab-loss-throughput} shows the latency measurements results. Each implementation
exhibits the same latency for both 0.004\% (\autoref{fig:04-lab-loss-latency-a}) and 0.1\%
(\autoref{fig:04-lab-loss-latency-b}) packet loss. This means that 0.1\% is still low enough to not
affect the 99th percentile of latency.

\begin{myFigure}{fig:04-lab-loss-latency}{Single stream latency measurements in simulated network}
\begin{mySubfigure}{0.49\linewidth}{fig:04-lab-loss-latency-a}{\SI{25}{\milli\second} Lag, 0.004\% Loss}
\footnotesize
\input{plots/04-lab-loss-latency-a.tex}
\end{mySubfigure}
\begin{mySubfigure}{0.49\linewidth}{fig:04-lab-loss-latency-b}{\SI{25}{\milli\second} Lag, 0.1\% Loss}
\footnotesize
\input{plots/04-lab-loss-latency-b.tex}
\end{mySubfigure}
\end{myFigure}

All implementations show almost the same latency in almost all tests. The single exception is our
implementation, which exhibits greater latencies for \SI{4096}{\byte} messages. This is because this
message size does not fit a single packet, our pacer implementation spaces out individual packet one
by one. The \libmsquic{}-based implementation, on the other hand, sends small batches of packets for
each pacer tick, and the small batches are enough to transmit entire \SI{4096}{\byte} messages.

By using only a single stream in a QUIC connection, we do not utilize one of the big advantages of
QUIC over TCP\@. Because QUIC moves stream multiplexing to the transport layer, it significantly
reduces the \gls{head-of-line-blocking} phenomenon.

\section{Interop between QUIC Implementations}

The benchmarking application can also be used to test interoperability of the managed QUIC
implementation and the \libmsquic{}-based implementation. This can be done by starting one
application in \texttt{server} mode using one implementation provider, and another instance of the
application in \texttt{client} mode using the other provider.

More exhaustive interop tests could be achieved using the open source QUIC Interop Test
Runner~\cite{QuicInteropRunner}. However, this level of testing is outside of scope of this thesis
and is subject for future work.

\section{Summary}

\todo{use different section name?}

When compared to \libmsquic{}-based QUIC implementation provider, the managed QUIC implementation
developed in this thesis can provide higher throughput when using small number of parallel
connection while maintaining comparable latencies. However, in the current state, the managed QUIC
implementation does not scale very well and the throughput and observed latencies degrade when large
number of connections is created.

Both QUIC implementation providers have measured lower single-stream performance when compared with
the TCP+TLS stack over the loopback interface. However, it should be noted that the implementations
may behave differently when used over real network.

In case of simulated lossy environment, we measured significantly higher single-stream throughput
using TCP+TLS stack than either QUIC implementation. For smaller messages, the measured latencies
were similar in all implementations, but for larger messages, both QUIC implementations showed
significantly lower latencies than the combination of TCP+TLS\@.

We conclude from these measurements that while QUIC both implementations show some interesting
results, there is still a lot of optimizations required before their performance is comparable to
that of the TCP protocol.
